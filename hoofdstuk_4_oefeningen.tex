\documentclass[lineaire_algebra_oplossingen.tex]{subfiles}
\begin{document}

\chapter{Oefeningen Hoofdstuk 4}
\section{Oefeningen 4.8}
\subsection{Oefening 1}
\subsubsection*{a)}
Deze afbeelding is niet lineair. Tegenvoorbeeld: Kies $\lambda = 2$\\
\[
\lambda L(v) = 2 . 2 . x + 2 \neq 2 . 2 x + 1 = L(\lambda v)
\]

\subsubsection*{b)}
Deze afbeelding is lineair.
\begin{proof}
Kies willekeurige $v_1, v_2 \in \mathbb{R}^2$ en $\lambda_1, \lambda_2 \in \mathbb{R}^2$
\[
L_2(\lambda_1v_1 + \lambda_2v_2)
\]
TODO afmaken
\end{proof}

\subsubsection*{c)}
$$
L_3:\mathbb{R}^2\rightarrow\mathbb{R}:(x,y)\mapsto |x-y|
$$
$$
Ker(L_3)= \{(x,y)|x=y \wedge x,y \in \mathbb{R} \}
$$
$$
Im(L_3)= \{(x,y)|x\neq y \wedge x,y \in \mathbb{R} \}
$$
\begin{center}
Niet isomorf want $L_3$ is niet bijectief.
\end{center}

\subsubsection*{d)}
$L_4$ is niet lineair want $sin(x)$ is niet lineair.
Tegenvoorbeeld:\\
Kies $v_1=(1,0),v_2=\vec{0},\lambda_1 = 2,\lambda_2=0$
\[
L_4(\lambda_1v_1+\lambda_2v_2) = L(2\cdot(1,0))= L((2,0))=(sin(2),0,0)
\]
\[
\neq (2\cdot sin(1),0,0) = 2\cdot L(1,0) = \lambda_1L(v_1)+\lambda_2L(v_2)
\]

\subsubsection*{e)}
$$
L_5:\mathbb{R}^3\rightarrow\mathbb{R}^2:(x,y,z)\mapsto (3y+z,x-y-z)
$$
$$
Ker(L_4)= \{(4\lambda ,\lambda ,-3\lambda)|\lambda \in \mathbb{R} \}
$$
$$
Im(L_4)= \ \{(x ,y ,z)|x,y,z \in \mathbb{R} \}
$$
\begin{center}
Niet isomorf want $L_4$ is niet bijectief.
\end{center}

\subsubsection*{h)}

Deze is niet lineair:
Nemen we $n=2$, dan beschouwen we het tegenvoorbeeld:
\[L_8(2) = 2^2 = 4\]
\[L_8(4*2) = 8^2 = 64\]
\[4*L_8(2) = 16\]
\[L_8(4*2) \neq 4*L_8(2)\]
Volgens Propositie 4.10.1 is dit geen lineaire afbeelding.\\

\subsubsection*{j)}
<TODO Aan Jorik: nakijken!>
\[
ker(ev_a) = \{ f \in \mathbb{R}^{\mathbb{R}} | f(a) = 0\}
\]
\[
Im(ev_a) = \mathbb{R}
\]
Hiervoor moeten we bewijzen dat $\mathbb{R} \subset Im(ev_a)$.
\begin{proof}
Om te bewijzen dat $ev_a$ lineair is nemen we twee vectoren $f_1,f_2 \in \mathbb{R}^{\mathbb{R}}$ en twee scalars $\lambda_1, \lambda_2 \in \mathbb{R}$.
\[
ev_a(\lambda_1f_1 + \lambda_2f_2) = (\lambda_1f_1 + \lambda_2f_2)(a) = \lambda_1f_1(a) + \lambda_2f_2(a) = \lambda_1ev_a(f_2) + \lambda_2ev_a(f_1)
\]
<TODO verwijs naar lemmaatje>
\end{proof}

\subsection{Oefening 2}
Om de kern te vinden zoeken we alle $X$ zodat de volgende vergelijking opgaat.
\[
\begin{pmatrix}
1 & -2 & 0 & 0 & 3\\
2 & -5 & -3 & -2 & 6\\
0 & 5 & 15 & 10 & 0\\
2 & 6 & 18 & 8 & 6
\end{pmatrix}
\cdot X
= \vec{0}
\]
Dit komt overeen met het volgende stelsel.
\[
\left\lbrace
\begin{array}{c c c c c c c c c c}
a &-& 2b &+& 0 &+& 0 &+& 3e &= 0\\
2a &-& 5b &-& 3c &-& 2d &+& 6e &= 0\\
0 &+& 5b &+& 15c &+& 10d &+& 0&= 0\\
2a &+& 6b &+& 18c &+& 8d &+& 6e &= 0
\end{array}
\right.
\]
TODO stelsel oplossen.

Om het beeld te vinden zoeken we alle $Y$ zodat de volgende vergelijking opgaat.
\[
\begin{pmatrix}
1 & -2 & 0 & 0 & 3\\
2 & -5 & -3 & -2 & 6\\
0 & 5 & 15 & 10 & 0\\
2 & 6 & 18 & 8 & 6
\end{pmatrix}
\cdot 
\begin{pmatrix}
a\\b\\c\\d\\e
\end{pmatrix}
= Y
\]
TODO stelsel oplossen.

\subsection{Oefening 3}
\subsubsection*{$K\circ L$}
\[
(K\circ L): \mathbb{R}^3\rightarrow\mathbb{R}^3 (x,y,z)\mapsto (x-y+z,-y,x+z)
\]
\subsubsection*{$L\circ K$}
\[
(K\circ L): \mathbb{R}^2\rightarrow\mathbb{R}^2 (x,y)\mapsto (2y,x+y)
\]

\subsection{Oefening 4}
Zij $(\mathbb{R},V,+)$ een $n$-dimensionale vectorruimte en zij $\alpha = \{a_1,a_2,...,a_n\}$ een basis van $V$.
\subsubsection*{Te Bewijzen}
\[co_\alpha:V\rightarrow \mathbb{R} \text{ is lineair.}\]
\[
\forall v_1,v_2 \in V, \lambda_1,\lambda_2\in \mathbb{R}: co_\alpha(\lambda_1v_1+\lambda_2v_2)=\lambda_1co_\alpha(v_1)+\lambda_2co_\alpha(v_2)
\]
\subsubsection*{Bewijs}
\begin{proof}
Direct bewijs.\\
Zij $v = \sum_{i=1}^n\lambda_ia_i \in V$ een vector geschreven als de lineaire combinatie van de basisvectoren, dan is $(\lambda_1,\lambda_2,...,\lambda_n)$ het beeld ervan door $co_\alpha$.
Kies nu willekeurig $v_1,v_2 \in V$ en $\lambda_1,\lambda_2\in \mathbb{R}$.
\[
co_\alpha(\lambda_1v_1+\lambda_2v_2)= co_\alpha(\lambda_1\sum_{i=1}\lambda_{1i}a_{1i} + \lambda_2\sum_{i=1}\lambda_{2i}a_{i}) = co_\alpha(\sum_{i=1}\lambda_1\lambda_{1i}a_{i} + \sum_{i=1}\lambda_2\lambda_{2i}a_{i})
\]
\[
= co_\alpha(\sum_{i=1}((\lambda_1\lambda_{1i}+\lambda_2\lambda_{2i})a_{i}) = ((\lambda_1\lambda_{11}+\lambda_2\lambda_{21}),(\lambda_1\lambda_{12}+\lambda_2\lambda_{22}),...,(\lambda_1\lambda_{1n}+\lambda_2\lambda_{2n}))
\]
\[
((\lambda_1\lambda_{11}),(\lambda_1\lambda_{12}),...,(\lambda_1\lambda_{1n}))+((\lambda_2\lambda_{21}),(\lambda_2\lambda_{22}),...,(\lambda_2\lambda_{2n}))
\]
\[
\lambda_1(\lambda_{11},\lambda_{12},...,\lambda_{1n})+ \lambda_2(\lambda_{21},\lambda_{22},...,\lambda_{2n}) = \lambda_1co_\alpha(v_1)+\lambda_2co_\alpha(v_2)
\]
\end{proof}
We kunnen $(co_\alpha)_\alpha^\epsilon$ nu generisch opschrijven.
\[
v(co_\alpha)_\alpha^\epsilon = 
\begin{pmatrix}
a_{11} & a_{12} & \cdots & a_{1n}\\
a_{21} & a_{22} & \cdots & a_{2n}\\
\vdots & \vdots & \ddots & \vdots\\
a_{n1} & a_{n2} & \cdots & a_{nn}\\
\end{pmatrix}
\]

\subsection{Oefening 5}
Zij $L:V \rightarrow W$ een lineaire afbeelding en zij $dim V = n$ en $dim W = m$.\\
\subsubsection*{a)}
\textbf{Te Bewijzen}
\[
n > m \Rightarrow L \text{ is niet injectief}
\]
\textbf{Bewijs}
\begin{proof}
Bewijs uit het ongerijmde.\\
Stel $n>m$ en $L$ is injectief.
\[
dim(V) = dim(ker(L)) + dim(Im(L))
\]
Als $L$ injectief is dan $ker(L)=\{0\}$.\footnote{Zie stelling 4.29 p 156} en $Im(L) \subset W$ dus $dim(Im(L))\le dim(W)$. We weten nu het volgende en dit is een contradictie met $n>m$.
\[
dim(V) \le dim(W)
\]
\end{proof}
\subsubsection*{b)}
\textbf{Te Bewijzen}
\[
n < m \Rightarrow L \text{ is niet surjectief}
\]
\textbf{Bewijs}
\begin{proof}
Bewijs uit het ongerijmde.\\
Stel $n<m$ en $L$ is surjectief.
\[
\forall w\in W:\exists v\in V: L(v)=w
\]
Hieruit volgt dat $Im(V)=W$ dus ook dat $dim(Im(V))=dim(W)$.
We kijken nu opnieuw naar stelling 4.31 p 157.
\[
dim(V) = dim(ker(L)) + dim(Im(L))
\]
Als we dit invullen krijgen we het volgende.
\[
dim(V) = dim(ker(L)) + dim(W)
\]
Hieruit volgt dat $dim(V) \ge dim(W)$ want $dim(ker(L)) \ge 0$. Dit is in contradictie met $n<m$.
\end{proof}


\subsection{Oefening 6}
\subsubsection*{(a)}
Zij $L: V \rightarrow V$ een surjectieve lineaire afbeelding. Zij $\beta = \{v_1,v_2,...,v_n\}$ een voortbrengend deel voor $V$.
\[
\forall w \in W, \exists v \in V: w=L(v)
\]
\emph{Te Bewijzen}\\
$W$ wordt voortgebracht door $L(\beta) = \{L(v_1),L(v_2),...,L(v_n)\}$.
\begin{proof}
Elk willekeurig element $v\in V$ kan geschreven worden als lineaire combinatie van de elementen in $\beta$. Er bestaan dus $\lambda_i$ zodat volgende gelijkheid geldt.
\[
v = \sum_{i=1}^n\lambda_iv_i
\]
Beelden we nu beide leden af dan bekomen we volgende gelijkheid.
\[
L(v) = L\left(\sum_{i=1}^n\lambda_iv_i\right)
\]
Omdat $L$ surjectief is bereiken we op deze manier elke $w\in W$.
Omdat $L$ lineair is, is deze bewering equivalent met de volgende.
\[
L(v) = \sum_{i=1}^n\lambda_iL(v_i)
\]
Bovenstaande bewering houdt precies in dat $W$ voortgebracht wordt door $L(\beta)$.
\end{proof}

\subsubsection*{(b)}
Zij $L: V \rightarrow V$ een injectieve lineaire afbeelding. Zij $\beta = \{v_1,v_2,...,v_n\}$ een voortbrengend deel voor $V$.
\[
L(v) = L(w) \Rightarrow v = w
\]
\emph{Te Bewijzen}\\
$L(\beta) = \{L(v_1),L(v_2),...,L(v_n)\}$ is vrij in $W$.
\begin{proof}
Bewijs uit het ongerijmde.\\
Stel dat $L(\beta)$ niet vrij is in $W$, dan bestaan er $\lambda_i$ verschillend van nul zodat volgende gelijkheid geldt.
\[
\sum_{i=1}^n \lambda_i L(v_i) = \vec{0}
\]
Omdat $L$ lineair is geldt $\vec{0} = L(\vec{0}$. Nu zullen we de injectiviteit van $L$ moeten gebruiken om volgende gelijkheid te bekomen.
\[
\sum_{i=1}^n \lambda_i v_i = \vec{0}
\]
Bovenstaande gelijkheid houdt echter in dat $\beta$ niet vrij is in $V$. Contradictie.
\end{proof}

\subsubsection*{(c)}
Zij $L: V \rightarrow V$ een isomorfisme. Zij $\beta = \{v_1,v_2,...,v_n\}$ een basis voor $V$.\\
\emph{Te Bewijzen}\\
$L(\beta) = \{L(v_1),L(v_2),...,L(v_n)\}$ is een basis voor $W$.
\begin{proof}
$L$ is een isomorfisme, dus bijectief en lineair. Bijectiviteit houdt injectiviteit en surjectiviteit in \footnote{Zie \ref{bijectief}}. Een basis in $V$ is vrij in en voortbrengend voor $V$ \footnote{Zie Definitie 3.31 p 104}. Nu kunnen we vorige twee bewezen stellingen toepassen om te bekomen dat $L(\beta)$ ook voortbrengend is voor en vrij is in $W$, wat precies inhoudt dat $L(\beta)$ een basis vormt voor $W$.
\end{proof}

\subsection{Oefening 7}
\subsubsection*{a)}
Omdat $L_1$ lineair is wordt $L_1$ beschreven door een matrix $A$.
\[
A = 
\begin{pmatrix}
a & b \\
c & d
\end{pmatrix}
\text{ zodat }
A
\cdot
\begin{pmatrix}1\\2\end{pmatrix}
=\begin{pmatrix}0\\-1\end{pmatrix}
\text{ en }
A
\cdot
\begin{pmatrix}-1\\-1\end{pmatrix}
=\begin{pmatrix}2\\1\end{pmatrix}
\]
Dit komt overeen met een groter stelsel $B$ waarvan we de oplossingen zoeken.
\[
\left(
\begin{array}{c c c c | c}
1 & 2 & 0 & 0 & 0\\
0 & 0 & 1 & 2 & -1\\
-1 & -1 & 0 & 0 & 2\\
0 & 0 & -1 & -1 & 1\\
\end{array}
\right)
\]
Dit stelsel heeft de volgende oplossingsverzameling.
\[
\left\lbrace
\begin{pmatrix}
-4\\2\\-1\\0
\end{pmatrix}
\right\rbrace
\]
We kunnen de lineaire afbeelding $L_1$ dus als volgt beschrijven.
\[
L_1:\mathbb{R}^2\rightarrow\mathbb{R}^2: (x,y)\mapsto (-4x+2y,-x)
\]

\subsection{Oefening 8}
Om de matrices van de basisverandering te berekenen stellen we voor beide bases een matrix op waarbij ze beide uitgeschreven worden ten opzichte van eenzelfde basis, bijvoorbeeld, de standaardbasis, en als kolommen gebruikt worden in bijhorende matrix. Deze matrices plaatsen we naast elkaar en rijreduceren we tot de linkse matrix de eenheidsmatrix is. De eerste oefening is relatief triviaal, zoals we zullen zien.
\subsubsection*{a)}
We schrijven beide uit in functie van standaardbasis \{(0,1), (1,0)\}, zetten de bekomen co\"ordinaten in een matrix als kolommen (wat hier niet overdreven duidelijk is, maar in $b$ en $c$ hopelijk wel, waarna we beide matrices aan elkaar te zetten en komen tot

\[
\left[
\begin{array}{ c c | c c }
1 & 0 & 5 & 1\\
0 & 1 & 1 & 2
\end{array}
\right]
\]

De linkerkant is reeds de eenheidsmatrix dus geldt dat de matrix 
\[
\left[
\begin{array}{ c c }
5 & 1\\
1 & 2
\end{array}
\right]
\]
de matrix voor de basisverandering van $\alpha$ naar $\beta$. We zeggen altijd dat het de basisverandering van onze basis van de rechtse matrix is naar die van de linkse matrix. Voor de andere matrix, van $\beta$ naar $\alpha$ rijreduceren we dezelfde matrix, maar met de linker en rechterhelft omgewisseld, wat dus neerkomt op de inverse zoeken van zijn andere basisveranderingmatrix. We vinden

\[
\left[
\begin{array}{ c c | c c }
5 & 1 & 1 & 0\\
1 & 2 & 0 & 1
\end{array}
\right]
\]

\[R1 \leftrightarrow R2 \]

\[
\left[
\begin{array}{ c c | c c }
1 & 2 & 0 & 1\\
5 & 1 & 1 & 0
\end{array}
\right]
\]

\[R2 \mapsto R2 - 5*R1 \]

\[
\left[
\begin{array}{ c c | c c }
1 & 2 & 0 & 1\\
0 & -9 & 1 & -5
\end{array}
\right]
\]

\[R2 \mapsto \frac{R2}{-9} \]

\[
\left[
\begin{array}{ c c | c c }
1 & 2 & 0 & 1\\
0 & 1 & \frac{-1}{9} & \frac{5}{9}
\end{array}
\right]
\]

\[R1 \mapsto R1 - 2*R2 \]

\[
\left[
\begin{array}{ c c | c c }
1 & 0 & \frac{2}{9} & \frac{-1}{9}\\
0 & 1 & \frac{-1}{9} & \frac{5}{9}
\end{array}
\right]
\]

\subsubsection*{b)}

We nemen deze keer als standaardbasis \{$\bigl(
\begin{smallmatrix}
1&0\\ 0&0
\end{smallmatrix}
\bigr)$,
$\bigl(
\begin{smallmatrix}
0&1\\ 0&0
\end{smallmatrix}
\bigr)$,
$\bigl(
\begin{smallmatrix}
0&0\\ 1&0
\end{smallmatrix}
\bigr)$,
$\bigl(
\begin{smallmatrix}
0&0\\ 0&1
\end{smallmatrix}
\bigr)$\}, waardoor we de opgegeven bases in volgende matrix kunnen weergeven (met $\beta$ links):

\[
\left[
\begin{array}{ c c c c | c c c c }
1 & 0 & 0 & 0 & 1 & 3 & 1 & 2\\
0 & 0 & 0 & 1 & 1 & 1 & -1 & 1\\
0 & 1 & 0 & 0 & -1 & -1 & 2 & -1\\
0 & 0 & 1 & 0 & 2 & 2 & -3 & 0
\end{array}
\right]
\]

dewelke we door achtereenvolgens $R2 \leftrightarrow R4$ en $R2 \leftrightarrow R3$ toe te passen omzetten tot

\[
\left[
\begin{array}{ c c c c | c c c c }
1 & 0 & 0 & 0 & 1 & 3 & 1 & 2\\
0 & 1 & 0 & 0 & -1 & -1 & 2 & -1\\
0 & 0 & 1 & 0 & 2 & 2 & -3 & 0\\
0 & 0 & 0 & 1 & 1 & 1 & -1 & 1
\end{array}
\right]
\]

De rechtse kant is dus de basisveranderingsmatrix van $\alpha$ naar $\beta$. Rekenen we nu de andere kant uit hebben we

\[
\left[
\begin{array}{ c c c c | c c c c }
1 & 3 & 1 & 2 & 1 & 0 & 0 & 0\\
-1 & -1 & 2 & -1 & 0 & 1 & 0 & 0\\
2 & 2 & -3 & 0 & 0 & 0 & 1 & 0\\
1 & 1 & -1 & 1 & 0 & 0 & 0 & 1
\end{array}
\right]
\]

dewelke we door rijreductie terugbrengen tot

\[
\left[
\begin{array}{ c c c c | c c c c }
1 & 0 & 0 & 0 & \frac{-1}{2} & \frac{-3}{4} & \frac{1}{4} & \frac{1}{4}\\
0 & 1 & 0 & 0 & \frac{1}{2} & \frac{1}{4} & \frac{1}{4} & \frac{-3}{4}\\
0 & 0 & 1 & 0 & 0 & \frac{-1}{3} & 0 & \frac{-1}{3}\\
0 & 0 & 0 & 1 & 0 & \frac{1}{6} & \frac{-1}{2} & \frac{7}{6}
\end{array}
\right]
\]

De rechterkant van deze laatste is dus de basisveranderingsmatrix van basis $\beta$ naar $\alpha$.

\subsubsection*{c)}
We nemen als standaardbasis \{1, X, X$^2$\}, waardoor we als beginmatrix voor ons rekenwerk als makkelijkste deze krijgen:

\[
\left[
\begin{array}{ c c c | c c c }
0 & 0 & 1 & 3 & 4 & 2\\
0 & 1 & 0 & 2 & 0 & 1\\
1 & 0 & 0 & 1 & -1 & 0
\end{array}
\right]
\]

\[R1 \leftrightarrow R3 \]

\[
\left[
\begin{array}{ c c c | c c c }
1 & 0 & 0 & 1 & -1 & 0\\
0 & 1 & 0 & 2 & 0 & 1\\
0 & 0 & 1 & 3 & 4 & 2
\end{array}
\right]
\]

Deze laatste is dus onze basisveranderingsmatrix van $\alpha$ naar $\beta$. Rekenen we de andere richting uit, bekomen we

\[
\left[
\begin{array}{ c c c | c c c }
1 & -1 & 0 & 1 & 0 & 0\\
2 & 0 & 1 & 0 & 1 & 0\\
3 & 4 & 2 & 0 & 0 & 1
\end{array}
\right]
\]

Wat we door rijreductie omvormen tot

\[
\left[
\begin{array}{ c c c | c c c }
1 & 0 & 0 & \frac{4}{3} & \frac{-2}{3} & \frac{1}{3}\\
0 & 1 & 0 & \frac{1}{3} & \frac{-2}{3} & \frac{1}{3}\\
0 & 0 & 1 & \frac{-8}{3} & \frac{7}{3} & \frac{-2}{3}
\end{array}
\right]
\]

De rechterkant van deze laatste is dus de basisveranderingsmatrix van basis $\beta$ naar $\alpha$.

\subsection{Oefening 9}

Berekenen we voor de elementen van de basis hun lineaire afbeelding als volgt:\\

$T(e1) = T( (1,0,0) ) = (2,5,4) = 2 * e1 + 5 * e2 + 4 * e3$\\

$T(e2) = T( (0,1,0) ) = (-3,-1,7) = -3 * e1 + -1 * e2 + 7 * e3$\\

$T(e3) = T( (0,0,1) ) = (4,2,0) = 4 * e1 + 2 * e2$\\

We plaatsen dezen in de matrix van de lineaire afbeelding als kolommen en bekomen zo\\

\[
T^{\alpha}_{\alpha} =
\left[
\begin{array}{c c c}
2 & -3 & -4\\
5 & -1 & 2\\
4 & 7 & 0
\end{array}
\right]
\]\\

Voor de tweede basis zijn 2 elementen van plaats verwisseld. We kunnen deze verwisselen bij de berekening en bekomen dan de vergelijkbare matrix

\[
T^{\beta}_{\beta} =
\left[
\begin{array}{c c c}
4 & -3 & 2\\
2 & -1 & 5\\
0 & 7 & 4
\end{array}
\right]
\]


\subsection{Oefening 10}
\textbf{VERBETERING}
\[
\left\lbrace
\begin{array}{r l}
T(X) &= X-4\\
T(1+X) &= X-3\\
T(X+X^2) &= X^2-7X-3\\
T(X^3) &= X^3-12X^2+6X\\
\end{array}
\right.
\]

\[
T^{\alpha}_{\alpha} =
\left[
\begin{array}{c c c c}
5 & 4 & 0-4 & 18\\
-4 & -3 & -3 & 0\\
0 & 0 & 1 & -12\\
0 & 0 & 0 & 1
\end{array}
\right]
\]\\

\subsection{Oefening 11}
\subsubsection*{a)}
We berekenen hoe we de vectoren in $\alpha$ schrijven als lineaire combinatie van de vectoren in $\beta$.
\[
\left\lbrace
\begin{array}{c c c c}
(1,0,0) &= 1\cdot (1,0,0) &+ 0\cdot (1,1,0) &+ 0\cdot (1,1,1)\\
(0,1,0) &= -1\cdot(1,0,0) &+ 1\cdot (1,1,0) &+ 0\cdot (1,1,1)\\
(0,0,1) &= 0\cdot(1,0,0) &-1\cdot (1,1,0) &+ 1\cdot (1,1,1)
\end{array}
\right.
\]
Hieruit volgt het volgende.
\[
Id_\alpha^\beta = 
\begin{pmatrix}
1 & -1 & 0\\
0 & 1 & -1\\
0 & 0 & 1\\
\end{pmatrix}
\]
Omgekeerd gaat het analoog.
\[
\left\lbrace
\begin{array}{c c c c}
(1,0,0) &= 1\cdot (1,0,0) &+ 0\cdot (0,1,0) &+ 0\cdot (0,0,1)\\
(1,1,0) &= 1\cdot (1,0,0) &+ 1\cdot (0,1,0) &+ 0\cdot (0,0,1)\\
(1,1,1) &= 1\cdot (1,0,0) &+ 1\cdot (0,1,0) &+ 1\cdot (0,0,1)
\end{array}
\right.
\]
\[
Id_\beta^\alpha = 
\begin{pmatrix}
1 & 1 & 1\\
0 & 1 & 1\\
0 & 0 & 1\\
\end{pmatrix}
\]
\subsubsection*{b)}
\[
\left\lbrace
\begin{array}{c l c c c}
T((1,0,0)) &= (1,0,1) &= 1\cdot (1,0,0) &+ 0\cdot (0,1,0) &+ 1\cdot (0,0,1) \\
T((0,1,0)) &= (2,-1,0)&= 2\cdot (1,0,0) &- 1\cdot (0,1,0) &+ 0\cdot (0,0,1)\\
T((0,0,1)) &= (1,0,4) &= 1\cdot (1,0,0) &+ 0\cdot (0,1,0) &+ 4\cdot (0,0,1)\\ 
\end{array}
\right.
\]
\[
Id_\alpha^\alpha = 
\begin{pmatrix}
1 & 2 & 1\\
0 & -1 & 0\\
1 & 0 & 4\\
\end{pmatrix}
\]

\subsubsection*{c)}
\[
\left\lbrace
\begin{array}{c l c c c}
T((1,0,0)) &= (1,0,1) &= 1\cdot (1,0,0) &- 1\cdot (1,1,0) &+ 1\cdot (1,1,1)\\
T((1,1,0)) &= (3,-1,1)&= 4\cdot (1,0,0) &- 2\cdot (1,1,0) &+ 1\cdot (1,1,1)\\
T((1,1,1)) &= (4,-1,5)&= 5\cdot (1,0,0) &- 6\cdot (1,1,0) &+ 5\cdot (1,1,1)\\
\end{array}
\right.
\]
\[
Id_\beta^\beta = 
\begin{pmatrix}
1 & 4 & 5\\
-1 & -2 & -6\\
1 & 1 & 5\\
\end{pmatrix}
\]

\subsection{Oefening 14}
\subsubsection*{a)}
De oplossing bekomen we door de volgende matrix te rijreduceren en de oplossing aan de rechterkant af te lezen. Merk op dat de linkerkant van deze matrix precies de matrix van basisverandering van $\epsilon_{1}$ naar $\alpha$ is.
\[
\left(
\begin{array}{c c c c c | c c c c c}
1 & 1 & 1 & 1 & 1 & 1 & 0 & 0 & 0 & 0\\ 
0 & 1 & 2 & 3 & 4 & 0 & 1 & 0 & 0 & 0\\ 
0 & 0 & 1 & 3 & 6 & 0 & 0 & 1 & 0 & 0\\ 
0 & 0 & 0 & 1 & 4 & 0 & 0 & 0 & 1 & 0\\ 
0 & 0 & 0 & 0 & 1 & 0 & 0 & 0 & 0 & 1\\ 
\end{array}
\right)
\]
\[
\rightarrow
\left(
\begin{array}{c c c c c | c c c c c}
1 & 0 & 0 & 0 & 0 & 1 & -1 & 1 & -1 & 1 \\ 
0 & 1 & 0 & 0 & 0 & 0 & 1 & -2 & 3 & -4\\ 
0 & 0 & 1 & 0 & 0 & 0 & 0 & 1 & -3 & 6\\ 
0 & 0 & 0 & 1 & 0 & 0 & 0 & 0 & 1 & -4\\ 
0 & 0 & 0 & 0 & 1 & 0 & 0 & 0 & 0 & 1\\ 
\end{array}
\right)
\]
De matrix die we zoeken $L_\alpha^{\epsilon_{1}}$.
\[
L_\alpha^{\epsilon_{1}}
=
\left(
\begin{array}{c c c c c}
1 & -1 & 1 & -1 & 1 \\ 
0 & 1 & -2 & 3 & -4\\ 
0 & 0 & 1 & -3 & 6\\ 
0 & 0 & 0 & 1 & -4\\ 
0 & 0 & 0 & 0 & 1\\ 
\end{array}
\right)
\]
Voor de volledigheid berekenen we ook de matrices van basisverandering van $\epsilon_{2}$ naar $\beta$ en omgekeerd, we hebben deze verder nog nodig.
\[
\left(
\begin{array}{c c | c c}
1 & 1 & 1 & 0\\ 
1 & -1 & 0 & 1\\ 
\end{array}
\right)
\rightarrow
\left(
\begin{array}{c c | c c}
1 & 0 & \frac{1}{2} & \frac{1}{2}\\ 
0 & 1 & \frac{1}{2} & -\frac{1}{2}\\ 
\end{array}
\right)
\]
\[
L_{\epsilon_{2}}^\beta =
\begin{pmatrix}
1 & 1\\ 
1 & -1\\
\end{pmatrix}
\text{ en }
L_\beta^{\epsilon_{2}} =
\begin{pmatrix}
\frac{1}{2} & \frac{1}{2}\\ 
\frac{1}{2} & -\frac{1}{2}\\ 
\end{pmatrix}
\]

\subsubsection*{b)}
\[
Id_{\epsilon_1}^\alpha\cdot X  =
\left(
\begin{array}{c c c c c}
1 & 1 & 1 & 1 & 1\\ 
0 & 1 & 2 & 3 & 4\\ 
0 & 0 & 1 & 3 & 6\\ 
0 & 0 & 0 & 1 & 4\\ 
0 & 0 & 0 & 0 & 1\\ 
\end{array}
\right)
\cdot 
\begin{pmatrix}
a\\b\\c\\d\\e
\end{pmatrix}
=
\begin{pmatrix}
0\\1\\0\\1\\1
\end{pmatrix}
=
b
\]
Merk op dat de bovenstaande matrix net de matrix van basisverandering van de standaardbasis naar $\alpha$ omdat $X+X^3+X^4$ volgens de standaard matrix is geschreven in $b$.
De oplossing is de volgende.
\[
X = 
\begin{pmatrix}
-1 \\ 0 \\ 3 \\ -3 \\ 1
\end{pmatrix}
\]

\subsubsection*{c)}
We zien in het voorschrift van $L$ eenvoudig welke lineaire combinaties van de standaardbasissen nodig zijn om $L$ voor te stellen.
\[
L_{\epsilon_{1}}^{\epsilon_2}=
\begin{pmatrix}
1 & 1 & 0 & 0 & 0\\
0 & 0 & 1 & 1 & 1\\
\end{pmatrix}
\]

\subsubsection*{d)}
De matrix van $L$ kunnen we berekenen met de matrices die we al hebben gevonden in vorige delen van de oefening. Kijk naar pagina 149 en \ref{matrix_van_lineaire_afbeeldingen_tov_gegeven_basissen} voor meer uitleg.
De gevraagde matrix $L_\alpha^\beta$ is nu de volgende.
\[
L_\alpha^\beta = Id_{\epsilon_2}^\beta\cdot L_{\epsilon_1}^{\epsilon_2}\cdot Id_{\alpha}^{\epsilon_1}
\]
We verklaren deze formule door hem eens te analyseren. De matrix van de lineaire afbeelding van basis $\alpha$ naar $\beta$ is te herschrijven als, de basisverandering van $\alpha$ naar $\epsilon_1$, gevolgd door de (matrix van de) lineaire afbeelding van $\epsilon_1$ naar $\epsilon_2$, en tenslotte de basisverandering van $\epsilon_2$ naar $\beta$. We hebben als het ware het probleem opgesplitst in tussenstappen die we reeds berekend hebben. Zo bekomen we: 
\[
L = 
\begin{pmatrix}
1 & 1\\ 
1 & -1\\
\end{pmatrix}
\cdot
\begin{pmatrix}
1 & 1 & 0 & 0 & 0\\
0 & 0 & 1 & 1 & 1\\
\end{pmatrix}
\cdot
\begin{pmatrix}
1 & -1 & 1 & -1 & 1 \\ 
0 & 1 & -2 & 3 & -4\\ 
0 & 0 & 1 & -3 & 6\\ 
0 & 0 & 0 & 1 & -4\\ 
0 & 0 & 0 & 0 & 1\\ 
\end{pmatrix}
=
\begin{pmatrix}
0 & 0 & 0 & 0 & 0\\
0 & 0 & -2 & 4 & -6\\
\end{pmatrix}
\]

\subsection{Oefening 16}
Bij deze oefening gebruiken we hetzelfde idee als de laatste subvraag van oefening 14. We willen een lineaire afbeelding van een gegeven (niet-standaard) basis naar een andere (niet-standaard) basis bepalen in de vorm van een matrix.\\
Beschouwen we gegeven bases $\alpha$ en $\beta$ kunnen we ook bijhorende standaardbases kiezen als respectievelijk:
\[ \epsilon_\alpha = \{(1,0,0,0),(0,1,0,0),(0,0,1,0),(0,0,0,1)\}\]
\[ \epsilon_\beta = \{\bigl(\begin{smallmatrix} 1&0\\ 0&0 \end{smallmatrix} \bigr),\bigl(\begin{smallmatrix} 0&1\\ 0&0 \end{smallmatrix} \bigr),\bigl(\begin{smallmatrix} 0&0\\ 1&0 \end{smallmatrix} \bigr),\bigl(\begin{smallmatrix} 0&0\\ 0&1 \end{smallmatrix} \bigr)\}\]
We kunnen nu de gezochte afbeelding $L^\beta_\alpha$, de afbeelding $L$ van basis $\alpha$ naar $\beta$, zo defini\"eren:

\[
L^\beta_\alpha = Id^\beta_{\epsilon_\beta}\cdot L^{\epsilon_\beta}_{\epsilon_\alpha}\cdot Id^{\epsilon_\alpha}_{\alpha}
\]

We zoeken de drie componenten van deze matrixvermenigvuldiging. $Id^\beta_{\epsilon_\beta}$, de basisverandering van $\epsilon_\beta$ naar $\beta$, vinden we door beide bases in een matrix te plaatsen, kolomgewijs, geschreven volgens de standaardbasis, zoals eerder reeds uitgevoerd. We bekomen
\[
\left(
\begin{array}{c c c c | c c c c}
0 & 0 & 1 & -1 & 1 & 0 & 0 & 0 \\ 
0 & 1 & -1 & 1 & 0 & 1 & 0 & 0 \\ 
1 & -1 & 1 & 0 & 0 & 0 & 1 & 0 \\ 
-1 & 1 & 0 & 0 & 0 & 0 & 0 & 1 \\ 
\end{array}
\right)
\]
Waarbij de het linkse deel de basis $\beta$ ten opzichte van de standaardbasis is. Dus:
\[Id^\beta_{\epsilon_\beta} = \left(
\begin{array}{c c c c}
0 & 0 & 1 & -1 \\ 
0 & 1 & -1 & 1 \\ 
1 & -1 & 1 & 0 \\ 
-1 & 1 & 0 & 0 \\ 
\end{array}
\right)
\]
Vervolgens berekenen we $L^{\epsilon_\beta}_{\epsilon_\alpha}$, de afbeelding, gevormd met de standaardbases. We bepalen de beelden van elk element van de eerste basis, en schrijven dit als co\"ordinaten voor de tweede basis, waarna we deze co\"ordinaten als kolommen van een matrix gebruiken. Dit is de afbeeldingsmatrix. Dus:
\[L(1,0,0,0)=\bigl(\begin{smallmatrix} 1&1\\ 0&0 \end{smallmatrix} \bigr) = 1*\bigl(\begin{smallmatrix} 1&0\\ 0&0 \end{smallmatrix} \bigr) + 1*\bigl(\begin{smallmatrix} 0&1\\ 0&0 \end{smallmatrix} \bigr)\]
\[L(0,1,0,0)=\bigl(\begin{smallmatrix}0&0\\ 1&1 \end{smallmatrix} \bigr) = 1*\bigl(\begin{smallmatrix} 0&0\\ 1&0 \end{smallmatrix} \bigr) + 1*\bigl(\begin{smallmatrix} 0&0\\ 0&1 \end{smallmatrix} \bigr)\]
\[L(0,0,1,0)=\bigl(\begin{smallmatrix} 1&0\\ 0&1 \end{smallmatrix} \bigr) = 1*\bigl(\begin{smallmatrix} 1&0\\ 0&0 \end{smallmatrix} \bigr) + 1*\bigl(\begin{smallmatrix} 0&0\\ 0&1 \end{smallmatrix} \bigr)\]
\[L(0,0,0,1)=\bigl(\begin{smallmatrix} 0&1\\ -1&0 \end{smallmatrix} \bigr) = 1*\bigl(\begin{smallmatrix} 0&1\\ 0&0 \end{smallmatrix} \bigr) + (-1)*\bigl(\begin{smallmatrix} 0&0\\ 1&0 \end{smallmatrix} \bigr)\]
Waaruit we de matrix bekomen:
\[L^{\epsilon_\beta}_{\epsilon_\alpha} = 
\left(\begin{array}{c c c c}
1 & 0 & 1 & 0 \\ 
1 & 0 & 0 & 1 \\ 
0 & 1 & 0 & -1 \\ 
0 & 1 & 1 & 0 \\ 
\end{array}
\right)
\]
Rest ons nu nog $Id^{\epsilon_\alpha}_{\alpha}$ te bepalen, dewelke we vinden door elementen van beide bases te schrijven als co\"ordinaten volgens de standaardbasis, en als kolommen voor een matrix te gebruiken. Vervolgens rijreduceren we de kant die de originele basis van de basisverandering voorstelt en geeft de andere kant ons de matrix voor basisverandering. We bekomen:
\[
\left(
\begin{array}{c c c c | c c c c}
1 & 1 & 1 & 1 & 1 & 0 & 0 & 0 \\ 
1 & 1 & 1 & 0 & 0 & 1 & 0 & 0 \\ 
1 & 1 & 0 & 0 & 0 & 0 & 1 & 0 \\ 
1 & 0 & 0 & 0 & 0 & 0 & 0 & 1 \\ 
\end{array}
\right)
\]
\[
\rightarrow
\left(
\begin{array}{c c c c | c c c c}
1 & 0 & 0 & 0 & 0 & 0 & 0 & 1 \\ 
0 & 1 & 0 & 0 & 0 & 0 & 1 & -1 \\ 
0 & 0 & 1 & 0 & 0 & 1 & -1 & 0 \\ 
0 & 0 & 0 & 1 & 1 & -1 & 0 & 0 \\
\end{array}
\right)
\]
We beslissen dus dat
\[Id^{\epsilon_\alpha}_{\alpha} = \left(
\begin{array}{c c c c}
0 & 0 & 0 & 1 \\ 
0 & 0 & 1 & -1 \\ 
0 & 1 & -1 & 0 \\ 
1 & -1 & 0 & 0 \\
\end{array}
\right)
\]

Nu, terug naar onze originele formule, we bekomen door invulling:
\[
L^\beta_\alpha = Id^\beta_{\epsilon_\beta}\cdot L^{\epsilon_\beta}_{\epsilon_\alpha}\cdot Id^{\epsilon_\alpha}_{\alpha}
\]
\[L^\beta_\alpha = \left(
\begin{array}{c c c c}
0 & 0 & 1 & -1 \\ 
0 & 1 & -1 & 1 \\ 
1 & -1 & 1 & 0 \\ 
-1 & 1 & 0 & 0 \\ 
\end{array}
\right) \cdot \left(\begin{array}{c c c c}
1 & 0 & 1 & 0 \\ 
1 & 0 & 0 & 1 \\ 
0 & 1 & 0 & -1 \\ 
0 & 1 & 1 & 0 \\ 
\end{array}
\right) \cdot \left(
\begin{array}{c c c c}
0 & 0 & 0 & 1 \\ 
0 & 0 & 1 & -1 \\ 
0 & 1 & -1 & 0 \\ 
1 & -1 & 0 & 0 \\
\end{array}
\right)
\]
\[L^\beta_\alpha = \left(
\begin{array}{c c c c}
-1 & 0 & 1 & 0 \\ 
2 & -1 & -1 & 1 \\ 
-2 & 3 & 0 & -1 \\ 
1 & -2 & 1 & 0 \\
\end{array}
\right)
\]


\subsection{Oefening 17}
AAN WARD: nog nakijken, gewoon opgeschreven wat er in de oz gezegd werd.


\subsubsection*{a)}
Voor alle $v \in V$ geldt het volgende.
\[
v
=
a_1
\begin{pmatrix}
1\\0\\0\\
\end{pmatrix}
+
a_2
\begin{pmatrix}
1\\0\\0\\
\end{pmatrix}
+
a_3
\begin{pmatrix}
0\\0\\1\\
\end{pmatrix}
\]
Als $v$ nu in $ker(L)$ zit dan geldt ook het volgende.
\[
L(v) = 0 = a_1 \cdot (4\beta_1) + a_2 \cdot (2 \beta_1 + 1\beta_2) + a_3 \cdot (\beta_1 + 3\beta_2)
\]
\[
= (4a_1+2a_2+a_3)\cdot\beta_1 + (a_2+3a_3)\cdot\beta_2
\]
\[
\rightarrow
\left\lbrace
\begin{array}{c c}
(4a_1+2a_2+a_3) &= 0\\
(a_2+3a_3) &= 0\\
\end{array}
\right.
\]
Als we dit oplossen vinden we de coordinaten van kern van $L$ volgens basis $\alpha$.
\[
(a_1,a_2,a_3) = 
\left\{(\frac{5}{4}\lambda,-3\lambda,\lambda) | \lambda \in \mathbb{R}\right\}
\]
Dit moeten we echter nog omvormen tot de kern van $L$.
\[
\frac{5}{4}\cdot
\begin{pmatrix}
1\\0\\0\\
\end{pmatrix}
+
-3
\begin{pmatrix}
1\\0\\0\\
\end{pmatrix}
+
\begin{pmatrix}
0\\0\\1\\
\end{pmatrix}
=
\begin{pmatrix}
-\frac{3}{4}\\-2\\1\\
\end{pmatrix}
\]
De dimensie van $ker(L)$ is $1$ met $\left\{\begin{pmatrix}-\frac{3}{4}\\-2\\1\\\end{pmatrix}\right\}$ als basis.
We weten nu dat de dimensie van het beeld van $L$ gelijk is aan de dimensie van $\mathbb{R}^3$ min de dimensie van de kern\footnote{Zie stelling 4.31 p 157}. $dim(Im(L))= 2$ met basis ... <TODO aanvullen, verduidelijken...>

\subsubsection*{b)}
\[
L(x,y,z) = (4x-y+z,y+2z)
\]

\subsection{Oefening 20}
%TODO THUIS HERMAKEN
We berekenen de rang van $L_{\epsilon_2}^{\epsilon_3}$. Deze is $1$.
Wanneer we nu kijken naar de afbeelding $L_{W}^V$ zien we dat $L(v_1) = w_1$ en $L(v_2) = 0$. $v_2$ Zit dus in de kern van $L$.
Zoek een $w_1,w_2,w_3$ die voldoen.


\subsection{Oefening 24}
\subsubsection*{a)}
We beschouwen een de lineaire afbeelding $L:V\rightarrow W$. We zoeken nu een deelruimte van de bronverzameling $V$ zodat deze verzameling $U$ de volgende eigenschappen heeft:\\
$ker(L)\cap U = \{0\}$, wat betekent dat het enige element dat $U$ en de kern van $L$ gemeenschappelijk hebben het element $0$ is. Dat betekent dus dat we alleen het nulelement nodig hebben uit $V$, en geen andere elementen van de kern van $L$, omdat onze doorsnede dan niet meer zou kloppen.\\
$Im(L) = \{L(u)| u\in U\}$, wat betekent dat het beeld van de originele afbeelding $L$ gelijk moet zijn aan het beeld van de afbeelding $L$ op $U$ in plaats van $V$.\\
We merken dus op dat het enige dat we moeten elimineren uit $V$ om $U$ te bekomen, alle elementen van de kern zijn, die niet het nulelement zijn.\\
We kunnen dit (dixit de assistente) doen door uit te gaan van het bewijs van de dimensiestelling, op pagina 157, waarbij we enkel de eerste paar regels beschouwen, en daarin het toevoegen van de basis van de kern, $\{v_{r+1},\cdots,v_n\}$ elimineren. We bekomen zo de gevraagde deelruimte, want we hebben alle overtollige elementen van de kern ge\"elimineerd.
\subsubsection*{b)}
Te Bewijzen:
Er bestaat een surjectieve lineaire afbeelding $L:V\rightarrow W \Leftrightarrow dimW\le dimV$
\begin{proof}
Bewijs van een equivalentie.
\begin{itemize}
\item $\Rightarrow$\\
We bewijzen dat $dimW$ zeker niet groter is dan $dimV$. Omdat er een surjectieve lineaire afbeelding bestaat van $W$ naar $V$, wordt elke $w\in W$ bereikt. Het beeld van $L$ kunnen we dus defini\"eren als $Im(L)=W$. Indien we dit invullen in de dimensiestelling bekomen we:
\[dim_{\mathbb{R}} V = dim_{\mathbb{R}} (kerL) + dim_{\mathbb{R}} (ImL)\]
\[\Rightarrow dim_{\mathbb{R}} V = dim_{\mathbb{R}} (kerL) + dim_{\mathbb{R}} W\]
\[\Rightarrow dim_{\mathbb{R}} V - dim_{\mathbb{R}} W= dim_{\mathbb{R}} (kerL)\]
Aangezien de dimensie per definitie een (niet-strikt) positief getal is, kan deze gelijkheid alleen gelden als $dim W \leq dim V$, anders wordt de dimensie van de kern negatief.
\item $\Leftarrow$\\
Nemen we $\{v_1,v_2,\cdots,v_n\}$ een basis van $V$ en $\{w_1,w_2,\cdots,w_m\}$ een basis van $W$ met $n=dim V$ en $m=dim W$ en $n \geq m$.\\
We defini\"eren nu de lineaire afbeelding $L:V \rightarrow W$ als:
\[ L(v_i) = \left\{
  \begin{array}{l l}
    w_i & \quad \text{als $i \leq m$}\\
    0 & \quad \text{als $i > m$}\\
  \end{array} \right.
\]
Deze afbeelding is surjectief omdat alle elementen van de basis van $W$ bereikt kunnen worden en via lineaire combinatie alle elementen van $W$ zelf. Zo hebben we dus een surjectieve afbeelding gedefini\"eerd die bestaat als $dim W \leq dim V$.
\end{itemize}
\end{proof}


\subsection{Oefening 25}
Zij $A \in \mathbb{R}^{m\times n}$ en $X,B \in \mathbb{R}^{n\times 1}$.

\subsubsection*{Te Bewijzen}
\begin{center}
$A\cdot X = B$ Heeft een oplossing.
\end{center}
\[\Leftrightarrow\]
\[Rang(A) = Rang([A B])\]

\subsubsection*{Bewijs}
\begin{proof}
Bewijs van een equivalentie.
\begin{itemize}
\item $\Rightarrow$\\
We weten dat $A\cdot X = B$ een oplossing heeft als en slechts als $B \in Im(L)$ geldt.\footnote{Zie Stelling 4.43 p 164 (\ref{4.43})} Hierin is $L$ de lineaire afbeelding gepaald door $A$ te beschouwen als de matrix van $L$ ten opzichte van de standaardbasissen.
Bovendien weten we dat de rang van een matrix gelijk is aan de kolomrang van die matrix.
Dit is precies de dimensie van de kolomruimte van die matrix.\\\\
Omdat er een $X$ bestaat zodat $X$ een oplossing is van $A\cdot X = B$ zit die $X$ in de kolomruimte van $A$. Dit houdt in dat $B$ een lineaire combinatie is van de kolommen van $A$. Bijgevolg is de kolomruimte van $A$ gelijk aan die van $[A B]$ en volgens de definitie van rang is de rang van $A$ dus gelijk aan die van $[A B]$\footnote{Zie Opmerking 4.38 p 160.}.

\item $\Leftarrow$\\
Omdat de rang van $A$ gelijk is aan die van $[A B]$, is de kolomruimte van $A$ gelijk aan die van $[A B]$. Dit betekent dat $B$ lineair afhankelijk is van de kolommen van $A$. Bijgevolg is de $X$ uit $A\cdot X = B$ niet onbestaand en is $A\cdot X = B$ dus oplosbaar.
\end{itemize}
\end{proof}


\subsection{Oefening 26}

Volgens de 2 stellingen $4.40$ en $4.41$ nxn matrix is uitgesloten(4.42)

gegeven: $ A ^ {m x n} $

1) Volgens 4.40: 

$rang(A) = n $ als rijen lineair onafhankelijk zijn.
Deze zijn vrij indien $x != -2 of -3$ Hierdoor zouden de rijen immers lineair afhankelijk worden! 

2) volgens 4.41

$rang(A) = m $ als de kolommen lineair onafhankelijk zijn. De eerste 2 rijen zijn duidelijk niet vrij, dus kunnen we dit geval uitsluiten.


\section{Opdrachten}

\subsection{Opdracht 4.5 p 136}
\label{4.5}
\subsubsection*{a)}
\[
\mathbb{C} = (\mathbb{R},\mathbb{C},+)
\]
\textbf{Te Bewijzen}\\
\[
\forall v_1,v_2 \in \mathbb{C}: L(v_1+v_2) = L(v_1)+L(v_2) 
\]
\[
\forall v\in \mathbb{C}, \lambda \in \mathbb{R}: L(\lambda v) = \lambda L(v)
\]
\textbf{Bewijs}\\
\begin{proof}
Rechtstreeks bewijs\\
Neem twee willekeurige vectoren $v_1 = x_1+y_1i$ en $v_2=x_2+y_2i$.
\[
L(v_1+v_2)=L( (x_1+y_1i) + (x_2+y_2i)) = L((x_1+x_2) + i(y_1+y_2))
\]
\[
= (x_1+x_2)-i(y_1+y_2) = (x_1-y_1i) + (x_2-y_2i)
\]
\[
= L(x_1+y_1i) + L(x_2+y_2i) = L(v_1)+L(v_2)
\]
Tot zover deel $1$ van het bewijs.\\
Neem nu een willekeurige vector $v=x+yi$ en een willekeurige scalar $\lambda$.
\[
L(\lambda v) = L(\lambda (x+yi)) = L(\lambda x + \lambda yi) = \lambda x - \lambda yi
\]
\[
\lambda (x-yi) = \lambda L(x+yi) = \lambda L(v)
\]
Nu hebben we bewezen dat $T$ een lineaire afbeelding is over $(\mathbb{R},\mathbb{C},+)$.
\end{proof}

\subsubsection*{b)}
\[
\mathbb{C} = (\mathbb{C},\mathbb{C},+)
\]
$T$ is geen lineaire afbeelding over $(\mathbb{C},\mathbb{C},+)$ want $T$ voldoet niet aan de definitie van een lineaire afbeelding\footnote{Zie definitie 4.1 p 130}. \\
De volgende formule geldt namelijk niet.
\[\forall v\in \mathbb{C}, \lambda \in \mathbb{C}: L(\lambda v) = \lambda L(v)\]
\textbf{Tegenvoorbeeld}\\
Kies $\lambda = i \in \mathbb{C}$ en $v = 2+5i\in \mathbb{C}$.
Het linkerlid is dan gelijk aan de volgende expressie.
\[
L(i(2+5i)) = L(2i-5) = -5 -2i
\]
Terwijl het rechterlid gelijk is aan de volgende expressie.
\[
iL(2+5i) = i(2-5i) = 2i+5
\]
Deze expressies zijn niet gelijk.


\subsection{Opdracht 4.9 p 140}
\label{4.9}

\subsubsection*{1)}
We bepalen eerst de co\"ordinaten van $(2,0,5)$ ten opzichte van de eerste basis. Aangezien het over een transformatie gaat is $\beta$ zowel de eerste als de tweede basis. 
Na het opstellen van een stelsel en rijreductie vinden we dat de co\"ordinaten $\left(\frac{12}{5},\frac{3}{5},\frac{-2}{5}\right)$ zijn.

Voeren we nu de vermenigvuldiging van de matrix van $L$ met de bekomen co\"ordinaten vinden we:\\
$$
\begin{pmatrix}
1&0&5\\
0&-2&2\\
1&-2&7
\end{pmatrix}
\cdot
\begin{pmatrix}
\frac{12}{\strut5}\\ \frac{3}{5}\\ \frac{\strut -2}{5}
\end{pmatrix}
=
\begin{pmatrix}
\strut \frac{2}{5}\\ -2\\ \strut \frac{-8}{5}
\end{pmatrix}
$$
Dit is dan ook het antwoord op de vraag.

\subsubsection*{2a)}
We gaan na of de verzameling lineair onafhankelijk is door de determinant te berekenen:
$$
\begin{vmatrix}
2 & 0 & -2\\
0&3&3\\
3&-2&-4
\end{vmatrix}
= 6
$$
Aangezien de determinant niet 0 is deze verzameling niet lineair afhankelijk en dus vrij.
\\
\\
Aangezien we weten dat de dimensie van $\mathbb{R}_{<2}$ gelijk is aan 3. En we hier 3 vrije vectoren hebben, kunnen we via Eig. 3.40 besluiten dat het een basis is.
\\
\\
Willen we nu toch voortbrengend bewijzen. Dan moeten we nagaan of er willekeurige $k,l,m$ bestaan zodat we voor een willekeurige vector $aX^2+bX+c$ met de gegeven basis $(v_1,v_2,v_3)$ een combinatie vinden:
$$
aX^2+bX+c = kv_1+lv_2+mv_3
$$
$$
\left\lbrace
\begin{array}{r c}
k-2m &= a\\
3l+m &= b\\
3k-2l+3m &= c
\end{array}
\right.
$$
Als we dit oplossen en we krijgen een oplossing dan wil dit zeggen dat er voor elke willekeurige vector zo'n combinatie bestaat, en dit wil zeggen dat het voortbrengend is.

\subsubsection*{2b)}
Beschouwen we de eerste basis in functie van de tweede bekomen we:\\

$D(X^2+3) = \bigl(
\begin{smallmatrix}
1&-3\\ 1&3
\end{smallmatrix}
\bigr) = \bigl(
\begin{smallmatrix}
1&0\\ 0&0
\end{smallmatrix}
\bigr) -3* \bigl(
\begin{smallmatrix}
0&1\\ 0&0
\end{smallmatrix}
\bigr) + \bigl(
\begin{smallmatrix}
0&0\\ 1&0
\end{smallmatrix}
\bigr) +3* \bigl(
\begin{smallmatrix}
0&0\\ 0&1
\end{smallmatrix}
\bigr)$\\

$D(-2X^2+3X-4) = \bigl(
\begin{smallmatrix}
1&7\\ -5&-1
\end{smallmatrix}
\bigr) = \bigl(
\begin{smallmatrix}
1&0\\ 0&0
\end{smallmatrix}
\bigr) + 7* \bigl(
\begin{smallmatrix}
0&1\\ 0&0
\end{smallmatrix}
\bigr) - 5* \bigl(
\begin{smallmatrix}
0&0\\ 1&0
\end{smallmatrix}
\bigr) - \bigl(
\begin{smallmatrix}
0&0\\ 0&1
\end{smallmatrix}
\bigr)$\\

$D(3X-2) = \bigl(
\begin{smallmatrix}
3&5\\ -3&1
\end{smallmatrix}
\bigr) = 3* \bigl(
\begin{smallmatrix}
1&0\\ 0&0
\end{smallmatrix}
\bigr) + 5* \bigl(
\begin{smallmatrix}
0&1\\ 0&0
\end{smallmatrix}
\bigr) - 3* \bigl(
\begin{smallmatrix}
0&0\\ 1&0
\end{smallmatrix}
\bigr) + \bigl(
\begin{smallmatrix}
0&0\\ 0&1
\end{smallmatrix}
\bigr)$\\

Nemen we dus de bekomen co\"effici\"enten met de tweede basis, vormen we de matrix van $L$ op de twee basissen als volgt:\\

\[
L^{\beta_1}_{\beta_2} =
\left[
\begin{array}{c c c}
1 & 1 & 3\\
-3 & 7 & 5\\
1 & -5 & -3\\
3 & -1 & 1
\end{array}
\right]
\]

\subsubsection*{2c)}
Door direct te berekenen bekomen we\\

$ L(-X^2+6X-3) = \bigl(
\begin{smallmatrix}
-1+6&6+3\\ -1-6&6-3
\end{smallmatrix}\bigr) = \bigl(
\begin{smallmatrix}
5&9\\ -7&3
\end{smallmatrix}\bigr)$\\

Om ditzelfde met de matrix van $L$ te berekenen bepalen we eerst de co\"ordinaten van $-X^2+6X-3$ ten opzichte van de eerste basis.
Na het opstellen van een stelsel en rijreductie vinden we dat de co\"ordinaten $(1,1,1)$ zijn.

Voeren we nu de vermenigvuldiging van de matrix van $L$ met de bekomen co\"ordinaten vinden we:\\

$\begin{pmatrix}
1 & 1 & 3\\
-3 & 7 & 5\\
1 & -5 & -3\\
3 & -1 & 1
\end{pmatrix} * \begin{pmatrix}
1\\
1\\
1
\end{pmatrix} = \begin{pmatrix}
5\\
9\\
-7\\
3
\end{pmatrix}$\\

Deze resulterende co\"rdinaten gebruiken we voor de tweede basis, waardoor we $\bigl(
\begin{smallmatrix}
5&9\\ -7&3
\end{smallmatrix}\bigr)$ bekomen, hetzelfde resultaat als de directe berekening.


\subsection{Opdracht 4.19 p 147}
\label{4.19}
Deze deelruimte is isomorf met $\mathbb{R}^n$. Dit is makkelijk in te zien omdat we enkel de $0$ componenten moeten weg laten. 


\subsection{Opdracht 4.25 p 154}
\label{4.25}
We stellen beide basissen voor ten opzichte van de standaardbasis, en zetten de co\"ordinaatvectoren die we dan bekomen in de kolommen van een matrix.
Vervolgens rijreduceren we die matrix en vinden we aan de rechterkant de matrix van basisverandering van $\beta_s$ naar $\beta$.
\[
\left(
\begin{array}{c c c c | c c c c}
1 & 0 & 0 & 0 & 1 & 0 & 0 & 0\\
0 & 1 & 0 & 0 & 0 & 1 & -1 & 2\\
0 & 0 & 1 & 0 & 0 & 0 & 1 & -2\\
0 & 0 & 0 & 1 & 0 & 0 & 0 & 1\\
\end{array}
\right)
\]
\[
Id_{\beta_s}^\beta = 
\begin{pmatrix}
1 & 0 & 0 & 0\\
0 & 1 & -1 & 2\\
0 & 0 & 1 & -2\\
0 & 0 & 0 & 1
\end{pmatrix}
\]
Inverteren we deze matrix, dan krijgen we de matrix van basisverandering in de omgekeerde richting.
\[
Id_{\beta}^{\beta_s} = 
\begin{pmatrix}
1 & 0 & 0 & 0\\
0 & 1 & 1 & 0\\
0 & 0 & 1 & 2\\
0 & 0 & 0 & 1
\end{pmatrix}
\]
Nu moeten we deze matrices, samen met de gegeven matrix nog samenstellen om de matrixvoorstelling van de lineaire transformatie te vinden.
\[
B = Id_{\beta}^{\beta_s}\cdot  A\cdot Id_{\beta_s}^\beta
\]
%TODO uitschrijven


\subsection{Opdracht 4.37 p 160}
\label{4.37}
In Gevolg 4.35 op pagina 159 staat net dat zo'n transformatie niet bestaat als hij eindig dimensionaal moet zijn. We moeten dus oneindigdimensionaal gaan denken.\\
Beschouw de vectorruimte $(\mathbb{R},\mathbb{R}^\mathbb{R},+)$. De integraalafbeelding is een lineaire transformatie van deze vectorruimte. De integraalafbeelding is injectief, maar niet surjectief want er bestaan niet afleidbare functies.
%TODO zeker?



\end{document}