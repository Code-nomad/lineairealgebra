\documentclass[lineaire_algebra_oplossingen.tex]{subfiles}
\begin{document}

\section{Zelfreflectie 5}
\subsection{Oefening 1}
De karacteristieke veelterm van een lineaire afbeelding is onafhankelijk van de gekozen basis. \footnote{Zie Gevolg 5.9 p 182.}
Het spectrum dus ook.

\subsection{Oefening 2}
Fout.
\[
1 \le d(\lambda) \le m(\lambda) \le n
\]

\subsection{Oefening 3}
Verschillende matrixvoorstellingen kunnen bekomen worden door er een matrix van basisverandering mee te vermenigvuldigen.
Die basisveranderingen zijn inverteerbaar. 
Verschillende matrixvoorstellingen zijn dus gelijkvormig.

\subsection{Oefening 4}
Enkel de derde is niet diagonaliseerbaar. De rest heeft een enkelvoudig spectrum.

\subsection{Oefening 5}





\end{document}