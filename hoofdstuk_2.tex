\documentclass[10pt,a4paper]{article}

\usepackage[english]{babel}
\usepackage{amsmath}
\usepackage{amsfonts}
\usepackage{amssymb}
\usepackage{amsthm}
\makeatletter
\renewcommand*\env@matrix[1][*\c@MaxMatrixCols c]{%
  \hskip -\arraycolsep
  \let\@ifnextchar\new@ifnextchar
  \array{#1}}
\makeatother
\title{Oplossingen Lineaire Algebra 2013}
\author{TODO}


\begin{document}

\maketitle
\pagebreak
\tableofcontents
\pagebreak

\section{2.4}
\subsection*{oef 1}
\subsubsection*{j)}
Door rijoperaties:\\
\[
\begin{vmatrix}
3 & -7 & 8 & 9 & -6\\
0 & 2 & -5 & 7 & 3\\
0 & 0 & 1 & 5 & 0\\
0 & 0 & 2 & 4 & -1\\
0 & 0 & 0 & -2 & 0
\end{vmatrix}
\overset{R4 \longmapsto R4-2R3}{=}
\begin{vmatrix}
3 & -7 & 8 & 9 & -6\\
0 & 2 & -5 & 7 & 3\\
0 & 0 & 1 & 5 & 0\\
0 & 0 & 0 & -6 & -1\\
0 & 0 & 0 & -2 & 0
\end{vmatrix}
\]
\[
\overset{R5 \longmapsto R5-\frac{-2}{6}R4}{=}
\begin{vmatrix}
3 & -7 & 8 & 9 & -6\\
0 & 2 & -5 & 7 & 3\\
0 & 0 & 1 & 5 & 0\\
0 & 0 & 0 & -6 & -1\\
0 & 0 & 0 & 0 & \frac{-2}{6}
\end{vmatrix}
=
3\cdot 2 \cdot 1 \cdot (-6) \cdot (-\frac{2}{6}) = 12
\]\\
Door ontwikkeling:
\[
\begin{vmatrix}
3 & -7 & 8 & 9 & -6\\
0 & 2 & -5 & 7 & 3\\
0 & 0 & 1 & 5 & 0\\
0 & 0 & 2 & 4 & -1\\
0 & 0 & 0 & -2 & 0
\end{vmatrix}
=
3 \cdot
\begin{vmatrix}
2 & -5 & 7 & 3\\
0 & 1 & 5 & 0\\
0 & 2 & 4 & -1\\
0 & 0 & -2 & 0
\end{vmatrix}
=
3 \cdot 2 \cdot
\begin{vmatrix}
1 & 5 & 0\\
2 & 4 & -1\\
0 & -2 & 0
\end{vmatrix}
\]
\[
3\cdot 2\cdot \left( 
\begin{vmatrix}
4 & -1\\
-2 & 0
\end{vmatrix}
-2\cdot
\begin{vmatrix}
5 & 0\\
-2 & 0
\end{vmatrix}
\right)
=
3\cdot 2 \cdot 2 = 12
\]

\subsection*{oef 3}
\begin{center}
$det(AB) = 0$ als $det(A) = 0$ of $det(B) = 0$
\\
\[det(A) = \begin{vmatrix}
x + 2 & 3x\\
3 & x + 2
\end{vmatrix}
 = (x + 2)^2 - 9x = x^2 -5 x + 4
\]
$x^2 -5 x + 4 = 0$ voor $x_1, x_2$ met $x_{1,2} = \frac{5 \pm 3}{2}$ ($x_1 = 1$ en $x_2 = 4$)
\\
\[det(B) = \begin{vmatrix}
x & 0\\
5 & x + 2
\end{vmatrix}
 = x \cdot (x + 2) = x^2 + 2x\]
$x^2 + 2x$ voor $x_3, x_4$ met $x_3 = -2$ en $x_4 = 0$
\end{center}
\subsubsection*{Besluit}
$det(AB) = 0$ voor $x$ met $x \in \{-2, 0, 1, 4\}$.

\section{Opdrachten}
\subsection*{2.14}
\subsection*{2.16}
\subsubsection*{1.}
Permutaties van $\{1,2,3,4\}$:
$$
\{\{1, 2, 3, 4\}, \{1, 2, 4, 3\}, \{1, 3, 2, 4\}, \{1, 3, 4, 2\},$$
$$ \{1, 4, 2, 3\}, \{1, 4, 3, 2\}, \{2, 1, 3, 4\}, \{2, 1, 4, 3\}, $$ 
$$\{2, 3, 1, 4\}, \{2, 3, 4, 1\}, \{2, 4, 1, 3\}, \{2, 4, 3, 1\}, $$
$$\{3, 1, 2, 4\}, \{3, 1, 4, 2\}, \{3, 2, 1, 4\}, \{3, 2, 4, 1\}, $$ $$
\{3, 4, 1, 2\}, \{3, 4, 2, 1\}, \{4, 1, 2, 3\}, \{4, 1, 3, 2\}, $$
$$\{4, 2, 1, 3\}, \{4, 2, 3, 1\}, \{4, 3, 1, 2\}, \{4, 3, 2, 1\}\}
$$
Wat men eigenlijk doet voor een algemene matrix met deze permutaties is het volgende: begin met de standaard volgorde dit is: $\{1,2,3,4\}$.\\ Zet hieronder telkens een permutatie van deze volgorde, bijvoorbeeld:
$$\begin{pmatrix}
1 & 2 & 3 & 4\\
2 & 3 & 4 & 1
\end{pmatrix}
$$
Nu zijn de paarsgewijze koppels telkens de elementen uit de $4x4$ matrix die je moet nemen: $a_{12}$, $a_{23}$, $a_{34}$, $a_{41}$.
\end{document}
