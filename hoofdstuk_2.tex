\documentclass[lineaire_algebra_oplossingen.tex]{subfiles}
\begin{document}


\part{Hoofdstuk 2}
\section{Bewijzen uit het boek}
\subsection{Stelling 2.2 p 57}
Zij $f : \mathbb{R}^{n\times n} \rightarrow \mathbb{R}:A\mapsto f(A)$ een afbeeldingen die aan definitie 2.1 op pagina 57 voldoet.
\subsubsection*{Te Bewijzen}
\begin{enumerate}
\item $f$ is lineair in de $i$de rij voor elke $i\in \{1,2,...,n\}$
\[
f\left(
\begin{pmatrix}
R_1 \\ \vdots \\ \lambda R_i + \mu R_i' \\ \vdots \\R_n
\end{pmatrix}
\right)
=
\lambda
f\left(
\begin{pmatrix}
R_1 \\ \vdots \\ R_i\\\vdots \\R_n
\end{pmatrix}
\right)
+
\mu 
f
\left(
\begin{pmatrix}
R_1 \\ \vdots \\ R_i' \\\vdots \\R_n
\end{pmatrix}
\right)
\] 
\item Als er in $A$ een nulrij is, of als er twee gelijke rijen zijn, dan geldt $f(A)=0$.
\end{enumerate}
\subsubsection*{Bewijs}
\begin{proof}
\begin{enumerate}
\item We verwisselen rij $1$ met rij $i$ om de volgende uitdrukking te bekomen.
\[
f\left(
\begin{pmatrix}
R_1 \\ \vdots \\ \lambda R_i + \mu R_i' \\ \vdots \\R_n
\end{pmatrix}
\right)
=
-
f\left(
\begin{pmatrix}
\lambda R_i + \mu R_i' \\ \vdots \\ R_1 \\ \vdots \\R_n
\end{pmatrix}
\right)
\] 
We weten dat lineariteit geldt voor de eerste rij\footnote{Zie D-3 in Definitie 2.1 p 57}.
\[
-
f\left(
\begin{pmatrix}
\lambda R_i + \mu R_i' \\ \vdots \\ R_1 \\ \vdots \\R_n
\end{pmatrix}
\right)
=
-
\lambda
f\left(
\begin{pmatrix}
R_i \\ \vdots \\ R_1\\\vdots \\R_n
\end{pmatrix}
\right)
-
\mu 
f
\left(
\begin{pmatrix}
R_i' \\ \vdots \\ R_1 \\\vdots \\R_n
\end{pmatrix}
\right)
\] 
We verwisselen in de matrixen in beide termen opnieuw rij $1$ met rij $i$.
\[
\lambda
f\left(
\begin{pmatrix}
R_1 \\ \vdots \\ R_i\\\vdots \\R_n
\end{pmatrix}
\right)
+
\mu 
f
\left(
\begin{pmatrix}
R_1 \\ \vdots \\ R_i' \\\vdots \\R_n
\end{pmatrix}
\right)
\] 
\item
Gevalsonderscheid\\
Geval 1: $A$ heeft twee gelijke rijen.\\
Stel $R_i = R_j$ zijn gelijke rijen. We kunnen nu $R_i$ en $R_j$ omwisselen om $A'$ te bekomen. Nu zal $f(A)$ het tegengestelde zijn van $f(A')$ maar $A$ en $A'$ zijn gelijk omdat de verwisselde rijen gelijk zijn.
\[
f(A) = - f(A) \Leftrightarrow f(A)=0
\]
Geval 2: $A$ heeft een nulrij.\\
Stel dat $R_i$ een nulrij is in $A$. Vervang nu $R_i$ door $R_i'  = \lambda R_i$ (met $\lambda \neq 0$) om $A'$ te bekomen. $A$ is gelijk aan $A'$ want $\forall \lambda \in \mathbb{R}: \lambda*0=0$. Bovendien geldt door lineariteit van de $i$ de rij (zie hierboven) het volgende.
\[
f(A) = \lambda f(A')
\]
Dit kan enkel als $f(A) = 0$.
\end{enumerate}
\end{proof}

\subsection{Stelling 2.3 p 58}
Zij $f : \mathbb{R}^{n\times n} \rightarrow \mathbb{R}:A\mapsto f(A)$ een afbeeldingen die aan definitie 2.1 (determinant afbeelding) op pagina 57 voldoet.
\subsubsection*{Te Bewijzen}
\begin{enumerate}
\item Als men op de matrix $A$ een elementaire rijoperatie uitvoert van het type $R_i\rightarrow R_i + \lambda R_j$ met $j\neq i$, dan verandert $f(A)$ niet.
\item Als $E$ een elementaire matrix is, dan is 
\[
\left\lbrace
\begin{array}{c l l}
f(E_1) &= 1 & \text{ als } E_1 \text{ correspondeert met } R_i \rightarrow R_i + \lambda R_j\\
f(E_2) &= -1 & \text{ als } E_2 \text{ correspondeert met } R_i \leftrightarrow R_j\\
f(E_3) &= \lambda & \text{ als } E_2 \text{ correspondeert met } R_i \rightarrow \lambda R_i\\
\end{array}
\right.
\]
\item Als $E$ een elementaire matrix is, geldt steeds dat $f(E\cdot A) = f(E)\cdot f(A)$.
\end{enumerate}
\subsubsection*{Bewijs}
\begin{proof}
\begin{enumerate}
\item We weten dat de afbeelding $f$ lineair is in de $i$-de rij\footnote{Zie Stelling 2.2 p 57 D-4} dus het volgende geldt.
\[
f
\left(
\begin{pmatrix}
R_1\\ \vdots\\R_i+\lambda R_j\\\vdots\\R_j\\\vdots\\R_n
\end{pmatrix}
\right)
=
f
\left(
\begin{pmatrix}
R_1\\ \vdots\\R_i\\\vdots\\R_j\\\vdots\\R_n
\end{pmatrix}
\right)
+
\lambda
f
\left(
\begin{pmatrix}
R_1\\ \vdots\\R_j\\\vdots\\R_j\\\vdots\\R_n
\end{pmatrix}
\right)
=
f(A)
\]
De laatste gelijkheid geldt omdat de matrix in de tweede term van het middelste lid twee gelijke rijen heeft, namelijk $R_i$ en $R_j$\footnote{Zie Stelling 2.2 p 57 D-5}.

\item
Kijk even terug naar pagina 36 om te zien hoe elementaire matrices er uitzien. Onthoud bovendien dat $f(\mathbb{I}_n)=1$ \footnote{Zie definitie 2.1 p 57 D-1}.
\begin{enumerate}
\item $E_1$ bekomen we door op $\mathbb{I}_n$ de elementaire rijoperatie uit te voeren van het type $R_i\rightarrow R_i + \lambda R_j$. In Het eerste deel van dit bewijs dat deze rijoperatie niets aan de waarde van $f$ verandert dus $f(E_1) = f(\mathbb{I}) = 1$.
\item $E_2$ bekomen we door in $\mathbb{I}_n$ twee rijen van plaats te veranderen. Wanneer we twee rijen van plaats veranderen keert het teken van $f$ om dus $f(E_2) = -f(\mathbb{I}_n) = -1$.\footnote{Zie Definitie 2.1 p 57 D-2}\\\\
\item 1$E_3$ bekomen we door in $\mathbb{I}_n$ rij $R_i$ te vervangen door $R_i' = \lambda R_i$. Uit de lineariteit van $f$ volgt dat $f(E_3) = \lambda f(\mathbb{I}_n) = \lambda$\footnote{Zie Stelling 2.2 p 57 D-4}.
\end{enumerate}

\item
We moeten een gevalsonderscheid maken.
\begin{enumerate}
\item $E$ correspondeert met  $R_i \rightarrow R_i + \lambda R_j$.\\
$A' = E\cdot A$ is de matrix waarin $E$ uitgevoerd is op $A$. In het eerste deel van dit bewijs hebben we aangetoond dat de waarde van $f$ gelijk blijft als we $E$ uitvoeren op $A$. Bovendien weten we dat $f(E)=1$ uit deel twee van dit bewijs.
\[
f(E\cdot A) = f(A') = f(A) = f(E)\cdot f(A)
\]
\item $E$ correspondeert met  $R_i \leftrightarrow R_j$.\\
$A' = E\cdot A$ is de matrix waarin twee rijen verwisselt zijn zoals beschreven door $E$. Als we twee rijen verwisselen in een matrix keert het teken van $f$ om\footnote{Zie definitie 2.1 p 57 D-2}.
\[
f(E\cdot A) = f(A') = -f(A) = f(E)\cdot f(A)
\]
De bovenstaande bewering geldt want $f(E) = -1$ zoals bewezen in deel twee van dit bewijs.
\item $E$ correspondeert met $R_i \rightarrow \lambda R_i$.
$A' = E\cdot A$ is de matrix waarin rij $R_i$ vervangen is door $R_i' = \lambda R_i$. Door lineariteit in de $i$-de rij geldt dan de volgende bewering\footnote{Zie Stelling 2.2 p 57 D-4}.
\[
f(A') = \lambda f(A)
\]
\end{enumerate}
\end{enumerate}
\end{proof}

\subsection{Stelling 2.4 p 59}
Zij $f : \mathbb{R}^{n\times n} \rightarrow \mathbb{R}:A\mapsto f(A)$ een afbeeldingen die aan definitie 2.1 (determinant afbeelding) op pagina 57 voldoet.
\subsubsection*{Te Bewijzen}
\begin{enumerate}
\item Als $A$ een driehoeksmatrix is, dan is $f(A)$ gelijk aan het product van de diagonaalelementen in $A$.
\[
f(A) = \prod_{i=1}^n (A)_{ii}
\]
\item $A$ is inverteerbaar $\Leftrightarrow f(A) \neq 0$.
\item $\forall A,B \in \mathbb{R}^{n\times n}: f(A\cdot B) = f(A)\cdot f(B)$
\item $f(A^T) = f(A)$.
\end{enumerate}
\subsubsection*{Bewijs}
\begin{enumerate}
\item Als $A$ een driehoeksmatrix is, is $A$ rij-equivalent (uitsluitend via rijoperaties van het type $R_i \rightarrow R_i + \lambda R_j$) met $A'$ zijnde de diagonaalmatrix met precies dezelfde elementen op de diagonaal als $A$.
\[
f(A) = f(A') = \prod_{i=1}^n (A)_{ii}
\]
De bovenstaande bewering volgt uit de lineariteit van de determinant afbeelding en uit de definitie waarin staat dat $f(\mathbb{I}_n)=1$\footnote{Zie Stelling 2.2 p 57 D-4} \footnote{Zie definitie 2.1 p 57 D-2}. $A'$ is namelijk gelijk aan de eenheidsmatrix waar herhaaldelijk de elementaire rijoperatie van type $R_i \rightarrow \lambda R_i$ op is uitgevoerd.
\item Elke inverteerbare matrix valt te bekomen door op de eenheidsmatrix $\mathbb{I}_n$ elementaire rijoperaties uit te voeren zonder ooit een rij te vervangen door een nulrij. Dit houdt in dat elke inverteerbare matrix $A$ geschreven kan worden als volgt.
\[
A = E_1\cdot ... \cdot E_k
\]
Nemen we nu van beide leden de determinant dan krijgen we de volgende uitdrukking.
\[
f(A) = f(E_1\cdot ... \cdot E_k) = f(E_1)\cdot ... \cdot f(E_k) \neq 0
\]
De tweede gelijkheid omdat $\forall A\in \mathbb{R}^{n\times n}: f(E\cdot A) = f(E)\cdot f(A)$\footnote{Zie stelling 2.3 p 58}. De derde gelijkheid geldt omdat geen enkele elementaire matrix een determinant heeft die nul is.\\\\
$A$ is altijd rij-equivalent aan een matrix $U$ in trapvorm\footnote{Zie Propositie 1.8 p 21}. Dit betekent dat er een aantal elementaire matrices $E_1',...,E_k'$ bestaan zodat de volgende bewering geldt.
\[
E_k'\cdot ...\cdot E_1' \cdot A = U
\]
Als $A$ niet inverteerbaar is zal $U$ dat ook niet zijn. Dit zou betekenen dat er op de diagonaal van $U$ een nul staat en volgens deel \'e\'en van dit bewijs is $f(A)$ dan gelijk aan nul.
\item
Als $A$ niet inverteerbaar is, dan is ook $A\cdot B$ niet inverteerbaar. Dan geldt de bewering want volgend deel twee van dit bewijs is $f(A)$ dan gelijk aan nul en $f(A)\cdot f(B) = 0\cdot f(B) = 0$ geldt.\\
Als $A$ wel inverteerbaar is is $A$ het product van elementaire matrices\footnote{Zie stelling 1.36 p 38}.
\[
A = E_1\cdot ...\cdot E_k
\]
Bijgevolg geldt de volgende bewering.
\[
f(A\cdot B) = f(E_1\cdot ...\cdot E_k \cdot B) = f(E_1\cdot ...\cdot E_k) \cdot f(B)) = f(A)\cdot f(B)
\]
De tweede gelijkheid volgt uit het feit dat $\forall A\in \mathbb{R}^{n\times n}: f(E\cdot A) = f(E)\cdot f(A)$ geldt\footnote{Zie stelling 2.3 p 58}.
\item 
Als $A$ inverteerbaar is dan is $A^T$ ook inverteerbaar en omgekeerd\footnote{Zie Opdracht 1.33 p 35}. Als $A$ niet inverteerbaar is geldt $f(A) = f(A^T)=0$. Als $A$ wel inverteerbaar is, is $A$ een product van elementaire matrices\footnote{Zie stelling 1.36 p 38}.
\[
A = E_1\cdot ...\cdot E_k \Leftrightarrow A^T =  E_k^T\cdot ...\cdot E_1^T
\]
Bovenstaande bewering geldt vanwege een eigenschap van de getransponeerde matrix van het product van matrices\footnote{Zie Eigenschap 1.22 p 32 (bovenaan)}. $f(E^T)=f(E)$ want $E$ is ofwel symmetrisch (dan is $E^T$ gelijk aan $E$) ofwel is $f(E)$ gelijk aan  $1$. De volgende bewering geldt dan dus ook.
\[
f(A) = f(E_1\cdot ...\cdot E_k) = f(E_1)\cdot ...\cdot f(E_k) = f(E_1^T)\cdot ...\cdot f(E_k^T) = f(E_k^T\cdot ...\cdot E_1^T) = f(A^T)
\]
\end{enumerate}

\subsection{Gevolg 2.5 p 60}
Enkel het tweede puntje is zinvol om te bewijzen. De rest is triviaal of al bewezen.\\
Zij $f$ een determinantafbeelding en $A$ inverteerbaar.
\subsubsection*{Te Bewijzen}
\[
f(A^{-1}) = \frac{1}{f(A)}
\]
\subsubsection*{Bewijs}
\begin{proof}
Rechtstreeks bewijs.\\
\[
A \cdot A^{-1} = \mathbb{I}
\]
\[
f(A \cdot A^{-1}) = f(\mathbb{I})
\]
\[
f(A) \cdot f(A^{-1}) = 1
\]
\[
f(A^{-1}) = \frac{1}{f(A)}
\]
\end{proof}

\subsection{Stelling 2.7 p 61}
\subsubsection*{Te Bewijzen}
Elke permutatie $S_n$ is een samenstelling van transposities.
\subsubsection*{Bewijs}
\begin{proof}
TODO hoe bewijzen we dit?
\end{proof}

\subsection{Stelling 2.10 p 62}
Zij $\sigma$ een willekeurige permutatie en $\tau$ een transpositie in $\mathbb{S}_n$
\subsubsection*{Te Bewijzen}
\[
sgn(\tau \circ \sigma) = - sgn(\sigma)
\]
\subsubsection*{Bewijs}
\begin{proof}
Bewijs in woorden.\\
Wanneer we op een willekeurige permutatie van $\mathbb{S}_n$ een transpositie uitvoeren, wisselen we twee elementen van plaatst. Nu zijn er twee gevallen. In het eerste gevallen waren die twee elementen nog niet ge\"inverteerd en na de transpositie wel. In het tweede geval waren de twee elementen al ge\"inverteerd. In dat geval zijn de elementen na de transpositie niet meer ge\"inverteerd. In het eerste geval was sgn $sgn(\tau \circ \sigma)$ $1$ voor de transpositie en $-1$ na de transpositie. In het tweede geval was $sgn(\tau \circ \sigma)$ $-1$ voor de transpositie en $1$ na de transpositie. In beide gevallen is het teken dus veranderd.
\[
sgn(\tau \circ \sigma) = - sgn(\sigma)
\]
\end{proof}

\subsection{Stelling 2.12 p 63}
Zij $\sigma$ in $\mathbb{S}_n$ een samenstelling van $m \in \mathbb{N}$ transposities.
\subsubsection*{Te Bewijzen}
\[
sgn(\sigma) = (-1)^{m}
\]
\subsubsection*{Bewijs}
\begin{proof}
Bewijs door inductie.\\
\textsf{Basis stap}\\
We bewijzen dat de bewering geldt voor $m=1$. Deze bewering is precies dezelfde als de bewering die bewezen is in stelling 2.10 p 62.\\\\
\textsf{Inductie stap}\\
Stel dat de bewering geldt voor een bepaalde $m \in \mathbb{N}$ dan bewijzen we nu dat ze geldt voor $m+1 \in \mathbb{N}$ zodat ze geldt voor alle $m \in \mathbb{N}$.
Nu is $\sigma$ een samenstelling van $m+1$ transposities. $\sigma$ kan ookgezien worden als de samenstelling van een permutatie in $\mathbb{S}_n$ en een transpositie want elke permutatie in $\mathbb{S}_n$ is een samenstelling van transposities.\footnote{Zie Stelling 2.7 p 61}. Noem deze permutatie $\sigma'$. We weten dat $sgn(\sigma') = (-1)^m$ volgens de basis stap. We weten ook dat $sgn(\tau \circ \sigma') = -(-1)^m$ waarin $\tau$ de $m+1$de transpositie is \footnote{Zie Stelling 2.10 p 62}.
\end{proof}

\subsection{Stelling 2.18 p 69}
Zij $A \in \mathbb{R}^{n\times n}$
\subsubsection*{Te Bewijzen}
\begin{enumerate}
\item
\[
det(A) = \sum_{j=1}^n(-1)^{i+j}a_{ij}det(M_{ij})
\]
\item
\[
det(A) = \sum_{i=1}^n(-1)^{i+j}a_{ij}det(M_{ij})
\]
\end{enumerate}
\subsubsection*{Bewijs}
\begin{proof}
\[
det(A) = \sum_{\sigma \in \mathbb{S}_n} a_{1\sigma(1)}a_{2\sigma(2)}...a_{n\sigma(n)}
\]
\[
C_{ij} = (-1)^{i+j}det_{M_{ij}} = \sum_{\sigma \in \mathbb{S}_n,\sigma(i)=j}sgn(\sigma)a_{2\sigma(2)}...a_{n\sigma(n)}
\]
\begin{enumerate}
\item

\item

\end{enumerate}
\end{proof}

\subsection{Stelling 2.21 p 71}
Zij $A \in \mathbb{R}^{n\times n}$.
\subsubsection*{Te Bewijzen}
\[
A\cdot Adj(A) = adj(A) \cdot A = det(A)\cdot\mathbb{I}_n
\]
\subsubsection*{Bewijs}
\begin{proof}
We zullen het matrixproduct $A\cdot Adj(A)$ en $adj(A) \cdot A$ berekenen.\\\\
\emph{Elementen op de hoofddiagonaal van het product}\\
Het element op de $i$-de rij en de $i$-de kolom van het product ziet er als volgt uit.
\[
(A\cdot Adj(A))_{ii} =
\begin{pmatrix}
a_{i1} & a_{i2}&\cdots&a_{in}\\
\end{pmatrix}
\begin{pmatrix}
C_{i1}\\C_{i2}\\\vdots\\C_{in}
\end{pmatrix}
=
\sum_{k=1}^na_{ik}C_{ik} = det(A)
\]
Deze uitdrukking geeft de ontwikkeling van $det(A)$ naar rij $i$ weer.
Analoog geldt voor $adj(A) \cdot A$ iets gelijkaardig.
\[
(adj(A) \cdot A)_{ii} =
\begin{pmatrix}
C_{i1}&C_{i2}&\cdots&C_{in}\\
\end{pmatrix}
\begin{pmatrix}
a_{i1}\\a_{i2}\\\cdots\\a_{in}\\
\end{pmatrix}
=
\sum_{k=1}^na_{ik}C_{ik} = det(A)
\]
\emph{Elementen niet op de hoofddiagonaal van het product}\\
Voor de elementen niet op de hoofddiagonaal rest er ons nu nog te bewijzen dat die nul zijn.\\
We berekenen het element van het product op positie $i\neq j$.
\[
(A\cdot Adj(A))_{ij} =
\begin{pmatrix}
a_{i1} & a_{i2}&\cdots&a_{in}\\
\end{pmatrix}
\begin{pmatrix}
C_{j1}\\C_{j2}\\\vdots\\C_{jn}
\end{pmatrix}
=
\sum_{k=1}^na_{ik}C_{jk}
=0
\]
Deze laatste uitdrukking vormt de ontwikkeling van rij $j$ van de determinant van een matrix $A'$ waarbij $A'$ een matrix is waarbij $R_i = R_j = \begin{pmatrix}a_{i1} & a_{i2}&\cdots&a_{in}\\\end{pmatrix}$. Deze determinant is evident nul omdat er in $A$ twee gelijke rijen voorkomen\footnote{Zie Stelling 2.2 p 57 D-5}.\\
Opnieuw is de uitdrukking voor $adj(A) \cdot A$ analoog.
\[
(A\cdot Adj(A))_{ij} =
\begin{pmatrix}
C_{j1}&C_{j2}&\cdots&C_{jn}\\
\end{pmatrix}
\begin{pmatrix}
a_{i1}\\a_{i2}\\\vdots\\a_{in}\\
\end{pmatrix}
=
\sum_{k=1}^na_{ik}C_{jk}
=0
\]
\end{proof}

\subsection{Gevolg 2.22 p 72}
Zij $A \in \mathbb{R}^{n\times n}$ inverteerbaar.
\subsubsection*{Te Bewijzen}
\[
A^{-1} = \frac{1}{det(A)}adj(A)
\]
\subsubsection*{Bewijs}
\begin{proof}
Rechtstreeks bewijs.\\
We weten dat $A\cdot Adj(A) = adj(A) \cdot A = det(A)\cdot\mathbb{I}_n$\footnote{Zie Stelling 2.21 p 71}.
\[
adj(A) \cdot A = det(A)\cdot\mathbb{I}_n
\]
\[
adj(A) = det(A) \cdot A_{-1}
\]
\[
A^{-1} = \frac{1}{det(A)}adj(A)
\]
\end{proof}

\subsection{Stelling 2.25 p 74}
\subsubsection*{Te Bewijzen}
Door $n+1$ gegeven punten $(x_0,y_0),(x_1,y_1),...,(x_n,y_n)$ met $x_0 < x_1 < ... < x_n$ gaat de grafiek van juist \'e\'en veelterm functie van graad hoogstens $n$.
\subsubsection*{Bewijs}
\begin{proof}
We zullen proberen de gezochte veelterm functie te construeren.
Stel dat de volgende uitdrukking deze veelterm functie voorstelt.
\[
f: \mathbb{R} \rightarrow \mathbb{R} : x \mapsto a_0 + a_1x + a_2x^2+...+a_nx^n
\]
Nu moeten we bewijzen dat de co\"eficienten $a_i$ bestaan en uniek zijn zodat het volgende stelsel geldt.
\[
\left\lbrace
\begin{array}{ c }
a_0 + a_1x_0 + a_2x_0^2 + \cdots + a_nx_0^n = y_0\\
a_0 + a_1x_1 + a_2x_1^2 + \cdots + a_nx_1^n = y_1\\
\vdots\\
a_0 + a_1x_n + a_2x_n^2 + \cdots + a_nx_n^n = y_n\\
\end{array}
\right.
\]
Deze coeffici\"enten bestaan inderdaad en zijn uniek want de volgende determinant is niet nul.
\[
\begin{vmatrix}
1 & x_0 & x_0^2 & \cdots & x_0^n\\
1 & x_1 & x_1^2 & \cdots & x_1^n\\
\vdots & \vdots & \vdots & \ddots & \vdots \\
1 & x_n & x_n^2 & \cdots & x_n^n\\
\end{vmatrix}
\neq 0
\]
\end{proof}


\section{2.4}
\subsection{oef 1}
\subsubsection*{a}
\[
\begin{vmatrix}
3 & 5\\
-2 & 4
\end{vmatrix}
= 22
\]
\subsubsection*{b)}
\[
\begin{vmatrix}
5 & 6\\
7 & 2
\end{vmatrix}
= -32
\]
\subsubsection*{c)}
\[
    \frac{1}{52} \begin{vmatrix}
        -2 & -6\\
        7 & -5
    \end{vmatrix} 
    = \frac{1}{52} \cdot 52 = 1
\]
\subsubsection*{d)}
\[
\begin{vmatrix}
1 & 5 & 0\\
2 & 4 & -1\\
0 & -2 & 0
\end{vmatrix}
= -2
\]
\subsubsection*{e)}
Door rijoperaties:
$$
\begin{vmatrix}
2 & 1 & 3\\
-1 & 2 & 0\\
1 & 8 & 6
\end{vmatrix}
=
\overset{R1 \leftrightarrow R3}{=}
\begin{vmatrix}
1 & 8 & 6\\
-1 & 2 & 0\\
2 & 1 & 3
\end{vmatrix}
\overset{R3 \longmapsto R3 + R1}{=}
\begin{vmatrix}
1 & 8 & 6\\
0 & 10 & 6\\
2 & 1 & 3
\end{vmatrix}
\overset{R3 \longmapsto R3 - 2\cdot R1}{=}
\begin{vmatrix}
1 & 8 & 6\\
0 & 10 & 6\\
0 & -15 & -9
\end{vmatrix}
= 0
$$
Merk op: $R3 = -\frac{3}{2} R2$ vandaar $0$.\\
\\
Door ontwikkeling:
$$
\begin{vmatrix}
2 & 1 & 3\\
-1 & 2 & 0\\
1 & 8 & 6
\end{vmatrix}
=
3 \cdot
\begin{vmatrix}
-1 & 2\\
1 & 8
\end{vmatrix}
-0
+6 \cdot
\begin{vmatrix}
2 & 1\\
-1 & 2
\end{vmatrix}
= 0
$$
\subsubsection*{f)}
Door rijoperaties:
$$
\begin{vmatrix}
1 & 1 & 3\\
2 & 3 & 4\\
1 & 5 & 7
\end{vmatrix}
\overset{R2 \longmapsto R2-2R1}{=}
\begin{vmatrix}
1 & 1 & 3\\
0 & 1 & -2\\
1 & 5 & 7
\end{vmatrix}
\overset{R3 \longmapsto R3-R1}{=}
\begin{vmatrix}
1 & 1 & 3\\
0 & 1 & -2\\
0 & 4 & 4
\end{vmatrix}
\overset{R3 \longmapsto R3-4R2}{=}
\begin{vmatrix}
1 & 1 & 3\\
0 & 1 & -2\\
0 & 0 & 12
\end{vmatrix}
$$
$$ = 1\cdot 1 \cdot 12 = 12$$
Door ontwikkeling:
$$
\begin{vmatrix}
1 & 1 & 3\\
2 & 3 & 4\\
1 & 5 & 7
\end{vmatrix}
= 1 \cdot
\begin{vmatrix}
3 & 4\\
5 & 7
\end{vmatrix}
- 1 \cdot 
\begin{vmatrix}
2 & 4\\
1 & 7
\end{vmatrix}
+3\cdot
\begin{vmatrix}
2 & 3\\
1 & 5
\end{vmatrix}
$$
$$
= 1 - 10 + 3\cdot 7 = 12
$$
\subsubsection*{g)}
1)
\[
\begin{pmatrix}
0 & 1 & 1 & 1\\
2 & 0 & 1 & 1\\
2 & 2 & 0 & 1\\
1 & 1 & 1 & 0
\end{pmatrix}
\]
\[R1 \leftrightarrow R4\]
\[R2 \leftrightarrow R3\]
\[
\begin{pmatrix}
1 & 1 & 1 & 0\\
2 & 2 & 0 & 1\\
2 & 0 & 1 & 1\\
0 & 1 & 1 & 1
\end{pmatrix}
\]
\[R2 \longmapsto R2- 2 \cdot R1 \]
\[R3 \longmapsto R3- 2 \cdot R1 \]
\[
\begin{pmatrix}
1 & 1 & 1 & 0\\
0 & 0 & -2 & 1\\
0 & -2 & -1 & 1\\
0 & 1 & 1 & 1
\end{pmatrix}
\]
\[R4 \leftrightarrow R2\]
\[R3 \longmapsto R3 + 2 \cdot R2\]
\[
\begin{pmatrix}
1 & 1 & 1 & 0\\
0 & 1 & 1 & 1\\
0 & 0 & 1 & 3\\
0 & 0 & -2 & 1
\end{pmatrix}
\]
\[R4 \longmapsto R4 + 2 \cdot R3\]
\[
\begin{pmatrix}
1 & 1 & 1 & 0\\
0 & 1 & 1 & 1\\
0 & 0 & 1 & 3\\
0 & 0 & 0 & 7
\end{pmatrix}
\]
\[det = -7 \text{(3 rijwissels)} \]
2)
\[ -2 \cdot det
\begin{pmatrix}
1 & 1 & 1\\
2 & 0 & 1\\
1 & 1 & 0
\end{pmatrix}
+2 \cdot det
\begin{pmatrix}
1 & 1 & 1\\
0 & 1 & 1\\
1 & 1 & 0
\end{pmatrix}
- det 
\begin{pmatrix}
1 & 1 & 1\\
0 & 1 & 1\\
2 & 0 & 1
\end{pmatrix}
=-7
\]
\subsubsection*{h)}
\[ det
\begin{pmatrix}
4 & 3 & 1 & 5\\
4 & 0 & -2 & 4\\
8 & 9 & 5 & -11\\
8 & 3 & -1 & 9
\end{pmatrix}
\]
\[ = -3 \cdot det
\begin{pmatrix}
4 & -2 & 4\\
8 & 5 & -11\\
8 & -1 & 9
\end{pmatrix}
-9 \cdot det 
\begin{pmatrix}
4 & 1 & 5\\
4 & -2 & 4\\
8 & -1 & 9
\end{pmatrix}
+3 \cdot det
\begin{pmatrix}
4 & 1 & 5\\
4 & -2 & 4\\
8 & 5 & -11
\end{pmatrix}
= 0
\]

\subsubsection*{i)}
\[
\begin{vmatrix}
5 & 0 & 3 & 2 & 0\\
0 & 0 & 0 & 0 & 4\\
0 & 0 & -3  & 1 & 0\\
0 & 0 & 0 & 1 & 1\\
0 & -2 & 4 &0  &8
\end{vmatrix}
\]
Wissel R2 en R5
\[
=-
\begin{vmatrix}
5 & 0 & 3 & 2 & 0\\
0 & -2 & 4 &0  &8\\
0 & 0 & -3  & 1 & 0\\
0 & 0 & 0 & 1 & 1\\
0 & 0 & 0 & 0 & 4
\end{vmatrix}
= - (5 \cdot -2 \cdot -3 \cdot 1 \cdot 4) = -120
\]

\subsubsection*{j)}
Door rijoperaties:\\
\[
\begin{vmatrix}
3 & -7 & 8 & 9 & -6\\
0 & 2 & -5 & 7 & 3\\
0 & 0 & 1 & 5 & 0\\
0 & 0 & 2 & 4 & -1\\
0 & 0 & 0 & -2 & 0
\end{vmatrix}
\overset{R4 \longmapsto R4-2R3}{=}
\begin{vmatrix}
3 & -7 & 8 & 9 & -6\\
0 & 2 & -5 & 7 & 3\\
0 & 0 & 1 & 5 & 0\\
0 & 0 & 0 & -6 & -1\\
0 & 0 & 0 & -2 & 0
\end{vmatrix}
\]
\[
\overset{R5 \longmapsto R5-\frac{-2}{6}R4}{=}
\begin{vmatrix}
3 & -7 & 8 & 9 & -6\\
0 & 2 & -5 & 7 & 3\\
0 & 0 & 1 & 5 & 0\\
0 & 0 & 0 & -6 & -1\\
0 & 0 & 0 & 0 & \frac{-2}{6}
\end{vmatrix}
=
3\cdot 2 \cdot 1 \cdot (-6) \cdot (-\frac{2}{6}) = 12
\]\\
Door ontwikkeling:
\[
\begin{vmatrix}
3 & -7 & 8 & 9 & -6\\
0 & 2 & -5 & 7 & 3\\
0 & 0 & 1 & 5 & 0\\
0 & 0 & 2 & 4 & -1\\
0 & 0 & 0 & -2 & 0
\end{vmatrix}
=
3 \cdot
\begin{vmatrix}
2 & -5 & 7 & 3\\
0 & 1 & 5 & 0\\
0 & 2 & 4 & -1\\
0 & 0 & -2 & 0
\end{vmatrix}
=
3 \cdot 2 \cdot
\begin{vmatrix}
1 & 5 & 0\\
2 & 4 & -1\\
0 & -2 & 0
\end{vmatrix}
\]
\[
3\cdot 2\cdot \left( 
\begin{vmatrix}
4 & -1\\
-2 & 0
\end{vmatrix}
-2\cdot
\begin{vmatrix}
5 & 0\\
-2 & 0
\end{vmatrix}
\right)
=
3\cdot 2 \cdot 2 = 12
\]

\subsection{oef 2}
\[det(A) = -7, det(B) = 3\]
\begin{gather*}
    det(A^2BA^{-1}) =
    det(A^2) \cdot det(B) \cdot det(A^{-1}) =\\
    det(A)^2 \cdot det(B) \cdot det(A)^{-1} = det(A) \cdot det(B) = -21
\end{gather*}
\begin{gather*}
    det(B^{-1}A^3) = det(B^{-1}) \cdot det(A^3) = det(B)^{-1} \cdot det(A)^3 = \frac{1}{3} \cdot (-7)^3 = \frac{-343}{3}
\end{gather*}

\subsection{oef 3}
$det(AB) = 0$ als $det(A) = 0$ of $det(B) = 0$
\\
\[det(A) = \begin{vmatrix}
x + 2 & 3x\\
3 & x + 2
\end{vmatrix}
 = (x + 2)^2 - 9x = x^2 -5 x + 4
\]
$x^2 -5 x + 4 = 0$ voor $x_1, x_2$ met $x_{1,2} = \frac{5 \pm 3}{2}$ ($x_1 = 1$ en $x_2 = 4$)
\\
\[det(B) = \begin{vmatrix}
x & 0\\
5 & x + 2
\end{vmatrix}
 = x \cdot (x + 2) = x^2 + 2x\]
$x^2 + 2x$ voor $x_3, x_4$ met $x_3 = -2$ en $x_4 = 0$v 
\subsubsection*{Besluit}
$det(AB) = 0$ voor $x$ met $x \in \{-2, 0, 1, 4\}$.
\subsection{oef 4}
\subsubsection*{a)}
Aangezien er twee maal een ERO van het verwisselen van rijen voorkomt wordt de uitkomst:
$$-1\cdot -1\cdot -6 = -6$$
\subsubsection*{b)}
$$3\cdot(-1)\cdot 4 \cdot 6 = -72$$
\subsubsection*{c)}
\[ \text{De determinant blijft -6 (Stelling 2.3)} \]
\subsection*{oef 5}
Voor een $2\times2$ matrix, is deze determinant $-1$.
Voor elke grotere vierkante matrix is deze determinant $0$, omdat door rij reductie twee gelijke rijen kunnen gemaakt worden.
\[
\begin{vmatrix}
2 & 3 & 4\\
3 & 4 & 5\\
4 & 5 & 6
\end{vmatrix}
\overset{R2 \longmapsto R2-R1 \text{ en } R3 \longmapsto R3 - R1}{=}
\begin{vmatrix}
2 & 3 & 4\\
1 & 1 & 1\\
1 & 1 & 1
\end{vmatrix}
\]
\subsection{oef 6}
\subsubsection*{a)}
\[
det(A^{-1}) = \frac{1}{det(A)} $$
dus:
$$
det(A^{-1}) = \frac{1}{-7} 
\]
\subsubsection*{b)}
$$ det(2(A^{-1}) = 4\cdot det(A^{-1}) = 4 \cdot \frac{1}{-7} = \frac{-4}{7}
$$
\subsubsection*{c)}
\[ det((2A)^{-1}) = det(2A)^{-1} = 2^{3}det(A)^{-1} = -8/7\]
\subsection{oef 7}
\begin{proof}
De determinant van een product is het product van de determinanten.
\[
det(AB) = det(A) det(B) = 0 det(B) = 0 
\]
\end{proof}

\subsection{oef 8}
\begin{align*}
    \begin{vmatrix}
        k+2 & 1 & 1\\
        1 & k+2 & 1\\
        1 & 1 & k+2
    \end{vmatrix}
    &= (k+2) 
    \begin{vmatrix}
        k+2 & 1\\
        1 & k+2    
    \end{vmatrix} - 
    \begin{vmatrix}
        1 & 1\\
        1 & k+2
    \end{vmatrix} + 
    \begin{vmatrix}
        1 & 1\\
        k+2 & 1\\
    \end{vmatrix}\\
    &= (k+2)((k+2)^2 - 1) - (k+1) + (-k-1)\\
    &= (k+2)(k^2 + 4k + 3) -2k - 2\\
    &= k^3 + 6k^2 + 9k + 4
\end{align*}
De matrix is inverteerbaar als de determinant niet gelijk is aan 0.
\begin{align*}
    k^3 + 6k^2 + 9k + 4 &= (k+1)(k^2 + 5k + 4)\\
    &= (k+1)(k+1)(k+4)
\end{align*}
De matrix is dus inverteerbaar als $k \neq -1$ of $k \neq -4$.

\subsection{oef 9}
Basisgeval: \\
1) $(n=3, F_{3}=2):$
\[ 
\begin{pmatrix}
2 & 3\\ 
1 & 2
\end{pmatrix}
= 1 = (-1)^{3-1} = (-1)^{n-1}
\]
2) $(n=4, F_{4}=3:$
\[
\begin{pmatrix}
3 & 5\\ 
2 & 3
\end{pmatrix}
= -1 = (-1)^{4-1} = (-1)^{n-1}
\]
We kunnen van geval 1) naar geval 2) gaan door in geval 1) R2 $\longmapsto $ R2 + R1 en vervolgens R1 te verwisselen met R2. Deze eerste elementaire operatie veranderd niets aan de determinant, de tweede operatie veranderd slechts het teken. \\
\\
Algemeen:\\
Stel dat het klopt voor $F_{n-1}$\\
Te Bewijzen: het klopt voor $F_{n}$\\
Dit kunnen we doen door de uitleg die eerder gegeven werd.

\subsection{oef 11}
\subsubsection*{a)}
Waar: We kunnen de determinant van de matrix bepalen door er een bovendriehoeksmatrix van te maken en dan de elementen op de diagonaal te vermenigvuldigen. Dit kunnen we doen door zowel van A als D een bovendriehoeksmatrix te maken. Als we dit doen worden enkel de elementen van respectievelijk A en D aangepast (alle nullen blijven) dus de determinant is gelijk aan de det(A). det(B)

\subsubsection*{b)}
\begin{proof}
De matrices op de diagonaal van de samengestelde matrix: $A$ en $D$. kunnen we rijreduceren naar een echelonvorm zodat de elk diagonaal matrices worden ($A'$ en $D'$). Eventueel worden er dan bepaalde factoren voorop gezet ($a$ en $d$).
\[
\begin{vmatrix}
A & B\\
0 & D
\end{vmatrix}
=
ad
\begin{vmatrix}
A' & B\\
0 & D'
\end{vmatrix}
\]
Nu is het gemakkelijk te zien dat de volledige samengestelde matrix een diagonaalmatrix is.
Van een $n\times n$ diagonaalmatrix $X$ weten we het volgende.
\[
\det \left({X}\right) = \prod_{i \mathop = 1}^n x_{ii}
\]
Onze gezochte determinant is dan als volgt.
\[
ad
\begin{vmatrix}
A' & B\\
0 & D'
\end{vmatrix} = a\prod_{i \mathop = 1}^n a'_{ii} \cdot d \prod_{i \mathop = 1}^m d'_{ii} = \det \left({A}\right)\cdot\det \left({D}\right)
\]
\end{proof}

\subsubsection*{c)}
Niet waar, tegenvoorbeeld:
\begin{gather*}
    A = 
    \begin{pmatrix}
        1 & 2\\
        5 & 6
    \end{pmatrix} 
    , B = 
    \begin{pmatrix}
        3 & 4\\
        7 & 8
    \end{pmatrix}
    , C = 
    \begin{pmatrix}
        10 & 9\\
        13 & 14
    \end{pmatrix}
    , D = 
    \begin{pmatrix}
        11 & 12\\
        15 & 16
    \end{pmatrix}\\
    \begin{vmatrix}
        A & B\\
        C & D
    \end{vmatrix}
    = 0\\
    |A| \cdot |D| - |B| \cdot |C| = (-4) \cdot (-4) - (-4) \cdot 23 = 108
\end{gather*}


\subsection{oef 12}
\[
\begin{vmatrix}
2 & -1 & 3\\
-1 & 2 & 2\\
3 & 2 & 1
\end{vmatrix}
\]
\[
\left\lbrace
\begin{array}{c c c}
A_{11}: &C_{11} &= (-1)^(1+1) \cdot (2 - 4) = -2\\
A_{12}: &C_{12} &= (-1)^(1+2) \cdot (-1 -6) = 7\\
A_{13}: &C_{13} &= -8\\
A_{33}: &C_{33} &= 2\\
\end{array}
\right.
\]
\subsection{oef 13}
$$
\begin{vmatrix}
a & a^4 & a^7\\
a^2 & a^5 & a^8\\
a^3 & a^6 & a^9
\end{vmatrix}
\overset{R2 \longmapsto R2 - a \cdot R1}{=}
\begin{vmatrix}
a & a^4 & a^7\\
0 & 0 & 0\\
a^3 & a^6 & a^9
\end{vmatrix}
= 0
$$
Dit kan ook zo gedaan worden voor de derde rij. Vanaf dat ereen nulrij in zit weten we dat de determinant nul is.
\subsection{oef 14}
Voor elke scheefsymmetrische matrix geldt:
\[
A^T= -A
\]
\subsubsection*{Te bewijzen}
Voor $A\in R^{m\times n}$ met $n$ oneven.
\[
det(A) = 0
\]
\subsubsection*{Bewijs}
\[
A^T = -A
\]
\[
det(A^T) = det(-A)
\]
We weten:
\[
det(A^T) = det(A) \text{ en } det(-A) = (-1)^n det(A)
\]
dus
\[
det(A) = det(-A) \Rightarrow det(A) = 0
\]

\subsubsection*{tegenvoorbeeld}
\[
\begin{vmatrix}
0 & -1 & 0 & 0\\
1 & 0 & -1 & 0\\
0 & 1 & 0 & -1\\
0 & 0 & 1 & 0
\end{vmatrix}
\]
\subsection{oef 15}
\[ det(A-nI_{n})=0 \longmapsto \]
\[
\begin{pmatrix}
-(n-1) & 1 & 1 &... & 1\\
1 & -(n-1) & 1 & ... & 1\\
1 & 1 & -(n-1) & ... & 1\\
... & ... & ... & ... & ...\\
1 & 1 & 1 & ... & -(n-1)
\end{pmatrix} 
\]
\[R1 \leftrightarrow R_{N} \]
\[R2 \leftrightarrow R_{N-1} \]
\[...\]
\[
\begin{pmatrix}
1 & 1 & 1 & ... & -(n-1)\\
... & ... & ... & ... & ...\\
1 & 1 & -(n-1) & ... & 1\\
1 & -(n-1) & 1 & ... & 1\\
-(n-1) & 1 & 1 &... & 1
\end{pmatrix} 
\]
\[ \text{Als we deze matrix naar echelonvorm brengen zullen altijd een nulrij krijgen. Dus det=0}\]

\subsection{oef 16}
\subsubsection*{b)}
\[
\begin{vmatrix}
1 & 1 & 3\\
2 & 3 & 4\\
1 & 5 & 7
\end{vmatrix}
det(A) = 12
adj(A) =
\begin{vmatrix}
C_{11} & C_{21} & C_{31}\\
C_{21} & C_{22} & C_{32}\\
C_{13} & C_{23} & C_{33}
\end{vmatrix}
\]
\[
= 
\begin{vmatrix}
1 & -8 & -5\\
10 & 4 & -2\\
7 & 4 & 1
\end{vmatrix}
det(adj A) = 152
A^-1 = \frac{1}{det(A)} \cdot adj(A)
	 = \frac{1}{12} \cdot \begin{vmatrix}
	 						1 & -8	-5\\
	 						10 & 4 & -2\\
	 						7 & 4 & 1
	 					  \end{vmatrix}
\]


\subsection{oef 17}
\[
adj(A)=
\begin{pmatrix}
-3i & 4 & 10+16i\\
0 & 1+i & -5-4i\\
0 & 0 & 3+3i
\end{pmatrix}
\]

\subsection{oef 19}
\subsubsection*{19b)}
De uitleg staat tss "(())"
\[adj(A).A = det(A).I_{n}\]
\[adj(A.(adj(A))) = det(A)^{n-1}.I_{n}\]
\[((adj(A.B)=adj(A).adj(B)))\]
\[adj(A).adj(adj(A)) = det(A)^{n-1}.I_{n}\] 
\[(( \text{Beide kanten } \cdot A )) \]
\[(( adj(A).A = det(A).I_{n} )) \]
\[det(A).I_{n}.adj(adj(A)) = det(A)^{n-1}.A\]
\[((\text{We gaan er van uit dat de det(A) niet gelijk is aan nul }))\]
\[adj(adj(A)) = det(A)^{n-2}.A\]

\subsection{oef 20}
\subsubsection*{b)}

WAAR.
\[
adj(A) = CofactorenMatrix(A)^T
\]
dan is : 
\[
(adj(A))^T = (CM(A)^T)^T
\]
wat gewoon $CM(A)$ oplevert.
vervolgens is : 
\[
adj(A^T) = CM(A^T)^T
\]
wat opnieuw $CM(A)$ oplevert.


\subsubsection*{oef 21}
\[A^{-1} = adj(A)/det(A)\]
\[adj(A) =
\begin{pmatrix}
3 & -5 & 2\\
0 & -1 & 1\\
-6 & 8 & -5
\end{pmatrix} 
\]
\[det(A)=-3\]

\subsubsection*{oef 23}
\[ A = 
\begin{pmatrix}
3 & -1 & 1\\
4 & 1 & 1\\
2 & -1 & 1
\end{pmatrix}
\]
\[ det(A) = 2\]
\[X1 = det 
\begin{pmatrix}
2 & -1 & 1\\
1 & 1 & 1\\
0 & -1 & 1
\end{pmatrix}
/2 = 4/2 = 2 \]
\[X2 = det
\begin{pmatrix}
3 & 2 & 1\\
4 & 1 & 1\\
2 & 0 & 1
\end{pmatrix}
/2 = -3/2 \]
\[X3 = det
\begin{pmatrix}
3 & -1 & 2\\
4 & 1 & 1\\
2 & -1 & 0
\end{pmatrix}
/2 = -11/2 \]

\subsection{oef 24}
\[ \left\{
     \begin{array}{lr}
       \frac{2}{x}-\frac{3}{y}+\frac{5}{z} = & 3 \\
       \frac{-4}{x}+\frac{7}{y}+\frac{2}{z} = & 0 \\
       \frac{2}{y}-\frac{1}{z} = & 2
     \end{array}
   \right. \Rightarrow\left\{
     \begin{array}{lr}
       2x'-3y'+5z' = & 3 \\
       -4x'+7y'+2z' = & 0 \\
       2y'-z' = & 2
     \end{array}
   \right.
\]
met $x' = \frac{1}{x}, y' = \frac{1}{y}, z' = \frac{1}{z}$

\[  d = \begin{vmatrix}
            2 & -3 & 5\\
            -4 & 7 & 2\\
            0 & 2 & -1
    \end{vmatrix} = -50,\]
\[  d_1 = \begin{vmatrix}
        3 & -3 & 5\\
        0 & 7 & 2\\
        2 & 2 & -1
    \end{vmatrix} = 73,\\
    d_2 = \begin{vmatrix}
        2 & 3 & 5\\
        -4 & 0 & 2\\
        0 & 2 & 1
    \end{vmatrix} = -36,\\
    d_3 = \begin{vmatrix}
        2 & -3 & 3\\
        -4 & 7 & 0\\
        0 & 2 & 2
    \end{vmatrix} = -20
\]

\[\Rightarrow x = (x')^{-1}= \frac{d}{d_1} = \frac{-50}{73},
    y = (y')^{-1}= \frac{d}{d_2} = \frac{50}{36},
    z = (z')^{-1}= \frac{d}{d_3} = \frac{50}{20}
\]

\subsection{oef 25}
De beschreven determinant ziet er als volgt uit.
\[
\begin{vmatrix}
1 & 0 & 0 & \cdots & 0 & x_1 & 0 & \cdots & 0 \\
0 & 1 & 0 & \cdots & 0 & x_2 & 0 & \cdots & 0 \\
0 & 0 & 1 & \cdots & 0 & x_3 & 0 & \cdots & 0 \\
\vdots & \vdots & \vdots & \ddots  & \vdots & \vdots & \vdots &  & \vdots\\
0 & 0 & 0 & \cdots & 1 & x_{i-1} & 0 &\cdots & 0 \\
0 & 0 & 0 & \cdots & 0 & x_i & 0 & \cdots & 0 \\
0 & 0 & 0 & \cdots & 0 & x_{i+1} & 1 & \cdots & 0 \\
\vdots & \vdots & \vdots &  & \vdots & \vdots & \vdots & \ddots & \vdots\\
0 & 0 & 0 & \cdots & 0 & x_n & 0 & \cdots & 1\\
\end{vmatrix}
\]
Nu ontwikkelen we naar de eerste $i-1$ kolommen.
\[
= 
\begin{vmatrix}
x_i & 0 & \cdots & 0 \\
x_{i+1} & 1 & \cdots & 0 \\
\vdots & \vdots & \ddots & \vdots\\
 x_n & 0 & \cdots & 1\\
\end{vmatrix}
\]
Nu ontwikkelen we naar de laatste $n -i$ kolommen.
\[
= 
\begin{vmatrix}
x_i
\end{vmatrix}
= x_i
\]


\subsection{oef 27}
\subsubsection*{c)}
\[ \text{Niet waar, Stel c=2, n=3 en A=}
\begin{pmatrix}
3 & 0 & 0\\
0 & 1 & 0\\
0 & 0 & 1
\end{pmatrix} 
\]
\[c^{n}-det(A) = 8 - 3 = 5\]
\[ det (cI_{n}-A) = det
\begin{pmatrix}
-1 & 0 & 0 \\
0 & 1 & 0\\
0 & 0 & 1
\end{pmatrix}
= -1\]

\subsubsection*{d)}
\begin{proof}
\[
det(cI_n - A^T) = det(cI_n^T - A^T) = det(cI_n - A)
\]
\end{proof}

\subsubsection*{h)}
Niet waar. Tegenvoorbeeld: $A=I$ en $k=3$
\[
\rightarrow det(3A) = 27 \neq 3
\]
\subsubsection*{g}
 WAAR. ONVERBETERD(voorbereidende oefeningen maar ik was niet in de oefzitting aanwezig
eigschap van nilpotente matrix is dat zo'n matrix niet inverteerbaar is. We weten dat een inverteerbare matrix A gekenmerkt is door het feit dat zijn determinant verschillend van 0 is. Dus het feit dat 
nilpotenteMatrix niet inverteerbaar is betekent dat zijn determinant dus gelijk gaat zijn aan 0.
\subsubsection*{i)}
Niet waar. Tegenvoorbeeld: 
\[
    \begin{vmatrix}
        1 & 2 & 3\\
        4 & 5 & 6\\
        7 & 8 & 9
    \end{vmatrix} \neq
    \begin{vmatrix}
        5 & 3 & 8\\
        6 & 2 & 3\\
        7 & 8 & 9
    \end{vmatrix} + 
    \begin{vmatrix}
        -4 & -1 & -5\\
        -2 & 3 & 3\\
        7 & 8 & 9\\
    \end{vmatrix}
\]


\subsubsection*{k)}

Dit is niet waar, hoewel best vaag geformuleerd. Beschouwen we het tegenvoorbeeld

\[
\begin{pmatrix}
1 & 1\\
1 & 1\\
\end{pmatrix}
* X = 
\begin{pmatrix}
2\\
2
\end{pmatrix}
\]

dan geldt als uitkomst $X = \bigl(\begin{smallmatrix} 1\\ 1 \end{smallmatrix} \bigr)$, en dat $Det(A) = 0$. Hierbij is de stelling tegenbewezen.

\subsubsection*{n)}
niet waar:
\[
\begin{vmatrix}
1 & 0 & 1\\
1 & 1 & 0\\
0 & 1 & 1
\end{vmatrix}
=2
\]

\subsection*{extra}
Schrijf de volgende uitdrukking als een product van vier niet triviale factoren.
\[
\begin{vmatrix}
1+x & 1+x+x^{2} & 1+x+x^{2}+x^{3}\\
1+y & 1+y+y^{2} & 1+y+y^{2}+y^{3}\\
1+z & 1+z+z^{2} & 1+z+z^{2}+z^{3}
\end{vmatrix}
\]

\section{Opdrachten}
\subsection{2.14}
\subsection{2.16}
\subsubsection*{1.}
Permutaties van $\{1,2,3,4\}$:
$$
\{\{1, 2, 3, 4\}, \{1, 2, 4, 3\}, \{1, 3, 2, 4\}, \{1, 3, 4, 2\},$$
$$ \{1, 4, 2, 3\}, \{1, 4, 3, 2\}, \{2, 1, 3, 4\}, \{2, 1, 4, 3\}, $$ 
$$\{2, 3, 1, 4\}, \{2, 3, 4, 1\}, \{2, 4, 1, 3\}, \{2, 4, 3, 1\}, $$
$$\{3, 1, 2, 4\}, \{3, 1, 4, 2\}, \{3, 2, 1, 4\}, \{3, 2, 4, 1\}, $$ $$
\{3, 4, 1, 2\}, \{3, 4, 2, 1\}, \{4, 1, 2, 3\}, \{4, 1, 3, 2\}, $$
$$\{4, 2, 1, 3\}, \{4, 2, 3, 1\}, \{4, 3, 1, 2\}, \{4, 3, 2, 1\}\}
$$
Wat men eigenlijk doet voor een algemene matrix met deze permutaties is het volgende: begin met de standaard volgorde dit is: $\{1,2,3,4\}$.\\ Zet hieronder telkens een permutatie van deze volgorde, bijvoorbeeld:
$$\begin{pmatrix}
1 & 2 & 3 & 4\\
2 & 3 & 4 & 1
\end{pmatrix}
$$
Nu zijn de paarsgewijze koppels telkens de elementen uit de $4x4$ matrix die je moet nemen: $a_{12}$, $a_{23}$, $a_{34}$, $a_{41}$.\\
We kunnen nu ook nog het teken bepalen door te kijken naar het aan inversies:\\
$$1 < 2 \rightarrow 2 < 3$$
$$1 < 3 \rightarrow 3 < 4$$
$$1 < 4 \rightarrow 2 > 1 \ inversie$$
$$2 < 3 \rightarrow 3 < 4$$
$$2 < 4 \rightarrow 3 > 1 \ inversie$$
$$3 < 4 \rightarrow 4 > 1 \ inversie$$
Er zijn dus in totaal 3 inversies, het teken van de permutatie is dus negatief:
$$-a_{12}\cdot a_{23}\cdot a_{34}\cdot a_{41}$$
\subsubsection*{2.}
Op een algemene matrix:
$$
\begin{pmatrix}
a & b & c & d\\
e & f & g & h\\
i & j & k & l\\
m & n & o & p
\end{pmatrix}
$$
zou dit dan zijn:
$$-b\cdot g\cdot l\cdot m$$
Voor de uiteindelijke determinant moeten we deze stappen dus herhalen voor al de permutaties.
\end{document}
