\documentclass[10pt,a4paper]{article}

\usepackage[english]{babel}
\usepackage{amsmath}
\usepackage{amsfonts}
\usepackage{amssymb}
\usepackage{amsthm}
\makeatletter
\renewcommand*\env@matrix[1][*\c@MaxMatrixCols c]{%
  \hskip -\arraycolsep
  \let\@ifnextchar\new@ifnextchar
  \array{#1}}
\makeatother
\title{Oplossingen Lineaire Algebra 2013}
\author{TODO}


\begin{document}

\maketitle
\pagebreak
\tableofcontents
\pagebreak

\section{2.4}
\subsection*{oef 1}
\subsubsection*{j)}
Door rijoperaties:\\
\[
\begin{vmatrix}
3 & -7 & 8 & 9 & -6\\
0 & 2 & -5 & 7 & 3\\
0 & 0 & 1 & 5 & 0\\
0 & 0 & 2 & 4 & -1\\
0 & 0 & 0 & -2 & 0
\end{vmatrix}
\overset{R4 \longmapsto R4-2R3}{=}
\begin{vmatrix}
3 & -7 & 8 & 9 & -6\\
0 & 2 & -5 & 7 & 3\\
0 & 0 & 1 & 5 & 0\\
0 & 0 & 0 & -6 & -1\\
0 & 0 & 0 & -2 & 0
\end{vmatrix}
\]
\[
\overset{R5 \longmapsto R5-\frac{-2}{6}R4}{=}
\begin{vmatrix}
3 & -7 & 8 & 9 & -6\\
0 & 2 & -5 & 7 & 3\\
0 & 0 & 1 & 5 & 0\\
0 & 0 & 0 & -6 & -1\\
0 & 0 & 0 & 0 & \frac{-2}{6}
\end{vmatrix}
=
3\cdot 2 \cdot 1 \cdot (-6) \cdot (-\frac{2}{6}) = 12
\]\\
Door ontwikkeling:
\[
\begin{vmatrix}
3 & -7 & 8 & 9 & -6\\
0 & 2 & -5 & 7 & 3\\
0 & 0 & 1 & 5 & 0\\
0 & 0 & 2 & 4 & -1\\
0 & 0 & 0 & -2 & 0
\end{vmatrix}
=
3 \cdot
\begin{vmatrix}
2 & -5 & 7 & 3\\
0 & 1 & 5 & 0\\
0 & 2 & 4 & -1\\
0 & 0 & -2 & 0
\end{vmatrix}
=
3 \cdot 2 \cdot
\begin{vmatrix}
1 & 5 & 0\\
2 & 4 & -1\\
0 & -2 & 0
\end{vmatrix}
\]
\[
3\cdot 2\cdot \left( 
\begin{vmatrix}
4 & -1\\
-2 & 0
\end{vmatrix}
-2\cdot
\begin{vmatrix}
5 & 0\\
-2 & 0
\end{vmatrix}
\right)
=
3\cdot 2 \cdot 2 = 12
\]
\section{Opdrachten}
\subsection{2.14}
\subsection{2.16}

\end{document}