\documentclass[lineaire_algebra_oplossingen.tex]{subfiles}
\begin{document}

\section{Zelfreflectie 1}
\subsection{Oefening 1}
\begin{itemize}
\item echelonvorm
\item trapvorm
\item rijgereduceert
\end{itemize}
\subsection{Oefening 2}
\subsubsection*{a)}
\[
\left[
\begin{array}{l l l l l | r}
0 & 0 & 0 & 0 & 0 & 1\\
0 & 0 & 0 & 0 & 0 & 0\\
0 & 0 & 0 & 0 & 0 & 0\\
0 & 0 & 0 & 0 & 0 & 0\\
\end{array}
\right]
\]
\subsubsection*{b)}
Niet mogelijk, daarvoor hebben we niet genoeg vergelijkingen
\subsubsection*{c)}
\[
\left[
\begin{array}{l l l l l | r}
1 & 0 & 0 & 0 & 0 & 1\\
0 & 1 & 0 & 0 & 0 & 2\\
0 & 0 & 1 & 0 & 0 & 3\\
0 & 0 & 0 & 1 & 0 & 4\\
\end{array}
\right]
\]
\subsubsection*{d)}
\[
\left[
\begin{array}{l l l l l | r}
1 & 0 & 0 & 0 & 0 & 1\\
0 & 1 & 0 & 0 & 0 & 2\\
0 & 0 & 0 & 0 & 0 & 0\\
0 & 0 & 0 & 0 & 0 & 0\\
\end{array}
\right]
\]
\subsubsection*{e)}
\[
\left[
\begin{array}{l l l l l | r}
0 & 0 & 0 & 0 & 0 & 0\\
0 & 0 & 0 & 0 & 0 & 0\\
0 & 0 & 0 & 0 & 0 & 0\\
0 & 0 & 0 & 0 & 0 & 0\\
\end{array}
\right]
\]

\subsection{Oefening 3}
\begin{itemize}
\item 1
\item 0
\item $\infty$
\item 1
\end{itemize}

\subsection{Oefening 4}
$\lambda$ is een vrije variabele, $a$ is een parameter.
\[
\{(a-\lambda, 2+a\lambda, \lambda) | \lambda \in \mathbb{R}\}
\]

\subsection{Oefening 5}
\subsubsection*{gegeven}
$B$ is de linkerinverse van $A$. $A,B \in R^{n\times n}$
\subsubsection*{te bewijzen}
$B$ is de inverse van $A$.
\subsubsection*{bewijs}
\begin{proof}
Rechtstreeks bewijs.\\
Hetgeen gegeven is valt formeler te bewoorden als volgt.
\[
B\cdot A = \mathbb{I}_n
\]
Hetgeen te bewijzen is houdt het volgende in.
\[
B\cdot A = \mathbb{I}_n = A \cdot B
\]
De eerste van deze twee vergelijkingen is dus al gegeven.
We bewijzen het tweede deel. We beginnen met iets dat triviaal is, en bewijzen daaruit rechtstreeks hetgeen te bewijzen valt.\\
\[A\cdot B = \mathbb{I}_n\]
\[A\cdot B \cdot A = \mathbb{I}_n \cdot A\]
\[A\cdot\mathbb{I}_n = A\]
\[A = A\]
\[True\]
\end{proof}
\subsection{Oefening 6}
\subsubsection*{gegeven}
\[
A \in R^{n\times m}
\]
\[
B \in R^{m\times k}
\]
\[
C \in R^{k\times l}
\]
\subsubsection*{a)}
\emph{te bewijzen}
\[
A(BC) = (AB)C
\]
\emph{bewijs}
\begin{proof}
Rechtstreeks bewijs.\\
Zie p 32, eigenschappen. Het te bewijzen geldt als de afmetingen kloppen.
We moeten dus enkel nog bewijzen dat de afmetingen kloppen.
\[
(AB) \ in R^{n\times k} \text{ en } (BC) \in R^{m\times l}
\]
\[
ABC \in  R^{n\times n}
\]
\end{proof}

\subsubsection*{b)}
\emph{te bewijzen}
\[
(AB)^{T} = B^{T}A^{T}
\]
Zie p 32, eigenschappen. Het te bewijzen geldt als de afmetingen kloppen.
We moeten dus enkel nog bewijzen dat de afmetingen kloppen.
\[
(AB) \in R^{n\times k}
\]
\[
(AB)^T \in R^{k\times n}
\]
\[
B^{T} \in B \in R^{k\times m}
\]
\[
A^{T} \in B \in R^{m\times n}
\]
\[
 B^{T}A^{T} \in R^{k\times n}
\]
\emph{bewijs}
\subsection{Oefening 7}
Geen idee, lijkt te evident om op te lossen.
%TODO

\subsection{Oefening 8}
\begin{proof}
\[
E_k\cdots E_1\cdot A = \mathbb{I}_n
\]
\[
E_k\cdots E_1\cdot A \cdot A^{-1} = \mathbb{I}_n \cdot A^{-1} 
\]
\[
E_k\cdots E_1\cdot \mathbb{I}_n =  \cdot A^{-1} 
\]
\end{proof}
\end{document}