\documentclass[lineaire_algebra_oplossingen.tex]{subfiles}
\begin{document}

\newpage
\part{Hoofdstuk 4}
\section{Bewijzen uit het boek}
\subsection{Lemma 4.2 p 130}
\subsubsection*{Te bewijzen}
Een afbeelding $L:V\rightarrow W$ tussen re\"ele vectorruimten is een lineaire afbeelding a.s.a.
$$L(\lambda_1v_1+\lambda_2v_2) = \lambda_1L(v_1)+\lambda_2L(V_2)\ \text{ voor alle } \lambda_1,\lambda_2 \in \mathbb{R} \text{ en voor alle } v_1,v_2 \text{ in V.}$$
\subsubsection*{Bewijs}
We moeten dus aantonen dat:
\begin{enumerate}
\item $L(v_1+v_2) = L(v_1) + L(v_2) \ \forall v_1,v_2 \in V$
\item $L(\lambda v) = \lambda L(v) \ \forall v \in V \ \text{en} \ \forall \lambda \in \mathbb{R}$
\end{enumerate}
\begin{align*}L(\lambda_1v_1+\lambda_2v_2) = \lambda_1L(v_1)+\lambda_2L(V_2)\ \text{voor alle $\lambda_1,\lambda_2 \in \mathbb{R}$ en voor alle $v_1,v_2 \in V$}. \tag{3.}
\end{align*}

\subsubsection*{$\Rightarrow$}
Kies $\lambda_1,\lambda_2 \in \mathbb{R}$ en $v_1,v_2 \in V$ willekeurig.
\begin{align*}
L(\lambda_1 v_1 + \lambda_2 v_2) = L(\lambda_1 v_1) + L(\lambda_2 v_2)\tag{wegens 1.}\\
=\lambda_1 L(v_1)+ \lambda_2 L(v_2) \tag{wegens 2.}
\end{align*}

\subsubsection*{$\Leftarrow$ voor 1.}
Kies $v_1,v_2 \in V$ willekeurig.

\begin{align*}
L(v_1 + v_2) = L(1.v_1+1.v_2) \tag{co\"effici\"ent 1 in V}\\
=1L(v_1)+1L(v_2) \tag{wegens 3.}\\
=L(v_1)+L(v_2) \tag{co\"effici\"ent 1 in W}
\end{align*}

\subsubsection*{$\Leftarrow$ voor 2.}
Neem $v\in V, \lambda \in \mathbb{R}$ willekeurig.
\begin{align*}
L(\lambda v)=L(\lambda v + 0) \tag{neutraal element in V}\\
=L(\lambda v + 0v)\tag{Lemma 3.8}\\
=\lambda L(v) + 0 L(v) \tag{wegens 3.}\\
=\lambda L(v) + 0 \tag{Lemma 3.8 in W}\\
=\lambda L(v) \tag{neutraal element in W}
\end{align*}

\subsection{Gevolg 4.3 p 130}
Zij $L:V\rightarrow W$ een lineaire afbeelding.
\subsubsection*{Te Bewijzen}
\begin{enumerate}
\item
\[\forall v \in V: L(\vec{0})=\vec{0}\]
\item
\[\forall v \in V: L(-v)=-L(v)\]
\item
\[\forall v_i\in V, \lambda_i \in \mathbb{R}: L\left(\sum_{i=1}^n\lambda_iv_i\right) = \sum_{i=1}^n\lambda_iL(v_i)\]
\item
Een lineaire afbeelding ligt volledig vast door de beelden van een basis.
\end{enumerate}
\subsubsection*{Bewijs}
\begin{proof}
Direct bewijs.
\begin{enumerate}
\item
Uit lineariteit van een lineaire afbeelding\footnote{Zie Definitie 4.1 p 130} (duh) volgt het volgende.
\[
L(\vec{0}) = L(\vec{0} + \vec{0}) = L(\vec{0}) + L(\vec{0})
\]
Detail: het volgende geldt enkel omdat $L(\vec{0})$ een element is uit een vectorruimte\footnote{Zie Lemma 3.7 p 93}.
\[
L(\vec{0}) = L(\vec{0}) + L(\vec{0}) \Rightarrow L(\vec{0}) = \vec{0}
\]
\item
We beginnen met iets triviaal.
\[\vec{0}=\vec{0}\]
Zie vorig deel van dit bewijs.
\[L(\vec{0}) = \vec{0}\]
Tegengesteld element in een vectorruimte\footnote{Zie Definitie 3.2 p 88 puntje 3}
\[L(v + (-v)) = \vec{0}\]
Uit lineariteit van een lineaire afbeelding\footnote{Zie Definitie 4.1 p 130} (duh) volgt het volgende.
\[L(v) + L(-v) = \vec{0}\]
Deze overgang lijk triviaal maar het werkt omdat $L(v)$ een element van een vectorruimte is.
\[L(-v) = - L(v)\]
\item
Dit volgt eenvoudig uit de definitie van lineaire afbeelding\footnote{Zie Definitie 4.1 p 130}. Kies willekeurige $v_i$ en $\lambda_i$.
\[L\left(\sum_{i=1}^n\lambda_iv_i\right) =  \sum_{i=1}^nL(\lambda_iv_i)= \sum_{i=1}^n\lambda_iL(v_i)\]
\item
Elke vector $v\in V$ kan geschreven worden als lineaire combinatie van de basisvectoren van $\beta = \{e_1,e_2,...,e_n\}$. Als we nu de basis van $V$ afbeelden krijgen we $L(\beta) = \{L(e_1),L(e_2),...,L(e_n)\}$ We bewijzen nu dat elke vector in $W$ geschreven kan worden als lineaire combinatie van $L(\beta)$.
\[
\forall v \in V:\exists\lambda_i \in \mathbb{R}: v= \sum\lambda_iv_i
\]
\[
\forall v \in V:\exists\lambda_i \in \mathbb{R}: L(v)= L(\sum\lambda_iv_i)
\]
\[
\forall v \in V:\exists\lambda_i \in \mathbb{R}: L(v)= \sum\lambda_iL(v_i)
\]
\end{enumerate}
\end{proof}

\subsection{Voorbeeld 4.4 p 131}
\begin{enumerate}
\item
\begin{enumerate}[a)]
\item De projectie op een as is lineair.
\begin{proof}
\[
p_x \left(\lambda_1
\begin{pmatrix}
u_{1x}\\u_{1y}
\end{pmatrix} 
+ \lambda_2
\begin{pmatrix}
u_{2x}\\u_{2y}
\end{pmatrix} \right)
 = 
p_x \left(
\begin{pmatrix}
\lambda_1u_{1x}\\\lambda_1u_{1y}
\end{pmatrix} 
+ 
\begin{pmatrix}
\lambda_2u_{2x}\\\lambda_2u_{2y}
\end{pmatrix} \right)
=
p_x \left(
\begin{pmatrix}
\lambda_1u_{1x}+\lambda_2u_{2x}\\\lambda_1u_{1y}+\lambda_2u_{2y}\\
\end{pmatrix} \right)
\]
\[
= 
\begin{pmatrix}
\lambda_1u_{1x}+\lambda_2u_{2x}\\0\\
\end{pmatrix}
=
\begin{pmatrix}
\lambda_1u_{1x}\\0\\
\end{pmatrix}
+
\begin{pmatrix}
\lambda_2u_{2x}\\0\\
\end{pmatrix}
=
\lambda_1
\begin{pmatrix}
u_{1x}\\0\\
\end{pmatrix}
+
\lambda_2
\begin{pmatrix}
u_{2x}\\0\\
\end{pmatrix}
\]
\[
=
\lambda_1p_x(u_1) + \lambda_2p_x(u_2)
\]
\end{proof}
\item De spiegeling t.o.v. een as is lineair.
\begin{proof}
\[
s_x \left(\lambda_1
\begin{pmatrix}
u_{1x}\\u_{1y}
\end{pmatrix} 
+ \lambda_2
\begin{pmatrix}
u_{2x}\\u_{2y}
\end{pmatrix} \right)
 = 
s_x \left(
\begin{pmatrix}
\lambda_1u_{1x}\\\lambda_1u_{1y}
\end{pmatrix} 
+ 
\begin{pmatrix}
\lambda_2u_{2x}\\\lambda_2u_{2y}
\end{pmatrix} \right)
=
s_x \left(
\begin{pmatrix}
\lambda_1u_{1x}+\lambda_2u_{2x}\\\lambda_1u_{1y}+\lambda_2u_{2y}\\
\end{pmatrix} \right)
\]
\[
= 
\begin{pmatrix}
\lambda_1u_{1x}+\lambda_2u_{2x}\\-(\lambda_1u_{1y}+\lambda_2u_{2y})\\
\end{pmatrix} 
=
\begin{pmatrix}
\lambda_1u_{1x}\\-\lambda_2u_{2x}\\
\end{pmatrix}
+
\begin{pmatrix}
\lambda_2u_{2x}\\-\lambda_2u_{2y}\\
\end{pmatrix}
=
\lambda_1
\begin{pmatrix}
u_{1x}\\-u_{2x}\\
\end{pmatrix}
+
\lambda_2
\begin{pmatrix}
u_{2x}\\-u_{2y}\\
\end{pmatrix}
\]
\[
=
\lambda_1s_x(u_1) + \lambda_2s_x(u_2)
\]
\end{proof}
(De spiegeling t.o.v. een rechte die niet door de oorsprong gaat is niet lineair.)
\item De puntspiegeling t.o.v. de oorsprong is lineair.
\begin{proof}
\[
s_O \left(\lambda_1
\begin{pmatrix}
u_{1x}\\u_{1y}
\end{pmatrix} 
+ \lambda_2
\begin{pmatrix}
u_{2x}\\u_{2y}
\end{pmatrix} \right)
 = 
s_O \left(
\begin{pmatrix}
\lambda_1u_{1x}\\\lambda_1u_{1y}
\end{pmatrix} 
+ 
\begin{pmatrix}
\lambda_2u_{2x}\\\lambda_2u_{2y}
\end{pmatrix} \right)
=
s_O \left(
\begin{pmatrix}
\lambda_1u_{1x}+\lambda_2u_{2x}\\\lambda_1u_{1y}+\lambda_2u_{2y}\\
\end{pmatrix} \right)
\]
\[
= 
\begin{pmatrix}
-(\lambda_1u_{1x}+\lambda_2u_{2x})\\-(\lambda_1u_{1y}+\lambda_2u_{2y})\\
\end{pmatrix} 
=
\begin{pmatrix}
-\lambda_1u_{1x}\\-\lambda_2u_{2x}\\
\end{pmatrix}
+
\begin{pmatrix}
-\lambda_2u_{2x}\\-\lambda_2u_{2y}\\
\end{pmatrix}
=
\lambda_1
\begin{pmatrix}
-u_{1x}\\-u_{2x}\\
\end{pmatrix}
+
\lambda_2
\begin{pmatrix}
-u_{2x}\\-u_{2y}\\
\end{pmatrix}
\]
\[
=
\lambda_1s_O(u_1) + \lambda_2s_O(u_2)
\]
\end{proof}
(De puntspiegeling t.o.v. een ander punt is niet lineair.)
\item De homothetie met factor $\lambda$ en centrum de oorsprong is lineair.
\begin{proof}
\[
h_\lambda \left(\lambda_1
\begin{pmatrix}
u_{1x}\\u_{1y}
\end{pmatrix} 
+ \lambda_2
\begin{pmatrix}
u_{2x}\\u_{2y}
\end{pmatrix} \right)
 = 
h_\lambda \left(
\begin{pmatrix}
\lambda_1u_{1x}\\\lambda_1u_{1y}
\end{pmatrix} 
+ 
\begin{pmatrix}
\lambda_2u_{2x}\\\lambda_2u_{2y}
\end{pmatrix} \right)
=
h_\lambda \left(
\begin{pmatrix}
\lambda_1u_{1x}+\lambda_2u_{2x}\\\lambda_1u_{1y}+\lambda_2u_{2y}\\
\end{pmatrix} \right)
\]
\[
= 
\begin{pmatrix}
\lambda\lambda_1u_{1x}+\lambda\lambda_2u_{2x}\\\lambda\lambda_1u_{1y}+\lambda\lambda_2u_{2y})\\
\end{pmatrix} 
=
\begin{pmatrix}
\lambda\lambda_1u_{1x}\\\lambda\lambda_2u_{2x}\\
\end{pmatrix}
+
\begin{pmatrix}
\lambda\lambda_2u_{2x}\\\lambda\lambda_2u_{2y}\\
\end{pmatrix}
=
\lambda_1
\begin{pmatrix}
\lambda u_{1x}\\\lambda u_{2x}\\
\end{pmatrix}
+
\lambda_2
\begin{pmatrix}
\lambda u_{2x}\\\lambda u_{2y}\\
\end{pmatrix}
\]
\[
=
\lambda_1p_x(u_1) + \lambda_2p_x(u_2)
\]
\end{proof}
(een homothetie waarbij het centrum niet de oorsprong is is geen lineaire afbeelding.)
\end{enumerate}
\item Translaties volgens een vector verschillend van de nulvector zijn \textbf{geen} lineaire afbeeldingen.
\begin{proof}
De nulvector $\vec{0}$ wordt niet op zichzelf afgebeeld.
\end{proof}
\item
\begin{enumerate}[a)]
\item De identieke afbeelding is een lineaire afbeelding.
\begin{proof}
\[
Id(\lambda_1v_1+\lambda_2v_2) = \lambda_1v_1+\lambda_2v_2 = \lambda_1Id(v_1)+\lambda_2Id(v_2))
\]
\end{proof}
\item De nulafbeelding is een lineaire afbeelding.
\begin{proof}
\[
(0)(\lambda_1v_1+\lambda_2v_2) = \vec{0} = \lambda_1(0)(v_1)+\lambda_2(0)(v_2))
\]
\end{proof}
\item De $(+1)$ afbeelding is lineair in de binaire ringruimte.
\begin{proof}
Dit is geen belangrijk bewijs, maar het is makkelijk in te zien.
\[
(+1)(\lambda_1v_1+\lambda_2v_2) = \lambda_1(+1)(v_1)+\lambda_2(+1)(v_2))
\]
\end{proof}
\end{enumerate}
\item De afgeleide afbeelding is een lineaire afbeelding
\begin{proof}
De optelling en de scalaire vermenigvuldiging gaan door de afgeleide.
\[
D(\lambda_1v_1+\lambda_2v_2) = \lambda_1D(v_1)+\lambda_2D(v_2))
\]
\end{proof}
\item Elke lineaire combinatie van afgeleiden is een lineare afbeelding.
\begin{proof}
Dit volgt meteen uit het vorige bewijs.
\end{proof}
\item De bepaalde integraal is een lineaire afbeelding.
\begin{proof}
De optelling en de scalaire vermenigvuldiging gaan door de bepaalde integraal.
\[
\int_a^b(\lambda_1v_1+\lambda_2v_2) = \lambda_1\int_a^b(v_1)+\lambda_2\int_a^b(v_2))
\]
\end{proof}
\item De volgende afbeelding is een lineaire afbeelding.
\[f: \mathbb{R}^2 \rightarrow \mathbb{R}: (x,y) \mapsto ax+by\]
\begin{proof}
\[
f(\lambda_1v_1+\lambda_2v_2) = (a\lambda_1v_{1x}+b\lambda_1(v_{1y})+(a\lambda_2v_{2x}+b\lambda_2v_{2y})
\]
\[
= \lambda_1(av_{1x}+bv_{1y})+\lambda_2(av_{2x}+bv_{2y}) = \lambda_1f(v_1)+\lambda_2f(v_2))
\]
\end{proof}
\item De volgende afbeelding is niet lineair
\[g: \mathbb{R}^2 \rightarrow \mathbb{R}: (x,y) \mapsto ax+by+c\]
\begin{proof}
De nulvector $\vec{0}$ wordt niet op zichzelf afgebeeld maar op $c$.
\end{proof}
\item De projecties $p_i$ die co\"ordinaatvectoren afbeelden op individuele co\"ordinaten zijn lineaire vormen.
\begin{proof}
Het is duidelijk dat deze afbeelding op $(\mathbb{R},\mathbb{R},+)$ wordt afgebeeld. Nu rest er ons nog te bewijzen dat $p_i$ lineair is.
\[
p_i(\lambda_1v_1+\lambda_2v_2) = (\lambda_1v_1+\lambda_2v_2)_i
= \lambda_1v_{1i}+\lambda_2v_{2i} = \lambda_1p_i(v_1)+\lambda_2p_i(v_2))
\]
\end{proof}
\item De spoorafbeelding is een lineaire afbeelding.
TODO BEWIJS
\item Is de overgang zoals beschreven op pagina 134 een lineaire afbeelding.
TODO BEWIJS
\item Elke matrix bepaalt een lineaire afbeelding.
TODO BEWIJS.
\item Bewijs dat dit een lineaire afbeelding is.
TODO BEWIJS
\item Bewijs dat $\Delta$ een lineare afbeelding is.
TODO BEWIJS
\end{enumerate}
\subsection{Propositie 4.10 p 141}
Zij $(\mathbb{R},v,+)$ en $(\mathbb{R},W,+)$ vectorruimten. Zij $f$ en $g$ lineaire afbeeldingen $V\rightarrow W$.
\subsubsection*{Te Bewijzen}
\begin{enumerate}
\item De volgende afbeelding is lineair.
\[f+g: V \rightarrow W: v\mapsto(f+g)(v)=f(v)+g(v)\]
\item De volgende afbeelding is eveneens lineair.
\[\lambda f: V\rightarrow W: v\mapsto (\lambda f)(v) = \lambda f(v)\]
\item De verzameling van alle lineaire afbeeldingen van $V$ naar $W$ is een lineaire afbeelding.
\end{enumerate}
\subsubsection*{Bewijs}
\begin{proof}
Direct bewijs.
\begin{enumerate}
\item
We veri\"eren de lineariteit van deze afbeelding.
\[
(f+g)(\lambda_1v_1+\lambda_2v_2) = f(\lambda_1v_1+\lambda_2v_2) + g(\lambda_1v_1+\lambda_2v_2)
\]
Nu geldt door de lineariteit van $f$ en $g$ het volgende.
\[
= \lambda_1f(v_1)+\lambda_2f(v_2) + \lambda_1g(v_1)+\lambda_2g(v_2) = \lambda_1(f(v_1)+g(v_1)) + \lambda_2(f(v_1)+g(v_1))
\]
Dit is precies wat nodig is voor lineariteit.
\[
\lambda_1(f+g)(v_1)+\lambda_2(f+g)(v_2)
\]
\item
We veri\"eren de lineariteit van deze afbeelding.
\[
(\lambda f)(\lambda_1v_1+\lambda_2v_2) = \lambda f(\lambda_1v_1+\lambda_2v_2)
\]
De volgende gelijkheid geldt door de lineariteit van $f$. 
\[
=  f(\lambda_1\lambda v_1+\lambda_2\lambda v_2) = \lambda_1\lambda f(v_1)+\lambda_2\lambda f(v_2) = 
\lambda_1(\lambda f)(v_1)+\lambda_2(\lambda f)(v_2)
\]
\item
TODO ga alle axioma's na.
\end{enumerate}
\end{proof}

\subsection{Stelling 4.12 p 142}
Zij $(\mathbb{R},V,+)$ en  $(\mathbb{R},W,+)$ eindigdimensionale vecotrruimten van dimensie respectievelijk $n$ en $m$.
\subsubsection*{Te Bewijzen}
Na basiskeuze in $V$ en $W$ is de vectorruimte van de lineaire afbeeldingen van $V$ naar $W$, namelijk $Hom_{\mathbb{R}}(V,W)$, in bijectief verband met de vectorruimte van de matrices $\mathbb{R}^{m\times n}$.
\subsubsection*{Bewijs}
\begin{proof}
TODO BEWIJS: HOE
\end{proof}

\subsection{Stelling 4.13 p 142}
Zij $K:U\rightarrow V$ en $L:V\rightarrow W$ lineaire afbeeldingen.
\subsubsection*{Te Bewijzen}
De samenstelling van twee lineaire afbeeldingen is opnieuw een lineaire afbeelding.
\[
\forall u_1,u_2\in U \lambda_1,\lambda_2\in\mathbb{R}: (L\circ K)(\lambda_1u_1 + \lambda_2u_2) = \lambda_1(L\circ K)u_1 + \lambda_2(L\circ K)u_2
\]
\subsubsection*{Bewijs}
\begin{proof}
Kies willekeurige $u_1,u_2\in U$ en $\lambda_1,\lambda_2\in\mathbb{R}$
\[
(L\circ K)(\lambda_1u_1 + \lambda_2u_2) = L(K(\lambda_1u_1 + \lambda_2u_2))
\]
Bovenstaande gelijkheid geldt volgens de definitie van samengestelde afbeeldingen. Zie \ref{samenstelling_van_afbeeldingen}. De volgende gelijkheden gelden door de lineariteit van $L$ en $K$
\[
= L(\lambda_1K(u_1) + \lambda_2K(u_2)) = \lambda_1L(K(u_1)) + \lambda_2L(K(u_2))
\]
Opnieuw volgens de definitie van samengestelde afbeeldingen geldt de volgende vergelijking.
\[
= \lambda_1(L\circ K)u_1 + \lambda_2(L\circ K)u_2
\]
Dit betekent precies dat de samenstelling van lineaire afbeeldingen lineair is.
\end{proof}

\subsection{Stelling 4.16 p 145}
Zij $L:V\rightarrow W$ een bijectieve lineaire afbeelding van de vectorruimte $(\mathbb{R},V,+)$ naar de vectorruimte $(\mathbb{R},W,+)$.
\subsubsection*{Te Bewijzen}
De inverse afbeelding van $L$ is (noem het $K$) is linear.
\subsubsection*{Bewijs}
\begin{proof}
$K$ is ook bijectief. Te bewijzen is nu nog dat $K$ lineair is.
\[
(L\circ K)(\lambda_1u_1 + \lambda_2u_2) = \lambda_1u_1 + \lambda_2u_2 = L(\lambda_1K(u_1) + \lambda_2K(u_2))
\]
Omdat $L$ injectief is geldt de volgende vergelijking.
\[
K(\lambda_1u_1 + \lambda_2u_2) = \lambda_1K(u_1) + \lambda_2K(u_2)
\]
\end{proof}

\section{Oefeningen 4.8}
\subsection{Oefening 1}
\subsubsection*{a)}
Deze afbeelding is niet lineair. Tegenvoorbeeld: Kies $\lambda = 2$\\
\[
\lambda L(v) = 2 . 2 . x + 2 \neq 2 . 2 x + 1 = L(\lambda v)
\]

\subsubsection*{b)}
Deze afbeelding is lineair.
\begin{proof}
Kies willekeurige $v_1, v_2 \in \mathbb{R}^2$ en $\lambda_1, \lambda_2 \in \mathbb{R}^2$
\[
L_2(\lambda_1v_1 + \lambda_2v_2)
\]
TODO afmaken
\end{proof}

\subsubsection*{c)}
$$
L_3:\mathbb{R}^2\rightarrow\mathbb{R}:(x,y)\mapsto |x-y|
$$
$$
\text{Kern van L:} \ \{(x,y)|x=y \wedge x,y \in \mathbb{R} \}
$$
$$
\text{Beeld van L:} \ \{(x,y)|x\neq y \wedge x,y \in \mathbb{R} \}
$$
\begin{center}
Niet isomorf, hoe moet ik dit verklaren?
\end{center}

\subsubsection*{e)}
$$
L_5:\mathbb{R}^3\rightarrow\mathbb{R}^2:(x,y,z)\mapsto (3y+z,x-y-z)
$$
$$
\text{Kern van L:} \ \{(4\lambda ,\lambda ,-3\lambda)|\lambda \in \mathbb{R} \}
$$
$$
\text{Beeld van L:} \ \{(x ,y ,z)|x,y,z \in \mathbb{R} \}
$$

\subsubsection*{j)}
<TODO Aan Jorik: nakijken!>
\[
ker(ev_a) = \{ f \in \mathbb{R}^{\mathbb{R}} | f(a) = 0\}
\]
\[
Im(ev_a) = \mathbb{R}
\]
Hiervoor moeten we bewijzen dat $\mathbb{R} \subset Im(ev_a)$.
\begin{proof}
Om te bewijzen dat $ev_a$ lineair is nemen we twee vectoren $f_1,f_2 \in \mathbb{R}^{\mathbb{R}}$ en twee scalars $\lambda_1, \lambda_2 \in \mathbb{R}$.
\[
ev_a(\lambda_1f_1 + \lambda_2f_2) = (\lambda_1f_1 + \lambda_2f_2)(a) = \lambda_1f_1(a) + \lambda_2f_2(a) = \lambda_1ev_a(f_2) + \lambda_2ev_a(f_1)
\]
<TODO verwijs naar lemmaatje>
\end{proof}

\subsection{Oefening 5}
Zij $L:V \rightarrow W$ een lineaire afbeelding en zij $dim V = n$ en $dim W = m$.\\
\subsubsection*{a)}
\textbf{Te Bewijzen}
\[
n > m \Rightarrow L \text{ is niet injectief}
\]
\textbf{Bewijs}
\begin{proof}
Bewijs uit het ongerijmde.\\
Stel $n>m$ en $L$ is injectief.
\[
dim(V) = dim(ker(L)) + dim(Im(L))
\]
Als $L$ injectief is dan $ker(L)=\{0\}$.\footnote{Zie stelling 4.29 p 156} en $Im(L) \subset W$ dus $dim(Im(L))\le dim(W)$. We weten nu het volgende en dit is een contradictie met $n>m$.
\[
dim(V) \le dim(W)
\]
\end{proof}
\subsubsection*{b)}
\textbf{Te Bewijzen}
\[
n < m \Rightarrow L \text{ is niet surjectief}
\]
\textbf{Bewijs}
\begin{proof}
Bewijs uit het ongerijmde.\\
Stel $n<m$ en $L$ is surjectief.
\[
\forall w\in W:\exists v\in V: L(v)=w
\]
Hieruit volgt dat $Im(V)=W$ dus ook dat $dim(Im(V))=dim(W)$.
We kijken nu opnieuw naar stelling 4.31 p 157.
\[
dim(V) = dim(ker(L)) + dim(Im(L))
\]
Als we dit invullen krijgen we het volgende.
\[
dim(V) = dim(ker(L)) + dim(W)
\]
Hieruit volgt dat $dim(V) \ge dim(W)$ want $dim(ker(L)) \ge 0$. Dit is in contradictie met $n<m$.
\end{proof}

\subsection{Oefening 8}
Om de matrices van de basisverandering te berekenen stellen we voor beide bases een matrix op waarbij ze beide uitgeschreven worden ten opzichte van eenzelfde basis, bijvoorbeeld, de standaardbasis, en als kolommen gebruikt worden in bijhorende matrix. Deze matrices plaatsen we naast elkaar en rijreduceren we tot de linkse matrix de eenheidsmatrix is. De eerste oefening is relatief triviaal, zoals we zullen zien.
\subsubsection*{a)}
We schrijven beide uit in functie van standaardbasis \{(0,1), (1,0)\}, zetten de bekomen co\"ordinaten in een matrix als kolommen (wat hier niet overdreven duidelijk is, maar in $b$ en $c$ hopelijk wel, waarna we beide matrices aan elkaar te zetten en komen tot

\[
\left[
\begin{array}{ c c | c c }
1 & 0 & 5 & 1\\
0 & 1 & 1 & 2
\end{array}
\right]
\]

De linkerkant is reeds de eenheidsmatrix dus geldt dat de matrix 
\[
\left[
\begin{array}{ c c }
5 & 1\\
1 & 2
\end{array}
\right]
\]
de matrix voor de basisverandering van $\alpha$ naar $\beta$. We zeggen altijd dat het de basisverandering van onze basis van de rechtse matrix is naar die van de linkse matrix. Voor de andere matrix, van $\beta$ naar $\alpha$ rijreduceren we dezelfde matrix, maar met de linker en rechterhelft omgewisseld, wat dus neerkomt op de inverse zoeken van zijn andere basisveranderingmatrix. We vinden

\[
\left[
\begin{array}{ c c | c c }
5 & 1 & 1 & 0\\
1 & 2 & 0 & 1
\end{array}
\right]
\]

\[R1 \leftrightarrow R2 \]

\[
\left[
\begin{array}{ c c | c c }
1 & 2 & 0 & 1\\
5 & 1 & 1 & 0
\end{array}
\right]
\]

\[R2 \mapsto R2 - 5*R1 \]

\[
\left[
\begin{array}{ c c | c c }
1 & 2 & 0 & 1\\
0 & -9 & 1 & -5
\end{array}
\right]
\]

\[R2 \mapsto \frac{R2}{-9} \]

\[
\left[
\begin{array}{ c c | c c }
1 & 2 & 0 & 1\\
0 & 1 & \frac{-1}{9} & \frac{5}{9}
\end{array}
\right]
\]

\[R1 \mapsto R1 - 2*R2 \]

\[
\left[
\begin{array}{ c c | c c }
1 & 0 & \frac{2}{9} & \frac{-1}{9}\\
0 & 1 & \frac{-1}{9} & \frac{5}{9}
\end{array}
\right]
\]

\subsubsection*{b)}

We nemen deze keer als standaardbasis \{$\bigl(
\begin{smallmatrix}
1&0\\ 0&0
\end{smallmatrix}
\bigr)$,
$\bigl(
\begin{smallmatrix}
0&1\\ 0&0
\end{smallmatrix}
\bigr)$,
$\bigl(
\begin{smallmatrix}
0&0\\ 1&0
\end{smallmatrix}
\bigr)$,
$\bigl(
\begin{smallmatrix}
0&0\\ 0&1
\end{smallmatrix}
\bigr)$\}, waardoor we de opgegeven bases in volgende matrix kunnen weergeven (met $\beta$ links):

\[
\left[
\begin{array}{ c c c c | c c c c }
1 & 0 & 0 & 0 & 1 & 3 & 1 & 2\\
0 & 0 & 0 & 1 & 1 & 1 & -1 & 1\\
0 & 1 & 0 & 0 & -1 & -1 & 2 & -1\\
0 & 0 & 1 & 0 & 2 & 2 & -3 & 0
\end{array}
\right]
\]

dewelke we door achtereenvolgens $R2 \leftrightarrow R4$ en $R2 \leftrightarrow R3$ toe te passen omzetten tot

\[
\left[
\begin{array}{ c c c c | c c c c }
1 & 0 & 0 & 0 & 1 & 3 & 1 & 2\\
0 & 1 & 0 & 0 & -1 & -1 & 2 & -1\\
0 & 0 & 1 & 0 & 2 & 2 & -3 & 0\\
0 & 0 & 0 & 1 & 1 & 1 & -1 & 1
\end{array}
\right]
\]

De rechtse kant is dus de basisveranderingsmatrix van $\alpha$ naar $\beta$. Rekenen we nu de andere kant uit hebben we

\[
\left[
\begin{array}{ c c c c | c c c c }
1 & 3 & 1 & 2 & 1 & 0 & 0 & 0\\
-1 & -1 & 2 & -1 & 0 & 1 & 0 & 0\\
2 & 2 & -3 & 0 & 0 & 0 & 1 & 0\\
1 & 1 & -1 & 1 & 0 & 0 & 0 & 1
\end{array}
\right]
\]

dewelke we door rijreductie terugbrengen tot

\[
\left[
\begin{array}{ c c c c | c c c c }
1 & 0 & 0 & 0 & \frac{-1}{2} & \frac{-3}{4} & \frac{1}{4} & \frac{1}{4}\\
0 & 1 & 0 & 0 & \frac{1}{2} & \frac{1}{4} & \frac{1}{4} & \frac{-3}{4}\\
0 & 0 & 1 & 0 & 0 & \frac{-1}{3} & 0 & \frac{-1}{3}\\
0 & 0 & 0 & 1 & 0 & \frac{1}{6} & \frac{-1}{2} & \frac{7}{6}
\end{array}
\right]
\]

De rechterkant van deze laatste is dus de basisveranderingsmatrix van basis $\beta$ naar $\alpha$.

\subsubsection*{c)}
We nemen als standaardbasis \{1, X, X$^2$\}, waardoor we als beginmatrix voor ons rekenwerk als makkelijkste deze krijgen:

\[
\left[
\begin{array}{ c c c | c c c }
0 & 0 & 1 & 3 & 4 & 2\\
0 & 1 & 0 & 2 & 0 & 1\\
1 & 0 & 0 & 1 & -1 & 0
\end{array}
\right]
\]

\[R1 \leftrightarrow R3 \]

\[
\left[
\begin{array}{ c c c | c c c }
1 & 0 & 0 & 1 & -1 & 0\\
0 & 1 & 0 & 2 & 0 & 1\\
0 & 0 & 1 & 3 & 4 & 2
\end{array}
\right]
\]

Deze laatste is dus onze basisveranderingsmatrix van $\alpha$ naar $\beta$. Rekenen we de andere richting uit, bekomen we

\[
\left[
\begin{array}{ c c c | c c c }
1 & -1 & 0 & 1 & 0 & 0\\
2 & 0 & 1 & 0 & 1 & 0\\
3 & 4 & 2 & 0 & 0 & 1
\end{array}
\right]
\]

Wat we door rijreductie omvormen tot

\[
\left[
\begin{array}{ c c c | c c c }
1 & 0 & 0 & \frac{4}{3} & \frac{-2}{3} & \frac{1}{3}\\
0 & 1 & 0 & \frac{1}{3} & \frac{-2}{3} & \frac{1}{3}\\
0 & 0 & 1 & \frac{-8}{3} & \frac{7}{3} & \frac{-2}{3}
\end{array}
\right]
\]

De rechterkant van deze laatste is dus de basisveranderingsmatrix van basis $\beta$ naar $\alpha$.

\subsection{Oefening 9}

Berekenen we voor de elementen van de basis hun lineaire afbeelding als volgt:\\

$T(e1) = T( (1,0,0) ) = (2,5,4) = 2 * e1 + 5 * e2 + 4 * e3$\\

$T(e2) = T( (0,1,0) ) = (-3,-1,7) = -3 * e1 + -1 * e2 + 7 * e3$\\

$T(e3) = T( (0,0,1) ) = (4,2,0) = 4 * e1 + 2 * e2$\\

We plaatsen dezen in de matrix van de lineaire afbeelding als kolommen en bekomen zo\\

\[
T^{\alpha}_{\alpha} =
\left[
\begin{array}{c c c}
2 & -3 & -4\\
5 & -1 & 2\\
4 & 7 & 0
\end{array}
\right]
\]\\

Voor de tweede basis zijn 2 elementen van plaats verwisseld. We kunnen deze verwisselen bij de berekening en bekomen dan de vergelijkbare matrix

\[
T^{\beta}_{\beta} =
\left[
\begin{array}{c c c}
4 & -3 & 2\\
2 & -1 & 5\\
0 & 7 & 4
\end{array}
\right]
\]


\subsection{Oefening 10}

\[
T^{\alpha}_{\alpha} =
\left[
\begin{array}{c c c c}
0 & 0 & 0 & 0\\
1 & 1 & 1 & 0\\
0 & 0 & 2 & 0\\
0 & 0 & 0 & 3
\end{array}
\right]
\]\\

\textbf{VERBETERING}
\[
\left\lbrace
\begin{array}{r l}
T(X) &= X-4\\
T(1+X) &= X-3\\
T(X+X^2) &= X^2-7X-3\\
T(X^3) &= X^3-12X^2+6X\\
\end{array}
\right.
\]

\[
T^{\alpha}_{\alpha} =
\left[
\begin{array}{c c c c}
5 & 4 & 0-4 & 18\\
-4 & -3 & -3 & 0\\
0 & 0 & 1 & -12\\
0 & 0 & 0 & 1
\end{array}
\right]
\]\\

\subsection{Oefening 11}
\subsubsection*{a)}
We berekenen hoe we de vectoren in $\alpha$ schrijven als lineaire combinatie van de vectoren in $\beta$.
\[
\left\lbrace
\begin{array}{c c c c}
(1,0,0) &= 1\cdot (1,0,0) &+ 0\cdot (1,1,0) &+ 0\cdot (1,1,1)\\
(0,1,0) &= -1\cdot(1,0,0) &+ 1\cdot (1,1,0) &+ 0\cdot (1,1,1)\\
(0,0,1) &= 0\cdot(1,0,0) &-1\cdot (1,1,0) &+ 1\cdot (1,1,1)
\end{array}
\right.
\]
Hieruit volgt het volgende.
\[
Id_\alpha^\beta = 
\begin{pmatrix}
1 & -1 & 0\\
0 & 1 & -1\\
0 & 0 & 1\\
\end{pmatrix}
\]
Omgekeerd gaat het analoog.
\[
\left\lbrace
\begin{array}{c c c c}
(1,0,0) &= 1\cdot (1,0,0) &+ 0\cdot (0,1,0) &+ 0\cdot (0,0,1)\\
(1,1,0) &= 1\cdot (1,0,0) &+ 1\cdot (0,1,0) &+ 0\cdot (0,0,1)\\
(1,1,1) &= 1\cdot (1,0,0) &+ 1\cdot (0,1,0) &+ 1\cdot (0,0,1)
\end{array}
\right.
\]
\[
Id_\beta^\alpha = 
\begin{pmatrix}
1 & 1 & 1\\
0 & 1 & 1\\
0 & 0 & 1\\
\end{pmatrix}
\]
\subsubsection*{b)}
\[
\left\lbrace
\begin{array}{c l c c c}
T((1,0,0)) &= (1,0,1) &= 1\cdot (1,0,0) &+ 0\cdot (0,1,0) &+ 1\cdot (0,0,1) \\
T((0,1,0)) &= (2,-1,0)&= 2\cdot (1,0,0) &- 1\cdot (0,1,0) &+ 0\cdot (0,0,1)\\
T((0,0,1)) &= (1,0,4) &= 1\cdot (1,0,0) &+ 0\cdot (0,1,0) &+ 4\cdot (0,0,1)\\ 
\end{array}
\right.
\]
\[
Id_\alpha^\alpha = 
\begin{pmatrix}
1 & 2 & 1\\
0 & -1 & 0\\
1 & 0 & 4\\
\end{pmatrix}
\]

\subsubsection*{c)}
\[
\left\lbrace
\begin{array}{c l c c c}
T((1,0,0)) &= (1,0,1) &= 1\cdot (1,0,0) &- 1\cdot (1,1,0) &+ 1\cdot (1,1,1)\\
T((1,1,0)) &= (3,-1,1)&= 4\cdot (1,0,0) &- 2\cdot (1,1,0) &+ 1\cdot (1,1,1)\\
T((1,1,1)) &= (4,-1,5)&= 5\cdot (1,0,0) &- 6\cdot (1,1,0) &+ 5\cdot (1,1,1)\\
\end{array}
\right.
\]
\[
Id_\beta^\beta = 
\begin{pmatrix}
1 & 4 & 5\\
-1 & -2 & -6\\
1 & 1 & 5\\
\end{pmatrix}
\]

\subsection{Oefening 17}
AAN WARD: nog nakijken, gewoon opgeschreven wat er in de oz gezegd werd.


\subsubsection*{a)}
Voor alle $v \in V$ geldt het volgende.
\[
v
=
a_1
\begin{pmatrix}
1\\0\\0\\
\end{pmatrix}
+
a_2
\begin{pmatrix}
1\\0\\0\\
\end{pmatrix}
+
a_3
\begin{pmatrix}
0\\0\\1\\
\end{pmatrix}
\]
Als $v$ nu in $ker(L)$ zit dan geldt ook het volgende.
\[
L(v) = 0 = a_1 \cdot (4\beta_1) + a_2 \cdot (2 \beta_1 + 1\beta_2) + a_3 \cdot (\beta_1 + 3\beta_2)
\]
\[
= (4a_1+2a_2+a_3)\cdot\beta_1 + (a_2+3a_3)\cdot\beta_2
\]
\[
\rightarrow
\left\lbrace
\begin{array}{c c}
(4a_1+2a_2+a_3) &= 0\\
(a_2+3a_3) &= 0\\
\end{array}
\right.
\]
Als we dit oplossen vinden we de coordinaten van kern van $L$ volgens basis $\alpha$.
\[
(a_1,a_2,a_3) = 
\left\{(\frac{5}{4}\lambda,-3\lambda,\lambda) | \lambda \in \mathbb{R}\right\}
\]
Dit moeten we echter nog omvormen tot de kern van $L$.
\[
\frac{5}{4}\cdot
\begin{pmatrix}
1\\0\\0\\
\end{pmatrix}
+
-3
\begin{pmatrix}
1\\0\\0\\
\end{pmatrix}
+
\begin{pmatrix}
0\\0\\1\\
\end{pmatrix}
=
\begin{pmatrix}
-\frac{3}{4}\\-2\\1\\
\end{pmatrix}
\]
De dimensie van $ker(L)$ is $1$ met $\left\{\begin{pmatrix}-\frac{3}{4}\\-2\\1\\\end{pmatrix}\right\}$ als basis.
We weten nu dat de dimensie van het beeld van $L$ gelijk is aan de dimensie van $\mathbb{R}^3$ min de dimensie van de kern\footnote{Zie stelling 4.31 p 157}. $dim(Im(L))= 2$ met basis ... <TODO aanvullen, verduidelijken...>

\subsubsection*{b)}
\[
L(x,y,z) = (4x-y+z,y+2z)
\]

\subsection{Oefening 20}
<TODO THUIS HERMAKEN>
We berekenen de rang van $L_{\epsilon_2}^{\epsilon_3}$. Deze is $1$.
Wanneer we nu kijken naar de afbeelding $L_{W}^V$ zien we dat $L(v_1) = w_1$ en $L(v_2) = 0$. $v_2$ Zit dus in de kern van $L$.
Zoek een $w_1,w_2,w_3$ die voldoen.

\section{Opdrachten}

\subsection{4.5 p 136}
\subsubsection*{a)}
\[
\mathbb{C} = (\mathbb{R},\mathbb{C},+)
\]
\textbf{Te Bewijzen}\\
\[
\forall v_1,v_2 \in \mathbb{C}: L(v_1+v_2) = L(v_1)+L(v_2) 
\]
\[
\forall v\in V, \lambda \in \mathbb{R}: L(\lambda v) = \lambda L(v)
\]
\textbf{Bewijs}\\
\begin{proof}
Rechtstreeks bewijs\\
Neem twee willekeurige vectoren $v_1 = x_1+y_1i$ en $v_2=x_2+y_2i$.
\[
L(v_1+v_2)=L( (x_1+y_1i) + (x_2+y_2i)) = L((x_1+x_2) + i(y_1+y_2))
\]
\[
= (x_1+x_2)-i(y_1+y_2) = (x_1-y_1i) + (x_2-y_2i)
\]
\[
= L(x_1+y_1i) + L(x_2+y_2i) = L(v_1)+L(v_2)
\]
Tot zover deel $1$ van het bewijs.\\
Neem nu een willekeurige vector $v=x+yi$ en een willekeurige scalar $\lambda$.
\[
L(\lambda v) = L(\lambda (x+yi)) = L(\lambda x + \lambda yi) = \lambda x - \lambda yi
\]
\[
\lambda (x-yi) = \lambda L(x+yi) = \lambda L(v)
\]
Nu hebben we bewezen dat $T$ een lineaire afbeelding is over $(\mathbb{R},\mathbb{C},+)$.
\end{proof}

\subsubsection*{b)}
\[
\mathbb{C} = (\mathbb{C},\mathbb{C},+)
\]
$T$ is geen lineaire afbeelding over $(\mathbb{C},\mathbb{C},+)$ want $T$ voldoet niet aan de definitie van een lineaire afbeelding\footnote{Zie definitie 4.1 p 130}. \\
De volgende formule geldt namelijk niet.
\[\forall v\in V, \lambda \in \mathbb{R}: L(\lambda v) = \lambda L(v)\]
\textbf{Tegenvoorbeeld}\\
Kies $\lambda = i \in \mathbb{C}$ en $v = 2+5i\in \mathbb{C}$.
Het linkerlid is dan gelijk aan de volgende expressie.
\[
L(i(2+5i)) = L(2i-5) = -5 -2i
\]
Terwijl het rechterlid gelijk is aan de volgende expressie.
\[
iL(2+5i) = i(2-5i) = 2i+5
\]
Deze expressies zijn niet gelijk.

\subsection*{oef 26}

Volgens de 2 stellingen $4.40$ en $4.41$ nxn matrix is uitgesloten(4.42)

gegeven: $ A ^ {m x n} $

1) Volgens 4.40: 

$rang(A) = n $ als rijen lineair onafhankelijk zijn.
Deze zijn vrij indien $x != -2 of -3$ Hierdoor zouden de rijen immers lineair afhankelijk worden! 

2) volgens 4.41

$rang(A) = m $ als de kolommen lineair onafhankelijk zijn. De eerste 2 rijen zijn duidelijk niet vrij, dus kunnen we dit geval uitsluiten.


\subsection{4.9 p 140}
\subsubsection*{1)}
We bepalen eerst de co\"ordinaten van $(2,0,5)$ ten opzichte van de eerste basis. Aangezien het over een transformatie gaat is $\beta$ zowel de eerste als de tweede basis. 
Na het opstellen van een stelsel en rijreductie vinden we dat de co\"ordinaten $\left(\frac{12}{5},\frac{3}{5},\frac{-2}{5}\right)$ zijn.

Voeren we nu de vermenigvuldiging van de matrix van $L$ met de bekomen co\"ordinaten vinden we:\\
$$
\begin{pmatrix}
1&0&5\\
0&-2&2\\
1&-2&7
\end{pmatrix}
\cdot
\begin{pmatrix}
\frac{12}{\strut5}\\ \frac{3}{5}\\ \frac{\strut -2}{5}
\end{pmatrix}
=
\begin{pmatrix}
\strut \frac{2}{5}\\ -2\\ \strut \frac{-8}{5}
\end{pmatrix}
$$
Dit is dan ook het antwoord op de vraag.
\subsubsection*{2a)}
We gaan na of de verzameling lineair onafhankelijk is door de determinant te berekenen:
$$
\begin{vmatrix}
2 & 0 & -2\\
0&3&3\\
3&-2&-4
\end{vmatrix}
= 6
$$
Aangezien de determinant niet 0 is deze verzameling niet lineair afhankelijk en dus vrij.
\\
\\
Aangezien we weten dat de dimensie van $\mathbb{R}_{<2}$ gelijk is aan 3. En we hier 3 vrije vectoren hebben, kunnen we via Eig. 3.40 besluiten dat het een basis is.
\\
\\
Willen we nu toch voortbrengend bewijzen. Dan moeten we nagaan of er willekeurige $k,l,m$ bestaan zodat we voor een willekeurige vector $aX^2+bX+c$ met de gegeven basis $(v_1,v_2,v_3)$ een combinatie vinden:
$$
aX^2+bX+c = kv_1+lv_2+mv_3
$$
$$
\left\lbrace
\begin{array}{r c}
k-2m &= a\\
3l+m &= b\\
3k-2l+3m &= c
\end{array}
\right.
$$
Als we dit oplossen en we krijgen een oplossing dan wil dit zeggen dat er voor elke willekeurige vector zo'n combinatie bestaat, en dit wil zeggen dat het voortbrengend is.
\subsubsection*{2b)}

Beschouwen we de eerste basis in functie van de tweede bekomen we:\\

$D(X^2+3) = \bigl(
\begin{smallmatrix}
1&-3\\ 1&3
\end{smallmatrix}
\bigr) = \bigl(
\begin{smallmatrix}
1&0\\ 0&0
\end{smallmatrix}
\bigr) -3* \bigl(
\begin{smallmatrix}
0&1\\ 0&0
\end{smallmatrix}
\bigr) + \bigl(
\begin{smallmatrix}
0&0\\ 1&0
\end{smallmatrix}
\bigr) +3* \bigl(
\begin{smallmatrix}
0&0\\ 0&1
\end{smallmatrix}
\bigr)$\\

$D(-2X^2+3X-4) = \bigl(
\begin{smallmatrix}
1&7\\ -5&-1
\end{smallmatrix}
\bigr) = \bigl(
\begin{smallmatrix}
1&0\\ 0&0
\end{smallmatrix}
\bigr) + 7* \bigl(
\begin{smallmatrix}
0&1\\ 0&0
\end{smallmatrix}
\bigr) - 5* \bigl(
\begin{smallmatrix}
0&0\\ 1&0
\end{smallmatrix}
\bigr) - \bigl(
\begin{smallmatrix}
0&0\\ 0&1
\end{smallmatrix}
\bigr)$\\

$D(3X-2) = \bigl(
\begin{smallmatrix}
3&5\\ -3&1
\end{smallmatrix}
\bigr) = 3* \bigl(
\begin{smallmatrix}
1&0\\ 0&0
\end{smallmatrix}
\bigr) + 5* \bigl(
\begin{smallmatrix}
0&1\\ 0&0
\end{smallmatrix}
\bigr) - 3* \bigl(
\begin{smallmatrix}
0&0\\ 1&0
\end{smallmatrix}
\bigr) + \bigl(
\begin{smallmatrix}
0&0\\ 0&1
\end{smallmatrix}
\bigr)$\\

Nemen we dus de bekomen co\"effici\"enten met de tweede basis, vormen we de matrix van $L$ op de twee basissen als volgt:\\

\[
L^{\beta_1}_{\beta_2} =
\left[
\begin{array}{c c c}
1 & 1 & 3\\
-3 & 7 & 5\\
1 & -5 & -3\\
3 & -1 & 1
\end{array}
\right]
\]

\subsubsection*{2c)}

Door direct te berekenen bekomen we\\

$ L(-X^2+6X-3) = \bigl(
\begin{smallmatrix}
-1+6&6+3\\ -1-6&6-3
\end{smallmatrix}\bigr) = \bigl(
\begin{smallmatrix}
5&9\\ -7&3
\end{smallmatrix}\bigr)$\\

Om ditzelfde met de matrix van $L$ te berekenen bepalen we eerst de co\"ordinaten van $-X^2+6X-3$ ten opzichte van de eerste basis.
Na het opstellen van een stelsel en rijreductie vinden we dat de co\"ordinaten $(1,1,1)$ zijn.

Voeren we nu de vermenigvuldiging van de matrix van $L$ met de bekomen co\"ordinaten vinden we:\\

$\begin{pmatrix}
1 & 1 & 3\\
-3 & 7 & 5\\
1 & -5 & -3\\
3 & -1 & 1
\end{pmatrix} * \begin{pmatrix}
1\\
1\\
1
\end{pmatrix} = \begin{pmatrix}
5\\
9\\
-7\\
3
\end{pmatrix}$\\

Deze resulterende co\"rdinaten gebruiken we voor de tweede basis, waardoor we $\bigl(
\begin{smallmatrix}
5&9\\ -7&3
\end{smallmatrix}\bigr)$ bekomen, hetzelfde resultaat als de directe berekening.
\end{document}
