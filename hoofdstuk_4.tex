\documentclass[lineaire_algebra_oplossingen.tex]{subfiles}
\begin{document}

\part{Hoofdstuk 4}
\section{Bewijzen uit het boek}
\subsection{Lemma 4.3}
\subsubsection*{Te bewijzen}
Een afbeelding $L:V\rightarrow W$ tussen re\"ele vectorruimten is een lineaire afbeelding a.s.a.
$$L(\lambda_1v_1+\lambda_2v_2) = \lambda_1L(v_1)+\lambda_2L(V_2)\ \text{voor alle $\lambda_1,\lambda_2 \in \mathbb{R}$ en voor alle $v_1,v_2 \in V$}.$$
\subsubsection*{Bewijs}
We moeten dus aantonen dat:
\begin{enumerate}
\item $L(v_1+v_2) = L(v_1) + L(v_2) \ \forall v_1,v_2 \in V$
\item $L(\lambda v) = \lambda L(v) \ \forall v \in V \ \text{en} \ \forall \lambda \in \mathbb{R}$
\end{enumerate}
$$\Leftrightarrow$$
$$L(\lambda_1v_1+\lambda_2v_2) = \lambda_1L(v_1)+\lambda_2L(V_2)\ \text{voor alle $\lambda_1,\lambda_2 \in \mathbb{R}$ en voor alle $v_1,v_2 \in V$}.$$
\subsubsection*{$\Rightarrow$}
Kies $\lambda_1,\lambda_2 \in \mathbb{R}$ en $v_1,v_2 \in V$ willekeurig.
\begin{align*}
L(\lambda_1 v_1 + \lambda_2 v_2) = L(\lambda_1) \tag{test}
\end{align*}
\section{Oefeningen 4.8}
\subsection{Oefening 1}
\subsubsection{c)}
$$
L_3:\mathbb{R}^2\rightarrow\mathbb{R}:(x,y)\mapsto |x-y|
$$
$$
\text{Kern van L:} \ \{(x,y)|x=y \wedge x,y \in \mathbb{R} \}
$$
$$
\text{Beeld van L:} \ \{(x,y)|x\neq y \wedge x,y \in \mathbb{R} \}
$$
\begin{center}
Niet isomorf, hoe moet ik dit verklaren?
\end{center}
\subsubsection{e)}
$$
L_5:\mathbb{R}^3\rightarrow\mathbb{R}^2:(x,y,z)\mapsto (3y+z,x-y-z)
$$
$$
\text{Kern van L:} \ \{(4\lambda ,\lambda ,-3\lambda)|\lambda \in \mathbb{R} \}
$$
$$
\text{Beeld van L:} \ \{(x ,y ,z)|x,y,z \in \mathbb{R} \}
$$
\subsection{Oefening 8}
Om de matrices van de basisverandering te berekenen stellen we voor beide bases een matrix op waarbij ze beide uitgeschreven worden ten opzichte van eenzelfde basis, bijvoorbeeld, de standaardbasis, dezen naast elkaar plaatsen en rijreduceren tot de linkse matrix de eenheidsmatrix is. De eerste oefening is relatief triviaal, zoals we zullen zien.
\subsubsection*{a)}
We schrijven beide uit in functie van standaardbasis \{(0,1), (1,0)\}, waarbij we na ze aan elkaar te zetten komen tot

\[
\left[
\begin{array}{ c c | c c }
1 & 0 & 5 & 1\\
0 & 1 & 1 & 2
\end{array}
\right]
\]

De linkerkant is reeds de eenheidsmatrix dus geldt dat de matrix 
\[
\left[
\begin{array}{ c c }
5 & 1\\
1 & 2
\end{array}
\right]
\]
de matrix voor de basisverandering van $\beta$ naar $\alpha$. Voor de andere matrix, van $\alpha$ naar $\beta$ rijreduceren we dezelfde matrix, maar met de linker en rechterhelft omgewisseld, wat dus neerkomt op de inverse zoeken van zijn andere basisveranderingmatrix. We vinden

\[
\left[
\begin{array}{ c c | c c }
5 & 1 & 1 & 0\\
1 & 2 & 0 & 1
\end{array}
\right]
\]

\[R1 \leftrightarrow R2 \]

\[
\left[
\begin{array}{ c c | c c }
1 & 2 & 0 & 1\\
5 & 1 & 1 & 0
\end{array}
\right]
\]

\[R2 \mapsto R2 - 5*R1 \]

\[
\left[
\begin{array}{ c c | c c }
1 & 2 & 0 & 1\\
0 & -9 & 1 & -5
\end{array}
\right]
\]

\[R2 \mapsto \frac{R2}{-9} \]

\[
\left[
\begin{array}{ c c | c c }
1 & 2 & 0 & 1\\
0 & 1 & \frac{-1}{9} & \frac{5}{9}
\end{array}
\right]
\]

\[R1 \mapsto R1 - 2*R2 \]

\[
\left[
\begin{array}{ c c | c c }
1 & 0 & \frac{2}{9} & \frac{-1}{9}\\
0 & 1 & \frac{-1}{9} & \frac{5}{9}
\end{array}
\right]
\]

\subsubsection*{b)}

We nemen deze keer als standaardbasis \{$\bigl(
\begin{smallmatrix}
1&0\\ 0&0
\end{smallmatrix}
\bigr)$,
$\bigl(
\begin{smallmatrix}
0&1\\ 0&0
\end{smallmatrix}
\bigr)$,
$\bigl(
\begin{smallmatrix}
0&0\\ 1&0
\end{smallmatrix}
\bigr)$,
$\bigl(
\begin{smallmatrix}
0&0\\ 0&1
\end{smallmatrix}
\bigr)$\}, waardoor we de opgegeven bases in volgende matrix kunnen weergeven (met $\beta$ links):

\[
\left[
\begin{array}{ c c c c | c c c c }
1 & 0 & 0 & 0 & 1 & 1 & -1 & 2\\
0 & 0 & 1 & 0 & 3 & 1 & -1 & 2\\
0 & 0 & 0 & 1 & 1 & -1 & 2 & -3\\
0 & 1 & 0 & 0 & 2 & 1 & -1 & 0
\end{array}
\right]
\]

dewelke we door achtereenvolgens $R3 \leftrightarrow R4$ en $R2 \leftrightarrow R3$ toe te passen omzetten tot

\[
\left[
\begin{array}{ c c c c | c c c c }
1 & 0 & 0 & 0 & 1 & 1 & -1 & 2\\
0 & 1 & 0 & 0 & 2 & 1 & -1 & 0\\
0 & 0 & 1 & 0 & 3 & 1 & -1 & 2\\
0 & 0 & 0 & 1 & 1 & -1 & 2 & -3
\end{array}
\right]
\]

De rechtse kant is dus de basisveranderingsmatrix van $\beta$ naar $\alpha$. Rekenen we nu de andere kant uit hebben we

\[
\left[
\begin{array}{ c c c c | c c c c }
1 & 1 & -1 & 2 & 1 & 0 & 0 & 0\\
2 & 1 & -1 & 0 & 0 & 1 & 0 & 0\\
3 & 1 & -1 & 2 & 0 & 0 & 1 & 0\\
1 & -1 & 2 & -3 & 0 & 0 & 0 & 1
\end{array}
\right]
\]

\[R2 \mapsto R2 - 2*R1 \]
\[R3 \mapsto R3 - 3*R1 \]
\[R4 \mapsto R4 - R1 \]

\[
\left[
\begin{array}{ c c c c | c c c c }
1 & 1 & -1 & 2 & 1 & 0 & 0 & 0\\
0 & -1 & 1 & -4 & -2 & 1 & 0 & 0\\
0 & -2 & 2 & -4 & -3 & 0 & 1 & 0\\
0 & -2 & 3 & -5 & -1 & 0 & 0 & 1
\end{array}
\right]
\]

\[R3 \mapsto R3 - 2*R2 \]
\[R4 \mapsto R4 - 2*R2 \]

\[
\left[
\begin{array}{ c c c c | c c c c }
1 & 1 & -1 & 2 & 1 & 0 & 0 & 0\\
0 & -1 & 1 & -4 & -2 & 1 & 0 & 0\\
0 & 0 & 0 & 4 & 1 & -2 & 1 & 0\\
0 & 0 & 1 & 3 & 3 & -2 & 0 & 1
\end{array}
\right]
\]

\[R3 \mapsto \frac{R3}{4} \]
\[R4 \leftrightarrow R3 \]

\[
\left[
\begin{array}{ c c c c | c c c c }
1 & 1 & -1 & 2 & 1 & 0 & 0 & 0\\
0 & -1 & 1 & -4 & -2 & 1 & 0 & 0\\
0 & 0 & 1 & 3 & 3 & -2 & 0 & 1\\
0 & 0 & 0 & 1 & \frac{1}{4} & \frac{-1}{2} & \frac{1}{4} & 0
\end{array}
\right]
\]

\[R1 \mapsto R1 - 2*R4 \]
\[R2 \mapsto R2 + 4*R4 \]
\[R3 \mapsto R3 - 3*R4 \]

\[
\left[
\begin{array}{ c c c c | c c c c }
1 & 1 & -1 & 0 & \frac{1}{2} & 1 & \frac{-1}{2} & 0\\
0 & -1 & 1 & 0 & -1 & -1 & 1 & 0\\
0 & 0 & 1 & 0 & \frac{9}{4} & \frac{-1}{2} & \frac{-3}{4} & 1\\
0 & 0 & 0 & 1 & \frac{1}{4} & \frac{-1}{2} & \frac{1}{4} & 0
\end{array}
\right]
\]

\[R1 \mapsto R1 + R3 \]
\[R2 \mapsto -R2 + R3 \]

\[
\left[
\begin{array}{ c c c c | c c c c }
1 & 1 & 0 & 0 & \frac{11}{4} & \frac{1}{2} & \frac{-5}{4} & 1\\
0 & 1 & 0 & 0 & \frac{13}{4} & \frac{1}{2} & \frac{-7}{4} & 1\\
0 & 0 & 1 & 0 & \frac{9}{4} & \frac{-1}{2} & \frac{-3}{4} & 1\\
0 & 0 & 0 & 1 & \frac{1}{4} & \frac{-1}{2} & \frac{1}{4} & 0
\end{array}
\right]
\]

\[R1 \mapsto R1 - R2 \]

\[
\left[
\begin{array}{ c c c c | c c c c }
1 & 0 & 0 & 0 & \frac{-1}{2} & 0 & \frac{1}{2} & 0\\
0 & 1 & 0 & 0 & \frac{13}{4} & \frac{1}{2} & \frac{-7}{4} & 1\\
0 & 0 & 1 & 0 & \frac{9}{4} & \frac{-1}{2} & \frac{-3}{4} & 1\\
0 & 0 & 0 & 1 & \frac{1}{4} & \frac{-1}{2} & \frac{1}{4} & 0
\end{array}
\right]
\]

De rechterkant van deze laatste is dus de basisveranderingsmatrix van basis $\alpha$ naar $\beta$.

\subsubsection*{c)}
We nemen als standaardbasis \{1, X, X$^2$\}, waardoor we als beginmatrix voor ons rekenwerk als makkelijkste deze krijgen:

\[
\left[
\begin{array}{ c c c | c c c }
0 & 0 & 1 & 3 & 2 & 1\\
0 & 1 & 0 & 4 & 0 & -1\\
1 & 0 & 0 & 2 & 1 & 0
\end{array}
\right]
\]

\[R1 \leftrightarrow R3 \]

\[
\left[
\begin{array}{ c c c | c c c }
1 & 0 & 0 & 2 & 1 & 0\\
0 & 1 & 0 & 4 & 0 & -1\\
0 & 0 & 1 & 3 & 2 & 1
\end{array}
\right]
\]

Deze laatste is dus onze basisveranderingsmatrix van $\beta$ naar $\alpha$ (Ik ben nog steeds niet zeker van de "richting" van de basisveranderingsmatrix?). Rekenen we de andere richting uit, bekomen we

\[
\left[
\begin{array}{ c c c | c c c }
2 & 1 & 0 & 1 & 0 & 0\\
4 & 0 & -1 & 0 & 1 & 0\\
3 & 2 & 1 & 0 & 0 & 1
\end{array}
\right]
\]

\[R2 \mapsto R2 - 2*R1 \]
\[R3 \mapsto 2*R3 - 3*R1 \]

\[
\left[
\begin{array}{ c c c | c c c }
2 & 1 & 0 & 1 & 0 & 0\\
0 & -2 & -1 & -2 & 1 & 0\\
0 & 1 & 2 & -3 & 0 & 2
\end{array}
\right]
\]

\[R2 \leftrightarrow R3 \]

\[
\left[
\begin{array}{ c c c | c c c }
2 & 1 & 0 & 1 & 0 & 0\\
0 & 1 & 2 & -3 & 0 & 2\\
0 & -2 & -1 & -2 & 1 & 0
\end{array}
\right]
\]

\[R3 \mapsto R3 + 2*R2 \]

\[
\left[
\begin{array}{ c c c | c c c }
2 & 1 & 0 & 1 & 0 & 0\\
0 & 1 & 2 & -3 & 0 & 2\\
0 & 0 & 3 & -8 & 1 & 4
\end{array}
\right]
\]

\[R3 \mapsto \frac{R3}{3} \]

\[
\left[
\begin{array}{ c c c | c c c }
2 & 1 & 0 & 1 & 0 & 0\\
0 & 1 & 2 & -3 & 0 & 2\\
0 & 0 & 1 & \frac{-8}{3} & \frac{1}{3} & \frac{4}{3}
\end{array}
\right]
\]

\[R2 \mapsto R2 - 2*R3 \]

\[
\left[
\begin{array}{ c c c | c c c }
2 & 1 & 0 & 1 & 0 & 0\\
0 & 1 & 0 & \frac{7}{3} & \frac{-2}{3} & \frac{-2}{3}\\
0 & 0 & 1 & \frac{-8}{3} & \frac{1}{3} & \frac{4}{3}
\end{array}
\right]
\]

\[R1 \mapsto R1 - R2 \]

\[
\left[
\begin{array}{ c c c | c c c }
2 & 0 & 0 & \frac{-4}{3} & \frac{2}{3} & \frac{2}{3}\\
0 & 1 & 0 & \frac{7}{3} & \frac{-2}{3} & \frac{-2}{3}\\
0 & 0 & 1 & \frac{-8}{3} & \frac{1}{3} & \frac{4}{3}
\end{array}
\right]
\]

\[R1 \mapsto \frac{R1}{2}\]

\[
\left[
\begin{array}{ c c c | c c c }
1 & 0 & 0 & \frac{-2}{3} & \frac{1}{3} & \frac{1}{3}\\
0 & 1 & 0 & \frac{7}{3} & \frac{-2}{3} & \frac{-2}{3}\\
0 & 0 & 1 & \frac{-8}{3} & \frac{1}{3} & \frac{4}{3}
\end{array}
\right]
\]

De rechterkant van deze laatste is dus de basisveranderingsmatrix van basis $\alpha$ naar $\beta$.
\subsection{Oefening 9}

Berekenen we voor de elementen van de basis hun lineaire afbeelding als volgt:\\

$T(e1) = T( (1,0,0) ) = (2,5,4) = 2 * e1 + 5 * e2 + 4 * e3$\\

$T(e2) = T( (0,1,0) ) = (-3,-1,7) = -3 * e1 + -1 * e2 + 7 * e3$\\

$T(e3) = T( (0,0,1) ) = (4,2,0) = 4 * e1 + 2 * e2$\\

We plaatsen dezen in de matrix van de lineaire afbeelding als kolommen en bekomen zo\\

\[
T^{\alpha}_{\alpha} =
\left[
\begin{array}{c c c}
2 & -3 & -4\\
5 & -1 & 2\\
4 & 7 & 0
\end{array}
\right]
\]\\

Voor de tweede basis zijn 2 elementen van plaats verwisseld. We kunnen deze verwisselen bij de berekening en bekomen dan de vergelijkbare matrix

\[
T^{\beta}_{\beta} =
\left[
\begin{array}{c c c}
4 & -3 & 2\\
2 & -1 & 5\\
0 & 7 & 4
\end{array}
\right]
\]
\section{Opdrachten}

\subsection{4.9}
\subsubsection*{1)}
We bepalen eerst de co\"ordinaten van $(2,0,5)$ ten opzichte van de eerste basis. Aangezien het over een transformatie gaat is $\beta$ zowel de eerste als de tweede basis. 
Na het opstellen van een stelsel en rijreductie vinden we dat de co\"ordinaten $\left(\frac{12}{5},\frac{3}{5},\frac{-2}{5}\right)$ zijn.

Voeren we nu de vermenigvuldiging van de matrix van $L$ met de bekomen co\"ordinaten vinden we:\\
$$
\begin{pmatrix}
1&0&5\\
0&-2&2\\
1&-2&7
\end{pmatrix}
\cdot
\begin{pmatrix}
\frac{12}{\strut5}\\ \frac{3}{5}\\ \frac{\strut -2}{5}
\end{pmatrix}
=
\begin{pmatrix}
\strut \frac{2}{5}\\ -2\\ \strut \frac{-8}{5}
\end{pmatrix}
$$
Dit is dan ook het antwoord op de vraag.
\subsubsection*{2a)}
We gaan na of de verzameling lineair onafhankelijk is door de determinant te berekenen:
$$
\begin{vmatrix}
2 & 0 & -2\\
0&3&3\\
3&-2&-4
\end{vmatrix}
= 6
$$
Aangezien de determinant niet 0 is deze verzameling niet lineair afhankelijk en dus vrij.
\\
\\
We bekijken nu of de verzameling voortbrengend is.
\subsubsection*{2b)}

Beschouwen we de eerste basis in functie van de tweede bekomen we:\\

$D(X^2+3) = \bigl(
\begin{smallmatrix}
1&-3\\ 1&3
\end{smallmatrix}
\bigr) = \bigl(
\begin{smallmatrix}
1&0\\ 0&0
\end{smallmatrix}
\bigr) -3* \bigl(
\begin{smallmatrix}
0&1\\ 0&0
\end{smallmatrix}
\bigr) + \bigl(
\begin{smallmatrix}
0&0\\ 1&0
\end{smallmatrix}
\bigr) +3* \bigl(
\begin{smallmatrix}
0&0\\ 0&1
\end{smallmatrix}
\bigr)$\\

$D(-2X^2+3X-4) = \bigl(
\begin{smallmatrix}
1&7\\ -5&-1
\end{smallmatrix}
\bigr) = \bigl(
\begin{smallmatrix}
1&0\\ 0&0
\end{smallmatrix}
\bigr) + 7* \bigl(
\begin{smallmatrix}
0&1\\ 0&0
\end{smallmatrix}
\bigr) - 5* \bigl(
\begin{smallmatrix}
0&0\\ 1&0
\end{smallmatrix}
\bigr) - \bigl(
\begin{smallmatrix}
0&0\\ 0&1
\end{smallmatrix}
\bigr)$\\

$D(3X-2) = \bigl(
\begin{smallmatrix}
3&5\\ -3&1
\end{smallmatrix}
\bigr) = 3* \bigl(
\begin{smallmatrix}
1&0\\ 0&0
\end{smallmatrix}
\bigr) + 5* \bigl(
\begin{smallmatrix}
0&1\\ 0&0
\end{smallmatrix}
\bigr) - 3* \bigl(
\begin{smallmatrix}
0&0\\ 1&0
\end{smallmatrix}
\bigr) + \bigl(
\begin{smallmatrix}
0&0\\ 0&1
\end{smallmatrix}
\bigr)$\\

Nemen we dus de bekomen co\"effici\"enten met de tweede basis, vormen we de matrix van $L$ op de twee basissen als volgt:\\

\[
L^{\beta_1}_{\beta_2} =
\left[
\begin{array}{c c c}
1 & 1 & 3\\
-3 & 7 & 5\\
1 & -5 & -3\\
3 & -1 & 1
\end{array}
\right]
\]

\subsubsection*{2c)}

Door direct te berekenen bekomen we\\

$ L(-X^2+6X-3) = \bigl(
\begin{smallmatrix}
-1+6&6+3\\ -1-6&6-3
\end{smallmatrix}\bigr) = \bigl(
\begin{smallmatrix}
5&9\\ -7&3
\end{smallmatrix}\bigr)$\\

Om ditzelfde met de matrix van $L$ te berekenen bepalen we eerst de co\"ordinaten van $-X^2+6X-3$ ten opzichte van de eerste basis.
Na het opstellen van een stelsel en rijreductie vinden we dat de co\"ordinaten $(1,1,1)$ zijn.

Voeren we nu de vermenigvuldiging van de matrix van $L$ met de bekomen co\"ordinaten vinden we:\\

$\begin{pmatrix}
1 & 1 & 3\\
-3 & 7 & 5\\
1 & -5 & -3\\
3 & -1 & 1
\end{pmatrix} * \begin{pmatrix}
1\\
1\\
1
\end{pmatrix} = \begin{pmatrix}
5\\
9\\
-7\\
3
\end{pmatrix}$\\

Deze resulterende co\"rdinaten gebruiken we voor de tweede basis, waardoor we $\bigl(
\begin{smallmatrix}
5&9\\ -7&3
\end{smallmatrix}\bigr)$ bekomen, hetzelfde resultaat als de directe berekening.

\end{document}
