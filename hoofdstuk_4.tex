\documentclass[lineaire_algebra_oplossingen.tex]{subfiles}
\begin{document}

\part{Hoofdstuk 4}
\section{Oefeningen 4.8}

\subsection{Oefening 8}
Om de matrices van de basisverandering te berekenen stellen we voor beide bases een matrix op waarbij ze beide uitgeschreven worden ten opzichte van eenzelfde basis, bijvoorbeeld, de standaardbasis, dezen naast elkaar plaatsen en rijreduceren tot de linkse matrix de eenheidsmatrix is. De eerste oefening is relatief triviaal, zoals we zullen zien.
\subsubsection*{a)}
We schrijven beide uit in functie van standaardbasis \{(0,1), (1,0)\}, waarbij we na ze aan elkaar te zetten komen tot

\[
\left[
\begin{array}{ c c | c c }
1 & 0 & 5 & 1\\
0 & 1 & 1 & 2
\end{array}
\right]
\]

De linkerkant is reeds de eenheidsmatrix dus geldt dat de matrix 
\[
\left[
\begin{array}{ c c }
5 & 1\\
1 & 2
\end{array}
\right]
\]
de matrix voor de basisverandering van $\beta$ naar $\alpha$. Voor de andere matrix, van $\alpha$ naar $\beta$ rijreduceren we dezelfde matrix, maar met de linker en rechterhelft omgewisseld, wat dus neerkomt op de inverse zoeken van zijn andere basisveranderingmatrix. We vinden

\[
\left[
\begin{array}{ c c | c c }
5 & 1 & 1 & 0\\
1 & 2 & 0 & 1
\end{array}
\right]
\]

\[R1 \leftrightarrow R2 \]

\[
\left[
\begin{array}{ c c | c c }
1 & 2 & 0 & 1\\
5 & 1 & 1 & 0
\end{array}
\right]
\]

\[R2 \mapsto R2 - 5*R1 \]

\[
\left[
\begin{array}{ c c | c c }
1 & 2 & 0 & 1\\
0 & -9 & 1 & -5
\end{array}
\right]
\]

\[R2 \mapsto \frac{R2}{-9} \]

\[
\left[
\begin{array}{ c c | c c }
1 & 2 & 0 & 1\\
0 & 1 & \frac{-1}{9} & \frac{5}{9}
\end{array}
\right]
\]

\[R1 \mapsto R1 - 2*R2 \]

\[
\left[
\begin{array}{ c c | c c }
1 & 0 & \frac{2}{9} & \frac{-1}{9}\\
0 & 1 & \frac{-1}{9} & \frac{5}{9}
\end{array}
\right]
\]

\subsubsection*{b)}

We nemen deze keer als standaardbasis \{$\bigl(
\begin{smallmatrix}
1&0\\ 0&0
\end{smallmatrix}
\bigr)$,
$\bigl(
\begin{smallmatrix}
0&1\\ 0&0
\end{smallmatrix}
\bigr)$,
$\bigl(
\begin{smallmatrix}
0&0\\ 1&0
\end{smallmatrix}
\bigr)$,
$\bigl(
\begin{smallmatrix}
0&0\\ 0&1
\end{smallmatrix}
\bigr)$\}, waardoor we de opgegeven bases in volgende matrix kunnen weergeven (met $\beta$ links):

\[
\left[
\begin{array}{ c c c c | c c c c }
1 & 0 & 0 & 0 & 1 & 1 & -1 & 2\\
0 & 0 & 1 & 0 & 3 & 1 & -1 & 2\\
0 & 0 & 0 & 1 & 1 & -1 & 2 & -3\\
0 & 1 & 0 & 0 & 2 & 1 & -1 & 0
\end{array}
\right]
\]

dewelke we door achtereenvolgens $R3 \leftrightarrow R4$ en $R2 \leftrightarrow R3$ toe te passen omzetten tot

\[
\left[
\begin{array}{ c c c c | c c c c }
1 & 0 & 0 & 0 & 1 & 1 & -1 & 2\\
0 & 1 & 0 & 0 & 2 & 1 & -1 & 0\\
0 & 0 & 1 & 0 & 3 & 1 & -1 & 2\\
0 & 0 & 0 & 1 & 1 & -1 & 2 & -3
\end{array}
\right]
\]

De rechtse kant is dus de basisveranderingsmatrix van $\beta$ naar $\alpha$. Rekenen we nu de andere kant uit hebben we

\[
\left[
\begin{array}{ c c c c | c c c c }
1 & 1 & -1 & 2 & 1 & 0 & 0 & 0\\
2 & 1 & -1 & 0 & 0 & 1 & 0 & 0\\
3 & 1 & -1 & 2 & 0 & 0 & 1 & 0\\
1 & -1 & 2 & -3 & 0 & 0 & 0 & 1
\end{array}
\right]
\]

\[R2 \mapsto R2 - 2*R1 \]
\[R3 \mapsto R3 - 3*R1 \]
\[R4 \mapsto R4 - R1 \]

\[
\left[
\begin{array}{ c c c c | c c c c }
1 & 1 & -1 & 2 & 1 & 0 & 0 & 0\\
0 & -1 & 1 & -4 & -2 & 1 & 0 & 0\\
0 & -2 & 2 & -4 & -3 & 0 & 1 & 0\\
0 & -2 & 3 & -5 & -1 & 0 & 0 & 1
\end{array}
\right]
\]

\[R3 \mapsto R3 - 2*R2 \]
\[R4 \mapsto R4 - 2*R2 \]

\[
\left[
\begin{array}{ c c c c | c c c c }
1 & 1 & -1 & 2 & 1 & 0 & 0 & 0\\
0 & -1 & 1 & -4 & -2 & 1 & 0 & 0\\
0 & 0 & 0 & 4 & 1 & -2 & 1 & 0\\
0 & 0 & 1 & 3 & 3 & -2 & 0 & 1
\end{array}
\right]
\]

\[R3 \mapsto \frac{R3}{4} \]
\[R4 \leftrightarrow R3 \]

\[
\left[
\begin{array}{ c c c c | c c c c }
1 & 1 & -1 & 2 & 1 & 0 & 0 & 0\\
0 & -1 & 1 & -4 & -2 & 1 & 0 & 0\\
0 & 0 & 1 & 3 & 3 & -2 & 0 & 1\\
0 & 0 & 0 & 1 & \frac{1}{4} & \frac{-1}{2} & \frac{1}{4} & 0
\end{array}
\right]
\]

\[R1 \mapsto R1 - 2*R4 \]
\[R2 \mapsto R2 + 4*R4 \]
\[R3 \mapsto R3 - 3*R4 \]

\[
\left[
\begin{array}{ c c c c | c c c c }
1 & 1 & -1 & 0 & \frac{1}{2} & 1 & \frac{-1}{2} & 0\\
0 & -1 & 1 & 0 & -1 & -1 & 1 & 0\\
0 & 0 & 1 & 0 & \frac{9}{4} & \frac{-1}{2} & \frac{-3}{4} & 1\\
0 & 0 & 0 & 1 & \frac{1}{4} & \frac{-1}{2} & \frac{1}{4} & 0
\end{array}
\right]
\]

\[R1 \mapsto R1 + R3 \]
\[R2 \mapsto -R2 + R3 \]

\[
\left[
\begin{array}{ c c c c | c c c c }
1 & 1 & 0 & 0 & \frac{11}{4} & \frac{1}{2} & \frac{-5}{4} & 1\\
0 & 1 & 0 & 0 & \frac{13}{4} & \frac{1}{2} & \frac{-7}{4} & 1\\
0 & 0 & 1 & 0 & \frac{9}{4} & \frac{-1}{2} & \frac{-3}{4} & 1\\
0 & 0 & 0 & 1 & \frac{1}{4} & \frac{-1}{2} & \frac{1}{4} & 0
\end{array}
\right]
\]

\[R1 \mapsto R1 - R2 \]

\[
\left[
\begin{array}{ c c c c | c c c c }
1 & 0 & 0 & 0 & \frac{-1}{2} & 0 & \frac{1}{2} & 0\\
0 & 1 & 0 & 0 & \frac{13}{4} & \frac{1}{2} & \frac{-7}{4} & 1\\
0 & 0 & 1 & 0 & \frac{9}{4} & \frac{-1}{2} & \frac{-3}{4} & 1\\
0 & 0 & 0 & 1 & \frac{1}{4} & \frac{-1}{2} & \frac{1}{4} & 0
\end{array}
\right]
\]

De rechterkant van deze laatste is dus de basisveranderingsmatrix van basis $\alpha$ naar $\beta$.

\subsubsection*{c)}
We nemen als standaardbasis \{1, X, X$^2$\}, waardoor we als beginmatrix voor ons rekenwerk als makkelijkste deze krijgen:

\[
\left[
\begin{array}{ c c c | c c c }
0 & 0 & 1 & 3 & 2 & 1\\
0 & 1 & 0 & 4 & 0 & -1\\
1 & 0 & 0 & 2 & 1 & 0
\end{array}
\right]
\]

\[R1 \leftrightarrow R3 \]

\[
\left[
\begin{array}{ c c c | c c c }
1 & 0 & 0 & 2 & 1 & 0\\
0 & 1 & 0 & 4 & 0 & -1\\
0 & 0 & 1 & 3 & 2 & 1
\end{array}
\right]
\]

Deze laatste is dus onze basisveranderingsmatrix van $\beta$ naar $\alpha$ (Ik ben nog steeds niet zeker van de "richting" van de basisveranderingsmatrix?). Rekenen we de andere richting uit, bekomen we

\[
\left[
\begin{array}{ c c c | c c c }
2 & 1 & 0 & 1 & 0 & 0\\
4 & 0 & -1 & 0 & 1 & 0\\
3 & 2 & 1 & 0 & 0 & 1
\end{array}
\right]
\]

\[R2 \mapsto R2 - 2*R1 \]
\[R3 \mapsto 2*R3 - 3*R1 \]

\[
\left[
\begin{array}{ c c c | c c c }
2 & 1 & 0 & 1 & 0 & 0\\
0 & -2 & -1 & -2 & 1 & 0\\
0 & 1 & 2 & -3 & 0 & 2
\end{array}
\right]
\]

\[R2 \leftrightarrow R3 \]

\[
\left[
\begin{array}{ c c c | c c c }
2 & 1 & 0 & 1 & 0 & 0\\
0 & 1 & 2 & -3 & 0 & 2\\
0 & -2 & -1 & -2 & 1 & 0
\end{array}
\right]
\]

\[R3 \mapsto R3 + 2*R2 \]

\[
\left[
\begin{array}{ c c c | c c c }
2 & 1 & 0 & 1 & 0 & 0\\
0 & 1 & 2 & -3 & 0 & 2\\
0 & 0 & 3 & -8 & 1 & 4
\end{array}
\right]
\]

\[R3 \mapsto \frac{R3}{3} \]

\[
\left[
\begin{array}{ c c c | c c c }
2 & 1 & 0 & 1 & 0 & 0\\
0 & 1 & 2 & -3 & 0 & 2\\
0 & 0 & 1 & \frac{-8}{3} & \frac{1}{3} & \frac{4}{3}
\end{array}
\right]
\]

\[R2 \mapsto R2 - 2*R3 \]

\[
\left[
\begin{array}{ c c c | c c c }
2 & 1 & 0 & 1 & 0 & 0\\
0 & 1 & 0 & \frac{7}{3} & \frac{-2}{3} & \frac{-2}{3}\\
0 & 0 & 1 & \frac{-8}{3} & \frac{1}{3} & \frac{4}{3}
\end{array}
\right]
\]

\[R1 \mapsto R1 - R2 \]

\[
\left[
\begin{array}{ c c c | c c c }
2 & 0 & 0 & \frac{-4}{3} & \frac{2}{3} & \frac{2}{3}\\
0 & 1 & 0 & \frac{7}{3} & \frac{-2}{3} & \frac{-2}{3}\\
0 & 0 & 1 & \frac{-8}{3} & \frac{1}{3} & \frac{4}{3}
\end{array}
\right]
\]

\[R1 \mapsto \frac{R1}{2}\]

\[
\left[
\begin{array}{ c c c | c c c }
1 & 0 & 0 & \frac{-2}{3} & \frac{1}{3} & \frac{1}{3}\\
0 & 1 & 0 & \frac{7}{3} & \frac{-2}{3} & \frac{-2}{3}\\
0 & 0 & 1 & \frac{-8}{3} & \frac{1}{3} & \frac{4}{3}
\end{array}
\right]
\]

De rechterkant van deze laatste is dus de basisveranderingsmatrix van basis $\alpha$ naar $\beta$.

\end{document}
