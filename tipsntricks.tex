\documentclass[lineaire_algebra_oplossingen.tex]{subfiles}
\begin{document}

\part{Tips}
\section{Hulpvolle algoritmes}
\subsection{Een vrije deelverzameling uitbreiden tot een basis}
Stel dat $D = \{v_1,v_2,...,v_k\}$ de vrije verzameling is die we willen uitbreiden tot een basis van een vectorruimte $(\mathbb{R},V,+)$ van dimensie $n$ ($k\le n$).
\begin{enumerate}
\item Kijk na of $D$ voortbrengend is.
\begin{itemize}
\item $D$ is voortbrengend $\rightarrow$ Stop, $D$ is een  basis voor $V$.
\item $D$ is niet voortbrengend $\rightarrow$ Ga door naar stap $2$.
\end{itemize}
\item Zoek een vector $v_{k+1}$ die lineair onafhankelijk is van de vectoren in $D$ en voeg deze toe aan $D$.
\item Ga terug naar stap $1$.
\end{enumerate}
\subsection{Een voortbrengende deelverzameling uitdunnen tot een basis}
Stel dat $D = \{v_1,v_2,...,v_k\}$ de voortbrengende verzameling is die we willen uitdunnen tot een basis van een vectorruimte $(\mathbb{R},V,+)$ van dimensie $n$ ($k\ge n$).
\begin{enumerate}
\item Verwijder alle nulvectoren uit $D$.
\item Overloop, in volgorde, alle vectoren uit $D$ en verwijder elke vector die lineair afhankelijk is van de vorige.
\item De resulterende deelverzameling van $D$ is een basis van $V$.
\end{enumerate}

\section{Hulpvolle definities}
\subsection{Injectief}
Een afbeelding $L: V \rightarrow W$ is injectief.
\[
L(v) = L(w) \Rightarrow v = w
\]
\subsection{Surjectief}
Een afbeelding $L: V \rightarrow W$ is surjectief.
\[
\forall w \in W \exists v \in V: w=L(v)
\]
\subsection{Bijectief}
Een afbeelding $L: V \rightarrow W$ is bijectief als en slechts als deze afbeelding injectief en surjectief is.



\end{document}