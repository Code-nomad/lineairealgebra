\documentclass[lineaire_algebra_oplossingen.tex]{subfiles}
\begin{document}

\newpage
\part{Tips}
\section{Extra uitleg}
\subsection{Basisverandering}
Omdat de keuze van basis zeer belangrijk is moeten we co\"ordinaatvectoren t.o.v een basis om kunnen zetten naar co\"ordinaatvectoren t.o.v een andere basis. De transformatie van een basis naar een andere kan voorgesteld worden met een matrix $L$. 

Zij $\vec{a}$ de co\"ordinaatvector van een vector t.o.v. $\alpha$ dan is $\vec{b}$ de de co\"ordinaatvector van $\vec{a}$ vector t.o.v. $\beta$.
\[
\vec{b} = L_\alpha^\beta \vec{a}
\]

\emph{Elk lineaire afbeelding kan voorgesteld worden met een bepaalde $L_{\beta_V}^{\beta_W}$ mits gekozen basissen.}

\subsubsection{Identieke transformatie}
Voor de transformatie binnen \'e\'enzelfde vectorruimte gebruiken we een matrix $L = Id_\alpha^\beta$ om van co\"ordinaatvectoren t.o.v  $\alpha$ naar co\"ordinaatvectoren t.o.v $\beta$ te transformeren. Zij $\vec{a}$ de co\"ordinaatvector van een vector t.o.v. $\alpha$ dan is $\vec{b}$ de de co\"ordinaatvector van $\vec{a}$ vector t.o.v. $\beta$. Als $\alpha$ en $\beta$ basissen zijn van dezelfde vectorruimte geldt het volgende over $b$.
\[
\vec{b} = Id_\alpha^\beta \vec{a}
\]
$Id_\alpha^\beta$ is natuurlijk inverteerbaar. De inverse hiervan transformeert de nieuwe co\"ordinaatvector terug.

\section{Handige algoritmes}
\subsection{Een vrije deelverzameling uitbreiden tot een basis}
Stel dat $D = \{v_1,v_2,...,v_k\}$ de vrije verzameling is die we willen uitbreiden tot een basis van een vectorruimte $(\mathbb{R},V,+)$ van dimensie $n$ ($k\le n$).
\begin{enumerate}
\item Kijk na of $D$ voortbrengend is.
\begin{itemize}
\item $D$ is voortbrengend $\rightarrow$ Stop, $D$ is een  basis voor $V$.
\item $D$ is niet voortbrengend $\rightarrow$ Ga door naar stap $2$.
\end{itemize}
\item Zoek een vector $v_{k+1}$ die lineair onafhankelijk is van de vectoren in $D$ en voeg deze toe aan $D$.
\item Ga terug naar stap $1$.
\end{enumerate}
\subsection{Een voortbrengende deelverzameling uitdunnen tot een basis}
Stel dat $D = \{v_1,v_2,...,v_k\}$ de voortbrengende verzameling is die we willen uitdunnen tot een basis van een vectorruimte $(\mathbb{R},V,+)$ van dimensie $n$ ($k\ge n$).
\begin{enumerate}
\item Verwijder alle nulvectoren uit $D$.
\item Overloop, in volgorde, alle vectoren uit $D$ en verwijder elke vector die lineair afhankelijk is van de vorige.
\item De resulterende deelverzameling van $D$ is een basis van $V$.
\end{enumerate}

\section{Handige definities}
\subsection{Samenstelling van afbeeldingen}
\label{samenstelling_van_afbeeldingen}
Zij $f: U \rightarrow V$ en $g: V\rightarrow W$ twee afbeeldingen.
\[
\forall x\in U: (f \circ g)(x) = f(g(x))
\]
\subsection{Injectief}
\label{injectief}
Een afbeelding $L: V \rightarrow W$ is injectief.
\[
L(v) = L(w) \Rightarrow v = w
\]
\subsection{Surjectief}
\label{surjectief}
Een afbeelding $L: V \rightarrow W$ is surjectief.
\[
\forall w \in W \exists v \in V: w=L(v)
\]
\subsection{Bijectief}
\label{bijectief}
Een afbeelding $L: V \rightarrow W$ is bijectief als en slechts als deze afbeelding injectief en surjectief is.


\end{document}