\documentclass[lineaire_algebra_oplossingen.tex]{subfiles}
\begin{document}

\newpage
\part{Tips}
%\section{Extra uitleg}


\section{Handige algoritmes}
\subsection{Een vrije deelverzameling uitbreiden tot een basis}
Stel dat $D = \{v_1,v_2,...,v_k\}$ de vrije verzameling is die we willen uitbreiden tot een basis van een vectorruimte $(\mathbb{R},V,+)$ van dimensie $n$ ($k\le n$).
\begin{enumerate}
\item Kijk na of $D$ voortbrengend is.
\begin{itemize}
\item $D$ is voortbrengend $\rightarrow$ Stop, $D$ is een  basis voor $V$.
\item $D$ is niet voortbrengend $\rightarrow$ Ga door naar stap $2$.
\end{itemize}
\item Zoek een vector $v_{k+1}$ die lineair onafhankelijk is van de vectoren in $D$ en voeg deze toe aan $D$.
\item Ga terug naar stap $1$.
\end{enumerate}
\subsection{Een voortbrengende deelverzameling uitdunnen tot een basis}
Stel dat $D = \{v_1,v_2,...,v_k\}$ de voortbrengende verzameling is die we willen uitdunnen tot een basis van een vectorruimte $(\mathbb{R},V,+)$ van dimensie $n$ ($k\ge n$).
\begin{enumerate}
\item Verwijder alle nulvectoren uit $D$.
\item Overloop, in volgorde, alle vectoren uit $D$ en verwijder elke vector die lineair afhankelijk is van de vorige.
\item De resulterende deelverzameling van $D$ is een basis van $V$.
\end{enumerate}

\subsection{Matrix van basisverandering opstellen}
Zij $\alpha = \{a_1,a_2,...,a_n\}$ en $\beta = \{b_1,b_2,...,b_n\}$ twee basissen van de $n$-dimensionale vectorruimte $(\mathbb{R},V,+)$. Willen we nu de matrix van basisverandering $Id_\alpha^\beta$ opstellen, dan voeren we volgend algoritme uit.
\subsubsection*{Generiek}
We stellen beide basissen voor volgens dezelfde basis. Vaak is de standaardbasis het makkelijkst. %TODO schrijf een voorbeeld waarvoor dit niet zo is.
De co\"ordinaten van de vectoren $(a_1,a_2,...,a_n)$ en $(b_1,b_2,...,b_n)$ standaardbasissen $\alpha$ en $\beta$ zetten we in de volgende matrix in de kolommen.
\[
\left(
\begin{array}{c c c c | c c c c}
x_{11} & x_{12} & \cdots & x_{1n} & y_{11} & y_{12} & \cdots & y_{1n}\\
x_{21} & x_{22} & \cdots & x_{2n} & y_{21} & y_{22} & \cdots & y_{2n}\\
\vdots & \vdots & \ddots & \vdots & \vdots & \vdots & \ddots & \vdots\\
x_{n1} & x_{n2} & \cdots & x_{nn} & y_{n1} & y_{n2} & \cdots & y_{nn}\\
\end{array}
\right)
\]
We rijreduceren nu deze matrix om aan de rechterkant de gezochte matrix te vinden.
\[
\left(
\begin{array}{c c c c | c c c c}
1 & 0 & \cdots & 0 & z_{11} & z_{12} & \cdots & z_{1n}\\
0 & 1 & \cdots & 0 & z_{21} & z_{22} & \cdots & z_{2n}\\
\vdots & \vdots & \ddots & \vdots & \vdots & \vdots & \ddots & \vdots\\
0 & 0 & \cdots & 1 & z_{n1} & z_{n2} & \cdots & z_{nn}\\
\end{array}
\right)
\]
De gezochte matrix $Id_\alpha^\beta$ is dus de volgende.
\[
Id_\alpha^\beta
=
\begin{pmatrix}
z_{11} & z_{12} & \cdots & z_{1n}\\
z_{21} & z_{22} & \cdots & z_{2n}\\
\vdots & \vdots & \ddots & \vdots\\
z_{n1} & z_{n2} & \cdots & z_{nn}\\
\end{pmatrix}
\]

\subsubsection*{Voorbeeld}
Zij $\alpha = \{3+2X+X^2,4-X^2,2+X\}$ en $\beta = \{2X^2-1,X+3,-X^2+X\}$.
We schrijven de coordinaten van deze vectoren volgens de standaardbasis in de colommen van de volgende matrix.
\[
\left(
\begin{array}{c c c | c c c}
3 & 4 & 2 & -1 & 3 & 0\\
2 & 0 & 1 & 0 & 1 & 1\\
1 & -1 & 0 & 2 & 0 & 1\\
\end{array}
\right)
\]
Wanneer we dit rijreduceren verkrijgen we rechts de matrix van basisverandering $Id_\alpha^\beta$
\[
Id_\alpha^\beta =
\left(
\begin{array}{c c c | c c c}
1 & 0 & 0 & \frac{7}{3} & \frac{1}{3} & \frac{2}{3}\\
0 & 1 & 0 & \frac{1}{3} & \frac{1}{3} & -\frac{1}{3}\\
0 & 0 & 1 & -\frac{14}{3} & \frac{1}{3} & -\frac{1}{3}\\
\end{array}
\right)
\]


\subsection{Matrixvoorstelling van lineaire afbeelding ten opzichte van de standaardbasissen}
Zij $L:V\rightarrow W$ een lineaire afbeelding van de $n$-dimensionale vectorruimte $(\mathbb{R},V,+)$ naar de $m$-dimensionale vectorruimte $(\mathbb{R},W,+)$. Willen we nu de matrixvoorstelling van $L$ ten opzichte van de standaard basissen $\epsilon_V$ en $\epsilon_W$.

\subsubsection*{Generiek}
Als het functievoorschrift van $L$ gegeven is dan zijn de co\"ordinaten van een generische vector $v\in V$ en $w\in W$ makkelijk af te lezen.
De co\"ordinaten van $w$ ten opzichte van de standaardbasis van $V$ zijn een lineaire combinatie van de basisvectoren van $V$. ($w = \lambda_1v_1+\lambda_2v_1+...+\lambda_nv_n$) deze lineaire combinatie zetten we in de rijen van de volgende matrix.
\[
\begin{pmatrix}
\lambda_{11} & \lambda_{12} & \cdots & \lambda_{1n} \\
\lambda_{21} & \lambda_{22} & \cdots & \lambda_{2n} \\
\vdots & \vdots & \ddots & \vdots\\
\lambda_{m1} & \lambda_{m2} & \cdots & \lambda_{mn} \\
\end{pmatrix}
\]
Deze matrix is precies de matrix die we zochten.

\subsubsection*{Voorbeeld}
Zij $L:R[X]_{\le 4}\rightarrow \mathbb{R}^3: a+bX+cX^2+dX^3+eX^4\mapsto (a+b,c+d+e,-a+3b+c-4e)$.
We zien dat de co\"ordinaten van $(a+b,c+d+e,-a+3b+c-4e)$ ten opzichte van de standaardbasis van $\mathbb{R}^3$ precies in de vector staan en een lineaire combinatie zijn van $a$, $b$, $c$, $d$ en $e$.
Schrijven we deze lineaire combinaties als rijen in de volgende matrix, dan is dit de matrixvoorstelling van $L$ ten opzichte van de standaard basissen.
\[
\begin{pmatrix}
1 & 1 & 0 & 0 & 0\\
0 & 0 & 1 & 1 & 1\
-1 & 3 & 1 & 0 & -4\\
\end{pmatrix}
\]

\section{Handige definities}
\subsection{Samenstelling van afbeeldingen}
\label{samenstelling_van_afbeeldingen}
Zij $f: U \rightarrow V$ en $g: V\rightarrow W$ twee afbeeldingen.
\[
\forall x\in U: (f \circ g)(x) = f(g(x))
\]
\subsection{Injectief}
\label{injectief}
Een afbeelding $L: V \rightarrow W$ is injectief.
\[
L(v) = L(w) \Rightarrow v = w
\]
\subsection{Surjectief}
\label{surjectief}
Een afbeelding $L: V \rightarrow W$ is surjectief.
\[
\forall w \in W \exists v \in V: w=L(v)
\]
\subsection{Bijectief}
\label{bijectief}
Een afbeelding $L: V \rightarrow W$ is bijectief als en slechts als deze afbeelding injectief en surjectief is.


\end{document}