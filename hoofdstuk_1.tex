\documentclass[10pt,a4paper]{article}
\usepackage[utf8]{inputenc}
\usepackage[english]{babel}
\usepackage{amsmath}
\usepackage{amsfonts}
\usepackage{amssymb}

\title{Oplossingen Lineaire Algebra 2013}
\author{TODO}


\begin{document}

\maketitle
\pagebreak
\tableofcontents
\pagebreak


\section{Oefeningen}
\subsection*{oef 1}
\subsection*{oef 2}
\subsection*{oef 3}
\subsubsection*{Echelonvorm}
\[
\begin{pmatrix}
1 &  2 &  3 &  4 &  5\\
2 &  3 &  4 &  5 &  1\\
3 &  4 &  5 &  1 &  2\\
4 &  5 &  1 &  2 &  3\\
5 &  1 &  2 &  3 &  4 
\end{pmatrix}
\]
\[ R2 \longmapsto R2 -2\cdot R1\]
\[ R3 \longmapsto R3 -3\cdot R1\]
\[ R4 \longmapsto R4 -4\cdot R1\]
\[ R5 \longmapsto R5 -5\cdot R1\]
\[
\begin{pmatrix}
1 &  2 &  3 &  4 &  5 \\
0 & -1 & -2 & -3 & -9 \\
0 & -2 & -4 & -11& -13\\
0 & -3 & -11& -14& -17\\
0 & -9 & -13& -17& -21
\end{pmatrix}
\]
\[ R3 \longmapsto R3 -2\cdot R2\]
\[ R4 \longmapsto R4 -2\cdot R2\]
\[ R5 \longmapsto R5 -2\cdot R2\]

\[
\begin{pmatrix}
1 &  2 &  3 &  4 &  5\\
0 & -1 & -2 & -3 & -9\\
0 &  0 &  0 & -5 &  5\\
0 &  0 & -5 & -5 & 10\\
0 &  0 &  5 & 10 & 60 
\end{pmatrix}
\]
\begin{center}
Wissel R3 en R4
\end{center}
\[ R5 \longmapsto R5 + R4\]

\[
\begin{pmatrix}
1 &  2 &  3 &  4 &  5\\
0 & -1 & -2 & -3 & -9\\
0 &  0 & -5 & -5 & 10\\
0 &  0 &  0 & -5 &  5\\
0 &  0 &  0 &  0 & 75 
\end{pmatrix}
\]

\subsubsection*{Rij-geredeuceerde vorm}
\[
\begin{pmatrix}
1 &  0 &  0 &  0 & 0\\
0 &  1 &  0 &  0 & 0\\
0 &  0 &  1 &  0 & 0\\
0 &  0 &  0 &  1 & 0\\
0 &  0 &  0 &  0 & 1 
\end{pmatrix}
\]

\subsection*{oef 4}

\subsection*{oef 5}

\subsection*{oef 6}
\subsubsection*{a)}
\[
\begin{pmatrix}
1 &  2 &  4 &  6\\
3 &  8 & 14 & 16\\
2 &  6 & 11 & 12
\end{pmatrix}
\longrightarrow
\begin{pmatrix}
1 &  0 &  0 &  4\\
0 &  1 &  0 & -3\\
0 &  0 &  1 &  2
\end{pmatrix}
\]
Antwoord:
\[
V=\{(4,-3,2)\}
\]

\subsubsection*{b)}
\[
\begin{pmatrix}
3 &  2 &  4 &  5\\
1 &  1 & -3 &  2\\
4 &  3 &  1 &  7
\end{pmatrix}
\longrightarrow
\begin{pmatrix}
1 &  0 & 10 &  1\\
0 &  1 & -13&  1\\
0 &  0 &  0 &  0
\end{pmatrix}
\]
Antwoord:
\[
V=\{(1-10t,1+13t,t) | t \in \mathbb{R}\}
\]

\subsubsection*{c)}
\[
\begin{pmatrix}
1 &  2 & -3 & -1\\
3 & -1 &  2 &  7\\
5 &  3 & -4 &  2
\end{pmatrix}
\longrightarrow
\begin{pmatrix}
1 &  0 & \frac{1}{7} &  0\\
0 &  1 & \frac{-11}{7}&  0\\
0 &  0 &  0 &  1
\end{pmatrix}
\]
Antwoord:
\[
V=\emptyset
\]

\subsubsection*{d)}
\[
\begin{pmatrix}
1 &  1 & -2 &  1 &  2 & 1\\
2 & -1 &  2 &  2 &  6 & 2\\
3 &  2 & -4 & -3 & -9 & 3
\end{pmatrix}
\longrightarrow
\begin{pmatrix}
1 &  0 &  0 &  0 &  0 & 1\\
0 &  1 & -2 &  0 &  0 & 0\\
0 &  0 &  0 &  1 &  3 & 0
\end{pmatrix}
\]
Antwoord:
\[
V=\{(1,2a,a,-3b,b) | a,b \in \mathbb{R} \}
\]

\subsection*{oef 7}

\subsection*{oef 8}

\subsection*{oef 9}
\[
\begin{pmatrix}
2 &  1 &  7 & b_1\\
6 & -2 & 11 & b_2 \\
2 & -1 &  3 & b_3\\
\end{pmatrix}
\]
\[ R2 \longmapsto R2 -3\cdot R1\]
\[ R3 \longmapsto R3 - R1\]
\[
\begin{pmatrix}
2 &  1 &  7 & b_1\\
0 & -5 & -10& b_2-3b_1 \\
0 & -2 &  -4& b_3-b_1\\
\end{pmatrix}
\]
\[ R2 \longmapsto -\frac{1}{5} R2\]
\[
\begin{pmatrix}
2 &  1 &  7 & b_1\\
0 &  1 &  2 & -\frac{1}{5}(b_2-3b_1) \\
0 & -2 &  -4& b_3-b_1\\
\end{pmatrix}
\]
\[ R1 \longmapsto R1 - R2\]
\[ R3 \longmapsto R3 + 2 \cdot	R2\]
\[
\begin{pmatrix}
2 &  0 &  5 & b_1 + \frac{1}{5}(b_2-3b_1)\\
0 &  1 &  2 & -\frac{1}{5}(b_2-3b_1) \\
0 &  0 &  0 & (b_3-b_1) - \frac{2}{5}(b_2-3b_1)\\
\end{pmatrix}
\]
Antwoord:\\
Als $(b_3-b_1) - \frac{2}{5}(b_2-3b_1) = 0$ dan heeft het stelsel oneindig veel oplossingen.
Als $(b_3-b_1) - \frac{2}{5}(b_2-3b_1) \neq 0$ dan heeft het stelsel geen oplossingen ($V=\emptyset$)

\subsection*{oef 10}

\subsection*{oef 11}

\subsection*{oef 12}
\subsubsection*{a)}
\begin{center}
Wissel R1 en R3, en daarna R2 en R1.
\end{center}
\[
\begin{pmatrix}
1 &  1 &  k & 1\\
1 &  k &  1 & 1\\
k &  1 &  1 & 1
\end{pmatrix}
\]
\[ R2 \longmapsto R2 - R1\]
\[ R3 \longmapsto R3 - k \cdot	R1\]
\[
\begin{pmatrix}
1 &  1 &  k & 1\\
0 & k-1& 1-k& 0\\
0 & 1-k & 1-k^2 & 1-k
\end{pmatrix}
\]
Geval 1: $k=1$
\[
\begin{pmatrix}
1 &  1 & 1 & 1\\
0 &  0 & 0 & 0\\
0 &  0 & 0 & 0
\end{pmatrix}
\]
Antwoord:
\[
V=\{ (1-a-b,a,b) | a,b \in \mathbb{R}\}
\]\\
Geval 2: $k\neq 1$\\ 
\[ R2 \longmapsto \frac{1}{k-1}R2\]
\[ R3 \longmapsto \frac{1}{1-k}R3\]
\[
\begin{pmatrix}
1 &  1 &  k & 1\\
0 &  1 & -1 & 0\\
0 &  1 & k+1& 1
\end{pmatrix}
\]
\[ R1 \longmapsto R1-R2\]
\[ R3 \longmapsto R3-R1\]
\[
\begin{pmatrix}
1 &  0 &  k+1 & 1\\
0 &  1 & -1 & 0\\
0 &  0 & k+2& 1
\end{pmatrix}
\]\\
Geval 2a: $k = -2$\\
\[
\begin{pmatrix}
1 &  0 & -1 & 1\\
0 &  1 & -1 & 0\\
0 &  0 &  0 & 1
\end{pmatrix}
\]
Antwoord:
\[
V=\emptyset
\]\\
Geval 2b: $k\neq-2$\\
\[ R3 \longmapsto \frac{1}{k+2}R3\]
\[
\begin{pmatrix}
1 &  0 &  k+1 & 1\\
0 &  1 & -1 & 0\\
0 &  0 &  1 & \frac{1}{k+2}
\end{pmatrix}
\]
\[ R1 \longmapsto R1 - (k+1)\cdot R3\]
\[ R2 \longmapsto R2 + R3\]
\[
\begin{pmatrix}
1 &  0 &  0 & \frac{3}{k+2}\\
0 &  1 &  0 & \frac{1}{k+2}\\
0 &  0 &  1 & \frac{1}{k+2}
\end{pmatrix}
\]
Antwoord:
\[
V=\{(\frac{3}{k+2},\frac{1}{k+2},\frac{1}{k+2})\}
\]\\
\textbf{Samenvatting}:\\
Als $k=1$ dan heeft het stelsel oneindig veel oplossingen:
\[
V=\{ (1-a-b,a,b) | a,b \in \mathbb{R}\}
\]
Als $k=-2$ dan heeft het stelsel geen oplossingen:
\[
V=\emptyset
\]
Anders heeft het stelsel precies één oplossing:
\[
V=\{(\frac{3}{k+2},\frac{1}{k+2},\frac{1}{k+2})\}
\]

\subsubsection*{b)}
\begin{center}
Wissel R1 en R2
\end{center}
\[
\begin{pmatrix}
1 &  k &  k+1\\
k &  1 &  2
\end{pmatrix}
\]
\[ R2 \longmapsto R2 - k\cdot R3\]
\[
\begin{pmatrix}
1 &  k &  k+1\\
0 &  1-k^2 &  -k^2-k+2
\end{pmatrix}
\]\\
Geval 1: $k=1$\\
\[
\begin{pmatrix}
1 &  1 &  2\\
0 &  0 &  0
\end{pmatrix}
\]
Antwoord:
\[
V=\{(2-t,t) | t \in \mathbb{R}\}
\]\\
Geval 2: $k=-1$\\
\[
\begin{pmatrix}
1 & -1 &  0\\
0 &  0 &  2
\end{pmatrix}
\]
Antwoord:
\[
V=\emptyset
\]\\
Geval 3: $k\neq 1 \wedge k\neq -1$\\
\[ R2 \longmapsto \frac{1}{1-k^2}\cdot R2 \]
\[
\begin{pmatrix}
1 &  k &  k+1\\
0 &  1 &  -\frac{k+2}{k+1}
\end{pmatrix}
\]
\[ R1 \longmapsto R1-k\cdot R2 \]
\[
\begin{pmatrix}
1 &  0 &  \frac{2k^2+4k+1}{k+1}\\
0 &  1 &  -\frac{k+2}{k+1}
\end{pmatrix}
\]
Antwoord:
\[
V=\left\lbrace\left(\frac{2k^2+4k+1}{k+1},-\frac{k+2}{k+1}\right)\right\rbrace
\]\\
\textbf{Samenvatting}:\\
Als $k=1$ dan heeft het stelsel oneindig veel oplossingen:
\[
V=\{(2-t,t) | t \in \mathbb{R}\}
\]
Als $k=-1$ dan heeft het stelsel geen oplossingen:
\[
V=\emptyset
\]
Anders heeft het stelsel precies één oplossing:
\[
V=\left\lbrace\left(\frac{2k^2+4k+1}{k+1},-\frac{k+2}{k+1}\right)\right\rbrace
\]

\subsubsection*{c)}
\[
\begin{pmatrix}
k & k+1 & 1 & 0\\
k & 1 & k+1 & 0\\
2k & 1 & 1 & k+1\\
\end{pmatrix}
\]
Geval 1: $k=0$:
\[
\begin{pmatrix}
0 & 1 & 1 & 0\\
0 & 1 & 1 & 0\\
0 & 1 & 1 & 1\\
\end{pmatrix}
\]
\[ R1 \longmapsto R1 - R2 \]
\[ R3 \longmapsto R3 - R2 \]
\[
\begin{pmatrix}
0 & 0 & 0 & 0\\
0 & 1 & 1 & 0\\
0 & 0 & 0 & 1\\
\end{pmatrix}
\]
Antwoord:
\[
V=\emptyset
\]\\
Geval 2: $k\neq0$:
\[ R1 \longmapsto \frac{1}{k}R1 \]
\[
\begin{pmatrix}
1 & \frac{k+1}{k} & \frac{1}{k} & 0\\
k & 1 & k+1 & 0\\
2k & 1 & 1 & k+1\\
\end{pmatrix}
\]
\[ R2 \longmapsto R2 - k\cdot R1 \]
\[ R3 \longmapsto R3 - 2k\cdot R1 \]
\[
\begin{pmatrix}
1 & \frac{k+1}{k} & \frac{1}{k} & 0\\
0 & -k & k & 0\\
0 & -2k-1 & -1 & k+1\\
\end{pmatrix}
\]
\[ R2 \longmapsto \frac{-1}{k}R2\]
\[
\begin{pmatrix}
1 & \frac{k+1}{k} & \frac{1}{k} & 0\\
0 & 1 & -1 & 0\\
0 & -2k-1 & -1 & k+1\\
\end{pmatrix}
\]
\[ R1 \longmapsto R1 - \frac{k+1}{k}\cdot R2 \]
\[ R3 \longmapsto R3 + (2k+1)\cdot R2 \]
\[
\begin{pmatrix}
1 & 0 & \frac{k+2}{k} & 0\\
0 & 1 & -1 & 0\\
0 & 0 & -2k-2 & k+1\\
\end{pmatrix}
\]
Geval 2a $k=-1$:
\[
\begin{pmatrix}
1 & 0 & \frac{1}{k} & 0\\
0 & 1 & -1 & 0\\
0 & 0 & 0 & 0\\
\end{pmatrix}
\]
Antwoord:
\[
V=\left\lbrace\left(\frac{t}{k},t,t\right) | t \in \mathbb{R}\right\rbrace
\]\\
Geval 2b $k\neq-1$:
\[ R3 \longmapsto \frac{1}{-2k-2}R3\]
\[
\begin{pmatrix}
1 & 0 & \frac{k+2}{k} & 0\\
0 & 1 & -1 & 0\\
0 & 0 & 1 & -2\\
\end{pmatrix}
\]
\[ R1 \longmapsto R1 - \frac{k+2}{k}\cdot R3 \]
\[ R2 \longmapsto R2 + R3 \]
\[
\begin{pmatrix}
1 & 0 & 0 & 2\frac{k+2}{k}\\
0 & 1 & 0 & -2\\
0 & 0 & 1 & -2\\
\end{pmatrix}
\]
Antwoord:
\[
V=\left\lbrace\left(2\frac{k+2}{k},-2,-2\right)\right\rbrace
\]\\
\textbf{Samenvatting}:\\
Als $k=0$ dan heeft het stelsel geen oplossingen:
\[
V=\emptyset
\]
Als $k=-1$ dan heeft het stelsel geen oplossingen:
\[
V=\left\lbrace\left(\frac{t}{k},t,t\right) | t \in \mathbb{R}\right\rbrace
\]
Anders heeft het stelsel precies één oplossing:
\[
V=\left\lbrace\left(2\frac{k+2}{k},-2,-2\right)\right\rbrace
\]

\section{Opdrachten}
\subsection*{opdracht 1.2}

\subsection*{opdracht 1.23}

\subsection*{opdracht 1.33}

\section{Extra bewijzen}

\subsection*{1.10}

\subsection*{1.44}


\end{document}