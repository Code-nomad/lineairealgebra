\documentclass[10pt,a4paper]{article}

\usepackage[english]{babel}
\usepackage{amsmath}
\usepackage{amsfonts}
\usepackage{amssymb}

\title{Oplossingen Lineaire Algebra 2013}
\author{TODO}


\begin{document}

\maketitle
\pagebreak
\tableofcontents
\pagebreak


\section{Oefeningen}
\subsection*{oef 1}
\subsubsection*{Echelonvorm}
\subsubsection*{a)}
Do row reduction:
$\begin{pmatrix}
3 & -6 & 9\\
-2 & 7 & -2\\
0 & 1 & 5
\end{pmatrix}
$
\\
The following steps are done sequentially:

\begin{enumerate}
\item Add 2/3 x row 1 to row 2
\item Divide row 1 by 3:
\item Subtract 1/3 × (row 2) from row 3:
\item Multiply row 3 by 3/11:
\item Subtract 4 × (row 3) from row 2:
\item Subtract 3 × (row 3) from row 1:
\item Divide row 2 by 3:
\end{enumerate}

$
\begin{pmatrix}
1 & -2 & 0\\
0 & 1 & 0\\
0 & 0 & 1
\end{pmatrix}
$
\subsubsection*{b)}
Do row reduction:
$
\begin{pmatrix}
3 & -2 & -5 & 4\\
-5 & 2 & 8 & -5\\
-3 & 4 & 7 & -3\\
2 & -3 & -5 & 8
\end{pmatrix}
$
\\
The following steps are done sequentially:
\begin{enumerate}
\item Swap row 1 with row 2:
\item Add 3/5 × (row 1) to row 2:
\item Multiply row 1 by -1:
\item Multiply row 2 by -5:
\item Add 3/5 × (row 1) to row 3:
\item Multiply row 3 by 5:
\item Subtract 2/5 × (row 1) from row 4:
\item Multiply row 4 by 5:
\item Swap row 2 with row 3:
\item Subtract 2/7 × (row 2) from row 3:
\item Multiply row 3 by -7/5:
\item Add 11/14 × (row 2) to row 4:
\item Multiply row 4 by 14/5:
\item Add 1/3 × (row 3) to row 4:
\item Multiply row 4 by 3/259:
\item Subtract 7 × (row 4) from row 3:
\item Subtract 5 × (row 4) from row 1:
\item Divide row 3 by 3:
\item Subtract 11 × (row 3) from row 2:
\item Add 8 × (row 3) to row 1:
\item Divide row 2 by 14:
\item Add 2 × (row 2) to row 1:
\item Divide row 1 by 5:
\end{enumerate}
De matrix staat nu wel in gereduceerde echelon vorm, maar dat is ook een echelon vorm ;) :
\\
\[
\begin{pmatrix}
1 & 0 & 0 & 0\\
0 & 1 & 0 & 0\\
0 & 0 & 1 & 0\\
0 & 0 & 0 & 1
\end{pmatrix}
\]

\subsection*{oef 2}

\subsection*{oef 3}
\subsubsection*{Echelonvorm}
\[
\begin{pmatrix}
1 &  2 &  3 &  4 &  5\\
2 &  3 &  4 &  5 &  1\\
3 &  4 &  5 &  1 &  2\\
4 &  5 &  1 &  2 &  3\\
5 &  1 &  2 &  3 &  4 
\end{pmatrix}
\]
\[ R2 \longmapsto R2 -2\cdot R1\]
\[ R3 \longmapsto R3 -3\cdot R1\]
\[ R4 \longmapsto R4 -4\cdot R1\]
\[ R5 \longmapsto R5 -5\cdot R1\]
\[
\begin{pmatrix}
1 &  2 &  3 &  4 &  5 \\
0 & -1 & -2 & -3 & -9 \\
0 & -2 & -4 & -11& -13\\
0 & -3 & -11& -14& -17\\
0 & -9 & -13& -17& -21
\end{pmatrix}
\]
\[ R3 \longmapsto R3 -2\cdot R2\]
\[ R4 \longmapsto R4 -2\cdot R2\]
\[ R5 \longmapsto R5 -2\cdot R2\]

\[
\begin{pmatrix}
1 &  2 &  3 &  4 &  5\\
0 & -1 & -2 & -3 & -9\\
0 &  0 &  0 & -5 &  5\\
0 &  0 & -5 & -5 & 10\\
0 &  0 &  5 & 10 & 60 
\end{pmatrix}
\]
\begin{center}
Wissel R3 en R4
\end{center}
\[ R5 \longmapsto R5 + R4\]

\[
\begin{pmatrix}
1 &  2 &  3 &  4 &  5\\
0 & -1 & -2 & -3 & -9\\
0 &  0 & -5 & -5 & 10\\
0 &  0 &  0 & -5 &  5\\
0 &  0 &  0 &  0 & 75 
\end{pmatrix}
\]

\subsubsection*{Rij-geredeuceerde vorm}
\[
\begin{pmatrix}
1 &  0 &  0 &  0 & 0\\
0 &  1 &  0 &  0 & 0\\
0 &  0 &  1 &  0 & 0\\
0 &  0 &  0 &  1 & 0\\
0 &  0 &  0 &  0 & 1 
\end{pmatrix}
\]

\subsection*{oef 4}
\subsubsection*{a)}
Het oplossingsstelsel is bijna letterlijk af te lezen:\\
Stel $t=\lambda$ dan is:\\
\begin{center}
$x = -4\lambda - 1$\\
$y = -2\lambda + 6$\\
$z = -3\lambda + 2$\\
\end{center}
Hieruit volgt de oplossingsverzameling:
\[
V=\{(-4\lambda - 1,-2\lambda + 6,-3\lambda + 2, \lambda)|\lambda \in \mathbb{R}\}
\]
\subsubsection*{b)}
Dit triviaal direct te bepalen door de laatste rij:
\[
V=\emptyset
\]
\subsection*{oef 5}

\subsection*{oef 6}
\subsubsection*{a)}
\[
\begin{pmatrix}
1 &  2 &  4 &  6\\
3 &  8 & 14 & 16\\
2 &  6 & 11 & 12
\end{pmatrix}
\longrightarrow
\begin{pmatrix}
1 &  0 &  0 &  4\\
0 &  1 &  0 & -3\\
0 &  0 &  1 &  2
\end{pmatrix}
\]
Antwoord:
\[
V=\{(4,-3,2)\}
\]

\subsubsection*{b)}
\[
\begin{pmatrix}
3 &  2 &  4 &  5\\
1 &  1 & -3 &  2\\
4 &  3 &  1 &  7
\end{pmatrix}
\longrightarrow
\begin{pmatrix}
1 &  0 & 10 &  1\\
0 &  1 & -13&  1\\
0 &  0 &  0 &  0
\end{pmatrix}
\]
Antwoord:
\[
V=\{(1-10t,1+13t,t) | t \in \mathbb{R}\}
\]

\subsubsection*{c)}
\[
\begin{pmatrix}
1 &  2 & -3 & -1\\
3 & -1 &  2 &  7\\
5 &  3 & -4 &  2
\end{pmatrix}
\longrightarrow
\begin{pmatrix}
1 &  0 & \frac{1}{7} &  0\\
0 &  1 & \frac{-11}{7}&  0\\
0 &  0 &  0 &  1
\end{pmatrix}
\]
Antwoord:
\[
V=\emptyset
\]

\subsubsection*{d)}
\[
\begin{pmatrix}
1 &  1 & -2 &  1 &  2 & 1\\
2 & -1 &  2 &  2 &  6 & 2\\
3 &  2 & -4 & -3 & -9 & 3
\end{pmatrix}
\longrightarrow
\begin{pmatrix}
1 &  0 &  0 &  0 &  0 & 1\\
0 &  1 & -2 &  0 &  0 & 0\\
0 &  0 &  0 &  1 &  3 & 0
\end{pmatrix}
\]
Antwoord:
\[
V=\{(1,2a,a,-3b,b) | a,b \in \mathbb{R} \}
\]

\subsection*{oef 7}
\subsubsection*{a)}
\[
\begin{pmatrix}
2 & -3 & 0 & 8\\
4 & -5 & 1 & 0\\
2 & 0 & 4 & 1
\end{pmatrix}
\]
\[ R2 \longmapsto R2 - 2\cdot R1\]

\[
\begin{pmatrix}
2 & -3 & 0 & 8\\
0 & 1 & 1 & -16\\
0 & 3 & 4 & -7
\end{pmatrix}
\]
\[ R1 \longmapsto R1 / 2\]
\[ R3 \longmapsto R3 - 3\cdot R2\]

\[
\begin{pmatrix}
2 & -3 & 0 & 8\\
0 & 1 & 1 & -16\\
0 & 0 & 1 & 41
\end{pmatrix}
\]
\[
z = 41\]
\[y = -16 -41 = -57\]
\[2x = -(-3\cdot -57) + 8=-164\]
\[ x = -82\]

Antwoord:
\[
V = \{(-82,-57,41)\}
\]

\subsubsection*{b)}
\[
\begin{pmatrix}
0 & 2 & -1 & 1\\
4 & -10 & 3 & 5\\
3 & -3 & 0 & 6
\end{pmatrix}
\]

\[ R1 \leftrightarrow R3\]
\[
\begin{pmatrix}
3 & -3 & 0 & 6\\
4 & -10 & 3 & 5\\
0 & 2 & -1 & 1
\end{pmatrix}
\]

\[ R1 \longmapsto R1 / 3\]
\[
\begin{pmatrix}
1 & -1 & 0 & 2\\
4 & -10 & 3 & 5\\
0 & 2 & -1 & 1
\end{pmatrix}
\]
\[ R2 \longmapsto R2 - 4\cdot R1\]

\[
\begin{pmatrix}
1 & -1 & 0 & 2\\
0 & -6 & 3 & -3\\
0 & 2 & -1 & 1
\end{pmatrix}
\]
\[ R2 \longmapsto R2 + 3\cdot R3\]
\[
\begin{pmatrix}
1 & -1 & 0 & 2\\
0 & 0 & 0 & 0\\
0 & 2 & -1 & 1
\end{pmatrix}
\]
\[ R2 \leftrightarrow R3\]

\[
\begin{pmatrix}
1 & -1 & 0 & 2\\
0 & 2 & -1 & 1\\
0 & 0 & 0 & 0
\end{pmatrix}
\]
Antwoord:
\[
V = \{(-\frac{\lambda+1}{2} + 2,\frac{\lambda+1}{2},\lambda) | \lambda \in \mathbb{R}\}
\]
\subsection*{oef 8}

\subsection*{oef 9}
\[
\begin{pmatrix}
2 &  1 &  7 & b_1\\
6 & -2 & 11 & b_2 \\
2 & -1 &  3 & b_3\\
\end{pmatrix}
\]
\[ R2 \longmapsto R2 -3\cdot R1\]
\[ R3 \longmapsto R3 - R1\]
\[
\begin{pmatrix}
2 &  1 &  7 & b_1\\
0 & -5 & -10& b_2-3b_1 \\
0 & -2 &  -4& b_3-b_1\\
\end{pmatrix}
\]
\[ R2 \longmapsto -\frac{1}{5} R2\]
\[
\begin{pmatrix}
2 &  1 &  7 & b_1\\
0 &  1 &  2 & -\frac{1}{5}(b_2-3b_1) \\
0 & -2 &  -4& b_3-b_1\\
\end{pmatrix}
\]
\[ R1 \longmapsto R1 - R2\]
\[ R3 \longmapsto R3 + 2 \cdot	R2\]
\[
\begin{pmatrix}
2 &  0 &  5 & b_1 + \frac{1}{5}(b_2-3b_1)\\
0 &  1 &  2 & -\frac{1}{5}(b_2-3b_1) \\
0 &  0 &  0 & (b_3-b_1) - \frac{2}{5}(b_2-3b_1)\\
\end{pmatrix}
\]
Antwoord:\\
Als $(b_3-b_1) - \frac{2}{5}(b_2-3b_1) = 0$ dan heeft het stelsel oneindig veel oplossingen.
Als $(b_3-b_1) - \frac{2}{5}(b_2-3b_1) \neq 0$ dan heeft het stelsel geen oplossingen ($V=\emptyset$)

\subsection*{oef 10}

\subsection*{oef 11}

\subsection*{oef 12}
\subsubsection*{a)}
\begin{center}
Wissel R1 en R3, en daarna R2 en R1.
\end{center}
\[
\begin{pmatrix}
1 &  1 &  k & 1\\
1 &  k &  1 & 1\\
k &  1 &  1 & 1
\end{pmatrix}
\]
\[ R2 \longmapsto R2 - R1\]
\[ R3 \longmapsto R3 - k \cdot	R1\]
\[
\begin{pmatrix}
1 &  1 &  k & 1\\
0 & k-1& 1-k& 0\\
0 & 1-k & 1-k^2 & 1-k
\end{pmatrix}
\]
Geval 1: $k=1$
\[
\begin{pmatrix}
1 &  1 & 1 & 1\\
0 &  0 & 0 & 0\\
0 &  0 & 0 & 0
\end{pmatrix}
\]
Antwoord:
\[
V=\{ (1-a-b,a,b) | a,b \in \mathbb{R}\}
\]\\
Geval 2: $k\neq 1$\\ 
\[ R2 \longmapsto \frac{1}{k-1}R2\]
\[ R3 \longmapsto \frac{1}{1-k}R3\]
\[
\begin{pmatrix}
1 &  1 &  k & 1\\
0 &  1 & -1 & 0\\
0 &  1 & k+1& 1
\end{pmatrix}
\]
\[ R1 \longmapsto R1-R2\]
\[ R3 \longmapsto R3-R1\]
\[
\begin{pmatrix}
1 &  0 &  k+1 & 1\\
0 &  1 & -1 & 0\\
0 &  0 & k+2& 1
\end{pmatrix}
\]\\
Geval 2a: $k = -2$\\
\[
\begin{pmatrix}
1 &  0 & -1 & 1\\
0 &  1 & -1 & 0\\
0 &  0 &  0 & 1
\end{pmatrix}
\]
Antwoord:
\[
V=\emptyset
\]\\
Geval 2b: $k\neq-2$\\
\[ R3 \longmapsto \frac{1}{k+2}R3\]
\[
\begin{pmatrix}
1 &  0 &  k+1 & 1\\
0 &  1 & -1 & 0\\
0 &  0 &  1 & \frac{1}{k+2}
\end{pmatrix}
\]
\[ R1 \longmapsto R1 - (k+1)\cdot R3\]
\[ R2 \longmapsto R2 + R3\]
\[
\begin{pmatrix}
1 &  0 &  0 & \frac{3}{k+2}\\
0 &  1 &  0 & \frac{1}{k+2}\\
0 &  0 &  1 & \frac{1}{k+2}
\end{pmatrix}
\]
Antwoord:
\[
V=\{(\frac{3}{k+2},\frac{1}{k+2},\frac{1}{k+2})\}
\]\\
\textbf{Samenvatting}:\\
Als $k=1$ dan heeft het stelsel oneindig veel oplossingen:
\[
V=\{ (1-a-b,a,b) | a,b \in \mathbb{R}\}
\]
Als $k=-2$ dan heeft het stelsel geen oplossingen:
\[
V=\emptyset
\]
Anders heeft het stelsel precies één oplossing:
\[
V=\{(\frac{3}{k+2},\frac{1}{k+2},\frac{1}{k+2})\}
\]

\subsubsection*{b)}
\begin{center}
Wissel R1 en R2
\end{center}
\[
\begin{pmatrix}
1 &  k &  k+1\\
k &  1 &  2
\end{pmatrix}
\]
\[ R2 \longmapsto R2 - k\cdot R3\]
\[
\begin{pmatrix}
1 &  k &  k+1\\
0 &  1-k^2 &  -k^2-k+2
\end{pmatrix}
\]\\
Geval 1: $k=1$\\
\[
\begin{pmatrix}
1 &  1 &  2\\
0 &  0 &  0
\end{pmatrix}
\]
Antwoord:
\[
V=\{(2-t,t) | t \in \mathbb{R}\}
\]\\
Geval 2: $k=-1$\\
\[
\begin{pmatrix}
1 & -1 &  0\\
0 &  0 &  2
\end{pmatrix}
\]
Antwoord:
\[
V=\emptyset
\]\\
Geval 3: $k\neq 1 \wedge k\neq -1$\\
\[ R2 \longmapsto \frac{1}{1-k^2}\cdot R2 \]
\[
\begin{pmatrix}
1 &  k &  k+1\\
0 &  1 &  -\frac{k+2}{k+1}
\end{pmatrix}
\]
\[ R1 \longmapsto R1-k\cdot R2 \]
\[
\begin{pmatrix}
1 &  0 &  \frac{2k^2+4k+1}{k+1}\\
0 &  1 &  -\frac{k+2}{k+1}
\end{pmatrix}
\]
Antwoord:
\[
V=\left\lbrace\left(\frac{2k^2+4k+1}{k+1},-\frac{k+2}{k+1}\right)\right\rbrace
\]\\
\textbf{Samenvatting}:\\
Als $k=1$ dan heeft het stelsel oneindig veel oplossingen:
\[
V=\{(2-t,t) | t \in \mathbb{R}\}
\]
Als $k=-1$ dan heeft het stelsel geen oplossingen:
\[
V=\emptyset
\]
Anders heeft het stelsel precies één oplossing:
\[
V=\left\lbrace\left(\frac{2k^2+4k+1}{k+1},-\frac{k+2}{k+1}\right)\right\rbrace
\]

\subsubsection*{c)}
\[
\begin{pmatrix}
k & k+1 & 1 & 0\\
k & 1 & k+1 & 0\\
2k & 1 & 1 & k+1\\
\end{pmatrix}
\]
Geval 1: $k=0$:
\[
\begin{pmatrix}
0 & 1 & 1 & 0\\
0 & 1 & 1 & 0\\
0 & 1 & 1 & 1\\
\end{pmatrix}
\]
\[ R1 \longmapsto R1 - R2 \]
\[ R3 \longmapsto R3 - R2 \]
\[
\begin{pmatrix}
0 & 0 & 0 & 0\\
0 & 1 & 1 & 0\\
0 & 0 & 0 & 1\\
\end{pmatrix}
\]
Antwoord:
\[
V=\emptyset
\]\\
Geval 2: $k\neq0$:
\[ R1 \longmapsto \frac{1}{k}R1 \]
\[
\begin{pmatrix}
1 & \frac{k+1}{k} & \frac{1}{k} & 0\\
k & 1 & k+1 & 0\\
2k & 1 & 1 & k+1\\
\end{pmatrix}
\]
\[ R2 \longmapsto R2 - k\cdot R1 \]
\[ R3 \longmapsto R3 - 2k\cdot R1 \]
\[
\begin{pmatrix}
1 & \frac{k+1}{k} & \frac{1}{k} & 0\\
0 & -k & k & 0\\
0 & -2k-1 & -1 & k+1\\
\end{pmatrix}
\]
\[ R2 \longmapsto \frac{-1}{k}R2\]
\[
\begin{pmatrix}
1 & \frac{k+1}{k} & \frac{1}{k} & 0\\
0 & 1 & -1 & 0\\
0 & -2k-1 & -1 & k+1\\
\end{pmatrix}
\]
\[ R1 \longmapsto R1 - \frac{k+1}{k}\cdot R2 \]
\[ R3 \longmapsto R3 + (2k+1)\cdot R2 \]
\[
\begin{pmatrix}
1 & 0 & \frac{k+2}{k} & 0\\
0 & 1 & -1 & 0\\
0 & 0 & -2k-2 & k+1\\
\end{pmatrix}
\]
Geval 2a $k=-1$:
\[
\begin{pmatrix}
1 & 0 & \frac{1}{k} & 0\\
0 & 1 & -1 & 0\\
0 & 0 & 0 & 0\\
\end{pmatrix}
\]
Antwoord:
\[
V=\left\lbrace\left(\frac{t}{k},t,t\right) | t \in \mathbb{R}\right\rbrace
\]\\
Geval 2b $k\neq-1$:
\[ R3 \longmapsto \frac{1}{-2k-2}R3\]
\[
\begin{pmatrix}
1 & 0 & \frac{k+2}{k} & 0\\
0 & 1 & -1 & 0\\
0 & 0 & 1 & -2\\
\end{pmatrix}
\]
\[ R1 \longmapsto R1 - \frac{k+2}{k}\cdot R3 \]
\[ R2 \longmapsto R2 + R3 \]
\[
\begin{pmatrix}
1 & 0 & 0 & 2\frac{k+2}{k}\\
0 & 1 & 0 & -2\\
0 & 0 & 1 & -2\\
\end{pmatrix}
\]
Antwoord:
\[
V=\left\lbrace\left(2\frac{k+2}{k},-2,-2\right)\right\rbrace
\]\\
\textbf{Samenvatting}:\\
Als $k=0$ dan heeft het stelsel geen oplossingen:
\[
V=\emptyset
\]
Als $k=-1$ dan heeft het stelsel geen oplossingen:
\[
V=\left\lbrace\left(\frac{t}{k},t,t\right) | t \in \mathbb{R}\right\rbrace
\]
Anders heeft het stelsel precies één oplossing:
\[
V=\left\lbrace\left(2\frac{k+2}{k},-2,-2\right)\right\rbrace
\]

\subsection*{oef 13}

\subsection*{oef 14}

\subsection*{oef 15}
\begin{center}
Wissel R1 en R3
\end{center}
\[
\begin{pmatrix}
1 & 1 & ab & 1\\
a & 1 & b & 1\\
1 & a & b & 1\\
\end{pmatrix}
\]
\[ R2 \longmapsto R2 - a\cdot R1 \]
\[ R3 \longmapsto R3 - R1 \]
\[
\begin{pmatrix}
1 & 1 & ab & 1\\
0 & 1-a & b-a^2b & 1-a\\
0 & a-1 & b-ab & 0\\
\end{pmatrix}
\]
Geval 1: $a=1$
\[
\begin{pmatrix}
1 & 1 & b & 1\\
0 & 0 & 0 & 0\\
0 & 0 & 0 & 0\\
\end{pmatrix}
\]
Antwoord:
\[
V = \{ (1-p-bq,p,q) | p,q \in \mathbb{R} \}
\]\\
Geval 2: $a \neq -1$
\[ R2 \longmapsto \frac{1}{1-a}\cdot R2 \]
\[ R3 \longmapsto \frac{1}{1-a}\cdot R3 \]
\[
\begin{pmatrix}
1 & 1 & ab & 1\\
0 & 1 & b(1+a) & 1\\
0 & -1 & b & 0\\
\end{pmatrix}
\]
\[ R1 \longmapsto R1 - R2 \]
\[ R3 \longmapsto R3 + R2 \]
\[
\begin{pmatrix}
1 & 0 & -b & 0\\
0 & 1 & b(1+a) & 1\\
0 & 0 & b(2+a) & 1\\
\end{pmatrix}
\]
Geval 2a: $b=0 \vee a=-2$
\[
\begin{pmatrix}
1 & 0 & -b & 0\\
0 & 1 & b(1+a) & 1\\
0 & 0 & 0 & 1\\
\end{pmatrix}
\]
Antwoord:
\[
V=\emptyset
\]\\
Geval 2b: $b\neq \wedge a \neq -2$
\[ R3 \longmapsto \frac{1}{b(2+a)}\cdot R3 \]
\[
\begin{pmatrix}
1 & 0 & -b & 0\\
0 & 1 & b(1+a) & 1\\
0 & 0 & 1 & \frac{1}{b(2+a)}\\
\end{pmatrix}
\]
\[ R1 \longmapsto R2 + b\cdot R3 \]
\[ R2 \longmapsto R2 - b(1+a)\cdot R3 \]
\[
\begin{pmatrix}
1 & 0 & 0 & 0\\
0 & 1 & 0 & \frac{b(2+a)-1}{b(2+a)}\\
0 & 0 & 1 & \frac{1}{b(2+a)}\\
\end{pmatrix}
\]
Antwoord:
\[
V=\left\lbrace\left(0,\frac{b(2+a)-1}{b(2+a)}, \frac{1}{b(2+a)}\right)\right\rbrace
\]\\
\textbf{Samenvatting}:\\
Als  $b=0 \vee a=-2$ dan heeft het stelsel oneindig veel oplossingen:
\[
V = \{ (1-p-bq,p,q) | p,q \in \mathbb{R} \}
\]
Als $k=-1$ dan heeft het stelsel geen oplossingen:
\[
V=\emptyset
\]
Anders heeft het stelsel precies één oplossing:
\[
V=\left\lbrace\left(0,\frac{b(2+a)-1}{b(2+a)}, \frac{1}{b(2+a)}\right)\right\rbrace
\]

\section{Opdrachten}
\subsection*{opdracht 1.2}
Om op een matrix een ERO uit te voeren, berekenen we eigenlijk de vermenigvuldiging van de matrix met de corresponderende elementaire matrix. Dus: $M' = M \cdot E$. Om de ERO om te keren vermenigvuldigen we $M'$ met een matrix $E^{-1}$ zodat $M'\cdot E^{-1} = M \cdot I = M$.
Om aan te tonen dat alle elementaire rijoperaties inverteerbaar zijn, tonen we het aan voor elk van de EROs. Elk van de EROs komt overeen met een elementaire matrix (zie p 36). We bewijzen dat deze inverteerbaar zijn door de de inverse te construeren.
\subsubsection*{$R_i\rightarrow \lambda R_i$}
\[
E=
\begin{pmatrix}
1 & 0 & 0 & \cdots & 0 & 0\\
0 & 1 & 0 & \cdots & 0 & 0\\
\vdots & \vdots & \ddots & \vdots& & \vdots\\
0 & 0 & \cdots & \lambda & \cdots & 0\\
\vdots & \vdots & & \vdots& \ddots & \vdots\\
0 & 0 & \cdots & 0 & \cdots &1
\end{pmatrix}
\]
met de $\lambda$ op rij $i$.\\
De inverse hiervan is
\[
E^{-1}=
\begin{pmatrix}
1 & 0 & 0 & \cdots & 0 & 0\\
0 & 1 & 0 & \cdots & 0 & 0\\
\vdots & \vdots & \ddots & \vdots& & \vdots\\
0 & 0 & \cdots & \frac{1}{\lambda} & \cdots & 0\\
\vdots & \vdots & & \vdots& \ddots & \vdots\\
0 & 0 & \cdots & 0 & \cdots &1
\end{pmatrix}
\]

\subsubsection*{$R_i \leftrightarrow R_j$}
\[
E=
\begin{pmatrix}
1 & 0 & \cdots & 0 & \cdots & 0 & \cdots & 0\\
0 & 1 & \cdots & 0 & \cdots & 0 & \cdots & 0\\
\vdots & \vdots & \ddots & \vdots& & \vdots & &\vdots\\
0 & 0 & \cdots & 0 & \cdots & 1 & \cdots & 0\\
\vdots & \vdots & & \vdots& \ddots & \vdots & &\vdots\\
0 & 0 & \cdots & 1 & \cdots & 0 & \cdots & 0\\
\vdots & \vdots & & \vdots & & \vdots & \ddots & \vdots\\
0 & 0 & \cdots & 0 & \cdots & 0 & \cdots & 0
\end{pmatrix}
\]
met de eerste $1$, niet op de hoofddiagonaal, op rij $i$ en de tweede op rij $j$.\\
De inverse hiervan is $E^{-1} = E$

\subsubsection*{$R_i \rightarrow R_i \lambda R_j$}
\[
E=
\begin{pmatrix}
1 & 0 & \cdots & 0 & \cdots & 0 & \cdots & 0\\
0 & 1 & \cdots & 0 & \cdots & 0 & \cdots & 0\\
\vdots & \vdots & \ddots & \vdots& & \vdots & &\vdots\\
0 & 0 & \cdots & 1 & \cdots & \lambda & \cdots & 0\\
\vdots & \vdots & & \vdots& \ddots & \vdots & &\vdots\\
0 & 0 & \cdots & 0 & \cdots & 1 & \cdots & 0\\
\vdots & \vdots & & \vdots & & \vdots & \ddots & \vdots\\
0 & 0 & \cdots & 0 & \cdots & 0 & \cdots & 0
\end{pmatrix}
\]
met de $\lambda$ op rij $i$, kolom $j$.\\
De inverse hiervan is 
\[
E^{-1}=
\begin{pmatrix}
1 & 0 & \cdots & 0 & \cdots & 0 & \cdots & 0\\
0 & 1 & \cdots & 0 & \cdots & 0 & \cdots & 0\\
\vdots & \vdots & \ddots & \vdots& & \vdots & &\vdots\\
0 & 0 & \cdots & 1 & \cdots & -\lambda & \cdots & 0\\
\vdots & \vdots & & \vdots& \ddots & \vdots & &\vdots\\
0 & 0 & \cdots & 0 & \cdots & 1 & \cdots & 0\\
\vdots & \vdots & & \vdots & & \vdots & \ddots & \vdots\\
0 & 0 & \cdots & 0 & \cdots & 0 & \cdots & 0
\end{pmatrix}
\]

\subsection*{opdracht 1.23}

\subsection*{opdracht 1.33}
\end{document}