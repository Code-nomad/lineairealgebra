\documentclass[10pt,a4paper]{article}

\usepackage[english]{babel}
\usepackage{amsmath}
\usepackage{amsfonts}
\usepackage{amssymb}
\usepackage{amsthm}
\makeatletter
\renewcommand*\env@matrix[1][*\c@MaxMatrixCols c]{%
  \hskip -\arraycolsep
  \let\@ifnextchar\new@ifnextchar
  \array{#1}}
\makeatother
\title{Oplossingen Lineaire Algebra 2013}
\author{TODO}


\begin{document}

\maketitle
\pagebreak
\tableofcontents
\pagebreak


\section{Oefeningen 1.7.1}
\subsection*{oef 1}
\subsubsection*{Echelonvorm}
\subsubsection*{a)}
Do row reduction:
$\begin{pmatrix}
3 & -6 & 9\\
-2 & 7 & -2\\
0 & 1 & 5
\end{pmatrix}
$
\\
The following steps are done sequentially:

\begin{enumerate}
\item Add 2/3 x row 1 to row 2
\item Divide row 1 by 3:
\item Subtract 1/3 × (row 2) from row 3:
\item Multiply row 3 by 3/11:
\item Subtract 4 × (row 3) from row 2:
\item Subtract 3 × (row 3) from row 1:
\item Divide row 2 by 3:
\end{enumerate}

$
\begin{pmatrix}
1 & -2 & 0\\
0 & 1 & 0\\
0 & 0 & 1
\end{pmatrix}
$
\subsubsection*{b)}
Do row reduction:
$
\begin{pmatrix}
3 & -2 & -5 & 4\\
-5 & 2 & 8 & -5\\
-3 & 4 & 7 & -3\\
2 & -3 & -5 & 8
\end{pmatrix}
$
\\
The following steps are done sequentially:
\begin{enumerate}
\item Swap row 1 with row 2:
\item Add 3/5 × (row 1) to row 2:
\item Multiply row 1 by -1:
\item Multiply row 2 by -5:
\item Add 3/5 × (row 1) to row 3:
\item Multiply row 3 by 5:
\item Subtract 2/5 × (row 1) from row 4:
\item Multiply row 4 by 5:
\item Swap row 2 with row 3:
\item Subtract 2/7 × (row 2) from row 3:
\item Multiply row 3 by -7/5:
\item Add 11/14 × (row 2) to row 4:
\item Multiply row 4 by 14/5:
\item Add 1/3 × (row 3) to row 4:
\item Multiply row 4 by 3/259:
\item Subtract 7 × (row 4) from row 3:
\item Subtract 5 × (row 4) from row 1:
\item Divide row 3 by 3:
\item Subtract 11 × (row 3) from row 2:
\item Add 8 × (row 3) to row 1:
\item Divide row 2 by 14:
\item Add 2 × (row 2) to row 1:
\item Divide row 1 by 5:
\end{enumerate}
De matrix staat nu wel in gereduceerde echelon vorm, maar dat is ook een echelon vorm ;) :
\\
\[
\begin{pmatrix}
1 & 0 & 0 & 0\\
0 & 1 & 0 & 0\\
0 & 0 & 1 & 0\\
0 & 0 & 0 & 1
\end{pmatrix}
\]

\subsection*{oef 2}

\subsection*{oef 3}
\subsubsection*{Echelonvorm}
\[
\begin{pmatrix}
1 &  2 &  3 &  4 &  5\\
2 &  3 &  4 &  5 &  1\\
3 &  4 &  5 &  1 &  2\\
4 &  5 &  1 &  2 &  3\\
5 &  1 &  2 &  3 &  4 
\end{pmatrix}
\]
\[ R2 \longmapsto R2 -2\cdot R1\]
\[ R3 \longmapsto R3 -3\cdot R1\]
\[ R4 \longmapsto R4 -4\cdot R1\]
\[ R5 \longmapsto R5 -5\cdot R1\]
\[
\begin{pmatrix}
1 &  2 &  3 &  4 &  5 \\
0 & -1 & -2 & -3 & -9 \\
0 & -2 & -4 & -11& -13\\
0 & -3 & -11& -14& -17\\
0 & -9 & -13& -17& -21
\end{pmatrix}
\]
\[ R3 \longmapsto R3 -2\cdot R2\]
\[ R4 \longmapsto R4 -2\cdot R2\]
\[ R5 \longmapsto R5 -2\cdot R2\]

\[
\begin{pmatrix}
1 &  2 &  3 &  4 &  5\\
0 & -1 & -2 & -3 & -9\\
0 &  0 &  0 & -5 &  5\\
0 &  0 & -5 & -5 & 10\\
0 &  0 &  5 & 10 & 60 
\end{pmatrix}
\]
\begin{center}
Wissel R3 en R4
\end{center}
\[ R5 \longmapsto R5 + R4\]

\[
\begin{pmatrix}
1 &  2 &  3 &  4 &  5\\
0 & -1 & -2 & -3 & -9\\
0 &  0 & -5 & -5 & 10\\
0 &  0 &  0 & -5 &  5\\
0 &  0 &  0 &  0 & 75 
\end{pmatrix}
\]

\subsubsection*{Rij-geredeuceerde vorm}
\[
\begin{pmatrix}
1 &  0 &  0 &  0 & 0\\
0 &  1 &  0 &  0 & 0\\
0 &  0 &  1 &  0 & 0\\
0 &  0 &  0 &  1 & 0\\
0 &  0 &  0 &  0 & 1 
\end{pmatrix}
\]

\subsection*{oef 4}
\subsubsection*{a)}
Het oplossingsstelsel is bijna letterlijk af te lezen:\\
Stel $t=\lambda$ dan is:\\
\begin{center}
$x = -4\lambda - 1$\\
$y = -2\lambda + 6$\\
$z = -3\lambda + 2$\\
\end{center}
Hieruit volgt de oplossingsverzameling:
\[
V=\{(-4\lambda - 1,-2\lambda + 6,-3\lambda + 2, \lambda)|\lambda \in \mathbb{R}\}
\]
\subsubsection*{b)}
Dit triviaal direct te bepalen door de laatste rij:
\[
V=\emptyset
\]
\subsection*{oef 5}

\subsection*{oef 6}
\subsubsection*{a)}
\[
\begin{pmatrix}
1 &  2 &  4 &  6\\
3 &  8 & 14 & 16\\
2 &  6 & 11 & 12
\end{pmatrix}
\longrightarrow
\begin{pmatrix}
1 &  0 &  0 &  4\\
0 &  1 &  0 & -3\\
0 &  0 &  1 &  2
\end{pmatrix}
\]
Antwoord:
\[
V=\{(4,-3,2)\}
\]

\subsubsection*{b)}
\[
\begin{pmatrix}
3 &  2 &  4 &  5\\
1 &  1 & -3 &  2\\
4 &  3 &  1 &  7
\end{pmatrix}
\longrightarrow
\begin{pmatrix}
1 &  0 & 10 &  1\\
0 &  1 & -13&  1\\
0 &  0 &  0 &  0
\end{pmatrix}
\]
Antwoord:
\[
V=\{(1-10t,1+13t,t) | t \in \mathbb{R}\}
\]

\subsubsection*{c)}
\[
\begin{pmatrix}
1 &  2 & -3 & -1\\
3 & -1 &  2 &  7\\
5 &  3 & -4 &  2
\end{pmatrix}
\longrightarrow
\begin{pmatrix}
1 &  0 & \frac{1}{7} &  0\\
0 &  1 & \frac{-11}{7}&  0\\
0 &  0 &  0 &  1
\end{pmatrix}
\]
Antwoord:
\[
V=\emptyset
\]

\subsubsection*{d)}
\[
\begin{pmatrix}
1 &  1 & -2 &  1 &  2 & 1\\
2 & -1 &  2 &  2 &  6 & 2\\
3 &  2 & -4 & -3 & -9 & 3
\end{pmatrix}
\longrightarrow
\begin{pmatrix}
1 &  0 &  0 &  0 &  0 & 1\\
0 &  1 & -2 &  0 &  0 & 0\\
0 &  0 &  0 &  1 &  3 & 0
\end{pmatrix}
\]
Antwoord:
\[
V=\{(1,2a,a,-3b,b) | a,b \in \mathbb{R} \}
\]

\subsection*{oef 7}
\subsubsection*{a)}
\[
\begin{pmatrix}
2 & -3 & 0 & 8\\
4 & -5 & 1 & 15\\
2 & 0 & 4 & 1
\end{pmatrix}
\]
\[ R2 \longmapsto R2 - 2\cdot R1\]
\[ R3 \longmapsto R3 -  R1\]
\[
\begin{pmatrix}
2 & -3 & 0 & 8\\
0 & 1 & 1 & -1\\
0 & 3 & 4 & -7
\end{pmatrix}
\]

\[ R3 \longmapsto R3 - 3\cdot R2\]

\[
\begin{pmatrix}
2 & -3 & 0 & 8\\
0 & 1 & 1 & -1\\
0 & 0 & 1 & -4
\end{pmatrix}
\]
\[
z = 4\]
\[y = -4-1 = -5\]
\[2x = -(-3\cdot -5) + 8=-7\]
\[ x = -7/2\]

Antwoord:
\[
V = \{(-\frac{7}{2},-5,4)\}
\]

\subsubsection*{b)}
\[
\begin{pmatrix}
0 & 2 & -1 & 1\\
4 & -10 & 3 & 5\\
3 & -3 & 0 & 6
\end{pmatrix}
\]

\[ R1 \leftrightarrow R3\]
\[
\begin{pmatrix}
3 & -3 & 0 & 6\\
4 & -10 & 3 & 5\\
0 & 2 & -1 & 1
\end{pmatrix}
\]

\[ R1 \longmapsto R1 / 3\]
\[
\begin{pmatrix}
1 & -1 & 0 & 2\\
4 & -10 & 3 & 5\\
0 & 2 & -1 & 1
\end{pmatrix}
\]
\[ R2 \longmapsto R2 - 4\cdot R1\]

\[
\begin{pmatrix}
1 & -1 & 0 & 2\\
0 & -6 & 3 & -3\\
0 & 2 & -1 & 1
\end{pmatrix}
\]
\[ R2 \longmapsto R2 + 3\cdot R3\]
\[
\begin{pmatrix}
1 & -1 & 0 & 2\\
0 & 0 & 0 & 0\\
0 & 2 & -1 & 1
\end{pmatrix}
\]
\[ R2 \leftrightarrow R3\]

\[
\begin{pmatrix}
1 & -1 & 0 & 2\\
0 & 2 & -1 & 1\\
0 & 0 & 0 & 0
\end{pmatrix}
\]
Antwoord:
\[
V = \{(-\frac{\lambda+1}{2} + 2,\frac{\lambda+1}{2},\lambda) | \lambda \in \mathbb{R}\}
\]

\subsubsection*{c)}
\[
\begin{pmatrix}
1 & 1 & 1 &-1&-2\\
2 & -1 & 1 & -1 & 0\\
3 & 2 & -1 & -1 & 1\\
1 & 1 & 3 & -3 & -8
\end{pmatrix}
\]
\[ R2 \longmapsto R2 - 2\cdot R1\]
\[ R3 \longmapsto R3 - 3\cdot R1\]
\[ R4 \longmapsto R4 - R1\]

\[
\begin{pmatrix}
1 & 1 & 1 & -1 & -2\\
0 & -3 & -1 & 1 & 4\\
0 & -1 & -4 & 2 & 7\\
0 & 0 & 2 & -2 & -6
\end{pmatrix}
\]
\[ R2 \leftrightarrow R3\]

\[
\begin{pmatrix}
1 & 1 & 1 & -1 & -2\\
0 & -1 & -4 & 2 & 7\\
0 & -3 & -1 & 1 & 4\\
0 & 0 & 2 & -2 & -6
\end{pmatrix}
\]
\[ R2 \longmapsto  -1\cdot R2\]

\[
\begin{pmatrix}
1 & 1 & 1 & -1 & -2\\
0 & 1 & 4 & -2 & -7\\
0 & -3 & -1 & 1 & 4\\
0 & 0 & 2 & -2 & -6
\end{pmatrix}
\]
\[ R3 \longmapsto R3 + 3\cdot R2\]
\[
\begin{pmatrix}
1 & 1 & 1 & -1 & -2\\
0 & 1 & 4 & -2 & -7\\
0 & 0 & 11 & 5 & 17\\
0 & 0 & 2 & -2 & -6
\end{pmatrix}
\]
\[ R3 \leftrightarrow R4\]

\[
\begin{pmatrix}
1 & 1 & 1 & -1 & -2\\
0 & 1 & 4 & -2 & -7\\
0 & 0 & 2 & -2 & -6\\
0 & 0 & 11 & 5 & 17
\end{pmatrix}
\]
\[ R3 \longmapsto  \frac{R3}{2}
\]

\[
\begin{pmatrix}
1 & 1 & 1 & -1 & -2\\
0 & 1 & 4 & -2 & -7\\
0 & 0 & 1 & -1 & -3\\
0 & 0 & 11 & 5 & 17
\end{pmatrix}
\]
\[ R4 \longmapsto R4 - 11\cdot R3\]

\[
\begin{pmatrix}
1 & 1 & 1 & -1 & -2\\
0 & 1 & 4 & -2 & -7\\
0 & 0 & 1 & -1 & -3\\
0 & 0 & 0 & 16 & 17
\end{pmatrix}
\]
Antwoord:
\[
V = \{(-\frac{175}{8},-\frac{135}{8}, -\frac{-31}{16}, \frac{17}{16})\}
\]
\subsection*{oef 8}

\subsection*{oef 9}
\[
\begin{pmatrix}
2 &  1 &  7 & b_1\\
6 & -2 & 11 & b_2 \\
2 & -1 &  3 & b_3\\
\end{pmatrix}
\]
\[ R2 \longmapsto R2 -3\cdot R1\]
\[ R3 \longmapsto R3 - R1\]
\[
\begin{pmatrix}
2 &  1 &  7 & b_1\\
0 & -5 & -10& b_2-3b_1 \\
0 & -2 &  -4& b_3-b_1\\
\end{pmatrix}
\]
\[ R2 \longmapsto -\frac{1}{5} R2\]
\[
\begin{pmatrix}
2 &  1 &  7 & b_1\\
0 &  1 &  2 & -\frac{1}{5}(b_2-3b_1) \\
0 & -2 &  -4& b_3-b_1\\
\end{pmatrix}
\]
\[ R1 \longmapsto R1 - R2\]
\[ R3 \longmapsto R3 + 2 \cdot	R2\]
\[
\begin{pmatrix}
2 &  0 &  5 & b_1 + \frac{1}{5}(b_2-3b_1)\\
0 &  1 &  2 & -\frac{1}{5}(b_2-3b_1) \\
0 &  0 &  0 & (b_3-b_1) - \frac{2}{5}(b_2-3b_1)\\
\end{pmatrix}
\]
Antwoord:\\
Als $(b_3-b_1) - \frac{2}{5}(b_2-3b_1) = 0$ dan heeft het stelsel oneindig veel oplossingen.
Als $(b_3-b_1) - \frac{2}{5}(b_2-3b_1) \neq 0$ dan heeft het stelsel geen oplossingen ($V=\emptyset$)

\subsection*{oef 10}
\[
\begin{pmatrix}
1 & h & 1\\
2 & 3 & k\\
\end{pmatrix}
\]
\[ R2 \longmapsto R2 - 2 \cdot R1\]

\[
\begin{pmatrix}
1 & h & 1\\
0 & 3-2h & k-2\\
\end{pmatrix}
\]
\subsubsection*{a) Geen oplossing:}
Als $h = \frac{3}{2}$ en $k \neq 2$
\subsubsection*{b) Unieke oplossing:}
Als $h \neq \frac{3}{2}$
\subsubsection*{c) Meerdere oplossingen:}
Als $h = \frac{3}{2}$ en $k = 2$
\subsection*{oef 11}

\subsection*{oef 12}
\subsubsection*{a)}
\begin{center}
Wissel R1 en R3, en daarna R2 en R1.
\end{center}
\[
\begin{pmatrix}
1 &  1 &  k & 1\\
1 &  k &  1 & 1\\
k &  1 &  1 & 1
\end{pmatrix}
\]
\[ R2 \longmapsto R2 - R1\]
\[ R3 \longmapsto R3 - k \cdot	R1\]
\[
\begin{pmatrix}
1 &  1 &  k & 1\\
0 & k-1& 1-k& 0\\
0 & 1-k & 1-k^2 & 1-k
\end{pmatrix}
\]
Geval 1: $k=1$
\[
\begin{pmatrix}
1 &  1 & 1 & 1\\
0 &  0 & 0 & 0\\
0 &  0 & 0 & 0
\end{pmatrix}
\]
Antwoord:
\[
V=\{ (1-a-b,a,b) | a,b \in \mathbb{R}\}
\]\\
Geval 2: $k\neq 1$\\ 
\[ R2 \longmapsto \frac{1}{k-1}R2\]
\[ R3 \longmapsto \frac{1}{1-k}R3\]
\[
\begin{pmatrix}
1 &  1 &  k & 1\\
0 &  1 & -1 & 0\\
0 &  1 & k+1& 1
\end{pmatrix}
\]
\[ R1 \longmapsto R1-R2\]
\[ R3 \longmapsto R3-R1\]
\[
\begin{pmatrix}
1 &  0 &  k+1 & 1\\
0 &  1 & -1 & 0\\
0 &  0 & k+2& 1
\end{pmatrix}
\]\\
Geval 2a: $k = -2$\\
\[
\begin{pmatrix}
1 &  0 & -1 & 1\\
0 &  1 & -1 & 0\\
0 &  0 &  0 & 1
\end{pmatrix}
\]
Antwoord:
\[
V=\emptyset
\]\\
Geval 2b: $k\neq-2$\\
\[ R3 \longmapsto \frac{1}{k+2}R3\]
\[
\begin{pmatrix}
1 &  0 &  k+1 & 1\\
0 &  1 & -1 & 0\\
0 &  0 &  1 & \frac{1}{k+2}
\end{pmatrix}
\]
\[ R1 \longmapsto R1 - (k+1)\cdot R3\]
\[ R2 \longmapsto R2 + R3\]
\[
\begin{pmatrix}
1 &  0 &  0 & \frac{1}{k+2}\\
0 &  1 &  0 & \frac{1}{k+2}\\
0 &  0 &  1 & \frac{1}{k+2}
\end{pmatrix}
\]
Antwoord:
\[
V=\{(\frac{1}{k+2},\frac{1}{k+2},\frac{1}{k+2})\}
\]\\
\textbf{Samenvatting}:\\
Als $k=1$ dan heeft het stelsel oneindig veel oplossingen:
\[
V=\{ (1-a-b,a,b) | a,b \in \mathbb{R}\}
\]
Als $k=-2$ dan heeft het stelsel geen oplossingen:
\[
V=\emptyset
\]
Anders heeft het stelsel precies één oplossing:
\[
V=\{(\frac{1}{k+2},\frac{1}{k+2},\frac{1}{k+2})\}
\]

\subsubsection*{b)}
\begin{center}
Wissel R1 en R2
\end{center}
\[
\begin{pmatrix}
1 &  k &  k+1\\
k &  1 &  2
\end{pmatrix}
\]
\[ R2 \longmapsto R2 - k\cdot R3\]
\[
\begin{pmatrix}
1 &  k &  k+1\\
0 &  1-k^2 &  -k^2-k+2
\end{pmatrix}
\]\\
Geval 1: $k=1$\\
\[
\begin{pmatrix}
1 &  1 &  2\\
0 &  0 &  0
\end{pmatrix}
\]
Antwoord:
\[
V=\{(2-t,t) | t \in \mathbb{R}\}
\]\\
Geval 2: $k=-1$\\
\[
\begin{pmatrix}
1 & -1 &  0\\
0 &  0 &  2
\end{pmatrix}
\]
Antwoord:
\[
V=\emptyset
\]\\
Geval 3: $k\neq 1 \wedge k\neq -1$\\
\[ R2 \longmapsto \frac{1}{1-k^2}\cdot R2 \]
\[
\begin{pmatrix}
1 &  k &  k+1\\
0 &  1 &  -\frac{k+2}{k+1}
\end{pmatrix}
\]
\[ R1 \longmapsto R1-k\cdot R2 \]
\[
\begin{pmatrix}
1 &  0 &  \frac{2k^2+4k+1}{k+1}\\
0 &  1 &  -\frac{k+2}{k+1}
\end{pmatrix}
\]
Antwoord:
\[
V=\left\lbrace\left(\frac{2k^2+4k+1}{k+1},-\frac{k+2}{k+1}\right)\right\rbrace
\]\\
\textbf{Samenvatting}:\\
Als $k=1$ dan heeft het stelsel oneindig veel oplossinen:
\[
V=\{(2-t,t) | t \in \mathbb{R}\}
\]
Als $k=-1$ dan heeft het stelsel geen oplossingen:
\[
V=\emptyset
\]
Anders heeft het stelsel precies één oplossing:
\[
V=\left\lbrace\left(\frac{2k^2+4k+1}{k+1},-\frac{k+2}{k+1}\right)\right\rbrace
\]

\subsubsection*{c)}
\[
\begin{pmatrix}
k & k+1 & 1 & 0\\
k & 1 & k+1 & 0\\s
2k & 1 & 1 & k+1\\
\end{pmatrix}
\]
Geval 1: $k=0$:
\[
\begin{pmatrix}
0 & 1 & 1 & 0\\l
0 & 1 & 1 & 0\\
0 & 1 & 1 & 1\\
\end{pmatrix}
\]
\[ R1 \longmapsto R1 - R2 \]
\[ R3 \longmapsto R3 - R2 \]
\[
\begin{pmatrix}
0 & 0 & 0 & 0\\
0 & 1 & 1 & 0\\
0 & 0 & 0 & 1\\
\end{pmatrix}
\]
Antwoord:
\[
V=\emptyset
\]\\
Geval 2: $k\neq0$:
\[ R1 \longmapsto \frac{1}{k}R1 \]
\[
\begin{pmatrix}
1 & \frac{k+1}{k} & \frac{1}{k} & 0\\
k & 1 & k+1 & 0\\
2k & 1 & 1 & k+1\\
\end{pmatrix}
\]
\[ R2 \longmapsto R2 - k\cdot R1 \]
\[ R3 \longmapsto R3 - 2k\cdot R1 \]
\[
\begin{pmatrix}
1 & \frac{k+1}{k} & \frac{1}{k} & 0\\
0 & -k & k & 0\\
0 & -2k-1 & -1 & k+1\\
\end{pmatrix}
\]
\[ R2 \longmapsto \frac{-1}{k}R2\]
\[
\begin{pmatrix}
1 & \frac{k+1}{k} & \frac{1}{k} & 0\\
0 & 1 & -1 & 0\\
0 & -2k-1 & -1 & k+1\\
\end{pmatrix}
\]
\[ R1 \longmapsto R1 - \frac{k+1}{k}\cdot R2 \]
\[ R3 \longmapsto R3 + (2k+1)\cdot R2 \]
\[
\begin{pmatrix}
1 & 0 & \frac{k+2}{k} & 0\\
0 & 1 & -1 & 0\\
0 & 0 & -2k-2 & k+1\\
\end{pmatrix}
\]
Geval 2a $k=-1$:
\[
\begin{pmatrix}
1 & 0 & \frac{1}{k} & 0\\
0 & 1 & -1 & 0\\
0 & 0 & 0 & 0\\
\end{pmatrix}
\]
Antwoord:
\[
V=\left\lbrace\left(\frac{t}{k},t,t\right) | t \in \mathbb{R}\right\rbrace
\]\\
Geval 2b $k\neq-1$:
\[ R3 \longmapsto \frac{1}{-2k-2}R3\]
\[
\begin{pmatrix}
1 & 0 & \frac{k+2}{k} & 0\\
0 & 1 & -1 & 0\\
0 & 0 & 1 & -2\\
\end{pmatrix}
\]
\[ R1 \longmapsto R1 - \frac{k+2}{k}\cdot R3 \]
\[ R2 \longmapsto R2 + R3 \]
\[
\begin{pmatrix}
1 & 0 & 0 & 2\frac{k+2}{k}\\
0 & 1 & 0 & -2\\
0 & 0 & 1 & -2\\
\end{pmatrix}
\]
Antwoord:
\[
V=\left\lbrace\left(2\frac{k+2}{k},-2,-2\right)\right\rbrace
\]\\
\textbf{Samenvatting}:\\
Als $k=0$ dan heeft het stelsel geen oplossingen:
\[
V=\emptyset
\]
Als $k=-1$ dan heeft het stelsel geen oplossingen:
\[
V=\left\lbrace\left(\frac{t}{k},t,t\right) | t \in \mathbb{R}\right\rbrace
\]
Anders heeft het stelsel precies één oplossing:
\[
V=\left\lbrace\left(2\frac{k+2}{k},-2,-2\right)\right\rbrace
\]

\subsection*{oef 13}
De onderlinge stand van twee rechten bestaat uit 3 mogelijkheden:
\begin{enumerate}
\item Evenwijdig = geen oplossing voor het stelsel.\\
$1-ca = 0$ en $ d -bc \neq 0$
\item Samenvallend = oneindig veel oplossingen voor het stelsel.\\
$1-ca = 0$ en $ d -bc = 0$
\item Snijdend = een unieke oplossing voor het stelsel.\\
$1-ca \neq 0$
\end{enumerate}
\[
\begin{pmatrix}
1 & a & b\\
c & 1 & d\\
\end{pmatrix}
\]
\[ R2 \longmapsto R2 - c \cdot R1\]
\[
\begin{pmatrix}
1 & a & b\\
0 & 1-ca & d-cb\\
\end{pmatrix}
\]

\subsection*{oef 14}

\subsection*{oef 15}
\begin{center}
Wissel R1 en R3
\end{center}
\[
\begin{pmatrix}
1 & 1 & ab & 1\\
a & 1 & b & 1\\
1 & a & b & 1\\
\end{pmatrix}
\]
\[ R2 \longmapsto R2 - a\cdot R1 \]
\[ R3 \longmapsto R3 - R1 \]
\[
\begin{pmatrix}
1 & 1 & ab & 1\\
0 & 1-a & b-a^2b & 1-a\\
0 & a-1 & b-ab & 0\\
\end{pmatrix}
\]
Geval 1: $a=1$
\[
\begin{pmatrix}
1 & 1 & b & 1\\
0 & 0 & 0 & 0\\
0 & 0 & 0 & 0\\
\end{pmatrix}
\]
Antwoord:
\[
V = \{ (1-p-bq,p,q) | p,q \in \mathbb{R} \}
\]\\
Geval 2: $a \neq -1$
\[ R2 \longmapsto \frac{1}{1-a}\cdot R2 \]
\[ R3 \longmapsto \frac{1}{1-a}\cdot R3 \]
\[
\begin{pmatrix}
1 & 1 & ab & 1\\
0 & 1 & b(1+a) & 1\\
0 & -1 & b & 0\\
\end{pmatrix}
\]
\[ R1 \longmapsto R1 - R2 \]
\[ R3 \longmapsto R3 + R2 \]
\[
\begin{pmatrix}
1 & 0 & -b & 0\\
0 & 1 & b(1+a) & 1\\
0 & 0 & b(2+a) & 1\\
\end{pmatrix}
\]
Geval 2a: $b=0 \vee a=-2$
\[
\begin{pmatrix}
1 & 0 & -b & 0\\
0 & 1 & b(1+a) & 1\\
0 & 0 & 0 & 1\\
\end{pmatrix}
\]
Antwoord:
\[
V=\emptyset
\]\\
Geval 2b: $b\neq \wedge a \neq -2$
\[ R3 \longmapsto \frac{1}{b(2+a)}\cdot R3 \]
\[
\begin{pmatrix}
1 & 0 & -b & 0\\
0 & 1 & b(1+a) & 1\\
0 & 0 & 1 & \frac{1}{b(2+a)}\\
\end{pmatrix}
\]
\[ R1 \longmapsto R2 + b\cdot R3 \]
\[ R2 \longmapsto R2 - b(1+a)\cdot R3 \]
\[
\begin{pmatrix}
1 & 0 & 0 & 0\\
0 & 1 & 0 & \frac{b(2+a)-1}{b(2+a)}\\
0 & 0 & 1 & \frac{1}{b(2+a)}\\
\end{pmatrix}
\]
Antwoord:
\[
V=\left\lbrace\left(0,\frac{b(2+a)-1}{b(2+a)}, \frac{1}{b(2+a)}\right)\right\rbrace
\]\\
\textbf{Samenvatting}:\\
Als  $b=0 \vee a=-2$ dan heeft het stelsel oneindig veel oplossingen:
\[
V = \{ (1-p-bq,p,q) | p,q \in \mathbb{R} \}
\]
Als $k=-1$ dan heeft het stelsel geen oplossingen:
\[
V=\emptyset
\]
Anders heeft het stelsel precies één oplossing:
\[
V=\left\lbrace\left(0,\frac{b(2+a)-1}{b(2+a)}, \frac{1}{b(2+a)}\right)\right\rbrace
\]
\subsection*{oef 16}
Haal de gegevens van de twee tabel door de eerste:
\[
\begin{pmatrix}
4 & 1 & 1 & 1 & 16800\\
2 & 2 & 2 & 1 & 18500\\
1 & 3 & 1 & 3 & 21700\\
3 & 4 & 3 & 2 & 32100\\
\end{pmatrix}
\]
Dit oplossen geeft:
\[
\begin{pmatrix}
1 & 0 & 0 & 0 & 2100\\
0 & 1 & 0 & 0 & 3100\\
0 & 0 & 1 & 0 & 2800\\
0 & 0 & 0 & 1 & 2500\\
\end{pmatrix}
\]
Dit geeft voor Laurens:
\[
Premie = 2\cdot 2100 + 3\cdot 3100 + 2\cdot 2800 + 2\cdot 2500 = 24100
\]
\subsection*{oef 17}
\subsection*{Extra (oefenzitting)}
\[
\left\lbrace
\begin{array}{c c c c c c c}
ax &+& 4y &+& az &=& 0\\
x  &+& ay &+& 3z &=& b\\
(a+1)x &+& (a+4)y &+& (a-b^{2})z &=& b-2
\end{array}
\right\rbrace
\]
De matrix die hiermee overeen komt is:
\[
\begin{pmatrix}
a & 4 & a & 0\\
1 & a & 3 & b\\
(a+1) & (a+4) & (a-b^{2}) & (b-2)\\
\end{pmatrix}
\]
\begin{center}
Wissel R1 en R2
\end{center}
\[
\begin{pmatrix}
1 & a & 3 & b\\
a & 4 & a & 0\\
(a+1) & (a+4) & (a-b^{2}) & (b-2)\\
\end{pmatrix}
\]
\[ R2 \longmapsto R2 - a\cdot R1 \]
\[ R3 \longmapsto R3 - (a+1)\cdot R1 \]
\[
\begin{pmatrix}
1 & a & 3 & b\\
0 & 4-a^{2} & -2a & -ab\\
0 & 4-a^{2} & -b^{2}-2a-3 & -2-ab\\
\end{pmatrix}
\]
\\Geval 1: $a=2$
\[
\begin{pmatrix}
1 & 2 & 3 & b\\
0 & 0 & -4 & -2b\\
0 & 0 & -b^{2}-7 & -2-2b\\
\end{pmatrix}
\]
\[ R2 \longmapsto \frac{-1}{4}R2 \]
\[
\begin{pmatrix}
1 & 2 & 3 & b\\
0 & 0 & 1 & \frac{b}{2}\\
0 & 0 & -b^{2}-7 & -2-2b\\
\end{pmatrix}
\]
\[ R1 \longmapsto R1 - 3\cdot R2 \]
\[ R3 \longmapsto R3 + (b^{2}+7)\cdot R2 \]
\[
\begin{pmatrix}
1 & 2 & 0 & -\frac{b}{2}\\
0 & 0 & 1 & \frac{b}{2}\\
0 & 0 & 0 & \frac{b^{3}+3b-4}{2}\\
\end{pmatrix}
\]
\begin{center}
Vereenvoudig
\end{center}
\[
\begin{pmatrix}
1 & 2 & 0 & -\frac{b}{2}\\
0 & 0 & 1 & \frac{b}{2}\\
0 & 0 & 0 & b^{3}+3b-4\\
\end{pmatrix}
\]
\\Geval 1a: $b=1$
\[
\begin{pmatrix}
1 & 2 & 0 & -\frac{1}{2}\\
0 & 0 & 1 & \frac{1}{2}\\
0 & 0 & 0 & 0\\
\end{pmatrix}
\]
Antwoord:
\[
V = \left\lbrace \left(-\frac{1}{2}-2\lambda,\lambda,\frac{1}{2}\right)| \lambda \in \mathbb{R}\right\rbrace
\]
\\Geval 1b: $b\neq 1$
\[
\begin{pmatrix}
1 & 2 & 0 & -\frac{b}{2}\\
0 & 0 & 1 & \frac{b}{2}\\
0 & 0 & 0 & b^{3}+3b-4\\
\end{pmatrix}
\]
Antwoord:
\[
V = \emptyset
\]
\\Geval 2: $a=-2$
\[
\begin{pmatrix}
1 & -2 & 3 & b\\
0 & 0 & 4 & 2b\\
0 & 0 & 1-b^{2} & 2b-2\\
\end{pmatrix}
\]
\[ R2 \longmapsto \frac{1}{4}\cdot R2 \]
\[
\begin{pmatrix}
1 & -2 & 3 & b\\
0 & 0 & 1 & \frac{b}{2}\\
0 & 0 & 1-b^{2} & 2b-2\\
\end{pmatrix}
\]
\[ R1 \longmapsto R1 - 3\cdot R2 \]
\[ R3 \longmapsto R3 - (1-b^{2})\cdot R2 \]
\[
\begin{pmatrix}
1 & -2 & 0 & -\frac{b}{2}\\
0 & 0 & 1 & \frac{b}{2}\\
0 & 0 & 0 & \frac{b^3+3b-4}{2}\\
\end{pmatrix}
\]
\begin{center}
Vereenvoudig
\end{center}
\[
\begin{pmatrix}
1 & -2 & 0 & -\frac{b}{2}\\
0 & 0 & 1 & \frac{b}{2}\\
0 & 0 & 0 & b^3+3b-4\\
\end{pmatrix}
\]
\\Geval 2a: $b=1$
\[
\begin{pmatrix}
1 & -2 & 0 & -\frac{1}{2}\\
0 & 0 & 1 & \frac{1}{2}\\
0 & 0 & 0 & 0\\
\end{pmatrix}
\]
Antwoord:
\[
V= \left\lbrace \left(2\lambda-\frac{1}{2},\lambda,\frac{1}{2}\right)| \lambda \in \mathbb{R}\right\rbrace
\]
\\Geval 2b: $b \neq 1$
\[ R3 \longmapsto \frac{1}{b^3+3b-4}R3 \]
\[
\begin{pmatrix}
1 & -2 & 0 & -\frac{b}{2}\\
0 & 0 & 1 & \frac{b}{2}\\
0 & 0 & 0 & 1\\
\end{pmatrix}
\]
Antwoord:
\[
V= \emptyset
\]
\\Geval 3: $a\neq 2 \wedge a \neq -2$ 
\[ R2 \longmapsto \frac{1}{4-a^{2}}R2 \]
\[
\begin{pmatrix}
1 & a & 3 & b\\
0 & 1 & \frac{-2a}{4-a^{2}} & \frac{-ab}{4-a^{2}}\\
0 & 4-a^{2} & -b^{2}-2a-3 & -2-ab\\
\end{pmatrix}
\]
\[ R1 \longmapsto R1 - a \cdot R2 \]
\[ R3 \longmapsto R3 - (4-a^{2}) \cdot R2 \]
\[
\begin{pmatrix}
1 & 0 & \frac{12-a^{2}}{4-a^{2}} & \frac{4b}{4-a^{2}}\\
0 & 1 & \frac{-2a}{4-a^{2}} & \frac{-ab}{4-a^{2}}\\
0 & 0 & -b^{2}-3 & -2\\
\end{pmatrix}
\]
\[ R3 \longmapsto -\frac{1}{b^{2}+3}R3\]
\[
\begin{pmatrix}
1 & 0 & \frac{12-a^{2}}{4-a^{2}} & \frac{4b}{4-a^{2}}\\
0 & 1 & \frac{-2a}{4-a^{2}} & \frac{-ab}{4-a^{2}}\\
0 & 0 & 1 & \frac{2}{b^{2}+3}\\
\end{pmatrix}
\]
\[ R1 \longmapsto R1 - \frac{12-a^{2}}{4-a^{2}} \cdot R3 \]
\[ R2 \longmapsto R2 + \frac{2a}{4-a^{2}} \cdot R3 \]
\[
\begin{pmatrix}
1 & 0 & 0 & \frac{4b^{3}+2a^{2}-12}{(4-a^{2})(b^{2}+3)}\\
0 & 1 & 0 & \frac{-ab^{3}-3ab+4a}{(4-a^{2})(b^{2}+3)}\\
0 & 0 & 1 & \frac{2}{b^{2}+3}\\
\end{pmatrix}
\]
Antwoord:
\[
V = \left\lbrace\left(\frac{4b^{3}+2a^{2}-12}{(4-a^{2})(b^{2}+3)},\frac{-ab^{3}-3ab+4a}{(4-a^{2})(b^{2}+3)},\frac{2}{b^{2}+3}\right)\right\rbrace
\]
\\Samenvatting:
\begin{itemize}
\item $a=2$
\begin{itemize}
\item $b=1$
\[
V = \left\lbrace \left(-\frac{1}{2}-2\lambda,\lambda,\frac{1}{2}\right)| \lambda \in \mathbb{R}\right\rbrace
\]
\item $b\neq 1$
\[
V = \emptyset
\]
\end{itemize}
\item $a=-2$
\begin{itemize}
\item $b=1$
\[
V= \left\lbrace \left(2\lambda-\frac{1}{2},\lambda,\frac{1}{2}\right)| \lambda \in \mathbb{R}\right\rbrace
\]
\item $b\neq 1$
\[
V= \emptyset
\]
\end{itemize}
\item $a\neq 2 \wedge a \neq -2$ 
\[
V = \left\lbrace\left(\frac{4b^{3}+2a^{2}-12}{(4-a^{2})(b^{2}+3)},\frac{-ab^{3}-3ab+4a}{(4-a^{2})(b^{2}+3)},\frac{2}{b^{2}+3}\right)\right\rbrace
\]
\end{itemize}
\section{Oefeningen 1.7.2}
\subsection*{oef 3}
$$(A - B)^T = A^T - B^T$$
Dus:
$$A+B^T = A^T - B^T$$
$$A+2\cdot B^T = A^T$$
$$A - A^T = -2B^T$$
$$B = \left(-\frac{A - A^T}{2}\right)^T$$
$$ B = 
\begin{pmatrix}
0 & \frac{-7}{2} & \frac{1}{2}\\
\frac{7}{2} & 0 & \frac{3}{2}\\
\frac{-1}{2} & \frac{-3}{2} & 0
\end{pmatrix}
$$
\subsection*{oef 4}
\subsubsection*{a)}
\begin{proof}
Bewijs uit het ongerijmde:\\
Stel $A\neq 0$, neem $X_1$ en $X_2$ zodat $AX_1 = 0$ en $AX_2=0$ met $X_1\neq X_2$
\[
AX_1 = AX_2 = 0
\]
Omdat $A=A$ zijn $X_1$ en $X_2$ gelijk.
Contradictie.
\end{proof}

\subsubsection*{b)}
\begin{proof}
Rechtstreeks bewijs:\\
\[
AX=BX \Leftrightarrow AX - BX=0
\]
Dus $(A-B)X=0$.
Dit betekent volgens de stelling in 4.a dat $A-B=0$.
Nu zien we dat $A=B$.
\end{proof}

\subsection*{oef 6}
\subsubsection*{a)}
$$ AB =
\begin{pmatrix}
1 & 2 \\
0 & 1 \\
4 & 1
\end{pmatrix}
$$
$$ A^{-1} \cdot AB = A^{-1} \cdot 
\begin{pmatrix}
1 & 2 \\
0 & 1 \\
4 & 1
\end{pmatrix}
$$
$$ B = A^{-1} \cdot
\begin{pmatrix}
1 & 2 \\
0 & 1 \\
4 & 1
\end{pmatrix}
$$
$$ B = 
\begin{pmatrix}
1 & 1 & 2 \\
0 & 1 & 3 \\
4 & 2 & 1
\end{pmatrix}
\cdot
\begin{pmatrix}
1 & 2 \\
0 & 1 \\
4 & 1
\end{pmatrix}
$$
$$ B = 
\begin{pmatrix}
9 & 5 \\
12 & 4 \\
8 & 11
\end{pmatrix}
$$
\subsubsection*{b)}
$$ AC = A^2 + A$$
$$ A^{-1} \cdot AC = A^{-1} \cdot (A^2 + A)$$
$$ C = A + \mathbb{I} $$
$$ C = (A^{-1})^{-1} + \mathbb{I} $$
$$ C = 
\begin{pmatrix}
5 & -3 & -1 \\
-12 & 7 & 3 \\
4 & -2 & -1
\end{pmatrix}
+ \mathbb{I} $$
$$ C = 
\begin{pmatrix}
6 & -3 & -1 \\
-12 & 8 & 3 \\
4 & -2 & 0
\end{pmatrix}
$$

\subsection*{oef 8}
\subsubsection*{a)}
$$
\begin{pmatrix}[ccc|ccc]
 1 & 0 & 0 & 1 & 0 & 0\\
-2 & 1 & 0 & 0 & 1 & 0\\ 
 6 & 0 & 3 & 0 & 0 & 1 
\end{pmatrix}
$$
$$R2 \longmapsto R2 + 2\cdot R1$$
$$R3 \longmapsto R3 - 6\cdot R1$$

$$
\begin{pmatrix}[ccc|ccc]
 1 & 0 & 0 & 1 & 0 & 0\\
 0 & 1 & 0 & 2 & 1 & 0\\ 
 0 & 0 & 3 & -6 & 0 & 1 
\end{pmatrix}
$$
$$R3 \longmapsto \frac{1}{3}\cdot R3$$
$$
\begin{pmatrix}[ccc|ccc]
 1 & 0 & 0 & 1 & 0 & 0\\
 0 & 1 & 0 & 2 & 1 & 0\\ 
 0 & 0 & 1 & -2 & 0 & \frac{1}{3} 
\end{pmatrix}
$$
\subsubsection*{b)}

$$
\begin{pmatrix}[ccc|ccc]
1 & 4 & 0 & 1 & 0 & 0\\
0 & 1 & 0 & 0 & 1 & 0\\
0 & 0 & 3 & 0 & 0 & 1
\end{pmatrix}
$$
$$R1 \longmapsto R1 - 4\cdot R2$$
$$R3 \longmapsto \frac{1}{3}\cdot R3$$
$$
\begin{pmatrix}[ccc|ccc]
1 & 0 & 0 & 1 & -4 & 0\\
0 & 1 & 0 & 0 & 1 & 0\\
0 & 0 & 1 & 0 & 0 & \frac{1}{3}
\end{pmatrix}
$$

\subsubsection*{c)}
$$
\begin{pmatrix}
0 & 1 & 0 & 0\\
1 & 0 & 0 & 0\\
0 & 0 & 0 & 1\\
0 & 0 & 1 & 0
\end{pmatrix}
$$
Aangezien hier enkel rijopraties van type II zijn gebruikt is de inverse van deze matrix zichzelf.
\subsection*{oef 9}

\[
\begin{pmatrix}
a & b & 1 & 0\\
c & 0 & 0 & 1
\end{pmatrix}
\]
\[ R1 \longmapsto \frac{1}{a} R1 \]
\[
\begin{pmatrix}
1 & \frac{b}{a} & \frac{1}{a} & 0\\
c & 0 & 0 & 1
\end{pmatrix}
\]
\[ R2 \longmapsto R2 - c \cdot R1\]
\[
\begin{pmatrix}
1 & \frac{b}{a} & \frac{1}{a} & 0\\
0 & -\frac{bc}{a} & \frac{c}{a} & 1
\end{pmatrix}
\]
\[ R2 \longmapsto -\frac{a}{bc} R2 \]
\[
\begin{pmatrix}
1 & \frac{b}{a} & \frac{1}{a} & 0\\
0 & 1 & -\frac{1}{b} & -\frac{a}{bc}
\end{pmatrix}
\]
\[ R1 \longmapsto R1 - \frac{b}{a} \cdot R2\]
\[
\begin{pmatrix}
1 & 0 & \frac{2}{a} & -\frac{b}{a}\\
0 & 1 & -\frac{1}{b} & -\frac{a}{bc}
\end{pmatrix}
\]
\\De inverse van 
\[
\begin{pmatrix}
a & b\\
c & 0
\end{pmatrix}
\]
is dus
\[
\begin{pmatrix}
\frac{2}{a} & -\frac{b}{a}\\
-\frac{1}{b} & -\frac{a}{bc}
\end{pmatrix}
\]
Waarbij $a$,$b$,$c$ niet nul mogen zijn.\\
Nu hebben we dus het stelsel:
\[
\left\lbrace
\begin{array}{c c}
a &= \frac{2}{a}\\
b &= -\frac{b}{a}\\
c &= -\frac{1}{b}\\
d &= -\frac{a}{bc}
\end{array}
\right.
\]
De oplossingen van de stelsel zijn geven een antwoord op te vraag.


\subsection*{oef 16}
\subsubsection*{a)}
Te bewijzen: $A$ is inverteerbaar $\Rightarrow \not \exists k\; A^k=O$ 
\begin{proof}
We bewijzen:
\[
A \text{ is inverteerbaar } \Rightarrow A^{k} \text{ is inverteerbaar }
\]
Dit is equivalent.\\
Bewijs door volledige inductie:\\
Stap 1: $k=1$ (basis)
$A$ is inverteerbaar.\\
Stap 2: Stel dat de stelling waar is voor een bepaalde $n=k$. (inductiehypothese)\\
We bewijzen dat de stelling waar is voor $n=k+1$. (inductiestap)\\
\[ A^{k+1} =A^{k}\cdot A \]
Conclusie:\\
Door de inductiehypothese weten we dat $A^{k}$ inverteerbaar is.We weten ook dat $A$ inverteerbaar is. $A^{k} \cdot A$ is dus ook inverteerbaar.
\end{proof}

\subsubsection*{b)}
\begin{proof}
Als we een matrix $\mathbb{I}-A$ vermeningvuldigen met zijn inverse $(\mathbb{I}+A+A^2+...+A^{k+1})$ zouden we $\mathbb{I}$ moeten uitkomen.
\[ (\mathbb{I}-A)(\mathbb{I}+A+A^2+...+A^{k+1}) \]
\[ = \mathbb{I}^{2} -A+A-A^{2}+A^{2}-...-A^{k-1}+A^{k-1}-A^{k}\]
\[ = \mathbb{I}-A^{k}\]
Omdat A nilpotent is met $A^k=0$, is dit gelijk aan $\mathbb{I}$
\end{proof}
\subsection*{oef 20}
$$ A = LU = \begin{pmatrix}
1 & 0 & 0\\
-1 & 1 & 0\\
0 & -1 & 1
\end{pmatrix}
\cdot \begin{pmatrix}
1 & -1 & 0\\
0 & 1 & -1\\
0 & 0 & 1
\end{pmatrix}
=
\begin{pmatrix}
1 & -1 & 0\\
-1 & 2 & -1\\
0 & -1 & 2
\end{pmatrix}
$$
$$B = \begin{pmatrix}
2\\
-3\\
4
\end{pmatrix}
$$
We berekenen nu in $X$ in $AX = B$:
$$
\begin{pmatrix}[ccc|c]
1 & -1 & 0 & 2\\
-1 & 2 & -1 & -3\\
0 & -1 & 2 & 4
\end{pmatrix}
$$
De oplossing van dit stelsel is:\\
$$V = {(4,2,3)}$$
\subsection*{oef 22}
\[
A=
\begin{pmatrix}
1 & -1 & 0 \\
-1 & 2 & -1 \\
0 & -1 & 2 
\end{pmatrix}
\]
\[
R2 \longmapsto R2+R1 \sim
\begin{pmatrix}
1 & 0 & 0\\
1 & 1 & 0\\
0 & 0 & 1 
\end{pmatrix}
=E_1
\]
\[
\begin{pmatrix}
1 & -1 & 0 \\
0 & 1 & -1 \\
0 & -1 & 2  
\end{pmatrix}
\]
\[
R2 \longmapsto R3+R1 \sim
\begin{pmatrix}
1 & 0 & 0\\
0 & 1 & 0\\
1 & 0 & 1 
\end{pmatrix}
=E_2
\]
\[
\begin{pmatrix}
1 & -1 & 0 \\
0 & 1 & -1 \\
0 & 0 & 1 
\end{pmatrix}
\]
Nu weten we dat
\[
E_2 \cdot E_1 \cdot A 
=
\begin{pmatrix}
1 & 0 & 0\\
1 & 1 & 0\\
0 & 0 & 1 
\end{pmatrix}
\cdot
\begin{pmatrix}
1 & 0 & 0\\
0 & 1 & 0\\
1 & 0 & 1 
\end{pmatrix}
\cdot
\begin{pmatrix}
1 & -1 & 0 \\
-1 & 2 & -1 \\
0 & -1 & 2 
\end{pmatrix}
=
\begin{pmatrix}
1 & -1 & 0 \\
0 & 1 & -1 \\
0 & 0 & 1 
\end{pmatrix}
= U
\]
Dus
\[
A = E_1^{-1} \cdot E_2^{-1} \cdot U
=
\begin{pmatrix}
1 & 0 & 0\\
-1 & 1 & 0\\
0 & 0 & 1 
\end{pmatrix}
\cdot
\begin{pmatrix}
1 & 0 & 0\\
0 & 1 & 0\\
-1 & 0 & 1 
\end{pmatrix}
\cdot
U
=
\begin{pmatrix}
1 & 0 & 0\\
-1 & 1 & 0\\
-1 & 0 & 0
\end{pmatrix}
\cdot
\begin{pmatrix}
1 & -1 & 0 \\
0 & 1 & -1 \\
0 & 0 & 1 
\end{pmatrix}
\]
\subsubsection*{a)}
\[
\begin{pmatrix}
1 & 0 & 0\\
1 & 1 & 0\\
1 & 0 & 0
\end{pmatrix}
\cdot
\begin{pmatrix}
1\\1\\1
\end{pmatrix}
=
\begin{pmatrix}
1\\2\\1
\end{pmatrix}
\]
en
\[
\begin{pmatrix}
1 & -1 & 0 \\
0 & 1 & -1 \\
0 & 0 & 1 
\end{pmatrix}
\cdot
X
=
\begin{pmatrix}
1\\2\\1
\end{pmatrix}
\]
heeft als oplossingsverzameling:
\[
V=\{(4,3,1)\}
\]
\subsubsection*{b)}
\[
\begin{pmatrix}
1 & 0 & 0\\
1 & 1 & 0\\
1 & 0 & 0
\end{pmatrix}
\cdot
\begin{pmatrix}
2\\0\\-1
\end{pmatrix}
=
\begin{pmatrix}
2\\2\\2
\end{pmatrix}
\]
en
\[
\begin{pmatrix}
1 & -1 & 0 \\
0 & 1 & -1 \\
0 & 0 & 1 
\end{pmatrix}
\cdot
X
=
\begin{pmatrix}
2\\2\\2
\end{pmatrix}
\]
heeft als oplossingsverzameling:
\[
V=\{(6,4,2)\}
\]

\subsection*{Extra (oefenzitting)}


\section{Opdrachten}
\subsection*{opdracht 1.2}
Om op een matrix een ERO uit te voeren, berekenen we eigenlijk de vermenigvuldiging van de matrix met de corresponderende elementaire matrix. Dus: $M' = M \cdot E$. Om de ERO om te keren vermenigvuldigen we $M'$ met een matrix $E^{-1}$ zodat $M'\cdot E^{-1} = M \cdot I = M$.
Om aan te tonen dat alle elementaire rijoperaties inverteerbaar zijn, tonen we het aan voor elk van de EROs. Elk van de EROs komt overeen met een elementaire matrix (zie p 36). We bewijzen dat deze inverteerbaar zijn door de de inverse te construeren.
\subsubsection*{$R_i\rightarrow \lambda R_i$}
\[
E=
\begin{pmatrix}
1 & 0 & 0 & \cdots & 0 & 0\\
0 & 1 & 0 & \cdots & 0 & 0\\
\vdots & \vdots & \ddots & \vdots& & \vdots\\
0 & 0 & \cdots & \lambda & \cdots & 0\\
\vdots & \vdots & & \vdots& \ddots & \vdots\\
0 & 0 & \cdots & 0 & \cdots &1
\end{pmatrix}
\]
met de $\lambda$ op rij $i$.\\
De inverse hiervan is
\[
E^{-1}=
\begin{pmatrix}
1 & 0 & 0 & \cdots & 0 & 0\\
0 & 1 & 0 & \cdots & 0 & 0\\
\vdots & \vdots & \ddots & \vdots& & \vdots\\
0 & 0 & \cdots & \frac{1}{\lambda} & \cdots & 0\\
\vdots & \vdots & & \vdots& \ddots & \vdots\\
0 & 0 & \cdots & 0 & \cdots &1
\end{pmatrix}
\]

\subsubsection*{$R_i \leftrightarrow R_j$}
\[
E=
\begin{pmatrix}
1 & 0 & \cdots & 0 & \cdots & 0 & \cdots & 0\\
0 & 1 & \cdots & 0 & \cdots & 0 & \cdots & 0\\
\vdots & \vdots & \ddots & \vdots& & \vdots & &\vdots\\
0 & 0 & \cdots & 0 & \cdots & 1 & \cdots & 0\\
\vdots & \vdots & & \vdots& \ddots & \vdots & &\vdots\\
0 & 0 & \cdots & 1 & \cdots & 0 & \cdots & 0\\
\vdots & \vdots & & \vdots & & \vdots & \ddots & \vdots\\
0 & 0 & \cdots & 0 & \cdots & 0 & \cdots & 0
\end{pmatrix}
\]
met de eerste $1$, niet op de hoofddiagonaal, op rij $i$ en de tweede op rij $j$.\\
De inverse hiervan is $E^{-1} = E$

\subsubsection*{$R_i \rightarrow R_i \lambda R_j$}
\[
E=
\begin{pmatrix}
1 & 0 & \cdots & 0 & \cdots & 0 & \cdots & 0\\
0 & 1 & \cdots & 0 & \cdots & 0 & \cdots & 0\\
\vdots & \vdots & \ddots & \vdots& & \vdots & &\vdots\\
0 & 0 & \cdots & 1 & \cdots & \lambda & \cdots & 0\\
\vdots & \vdots & & \vdots& \ddots & \vdots & &\vdots\\
0 & 0 & \cdots & 0 & \cdots & 1 & \cdots & 0\\
\vdots & \vdots & & \vdots & & \vdots & \ddots & \vdots\\
0 & 0 & \cdots & 0 & \cdots & 0 & \cdots & 0
\end{pmatrix}
\]
met de $\lambda$ op rij $i$, kolom $j$.\\
De inverse hiervan is 
\[
E^{-1}=
\begin{pmatrix}
1 & 0 & \cdots & 0 & \cdots & 0 & \cdots & 0\\
0 & 1 & \cdots & 0 & \cdots & 0 & \cdots & 0\\
\vdots & \vdots & \ddots & \vdots& & \vdots & &\vdots\\
0 & 0 & \cdots & 1 & \cdots & -\lambda & \cdots & 0\\
\vdots & \vdots & & \vdots& \ddots & \vdots & &\vdots\\
0 & 0 & \cdots & 0 & \cdots & 1 & \cdots & 0\\
\vdots & \vdots & & \vdots & & \vdots & \ddots & \vdots\\
0 & 0 & \cdots & 0 & \cdots & 0 & \cdots & 0
\end{pmatrix}
\]

\subsection*{opdracht 1.23}
\subsubsection*{1.}
\[
AA^T \ en \ A + A^T zijn \ symmetrisch:
\]
\[
AA^T = (A\cdot A^T)^T
\]
$$eig: \ (A\cdot B)^T = B^T \cdot A^T $$
$$AA^T = (A^T)^T \cdot A^T$$
$$AA^T = A \cdot A^T$$
\\
$$A +A^T = (A +A^T)^T$$
$$eig: (A+B)^T = A^T + B^T$$
$$A +A^T = A^T + (A^T)^T$$
$$A +A^T = A^T + A$$
$$A +A^T = A + A^T$$
\subsubsection*{2.}
$$A - A^T \ is \ scheefsymmetrisch:$$
$$A - A^T = -(A+(-A^T))^T$$
$$A - A^T = -(A^T + (-A^T)^T)$$
$$A - A^T = -A^T + A$$
\subsubsection*{3.}
$$ A =  \frac{1}{2}(A + A^T) + \frac{1}{2}(A - A^T)$$
$$ A = \frac{1}{2}(A + A^T + A - A^T)$$
$$ A = \frac{1}{2}(2\cdot A)$$
$$ A = A$$
\subsubsection*{4.}
A is een vierkant matrix.\\
Stel A = X + Y waarbij X symmertrisch is en Y scheefsymmetrisch.\\
Dat is: $A^T$ = A en $B^T$ = -B.\\
$$A^T = X^T + Y^T$$
of: $$A^T = X - Y$$
$$A + A^T = X + Y + X - Y = 2X$$
$$A - A^T = X + Y - X + Y = 2Y$$
Dus:
$$X = \frac{A + A^T}{2}$$
$$Y = \frac{A - A^T}{2}$$
$$A = \frac{A + A^T}{2} + \frac{A - A^T}{2}$$
\\
$$
A = \begin{pmatrix}
1 & 3 & 4\\
0 & 2 & -1\\
0 & 0 & 3
\end{pmatrix}
$$
$$
A = \begin{pmatrix}
1 & \frac{3}{2} & 2\\
\frac{3}{2} & 2 & -\frac{1}{2}\\
2 & -\frac{1}{2} & 3
\end{pmatrix}
+
\begin{pmatrix}
0 & \frac{3}{2} & 2\\
-\frac{3}{2} & 0 & -\frac{1}{2}\\
-2 & -\frac{1}{2} & 0
\end{pmatrix}
$$
\subsection*{opdracht 1.25}
\[
Tr(A) = \sum^{n}_{i=1}(A)_{ii}
\]
Bij vierkante matrices:
\[
(A\cdot B)_{ij} = \sum^{n}_{j=1}(A)_{ij}(B)_{ji}
\]
Nu weten we dat:
\[
Tr(A\cdot B) =\sum^{n}_{i=1}(A\cdot B)_{ii}=\sum^{n}_{i=1}\sum^{n}_{j=1}(A)_{ij}(B)_{ji}
\]
en
\[
Tr(B\cdot A) =\sum^{n}_{i=1}(B\cdot A)_{ii}=\sum^{n}_{i=1}\sum^{n}_{j=1}(B)_{ij}(A)_{ji}
\]
We zien nu dat
\[
Tr(A\cdot B) = Tr(B\cdot A)
\]

\subsection*{opdracht 1.33}
\end{document}