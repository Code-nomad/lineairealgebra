\documentclass[10pt,a4paper]{article}

\usepackage[english]{babel}
\usepackage{amsmath}
\usepackage{amsfonts}
\usepackage{amssymb}
\usepackage{amsthm}
\makeatletter
\renewcommand*\env@matrix[1][*\c@MaxMatrixCols c]{%
  \hskip -\arraycolsep
  \let\@ifnextchar\new@ifnextchar
  \array{#1}}
\makeatother
\title{Oplossingen Lineaire Algebra 2013}
\author{TODO}


\begin{document}
\maketitle
\pagebreak
\tableofcontents
\pagebreak
\section*{Oefeningen 3.6}

\section*{Bewijzen}
\subsection*{Lemma 3.7 Correct???}
$$v+x = w + x \Rightarrow v=w$$
Uit definitie 3.2.4 volgt dat elke vector een tegengesteld element heeft. Dus ook voor $x$ en dit element tellen we op aan beide kanten van de gelijkheid:
$$(v+x)+x' = (w + x) + x'$$
associativiteit
\[
v+(x+x') = w + (x + x')
\]
neutraal element
\[
v + 0 = w + 0
\]
\[
v = w
\]

\subsection*{Lemma 3.8 punt 3}
\[(-\lambda)v = -(\lambda v) = \lambda(-v)\]
\[stel:w=0\]
\[(\lambda)(v+w) = \lambda(v) + \lambda(w) = \lambda(v) + \lambda(0) = \lambda(v) = (\lambda v) \]
\[\text{In punt 2 werd gesteld dat het tegengestelde element van v = -v}\]
\[Dus: -\lambda v = tegengstelde \cdot \lambda v\]
\[(-\lambda)v = -(\lambda v) = \lambda(-v) \]


\section*{Opdrachten}
\end{document}