\documentclass[10pt,a4paper]{article}

\usepackage[english]{babel}
\usepackage{amsmath}
\usepackage{amsfonts}
\usepackage{amssymb}
\usepackage{amsthm}
\makeatletter
\renewcommand*\env@matrix[1][*\c@MaxMatrixCols c]{%
  \hskip -\arraycolsep
  \let\@ifnextchar\new@ifnextchar
  \array{#1}}
\makeatother
\title{Oplossingen Lineaire Algebra 2013}
\author{TODO}


\begin{document}
\maketitle
\pagebreak
\tableofcontents
\pagebreak
\section*{Oefeningen 3.6}

\subsection*{oef 1}
\subsubsection*{b)}
distributiviteit-2 geldt niet.
\[
\lambda_1 + \lambda_2v = \lambda_1v = \lambda_2v
\]
en niet
\[
(\lambda_1 + \lambda_2)v = \lambda_1v+\lambda_2v
\]

\subsection*{oef 2}
\subsubsection*{f)}
$W_6$ is geen deelruimte van $R^{3\times 3}$ want de volgende eigenschap geldt niet: (zie p 94)
\[
\forall x,y\in W_6,\;\forall r,s\in\mathbb{R}:\; r\cdot x+s\cdot y\in U
\]
Tegenvoorbeeld:
stel $x=y$, $r=1$ en $s=-1$.\\
Concreet:
\[
x = I_3
\]
\subsubsection*{j}
$W_{10}$ is geen deelruimte van $R\rightarrow R$ want het neutraal element van $R\rightarrow R$  zit niet in $W_{10}$.

\subsection*{Oefenzitting 5}
\subsubsection*{1.3}
Tegengesteld element:
\[
\forall v \in V:\;\exists v' \in V:\; v+v'=v'+v=0
\]
Stel $g'$ is het tegengesteld element van $g$.
Noteer het neutraal element als $\odot$.
\[
g' = (-1)\bullet g
\]
\begin{proof}
\[
g\textbf{+}g' = g\textbf{+} (-1)\bullet g
\]
We evalueren dit in een willekeurige $x \in R$:
\[ 
(g\textbf{+} (-1)\bullet g)(x) \overset{def\;1}{=} g(x) -g(x) = 0 = \odot
\]
Vanwege de commutativiteit geldt ook:
\[
g(x) -g(x) = -g(x) + g(x) = 0 = \odot
\]
\end{proof}
\subsubsection*{1.4}
Stel $f:\mathbb{R} \rightarrow \mathbb{R}: x \mapsto f(x)$ en $g:\mathbb{R} \rightarrow \mathbb{R}: x \mapsto g(x)$, dan geldt voor iedere $x \in \mathbb{R}$ dat
\[ (\lambda \bullet (f \boldsymbol{+} g))(x) 
    = \lambda (f \boldsymbol{+} g)(x) 
    = \lambda (f(x) + g(x))
    = \lambda \cdot f(x) + \lambda \cdot g(x)
    = (\lambda \bullet f)(x) + (\lambda \bullet g)(x)
    = (\lambda \bullet f \boldsymbol{+} \lambda \bullet g)(x)\]
en dus $\lambda \bullet (f \boldsymbol{+} g) = \lambda \bullet f \boldsymbol{+} \lambda \bullet g $

\section*{Bewijzen}
\subsection*{Lemma 3.7}
$$v+x = w + x \Rightarrow v=w$$
Uit definitie 3.2.4 volgt dat elke vector een tegengesteld element heeft. Dus ook voor $x$ en dit element tellen we op aan beide kanten van de gelijkheid:
$$(v+x)+x' = (w + x) + x'$$
associativiteit
\[
v+(x+x') = w + (x + x')
\]
neutraal element
\[
v + 0 = w + 0
\]
\[
v = w
\]

\subsection*{Lemma 3.8 punt 3}
\[(-\lambda)v = -(\lambda v) = \lambda(-v)\]
\[stel:w=0\]
\[(\lambda)(v+w) = \lambda(v) + \lambda(w) = \lambda(v) + \lambda(0) = \lambda(v) = (\lambda v) \]
\[\text{In punt 2 werd gesteld dat het tegengestelde element van v = -v}\]
\[Dus: -\lambda v = tegengstelde \cdot \lambda v\]
\[(-\lambda)v = -(\lambda v) = \lambda(-v) \]


\section*{Opdrachten}
\end{document}