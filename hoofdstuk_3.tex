\documentclass[10pt,a4paper]{article}

\usepackage{amsmath}
\usepackage{amsfonts}
\usepackage{amssymb}
\usepackage{amsthm}
\makeatletter
\renewcommand*\env@matrix[1][*\c@MaxMatrixCols c]{%
  \hskip -\arraycolsep
  \let\@ifnextchar\new@ifnextchar
  \array{#1}}
\makeatother
\title{Oplossingen Lineaire Algebra 2013}
\author{TODO}


\begin{document}
\maketitle
\pagebreak
\tableofcontents
\pagebreak
\section*{Oefeningen 3.6}

\subsection*{oef 1}
\subsubsection*{b)}
distributiviteit-2 geldt niet.
\[
(\lambda_1 + \lambda_2)v = \lambda_1v = \lambda_2v
\]
en niet
\[
(\lambda_1 + \lambda_2)v = \lambda_1v+\lambda_2v
\]

\subsection*{oef 2}
\subsubsection*{b)}
$W_2$ is geen deelruimte van $\mathbb{R}^{3}$.\\
\underline{Tegenvoorbeeld:}\\
$x = (3, 4, 5) \in W_2$ en $y = (2, 2, \sqrt{8}) \in W_2$ maar $x + y = (5, 6, 5 + \sqrt{8}) \not \in W_2$

\subsubsection*{e) }
waar
*) $0 \in W_5$ \\
*) $A,B \in W_5 \longmapsto A+B \in W_5$ \\
$ \sum\limits_{i=1,j=1}^a (A+B)_{ij} = \sum\limits_{i=1,j=1}^a A_{ij} + \sum\limits_{i=1,j=1}^a B_{ij} = 0+0 = 0 $ \\
*) $ A \in W_5, \lambda \in R \longmapsto \lambda A \in W_5$ \\
$ \sum\limits_{i=1,j=1}^a (\lambda A)_{ij} = \lambda \sum\limits_{i=1,j=1}^a A_{ij} = \lambda 0 = 0$


\subsubsection*{f)}
$W_6$ is geen deelruimte van $R^{3\times 3}$ want de volgende eigenschap geldt niet: (zie p 94)
\[
\forall x,y\in W_6,\;\forall r,s\in\mathbb{R}:\; r\cdot x+s\cdot y\in U
\]
Tegenvoorbeeld:
stel $x=y$, $r=1$ en $s=-1$.\\
Concreet:
\[
x = I_3
\]
\subsubsection*{j)}
$W_{10}$ is geen deelruimte van $R\rightarrow R$ want het neutraal element van $R\rightarrow R$  zit niet in $W_{10}$.
\subsubsection*{l)}
Neem $f, g \in W_{12}$ en $\lambda, \mu \in \mathbb{R}$ willekeurig.\\
Omdat $f$ en $g$ integreerbaar zijn, is $\lambda f + \mu g$ integreerbaar.
\begin{align*}
  \int^1_0 (\lambda f + \mu g)(x)dx &= \int^1_0(\lambda f(x) + \mu g(x))dx\\
  &= \int^1_0 \lambda f(x)dx + \int^1_0 \mu g(x)dx\\
  &= \lambda \int^1_0 f(x)dx + \mu \int^1_0 g(x)dx = \lambda \cdot 0 + \mu \cdot 0 = 0
\end{align*}
$W_{12}$ is dus een deelruimte van $\mathbb{R} \rightarrow \mathbb{R}$.



\subsection*{oef 3}
\subsubsection*{Optelling}
\underline{Associativiteit:}\\
Stel $r_1$, $r_2$, $r_3$ willekeurige rijen met differentievergelijking $x_{n + 2} = x_{n + 1} + x_n$.\\
Dan geldt
\begin{align*}
    (r_1 + r_2) + r_3 &= (a_1 + b_1) + c_1 , (a_2 + b_2) + c_2 , ... , (a_n + b_n)+ c_n\\
    &= a_1 + b_1 + c_1, a_2 + b_2 + c_2, ... , a_n + b_n + c_n\\
    &= a_1 + (b_1 + c_1), a_2 + (b_2 + c_2), ..., a_n + (b_n + c_n)
\end{align*}
\underline{Neutraal element:}\\
We nemen als neutraal element de nulrij 0, met $x_1 = 0, x_2 = 0$ . 
Door de differentievergelijking is ieder volgend element ook nul.\\
Hierdoor geldt dan voor een willekeurige rij r
\begin{align*}
  r + 0 &= a_1 + 0, a_2 + 0, ..., a_n + 0\\
  &= a_1, a_2, ..., a_n\\
  &= r\\
  &= 0 + a_1, 0 + a_2, ..., 0 + a_n\\
  &= 0 + r
\end{align*}
\underline{Tegengesteld element:}\\
Als er een rij $r$ bestaat met $x_1 = a_1$, $x_2 = a_2$, dan kiezen we als tegengesteld element de rij $r'$ met $x_1 = -a_1$ en $x_2 = -a_2$.\\
Hierdoor geldt dan
\begin{align*}
    r + r' &= a_1 - a_1, a_2 - a_2, ...,a_n - a_n\\
    &= 0, 0, ..., 0\\
    &= -a_1 + a_1, -a_2 + a_2, ..., -a_n + a_n\\
    &= r' + r
\end{align*}
\underline{Commutativiteit:}\\
Stel $r_1$, $r_2$ willekeurige rijen met differentievergelijking $x_{n + 2} = x_{n + 1} + x_n$.\\
Dan geldt
\begin{align*}
    r_1 + r_2 &= a_1 + b_1, a_2 + b_2, ..., a_n + b_n\\
    &= b_1 + a_1, b_2 + a_2, ..., b_n + a_n\\
    &= r_2 + r_1
\end{align*}
\subsubsection*{Scalaire vermenigvuldiging}
\underline{Distributiviteit-1:}\\
Stel $r_1$, $r_2$ willekeurige rijen met differentievergelijking $x_{n + 2} = x_{n + 1} + x_n$, en een willekeurige $\lambda \in \mathbb{R}$.
\begin{align*}
    \lambda \cdot(r_1 + r_2) &= \lambda (a_1 + b_1), \lambda (a_2 + b_2), ..., \lambda (a_n + b_n)\\
    &= \lambda a_1 + \lambda b_1, \lambda a_2 + \lambda b_2, ..., \lambda a_n + \lambda b_n\\
    &= \lambda \cdot r_1 + \lambda \cdot r_2
\end{align*}
\underline{Distributiviteit-2:}\\
Stel $r$ een willekeurige rij met differentievergelijking $x_{n + 2} = x_{n + 1} + x_n$, en een willekeurige $\lambda_1 , \lambda_2 \in \mathbb{R}$.
\begin{align*}
    (\lambda_1 + \lambda_2) \cdot r &= (\lambda_1 + \lambda_2) a_1, (\lambda_1 + \lambda_2) a_2, ..., (\lambda_1 + \lambda_2) a_n\\
    &= \lambda_1 a_1 + \lambda_2 a_1, \lambda_1 a_2 + \lambda_2 a_2, ..., \lambda_1 a_n + \lambda_2 a_n\\
    &= \lambda_1 \cdot r + \lambda_2 \cdot r
\end{align*}
\underline{Gemengde associativiteit:}\\
Stel $r$ een willekeurige rij met differentievergelijking $x_{n + 2} = x_{n + 1} + x_n$, en een willekeurige $\lambda_1 , \lambda_2 \in \mathbb{R}$.
\begin{align*}
    \lambda_1 \cdot (\lambda_2 \cdot r) &= \lambda_1 \cdot (\lambda_2 a_1), \lambda_1 \cdot (\lambda_2 a_2), ..., \lambda_1 \cdot (\lambda_2 a_n)\\
    &= (\lambda_1 \cdot \lambda_2) a_1, (\lambda_1 \cdot \lambda_2) a_2, ..., (\lambda_1 \cdot \lambda_2) a_n\\
    &= (\lambda_1 \cdot \lambda_2) \cdot r
\end{align*}
\underline{Co\"effici\"ent 1:}\\
Stel $r$ een willekeurige rij met differentievergelijking $x_{n + 2} = x_{n + 1} + x_n$.
\begin{align*}
     1 \cdot r &= 1 \cdot a_1, 1 \cdot a_2, ..., 1 \cdot a_n\\
     &= a_1, a_2, ..., a_n\\
     &= r
\end{align*}

\subsection*{oef 6}
\subsubsection*{Gegeven}
$U$ en $W$ zijn deelruimten van een vectorruimte $(\mathbb{R},V,+)$.
\subsubsection*{Te Bewijzen}
\[
U \cup W \text{ is een deelruimte van }V \Leftrightarrow U \subset W \vee W \subset U
\]
\subsubsection*{Bewijs}
\begin{proof} Samengesteld bewijs.
\\Deel 1: $(\Rightarrow)$
\\ Te bewijzen:
\[
U \cup W \text{ is een deelruimte van }V \Rightarrow U \subset W \vee W \subset U
\]
Bewijs uit het ongerijmde.
\\Stel dat $\neg(U \subset W \wedge W \subset U)$.
\\Dit is equivalent met $U \not\subset W \wedge W \not\subset U$.
Vanuit de definitie van deelverzameling betekent deze bewering het volgende.
\[
(\exists w': w' \in W \wedge w' \not\in U) \wedge (\exists u': u' \in U \wedge u' \not\in W) 
\]
We moeten nu tot een contradictie komen met "$U\cup W \text{ is een deelruimte van }V$".
$U\cup W \text{ is een deelruimte van }V$ betekent vanuit de definitie het volgende.
\[
\forall w, u \in ĉ,\; \forall \lambda_1 , \lambda_2 \in \mathbb{R}: \lambda_1\cdot u + \lambda_2\cdot w \in U\cup W
\]
Aangezien deze bewering geldt voor elke $w,i \in U\cup W$, geldt ze ook voor $w'$ en $u'$.
We weten nu het volgende.
\[
\forall \lambda_1 , \lambda_2 \in \mathbb{R}: \lambda_1\cdot u' + \lambda_2\cdot w' \in U\cup W
\]
met $w' \in W \wedge w' \not\in U$ en $u' \in U \wedge u' \not\in W$.\\
Als dit geldt voor elke $\lambda_1$ en $\lambda_2$, geldt het ook voor $\lambda_1= \lambda_2 = 1$ en geldt dus het volgende.
\[
u' + w' \in U \cup W
\]
We weten dat dit niet klopt, want stel als $u' + w' \in U \cup W$ en $u' \in U$, dan $w' \in U$. Contradictie.
\\Deel 2: $(\Leftarrow)$
\\ Te bewijzen:
\[
U \cup W \text{ is een deelruimte van }V \Leftarrow U \subset W \vee W \subset U
\]
Direct bewijs.\\
Stel $U \subset W$. (Het andere geval is volledig analoog)\\
Als $U \subset W$, dan is $U\cup W = W$. Van $W$ is gegeven dat het een deelruimte is van $W$. Hiermee is bewezen dat $U\cup W$ een deelruimte is van $V$.6
\end{proof}

\subsection*{oef 7}
Om te zien of een vector $d$ tot een deelruimte behoort opgespannen door vectoren $v_1$ en $v_2$ en $v_3$ moeten we zien of er een lineaire combinatie bestaat zodat  $$a\cdot v_1 + b\cdot v_2 + c\cdot v_3 = d$$
Dus:
$$a(2,-1,3,2) + b(-1,1,1,-3) + c(1,1,9,-5) \overset{?}{=} (3,-1,0,-1)$$
We gieten dit in een matrix:
$$
\begin{pmatrix}
2 & -1 & 1 &3\\
-1 & 1 & 1 & -1\\
3 & 1 &9 & 0\\
2 & -3 &-5& -1
\end{pmatrix}
$$
Door rijreductie verkrijgen we volgende matrix:
$$
\begin{pmatrix}
1&0&2&0\\
0&1&3&0\\
0&0&0&1\\
0&0&0&0
\end{pmatrix}
$$
Dit is een ongeldige matrix en dus bestaat er geen lineaire combinatie zodat de vector $d$ gevormd wordt. De vector behoort dus niet tot de deelruimte.

\subsection*{oef 8}
We willen $p(X)$ opschrijven als lineaire combinatie van $p_1(X), p_2(X), p_3(X)$, dus $p(X) = \lambda_1 p_1(X) + \lambda_2 p_2(X) + \lambda_2 p_3(X)$.
Hiervoor lossen we het volgende stelsel op:
\[
\left\{
\begin{array}{l l l l l}
  -1 &= &\lambda_1 &+ 2 \lambda_2 &+ 3 \lambda_3\\
  -3 &= &2 \lambda_1 &+ 5 \lambda_2 &+ 8 \lambda_3\\
  3 &= &\lambda_1 &&- 2 \lambda_3
\end{array}
\right.
\]
Dit vullen we in in een matrix, die we hierna oplossen m.b.v. rijreductie.
\[
\begin{pmatrix}[ccc|c]
  1 & 2 & 3 & -1\\
  2 & 5 & 8 & -3\\
  1 & 0 & -2 & 3
\end{pmatrix}
\rightarrow
\begin{pmatrix}[ccc|c]
  1 & 0 & 0 & -1\\
  0 & 1 & 0 & 3\\
  0 & 0 & 1 & -2
\end{pmatrix}
\]
We resulteren dus dat $p(X) = - p_1(X) + 3 p_2(X) - 2 p_3(X)$.


\subsection*{oef 9}
$$f(x)+g(x)-h(x)+0\cdot exp(x) = 0$$
$$1 - 1 + 0 = 0$$
Dit is een lineaire combinatie waarbij niet alle co\"effi\"enten nul zijn en toch de nulvector als oplossing geeft. Hierdoor is
$\{f,g,h,exp\}$ geen lineair onafhankelijke deelruimte.
\subsection*{oef 12}
Moest dit waar zijn dan mag er uit de verzameling $\{ e_1-e_2,e_2-e_3,\dots ,e_{n-1}-e_n,e_n-e_1\}$ geen enkel element bestaan dat een combinatie is van de andere elementen.
$$
\text{We zien nu voor}\ 1 \leq i \leq n:
$$
$$ 
(e_1-e_2)+(e_2-e_3)+\dots +(e_{i-1}-e_i)+(e_{i+1}-e_{i+2})+\dots + (e_n - e_1)
$$
$$
= (\not e_1-\not e_2+\not e_2-\not e_3+\dots +\not e_{i-1}-e_i+e_{i+1}-\not e_{i+2}+\dots + \not e_n - \not e_1)
$$
$$
= -e_i + e_{i+1}
$$
Dus voor een willkeurige vector:
$$
e_i - e_{i+1} = -\left((e_1-e_2)+(e_2-e_3)+\dots +(e_{i-1}-e_i)+(e_{i+1}-e_{i+2})+\dots + (e_n - e_1)\right)
$$
kunnen we dus telkens een combinatie vinden uit de andere vectoren, hierdoor is het geen vrij deel. 
\subsection*{oef 13}
Hiervoor kunnen we aantonen dat $\{1+X,1+X^2,X+X^2\}$ minimaal voortbrengend is, hiervoor laten we zien dat elke vector uit  $(\mathbb{R},\mathbb{R}[X]_{\leq 2},+)$ een lineaire combinatie is van de basisvectoren:
$$aX^2 + bX + c = \lambda_{1} (1+X) + \lambda_{2} (1+X^2) + \lambda_{3} (X+X^2)$$
Als we deze vergelijking uitwerking uitwerken krijgen we:
$$a = \lambda_{1} + \lambda_2$$
$$b = \lambda_1 + \lambda_3$$
$$c = \lambda_1 + \lambda_2$$
Dit is een stelsel met exact 1 oplossing. Hierme tonen we aan dat het stelsel voortbrengend is en minimaal voortbrengend. Moesten we nog extra vergelijkingen er aan toevoegen zouden we redundantie invoeren of het stelsel onoplosbaar maken.
\\
\\
De co\"ordinaten van de vector $4 -3X + X^2$ kunnen we berekenen ze in een stelsel te gieten:
$$
\begin{pmatrix}
1&1&0&4\\
1&0&1&-3\\
0&1&1&1
\end{pmatrix}
$$
Via rijoperaties verkrijgen we dan:
$$
\begin{pmatrix}
1&0&0&0\\
0&1&0&4\\
0&0&1&-3
\end{pmatrix}
$$
Dit komt overeen met de co\"ordinaten $(0,4,-3)$.
\\
\\
Het co\"ordinatenstel $(2,-3,1)$ komt dan weer overeen met:
$$
2\cdot (1+X) -3 \cdot (1+X^2)+ 1\cdot (X+X^2)= -1 +3X-2X^2
$$
\section*{Oefenzitting 5}
\subsection*{oef 1}
\subsubsection*{1.1}
\underline{+} is de som van de functies\\
Neem $f,g,h \in \mathbb{R}^{\mathbb{R}}$ willekeurig.\\
Neem $x \in \mathbb{R}$ willekeurig dan is:
\begin{align*}
((f\underline{+}g)+h)(x) = (f\underline{+} g)(x) +h(x)\\
=(f(x)+g(x))+h(x)\\
=f(x)+(g(x)+h(x))\\
=f(x)+(g\underline{+}h)(x)\\
=(f\underline{+}(g\underline{+}h)(x))\\
\end{align*}
Dus:
\begin{align*}
(f\underline{+}g)\underline{+}h = f\underline{+}(g\underline{+}h)
\end{align*}
\subsubsection*{1.3}
Tegengesteld element:
\[
\forall v \in V:\;\exists v' \in V:\; v+v'=v'+v=0
\]
Stel $g'$ is het tegengesteld element van $g$.
Noteer het neutraal element als $\odot$.
\[
g' = (-1)\bullet g
\]
\begin{proof}
\[
g\textbf{+}g' = g\textbf{+} (-1)\bullet g
\]
We evalueren dit in een willekeurige $x \in R$:
\[ 
(g\textbf{+} (-1)\bullet g)(x) \overset{def\;1}{=} g(x) -g(x) = 0 = \odot
\]
Vanwege de commutativiteit geldt ook:
\[
g(x) -g(x) = -g(x) + g(x) = 0 = \odot
\]
\end{proof}
\subsubsection*{1.4}
Stel $f:\mathbb{R} \rightarrow \mathbb{R}: x \mapsto f(x)$ en $g:\mathbb{R} \rightarrow \mathbb{R}: x \mapsto g(x)$, dan geldt voor iedere $x \in \mathbb{R}$ dat
\[ (\lambda \bullet (f \boldsymbol{+} g))(x) 
    = \lambda (f \boldsymbol{+} g)(x) 
    = \lambda (f(x) + g(x))
    = \lambda \cdot f(x) + \lambda \cdot g(x)
    = (\lambda \bullet f)(x) + (\lambda \bullet g)(x)
    = (\lambda \bullet f \boldsymbol{+} \lambda \bullet g)(x)\]
en dus $\lambda \bullet (f \boldsymbol{+} g) = \lambda \bullet f \boldsymbol{+} \lambda \bullet g $
\section*{Bewijzen}
\subsection*{Lemma 3.7}

$$v+x = w + x \Rightarrow v=w$$
Neem $v,w,x \in V$ en we nemen aan dat $v+x = w+x$\\
\begin{align*}
v = v + 0 \tag{neutraal element}\\
= v + (x - x) \tag{invers element}\\
=(v+x) -x \tag{communativiteit}\\
=(w+x) -x \tag{dit hebben we aangenomen}\\
= w +(x-x) \tag{communitativiteit}\\
= w+0 \tag{invers element}\\
=w \tag{neutraal element}
\end{align*}

\subsection*{Lemma 3.8 punt 3}
\[(-\lambda)v = -(\lambda v) = \lambda(-v)\]
\[stel:w=0\]
\[(\lambda)(v+w) = \lambda(v) + \lambda(w) = \lambda(v) + \lambda(0) = \lambda(v) = (\lambda v) \]
\[\text{In punt 2 werd gesteld dat het tegengestelde element van v = -v}\]
\[Dus: -\lambda v = tegengstelde \cdot \lambda v\]
\[(-\lambda)v = -(\lambda v) = \lambda(-v) \]

\section*{Opdrachten}
\subsection*{3.9}
Neem aan dat $\lambda \cdot v = 0$ voor $\lambda \in \mathbb{R}$ en $v \in V$.\\
Er zijn dan twee mogelijkheden:\\
\begin{enumerate}
\item $\lambda = 0$ dan is het in orde.
\item $\lambda \neq 0$
\begin{align*}
v = 1\cdot v \tag{co\"efficient}
\\
= \frac{\lambda}{\lambda}\cdot v
\\
= \frac{1}{\lambda}(\lambda \cdot v) \tag{gemegde associativiteit}
\\
= \frac{1}{\lambda}(0) \tag{lemma 3.8.1}
\\
= 0
\end{align*}

\end{enumerate}
\subsection*{3.21}
Om aan te tonen dat $U_1 + U_2$ deelruimten zijn van $V$ dan moeten we de drie eigenschappen van een deelruimte aantonen:
\begin{enumerate}
\item We moeten aantonen dat $0 \in U_1 + U_2$.\\
We weten dat $U_1$ en $U_2$ deelruimten zijn van $V$.\\
Dus $0 \in U_1$ en $0 \in U_2$ samen met het feit dat $0 = 0 + 0$ volgt dat $0 \in U_1 + U_2$.

\item We moeten aantonen $\forall x,y \in U_1 + U_2$ geldt: $x + y \in U_1 + U_2$.\\
Veronderstel dat twee willekeurige vectoren $x$ en $y \in U_1 + U_2$ bestaan.\\
Dan volgt uit de definitie van $U_1 + U_2$ dat er twee vectoren $x_1$ en $y_1 \in U_1$ en $x_2$ en $y_2 \in U_2$ bestaan zodat $x = x_1 + x_2$ en $y = y_1 + y_2$.\\
Hieruit volgt:
$$x + y = (x_1 + x_2) + (y_1 + y_2) = (x_1 + y_1) + (x_2 + y_2)$$
Omdat $x_1, y_1 \in U_1$ en $U_1$ is een deelruimte van V, $x_1 + y_1 \in U_1$.\\
Omdat $x_2, y_2 \in U_2$ en $U_2$ is een deelruimte van V, $x_2 + y_2 \in U_2$.\\
Nu hebben we aangetoond dat $x+y$ de som is van een vector in $U_1$ en een vector in $U_2$. 
Dus $x + y \in U_1 + U_2$ door de definitie van $U_1 + U_2$ en aangezien $x$ en $y$ willekeurig waren geldt dit voor alle $x,y \in U_1 + U_2$.

\item We moeten aantonen dat $\forall x \in U_1 + U_2$ en $\forall r \in \mathbb{R}$ geldt: $rx \in U$.

Veronderstel dat een willekeurige $x \in U_1 + U_2$ en $r \in \mathbb{R}$ bestaan. Dan volgt uit de definitie van $U_1 + U_2$ dat er een $x_1 \in U_1$ en $x_2 \in U_2$ bestaan zodat $x = x_1 + x_2$.\\
Hieruit volgt:
$$rx = r(x_1+x_2) = rx_1+rx_2$$
Omdat $x_1 \in U_1$ en $U_1$ is een deelruimte van V, $rx_1 \in U_1$.\\
Omdat $x_2 \in U_2$ en $U_2$ is een deelruimte van V, $rx_2 \in U_2$.\\
Nu hebben we aangetoond dat $rx$ de som is van een vector in $U_1$ en een vector in $U_2$. 
Dus $rx \in U_1 + U_2$ door de definitie van $U_1 + U_2$ en aangezien $x$ en $r$ willekeurig waren geldt dit voor alle $r \in \mathbb{R}$ en $x \in U_1 + U_2$.
\end{enumerate}
\subsection*{3.24}

1) 
Neem \\ 
$$U_1 = \{(x,0,0)\} $$
$$U_2 =  \{ (0,y,0)\} $$
$$U_3 = \{ (x,y,0)\} $$
2) ???

\end{document}