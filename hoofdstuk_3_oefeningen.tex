\documentclass[lineaire_algebra_oplossingen.tex]{subfiles}
\begin{document}

\chapter{Oefeningen Hoofdstuk 3}
\section{Oefeningen 3.6}
\subsection{Oefening 1}
\subsubsection*{a)}
distributiviteit-1 geldt niet.
\[
\lambda((x_1,x_2) + (y_1,y_2)  = \lambda(x_1+y_1-1,x_2+y_2-1) = (\lambda x_1+\lambda y_1-\lambda 1,\lambda x_2+\lambda y_2-\lambda 1)
\]
Terwijl de volgende formule ook geldt, maar deze niet gelijk zijn voor bij voorbeeld $\lambda = 2$.
\[
\lambda(x_1,x_2) + \lambda(y_1,y_2) = (\lambda x_1+\lambda y_1- 1,\lambda x_2+\lambda y_2-1)
\]
\subsubsection*{b)}
distributiviteit-2 geldt niet:\\
Scalaire vermenigvuldiging volgens de gegeven definitie:
$$
(\lambda_1 + \lambda_2)(x,y) = (x,y)
$$
Volgens distributiviteit-2:
$$
(\lambda_1 + \lambda_2)(x,y) = \lambda_1(x,y) + \lambda_2(x,y)
$$
$$
= (x,y) + (x,y)
$$
Optelling volgens de gegeven definitie:
$$
= (x+x, y+y)
$$

\subsubsection*{c)}
optelling is niet commutatief.
\[
(x_1,y_1) + (x_2,y_2) = (x_1+2x_2,y_1+2y_2)
\]
\[ \neq \]
\[
(x_2,y_2) + (x_1,y_1) = (x_2+2x_1,y_2+2y_1)
\]

\subsubsection*{d)}
Dit lijkt te kloppen.

\subsection{Oefening 2}
\subsubsection*{a)}
Deze verzameling vormt de unie van de drie vlakken rond de assen. Dit is geen deelruimte want de optelling is niet intern.\\
\underline{Tegenvoorbeeld:}\\
\[
(1,0,0) + (0,1,1) = (1,1,1) \not \in W_1
\]
\subsubsection*{b)}
$W_2$ is geen deelruimte van $\mathbb{R}^{3}$.\\
\underline{Tegenvoorbeeld:}\\
$x = (3, 4, 5) \in W_2$ en $y = (2, 2, \sqrt{8}) \in W_2$ maar $x + y = (5, 6, 5 + \sqrt{8}) \not \in W_2$

\subsubsection*{c)}
$W_3$ is geen deelruimte van $R^3$ want de lineaire combinatie is niet intern.\\
Tegenvoorbeeld: Stel $\lambda = \pi$ en $v = (1,0,0) \in \mathbb{Q}$.
\[
\lambda v = \pi (1,0,0) = (\pi,0,0)
\]
$\pi$ is geen element van $\mathbb{Q}$.

\subsubsection*{d)}
$W_4$ is geen deelruimte van $R^3$. Intu\"itief is dit makkelijk te zien omdat $W_4$ precies een sfeer beschrijft.
Tegenvoorbeeld: Stel $v_1 = (1,0,0)$ en $v_2 =(1,0,0)$.
\[
v_1 + v_2 = (2,0,0) \not \in W_4
\]

\subsubsection*{e)}
waar\\
*) $0 \in W_5$ \\
*) $A,B \in W_5 \longmapsto A+B \in W_5$ \\
$ \sum\limits_{i=1,j=1}^a (A+B)_{ij} = \sum\limits_{i=1,j=1}^a A_{ij} + \sum\limits_{i=1,j=1}^a B_{ij} = 0+0 = 0 $ \\
*) $ A \in W_5, \lambda \in R \longmapsto \lambda A \in W_5$ \\
$ \sum\limits_{i=1,j=1}^a (\lambda A)_{ij} = \lambda \sum\limits_{i=1,j=1}^a A_{ij} = \lambda 0 = 0$


\subsubsection*{f)}
$W_6$ is geen deelruimte van $R^{3}$ want de volgende eigenschap geldt niet: (zie p 94)
\[
\forall x,y\in W_6,\;\forall r,s\in\mathbb{R}:\; r\cdot x+s\cdot y\in U
\]
Tegenvoorbeeld:
stel $x=y$, $r=1$ en $s=-1$.\\
Concreet:
\[
x = I_3
\]

\subsubsection*{g)}
$W_7$ is een deelruimte van $\mathbb{R}^{3x3}$.\\
\begin{proof} 

1) $W_7$ bevat het nulelement. De nulmatrix is immers ook een symmetrische matrix. \\
2) $ Symm(A1) + Symm(A2) = A3 $
	and $A3 = \mathbb{R}^{3x3}$ \\
3) $ \lambda \cdot Symm(A) = A2 $
	and $ A2 = \mathbb{R}^{3x3}$

Er is voldaan aan de drie eigenschappen.

\end{proof}


\subsubsection*{i)}
$W_9$ is geen deelruimte van $R^{3\times 3}$. De optelling is namelijk niet intern.
\[
(0,1,0) + (1,0,0) = (1,1,0) \not \in W_9
\]

\subsubsection*{j)}
$W_{10}$ is geen deelruimte van $R\rightarrow R$ want het neutraal element van $R\rightarrow R$  zit niet in $W_{10}$.
\subsubsection*{l)}
Neem $f, g \in W_{12}$ en $\lambda, \mu \in \mathbb{R}$ willekeurig.\\
Omdat $f$ en $g$ integreerbaar zijn, is $\lambda f + \mu g$ integreerbaar.
\begin{align*}
  \int^1_0 (\lambda f + \mu g)(x)dx &= \int^1_0(\lambda f(x) + \mu g(x))dx\\
  &= \int^1_0 \lambda f(x)dx + \int^1_0 \mu g(x)dx\\
  &= \lambda \int^1_0 f(x)dx + \mu \int^1_0 g(x)dx = \lambda \cdot 0 + \mu \cdot 0 = 0
\end{align*}
$W_{12}$ is dus een deelruimte van $\mathbb{R} \rightarrow \mathbb{R}$.



\subsection{Oefening 3}
\subsubsection*{Optelling}
\underline{Associativiteit:}\\
Stel $r_1$, $r_2$, $r_3$ willekeurige rijen met differentievergelijking $x_{n + 2} = x_{n + 1} + x_n$.\\
Dan geldt
\begin{align*}
    (r_1 + r_2) + r_3 &= (a_1 + b_1) + c_1 , (a_2 + b_2) + c_2 , ... , (a_n + b_n)+ c_n\\
    &= a_1 + b_1 + c_1, a_2 + b_2 + c_2, ... , a_n + b_n + c_n\\
    &= a_1 + (b_1 + c_1), a_2 + (b_2 + c_2), ..., a_n + (b_n + c_n)
\end{align*}
\underline{Neutraal element:}\\
We nemen als neutraal element de nulrij 0, met $x_1 = 0, x_2 = 0$ . 
Door de differentievergelijking is ieder volgend element ook nul.\\
Hierdoor geldt dan voor een willekeurige rij r
\begin{align*}
  r + 0 &= a_1 + 0, a_2 + 0, ..., a_n + 0\\
  &= a_1, a_2, ..., a_n\\
  &= r\\
  &= 0 + a_1, 0 + a_2, ..., 0 + a_n\\
  &= 0 + r
\end{align*}
\underline{Tegengesteld element:}\\
Als er een rij $r$ bestaat met $x_1 = a_1$, $x_2 = a_2$, dan kiezen we als tegengesteld element de rij $r'$ met $x_1 = -a_1$ en $x_2 = -a_2$.\\
Hierdoor geldt dan
\begin{align*}
    r + r' &= a_1 - a_1, a_2 - a_2, ...,a_n - a_n\\
    &= 0, 0, ..., 0\\
    &= -a_1 + a_1, -a_2 + a_2, ..., -a_n + a_n\\
    &= r' + r
\end{align*}
\underline{Commutativiteit:}\\
Stel $r_1$, $r_2$ willekeurige rijen met differentievergelijking $x_{n + 2} = x_{n + 1} + x_n$.\\
Dan geldt
\begin{align*}
    r_1 + r_2 &= a_1 + b_1, a_2 + b_2, ..., a_n + b_n\\
    &= b_1 + a_1, b_2 + a_2, ..., b_n + a_n\\
    &= r_2 + r_1
\end{align*}
\subsubsection*{Scalaire vermenigvuldiging}
\underline{Distributiviteit-1:}\\
Stel $r_1$, $r_2$ willekeurige rijen met differentievergelijking $x_{n + 2} = x_{n + 1} + x_n$, en een willekeurige $\lambda \in \mathbb{R}$.
\begin{align*}
    \lambda \cdot(r_1 + r_2) &= \lambda (a_1 + b_1), \lambda (a_2 + b_2), ..., \lambda (a_n + b_n)\\
    &= \lambda a_1 + \lambda b_1, \lambda a_2 + \lambda b_2, ..., \lambda a_n + \lambda b_n\\
    &= \lambda \cdot r_1 + \lambda \cdot r_2
\end{align*}
\underline{Distributiviteit-2:}\\
Stel $r$ een willekeurige rij met differentievergelijking $x_{n + 2} = x_{n + 1} + x_n$, en een willekeurige $\lambda_1 , \lambda_2 \in \mathbb{R}$.
\begin{align*}
    (\lambda_1 + \lambda_2) \cdot r &= (\lambda_1 + \lambda_2) a_1, (\lambda_1 + \lambda_2) a_2, ..., (\lambda_1 + \lambda_2) a_n\\
    &= \lambda_1 a_1 + \lambda_2 a_1, \lambda_1 a_2 + \lambda_2 a_2, ..., \lambda_1 a_n + \lambda_2 a_n\\
    &= \lambda_1 \cdot r + \lambda_2 \cdot r
\end{align*}
\underline{Gemengde associativiteit:}\\
Stel $r$ een willekeurige rij met differentievergelijking $x_{n + 2} = x_{n + 1} + x_n$, en een willekeurige $\lambda_1 , \lambda_2 \in \mathbb{R}$.
\begin{align*}
    \lambda_1 \cdot (\lambda_2 \cdot r) &= \lambda_1 \cdot (\lambda_2 a_1), \lambda_1 \cdot (\lambda_2 a_2), ..., \lambda_1 \cdot (\lambda_2 a_n)\\
    &= (\lambda_1 \cdot \lambda_2) a_1, (\lambda_1 \cdot \lambda_2) a_2, ..., (\lambda_1 \cdot \lambda_2) a_n\\
    &= (\lambda_1 \cdot \lambda_2) \cdot r
\end{align*}
\underline{Co\"effici\"ent 1:}\\
Stel $r$ een willekeurige rij met differentievergelijking $x_{n + 2} = x_{n + 1} + x_n$.
\begin{align*}
     1 \cdot r &= 1 \cdot a_1, 1 \cdot a_2, ..., 1 \cdot a_n\\
     &= a_1, a_2, ..., a_n\\
     &= r
\end{align*}

\subsection{Oefening 4}
Welke oef uit de definitie van een vectorruimte zijn voldaan? \\
1. De optelling is inwendig en overal bepaald, ja \\
2. De optelling is associatief,	ja \\
3. Er is een neutr element, ja \\
4. Elk element heeft een tegeng element, ja \\
5. De opt is comm, nee \\

1. Distr-1, \\
2. Distr-2, ja \\
3. gemeng asso, ja\\
4. nee

\subsection{Oefening 5}
Nee, $W$ is een deelruimte van $R[X]_{\le n}$ als $n\ge 1$.\\
\textbf{Tegenvoorbeeld: }\\Neem $f_1=x^2+x-3$ en neem $f_2=-x^2+x+3$.
Als we nu $f_1+f_2$ is dit geen veelterm meer van graad 2 maar van graad 1, het is dus geen deelruimte.
\cite{Tijl}



\subsection{Oefening 6}
\subsubsection*{Gegeven}
$U$ en $W$ zijn deelruimten van een vectorruimte $(\mathbb{R},V,+)$.
\subsubsection*{Te Bewijzen}
\[
U \cup W \text{ is een deelruimte van }V \Leftrightarrow U \subset W \vee W \subset U
\]
\subsubsection*{Bewijs}
\begin{proof} Samengesteld bewijs.
\\Deel 1: $(\Rightarrow)$
\\ Te bewijzen:
\[
U \cup W \text{ is een deelruimte van }V \Rightarrow U \subset W \vee W \subset U
\]
Bewijs uit het ongerijmde.
\\Stel dat $\neg(U \subset W \wedge W \subset U)$.
\\Dit is equivalent met $U \not\subset W \wedge W \not\subset U$.
Vanuit de definitie van deelverzameling betekent deze bewering het volgende.
\[
(\exists w': w' \in W \wedge w' \not\in U) \wedge (\exists u': u' \in U \wedge u' \not\in W) 
\]
We moeten nu tot een contradictie komen met ``$U\cup W \text{ is een deelruimte van }V$''.
$U\cup W$ is een deelruimte van $V$ betekent vanuit de definitie het volgende.
\[
\forall w, u \in ĉ,\; \forall \lambda_1 , \lambda_2 \in \mathbb{R}: \lambda_1\cdot u + \lambda_2\cdot w \in U\cup W
\]
Aangezien deze bewering geldt voor elke $w,i \in U\cup W$, geldt ze ook voor $w'$ en $u'$.
We weten nu het volgende.
\[
\forall \lambda_1 , \lambda_2 \in \mathbb{R}: \lambda_1\cdot u' + \lambda_2\cdot w' \in U\cup W
\]
met $w' \in W \wedge w' \not\in U$ en $u' \in U \wedge u' \not\in W$.\\
Als dit geldt voor elke $\lambda_1$ en $\lambda_2$, geldt het ook voor $\lambda_1= \lambda_2 = 1$ en geldt dus het volgende.
\[
u' + w' \in U \cup W
\]
We weten dat dit niet klopt, want stel als $u' + w' \in U \cup W$ en $u' \in U$, dan $w' \in U$. Contradictie.
\\Deel 2: $(\Leftarrow)$
\\ Te bewijzen:
\[
U \cup W \text{ is een deelruimte van }V \Leftarrow U \subset W \vee W \subset U
\]
Direct bewijs.\\
Stel $U \subset W$. (Het andere geval is volledig analoog)\\
Als $U \subset W$, dan is $U\cup W = W$. Van $W$ is gegeven dat het een deelruimte is van $W$. Hiermee is bewezen dat $U\cup W$ een deelruimte is van $V$.6
\end{proof}

\subsection{Oefening 7}
Om te zien of een vector $d$ tot een deelruimte behoort opgespannen door vectoren $v_1$ en $v_2$ en $v_3$ moeten we zien of er een lineaire combinatie bestaat zodat  $$a\cdot v_1 + b\cdot v_2 + c\cdot v_3 = d$$
Dus:
$$a(2,-1,3,2) + b(-1,1,1,-3) + c(1,1,9,-5) \overset{?}{=} (3,-1,0,-1)$$
We gieten dit in een matrix:
$$
\begin{pmatrix}
2 & -1 & 1 &3\\
-1 & 1 & 1 & -1\\
3 & 1 &9 & 0\\
2 & -3 &-5& -1
\end{pmatrix}
$$
Door rijreductie verkrijgen we volgende matrix:
$$
\begin{pmatrix}
1&0&2&0\\
0&1&3&0\\
0&0&0&1\\
0&0&0&0
\end{pmatrix}
$$
Dit is een ongeldige matrix en dus bestaat er geen lineaire combinatie zodat de vector $d$ gevormd wordt. De vector behoort dus niet tot de deelruimte.

\subsection{Oefening 8}
We willen $p(X)$ opschrijven als lineaire combinatie van $p_1(X), p_2(X), p_3(X)$, dus $p(X) = \lambda_1 p_1(X) + \lambda_2 p_2(X) + \lambda_2 p_3(X)$.
Hiervoor lossen we het volgende stelsel op:
\[
\left\{
\begin{array}{l l l l l}
  -1 &= &\lambda_1 &+ 2 \lambda_2 &+ 3 \lambda_3\\
  -3 &= &2 \lambda_1 &+ 5 \lambda_2 &+ 8 \lambda_3\\
  3 &= &\lambda_1 &&- 2 \lambda_3
\end{array}
\right.
\]
Dit vullen we in in een matrix, die we hierna oplossen m.b.v. rijreductie.
\[
\begin{pmatrix}[ccc|c]
  1 & 2 & 3 & -1\\
  2 & 5 & 8 & -3\\
  1 & 0 & -2 & 3
\end{pmatrix}
\rightarrow
\begin{pmatrix}[ccc|c]
  1 & 0 & 0 & -1\\
  0 & 1 & 0 & 3\\
  0 & 0 & 1 & -2
\end{pmatrix}
\]
We resulteren dus dat $p(X) = - p_1(X) + 3 p_2(X) - 2 p_3(X)$.


\subsection{Oefening 9}
$$f(x)+g(x)-h(x)+0\cdot exp(x) = 0$$
$$1 - 1 + 0 = 0$$
Dit is een lineaire combinatie waarbij niet alle co\"effici\"enten nul zijn en toch de nulvector als oplossing geeft. Hierdoor is
$\{f,g,h,exp\}$ geen lineair onafhankelijke deelruimte.

\subsection{Oefening 11}
\[v_1\cdots v_n, v_{n+1}\text{zijn lineair afhankelijk} \Leftrightarrow \text{v is een lineaire combinatie van } v_1\cdots v_n\]
Als
\[\sum_{i=1}^{n}{\lambda_{i1}v_i}=0\]
dan moeten alle $\lambda_{i1}$'s=0.\\
Als
\[\sum_{i=1}^{n+1}{\lambda_{i2}v_i}=0\]
Dan is het mogelijk dat niet alle $\lambda_{i1}$'s=0.
\[\sum_{i=1}^{n}{\lambda_{i2}v_i}+\lambda_{n+1}v_{n+1}=0\]
Daarom mogen we het volgende doen:
\[v_{n+1}=\frac{-1}{\lambda_{n+1}}.\sum_{i=1}^{n}{\lambda_{i2}v_i}\]
Dus is $v_{n+1}$ een lineaire combinatie:
 
\subsection{Oefening 12}
Moest dit waar zijn dan mag er uit de verzameling $\{ e_1-e_2,e_2-e_3,\dots ,e_{n-1}-e_n,e_n-e_1\}$ geen enkel element bestaan dat een combinatie is van de andere elementen.
$$
\text{We zien nu voor}\ 1 \leq i \leq n:
$$
$$ 
(e_1-e_2)+(e_2-e_3)+\dots +(e_{i-1}-e_i)+(e_{i+1}-e_{i+2})+\dots + (e_n - e_1)
$$
$$
= (\not e_1-\not e_2+\not e_2-\not e_3+\dots +\not e_{i-1}-e_i+e_{i+1}-\not e_{i+2}+\dots + \not e_n - \not e_1)
$$
$$
= -e_i + e_{i+1}
$$
Dus voor een willkeurige vector:
$$
e_i - e_{i+1} = -\left((e_1-e_2)+(e_2-e_3)+\dots +(e_{i-1}-e_i)+(e_{i+1}-e_{i+2})+\dots + (e_n - e_1)\right)
$$
kunnen we dus telkens een combinatie vinden uit de andere vectoren, hierdoor is het geen vrij deel. 
\subsection{Oefening 13}
Hiervoor kunnen we aantonen dat $\{1+X,1+X^2,X+X^2\}$ minimaal voortbrengend is, hiervoor laten we zien dat elke vector uit  $(\mathbb{R},\mathbb{R}[X]_{\leq 2},+)$ een lineaire combinatie is van de basisvectoren:
$$aX^2 + bX + c = \lambda_{1} (1+X) + \lambda_{2} (1+X^2) + \lambda_{3} (X+X^2)$$
Als we deze vergelijking uitwerking uitwerken krijgen we:
$$a = \lambda_{1} + \lambda_2$$
$$b = \lambda_1 + \lambda_3$$
$$c = \lambda_1 + \lambda_2$$
Dit is een stelsel met exact 1 oplossing. Hierme tonen we aan dat het stelsel voortbrengend is en minimaal voortbrengend. Moesten we nog extra vergelijkingen er aan toevoegen zouden we redundantie invoeren of het stelsel onoplosbaar maken.
\\
\\
De co\"ordinaten van de vector $4 -3X + X^2$ kunnen we berekenen ze in een stelsel te gieten:
$$
\begin{pmatrix}
1&1&0&4\\
1&0&1&-3\\
0&1&1&1
\end{pmatrix}
$$
Via rijoperaties verkrijgen we dan:
$$
\begin{pmatrix}
1&0&0&0\\
0&1&0&4\\
0&0&1&-3
\end{pmatrix}
$$
Dit komt overeen met de co\"ordinaten $(0,4,-3)$.
\\
\\
Het co\"ordinatenstel $(2,-3,1)$ komt dan weer overeen met:
$$
2\cdot (1+X) -3 \cdot (1+X^2)+ 1\cdot (X+X^2)= -1 +3X-2X^2
$$

\subsection{Oefening 14}

\subsubsection*{a)}
Beschouwen we eerst de verzameling met alleen het eerste element van de voorgestelde basis. We voegen nu stapsgewijs de daaropvolgende elementen toe, en proberen voor elk toegevoegd element dit element te schrijven als een lineaire combinatie van de reeds toegevoegden. Indien dit lukt maakt het de verzameling lineair afhankelijk en voegen we het niet toe. Indien het geen lineaire combinatie is kunnen we het toevoegen aan onze nieuwe verzameling zonder de vrijheid ervan in gevaar te brengen.\\

Het toevoegen van $(1,2,0)$ is triviaal. Geen veelvoud van $(1,2,3)$ kunnen we gelijk stellen hieraan. Het mag dus bij de verzameling. De volgende elementen zullen we meer systematisch aanpakken.\\

Bij het toevoegen van $(1,1,1)$ proberen we dit te schijven als een lineaire combinatie van de voorgaande als volgt:\\
\[
a * (1,2,3) + b * (1,2,0) = (1,1,1)
\]

Dit is equivalent met het stelsel
\[
\left\{
\begin{array}{l}
a + b = 1\\
2*a + 2*b = 1\\
3*a = 1
\end{array} \right.
\]

wat resulteert in het strijdig (in de eerste twee regels, maar ook indien men het verder probeert uit te werken) stelsel

\[
\left\{
\begin{array}{l}
a + b = 1\\
a + b = \frac{1}{2}\\
a = \frac{1}{3}
\end{array} \right.
\]

$(1,1,1)$ is dus geen lineaire combinatie van de eerste twee, en mogen we dus aan onze vrije verzameling toevoegen, deze blijft vrij. Passen we dit stramien toe op de volgende vector, namelijk $(1,1,0)$, vinden we dat deze een lineaire combinatie is, namelijk
\[
-\frac{1}{3} * (1,2,3) + \frac{1}{3} * (1,2,0) + (1,1,1) = (1,1,0)
\]
Dit kunnen we makkelijk met een stelsel bekomen. We zien dus dat dit een lineaire combinatie is, en voegen het niet toe aan de verzameling. Deze is dus nog steeds $\{(1,2,3), (1,2,0), (1,1,1)\}$.

De laatste vector schrijven we nog even met een stelsel op voor de duidelijkheid. Indien het stelsel

\[
\left\{
\begin{array}{l}
a + b + c = 4\\
2*a + 2*b + c = 5\\
3*a + c = 6
\end{array} \right.
\]

klopt, dan is de vector $(4,5,6)$ een lineaire combinatie van onze vrije verzameling.

We kunnen dit oplossen als matrix die we rijreduceren:

\[
\left[
\begin{array}{c c c | c}
1 & 1 & 1 & 4\\
2 & 2 & 1 & 5\\
3 & 0 & 1 & 6
\end{array}
\right]
\]

\[R_2 \mapsto -R_2 + 2*R_1 \]
\[R_3 \mapsto -R_3 + 2*R_1 \]
(Dit zijn niet echt volledig correcte overgangen op elementair niveau, maar eerder een samentrekking)

\[
\left[
\begin{array}{c c c | c}
1 & 1 & 1 & 4\\
0 & 0 & 1 & 3\\
0 & 3 & 2 & 6
\end{array}
\right]
\]

\[R_2 \leftrightarrow R_3\]

\[
\left[
\begin{array}{c c c | c}
1 & 1 & 1 & 4\\
0 & 3 & 2 & 6\\
0 & 0 & 1 & 3
\end{array}
\right]
\]

\[R_1 \mapsto R_1 - R_3 \]
\[R_2 \mapsto R_2 - 2*R_3 \]

\[
\left[
\begin{array}{c c c | c}
1 & 1 & 0 & 1\\
0 & 3 & 0 & 0\\
0 & 0 & 1 & 3
\end{array}
\right]
\]

\[R_2 \mapsto \frac{R_2}{3}\]

\[
\left[
\begin{array}{c c c | c}
1 & 1 & 0 & 1\\
0 & 1 & 0 & 0\\
0 & 0 & 1 & 3
\end{array}
\right]
\]

\[R_1 \mapsto R_1 - R_2\]

\[
\left[
\begin{array}{c c c | c}
1 & 0 & 0 & 1\\
0 & 1 & 0 & 0\\
0 & 0 & 1 & 3
\end{array}
\right]
\]

We besluiten dus dat $(4,5,6)$ een lineaire combinatie is uit onze verzameling, namelijk\\

\[
(1,2,3) + 3 * (1,1,1) = (4,5,6)
\]

Deze mag dus niet bij in de verzameling.\\

De uitgedunde verzameling is dus $\{(1,2,3), (1,2,0), (1,1,1)\}$.\\

Dit is de originele verzameling waar stapsgewijs de lineaire combinaties uitgehaald zijn, en deze is dus vrij.

\subsubsection*{b)}

We kunnen voor de verzameling via de vorige oefening bepalen dat deze vrij is, dat laten we even buiten beschouwing.\\

We willen deze verzameling voortbrengend maken. Dit betekent dat elk element van $\mathbb{R}^{2x2}$ moet kunnen geschreven worden als een combinatie van de elementen uit de verzameling.\\

Beschouwen we de standaardbasis $\{\bigl(
\begin{smallmatrix}
1&0\\ 0&0
\end{smallmatrix}
\bigr)$,
$\bigl(
\begin{smallmatrix}
0&1\\ 0&0
\end{smallmatrix}
\bigr)$,
$\bigl(
\begin{smallmatrix}
0&0\\ 1&0
\end{smallmatrix}
\bigr)$,
$\bigl(
\begin{smallmatrix}
0&0\\ 0&1
\end{smallmatrix}
\bigr)$\}, dan kunnen we stellen dat elk van deze elementen te vormen moet zijn uit onze voorgestelde basis, als lineaire combinatie. Indien dit voor alle elementen van de standaardbasis geldt is de verzameling voortbrengend, immers, we kunnen de elementen van de standaardbasis vormen en hier lineaire combinaties van maken om alle elementen uit $\mathbb{R}^{2x2}$ te schrijven.\\

Indien er een element van de standaardbasis is dat niet als lineaire combinatie geschreven kan worden mogen we dit toevoegen, het is immers niet lineair afhankelijk, de verzameling blijft dus vrij, en de verzameling brengt nu ook dat element van de standaardbasis voort.

Voor $\bigl(
\begin{smallmatrix}
1&0\\ 0&0
\end{smallmatrix}
\bigr)$ kijken we of we deze kunnen vormen als lineaire combinatie via de matrix

\[
\left[
\begin{array}{c c c | c}
1 & 2 & 0 & 1\\
-1 & 1 & -3 & 0\\
2 & -1 & 5 & 0\\
3 & 4 & 4 & 0
\end{array}
\right]
\]

\[R_2 \mapsto R_2 + R_1 \]
\[R_3 \mapsto R_3 - 2*R_1 \]
\[R_4 \mapsto R_4 - 3*R_1 \]

\[
\left[
\begin{array}{c c c | c}
1 & 2 & 0 & 1\\
0 & 3 & -3 & 1\\
0 & -5 & 5 & -2\\
0 & -2 & 4 & -2
\end{array}
\right]
\]

\[R_3 \mapsto 3*R_3 + 5*R_2 \]
\[R_4 \mapsto 3*R_4 + 3*R_2 \]

\[
\left[
\begin{array}{c c c | c}
1 & 2 & 0 & 1\\
0 & 3 & -3 & 1\\
0 & 0 & 0 & -1\\
0 & 0 & -6 & -4
\end{array}
\right]
\]

Het stelsel is strijdig, dus kunnen we beslissen dat $\bigl(
\begin{smallmatrix}
1&0\\ 0&0
\end{smallmatrix}
\bigr)$ geen lineaire combinatie is van onze verzameling. Deze mag er dus bij zonder dat de vrijheid van de verzameling verdwijnt.\\

Nu kunnen we al beslissen dat onze basis volledig is volgens Gevolg 3.34. De standaardbasis heeft namelijk 4 elementen en deze verzameling ook. Aangezien de verzameling tevens vrij is mogen we beslissen dat de verzameling $\{\bigl(
\begin{smallmatrix}
1&-1\\ 2&3
\end{smallmatrix}
\bigr)$,
$\bigl(
\begin{smallmatrix}
2&1\\ -1&4
\end{smallmatrix}
\bigr)$,
$\bigl(
\begin{smallmatrix}
0&-3\\ 5&4
\end{smallmatrix}
\bigr)$,
$\bigl(
\begin{smallmatrix}
1&0\\ 0&0
\end{smallmatrix}
\bigr)$\} een basis is van $\mathbb{R}^{2x2}$.\\

Indien $\bigl(
\begin{smallmatrix}
1&0\\ 0&0
\end{smallmatrix}
\bigr)$ wel een lineaire combinatie was zouden we deze niet toegevoegd hebben, en de andere elementen van de standaardbasis ge\"evalueerd hebben. Er moet immers volgens eerder aangehaald gevolg nog precies een element toegevoegd worden aan onze verzameling om een basis te kunnen zijn.

\subsection{Oefening 16}

Standaard basis : $\beta1 = {e1 =(1,0,0,0), e2 =(0,1,0,0), e3 = (0,0,1,0), e4 = (0,0,0,1) }$

tweede basis : $\beta2 = {f1 = (1,1,1,1), f2 =(0,1,1,1), f3 = (0,0,1,1), f4= (0,0,0,1) }$

volgens $\beta1  V = a1e1 + a2e2 + a3e3 + a4e4$
volgens $\beta2 V = \lambda1 \cdot f1 + \lambda2 \cdot f2 + \lambda3 \cdot f3 + \lambda4 \cdot f4 $ 

$=> (\lambda1, \lambda2, \lambda3, \lambda4) $

\subsection{Oefening 17}
\subsubsection*{a)}
$W_1$ vormt een vlak. We zoeken twee lineair onafhankelijke vectoren in dat vlak, die zijn dan een basis.
\[
\left\lbrace
\begin{pmatrix}
1\\-2\\1\\
\end{pmatrix}
,
\begin{pmatrix}
2\\-2\\0
\end{pmatrix}
\right\rbrace
\]

\subsubsection*{b)}
Voor de basis te kennen lossen we eerst volgend stelsel op want hieraan moet voldaan zijn:
$$
\begin{pmatrix}[c c c | c]
1 & 2&-1&0\\
2&1&1&0\\
\end{pmatrix}
$$
Hiervoor krijgen we de oplossing $x=-k$, $y= k$, $z=k$. Een oplossing is dus telkens van de vorm $(-k,k,k)$.\\
We krijgen dus als basis:
$$W_2 = \{(-\lambda, \lambda, \lambda)|\lambda \in \mathbb{R}\}$$
De dimensie is dus 1.
\subsection{Oefening 19}
\subsubsection*{b)}
\[ A =
\begin{pmatrix}
  1 & 1 & 1\\
  1 & 1 & -1\\
  1 & -1 & 1\\
  1 & -1 & -1
\end{pmatrix}
\]
Na Gauss-eliminatie krijgen we de rij-gereduceeerde vorm
\[ U =
\begin{pmatrix}
  1 & 0 & 0\\
  0 & 1 & 0\\
  0 & 0 & 1\\
  0 & 0 & 0
\end{pmatrix}
\]
Hierdoor krijgen we de oplossingsverzameling $\{(0,0,0)\}$.\\
De basis van de \underline{nulruimte} is dus $\emptyset$.\\
De rijen $r_1, r_2, r_3$ zijn niet-nulrijen, de basis van de \underline{rijruimte} is dus $vct\{(1,1,1), (1,1,-1), (1,-1,1)\}$.\\
De kolommen $c_1, c_2, c_3$ zijn onafhankelijk en voortbrengend, de basis van de \underline{kolomruimte} is dus $vct\{(1,1,1,1), (1,1,-1,-1), (1,-1,1,-1)\}$.

\subsubsection*{c)}
\[
\begin{pmatrix}
0 & 1 & -1 &  -2 & 1 \\
1 & 1 & -1 & 3   & 1 \\
2 & 1 & -1 & 8   & 3 \\
0 & 0 & -2 & 2   & 1 \\
3 & 5 & -5 & 5   & 10\\
\end{pmatrix}
\longrightarrow
\begin{pmatrix}
1 & 0 & 0 & 5 & 0 \\
0 & 1 & 0 & -3& 0 \\
0 & 0 & 1 & -1& 0 \\
0 & 0 & 0 & 0 & 0 \\
0 & 0 & 0 & 0 & 0 \\
\end{pmatrix}
\]
De eerste drie rijen van deze gereduceerde matrix vormen een basis voor de rijruimte.
De eerste drie kolommen van de gereduceerde matrix vormen een basis voor de kolomruimte. 
De oplossing van het homogeen stelsel van de gereduceerde matrix vormt de nullruimte van de matrix.
De oplossingsverzameling is de volgende.
\[
\{(-5\lambda_1,3\lambda_1,\lambda_1,\lambda_1,\lambda_2)|\lambda_1,\lambda_2\in \mathbb{R}\}
\]
We kiezen hieruit twee onafhankelijke vectoren ($2=5-3$).
\[
\left\lbrace
\begin{pmatrix}
-5\\3\\1\\1\\0\\
\end{pmatrix}
,
\begin{pmatrix}
0\\0\\0\\0\\1\\
\end{pmatrix}
\right\rbrace
\]

\subsection{Oefening 23}
Ik ga ervan uit dat we een bewijs moeten geven van deze stelling. Het staat er namelijk nergens bij.\\
Zij $V$ een vectorruimte met $\beta$ een basis ervan. Zij $\beta_1,...,\beta_k$ de disjuncte unie ervan. Noteer $U_i=vct(\beta_i)$.
\subsubsection*{Te Bewijzen}
\[
V = U_1 \oplus U_2 \oplus ... \oplus U_k
\]
\subsubsection*{Bewijs}
\begin{proof}
We weten dat voor elke $\beta_i,\beta_j$ de vectoren lineair onafhankelijk zijn, want samen vormen ze een basis. We weten dus al dat $\forall U_i,U_j: U_i \cap U_j = \{0\}$.\\
Nu hoeven we enkel nog te bewijzen dat $\sum_i U_i=V$. Voor elke $u_i \in U_i$ geldt dat $u_i = \sum_k\lambda_k\beta_{ik}$, in woorden: "elke $u_i$ is een lineaire combinatie van de basisvectoren van $U_i$. $\sum_i U_i$ bevat alle mogelijke lineaire combinaties van alle mogelijke $u_i$. Deze combinaties kunnen geschreven worden als lineaire combinaties van de vectoren in $\beta$. Nu hebben we dus bewezen dat $\sum_i U_i=V$ en $\forall U_i,U_j: U_i \cap U_j = \{0\}$. Hieruit volgt dat $\bigoplus_iU_i=V$.
\end{proof}



\subsection{Oefening 25}
Toon aan:
$ U \bigoplus W = \mathbb{R}^{n x n}$

\begin{proof}
Wat zijn de voorwaarden die moeten voldoen?
1) $ U + W = \mathbb{R}^{n x n}$
2) $ U \cap W = {0}$

nakijken 1) :
Het is duidelijk dat de doorsnede van U en W enkel de nulmatrix is.

nakijken 2) : 
De som van 2 vierkante matrices met grootte n is opniew een vierkante matrix met afmeting n. $U + W = \mathbb{R}^{n x n}$

QED

\end{proof}

\subsection{Oefening 26}
We kiezen $U = vct\{(0,1,0), (0,0,1)\}$.\\ Hierdoor is $(\mathbb{R}, \mathbb{R}^3, +) = U + W$ en $U \cap W = \{0\}$.\\
Dan is $v = v_U + v_W$ en $w = w_U + w_W$ met
\[
v_U = (0,-10,-8), 
v_W = (2,12,8), 
w_U = (0,4,4), 
w_W = (0,0,0)
\]

\subsection{Oefening 27}
\subsubsection*{a)}
$$\lambda_1 v_1 = \lambda_2 v_2 = 0$$
$$\lambda_1(1,0,1,1,2,3,5,8,...) + \lambda_2(0,1,1,2,3,5,8,...)=(0,0,0,0,0,...)$$
We nemen nu de eerste component van $v_1$ en $v_2$:
$$\lambda_1 \cdot 1 + \lambda_2 \cdot 0 = 0$$
$$\Rightarrow \lambda_1 = 0$$

$$\lambda_1 \cdot 0 + \lambda_2 \cdot 1 = 0$$
$$\Rightarrow \lambda_2 = 0$$
\subsubsection*{b)}
We beweren dat $v_3 = v_1 + v_2$.\\
We bewijzen dat $(v_3)_n = (v_1 + v_2)_n$ \hspace{5pt} $\forall n \in \mathbb{N}$\\
Dit tonen we aan per inductie:\\
\underline{Basisstap:} $(v_1 + v_2)_0 = (v_3)_0 \rightarrow (1 + 0) = 1$\\
\hspace*{4.5em} $(v_1 + v_2)_1 = (v_3)_1 \rightarrow (0 + 1) = 1$\\
\underline{Inductiestap:} Neem $n \geq 2$ en veronderstel dat $(v_1 + v_2)_{n-1} = (v_3)_{n-1}$ en $(v_1 + v_2)_{n-2} = (v_3)_{n-2}$.\\
\[
(v_1 + v_2)_n \overset{v_1 + v_2 \in V}{=} (v_1 + v_2)_{n-1} + (v_1 + v_2)_{n-2} \overset{inductiehypothese}{=} (v_3)_{n-1} + (v_3)_{n-2} = (v_3)_n
\]
\subsubsection*{c)}
Een basis is vrij en voortbrengend. $\beta = \{v_1,v_2\}$ is vrij (zie a).
Nu valt nog te bewijzen: "$\beta$ is voortbrengend voor $V$."
\begin{proof}
Neem een willekeurige vector $v \in V$. We tonen aan dat $v$ te schrijven valt als een lineaire combinatie van $v_1$ en $v_2$.
\[
v=(x_1,x_2,x_3,...)
\]
Waarbij geldt $\forall n \ge 1, n \in \mathbb{N}: x_{n+2}=x_{n+1}+x_n$
\[
v = \lambda_1v_1 + \lambda_2v_2
\]
\[
v = (\lambda_1,\lambda_2,\lambda_1+\lambda_2,\lambda_1+2\lambda_2,...)
\]
Elk element in $v$ voldoet dus aan de volgende recursieve formule.
\[
v_{n+2} = v_{n+1}+v_n=\lambda_1v_{1_{n+1}}+\lambda_2v_{2_{n+1}} + \lambda_1v_{1_{n}}+\lambda_2v_{2_{n}};
\]
Dit is juist volgens de definitie van vectorsom en scalaire vermenigvuldiging.
\end{proof}

\subsubsection*{d)}
$v_3 = (1,1)$

\subsection{Oefening 28}
\subsubsection*{b)}
Zij $V = \{(1,0,...),(0,1,0,...),(0,0,1,0,...),...)\}$.\\
\textbf{Te Bewijzen}\\
$V$ is geen basis van $\mathbb{R}^\mathbb{N}$.\\
\textbf{Bewijs}
\begin{proof}
In 28 a hebben we bewezen dat $V$ een vrij deel is van $\mathbb{R}^\mathbb{N}$. Om te bewijzen dat $V$ geen basis is van $\mathbb{R}^\mathbb{N}$ moeten we dus aantonen dat $V$ niet voortbrengend is voor $\mathbb{R}^\mathbb{N}$.
TODO <wat betekent $\mathbb{R}^\mathbb{N}$?>
\end{proof}


\subsection{Oefening 29}
\subsubsection*{(a)}
Niet waar. Tegenvoorbeeld:\\
Kies $\alpha$ en $\beta$ als volgt.
\[
\alpha =
\left\{
\begin{pmatrix}
1\\0
\end{pmatrix}
,
\begin{pmatrix}
0\\1
\end{pmatrix}
\right\}
\text{ en }
\beta
=
\begin{pmatrix}
1\\0
\end{pmatrix}
,
\begin{pmatrix}
0\\2
\end{pmatrix}
\]
$\alpha\cup\beta$ is niet vrij. want $2\alpha_2 = \beta_2$ geldt.
\[
\alpha\cup\beta = 
\left\{
\begin{pmatrix}
1\\0
\end{pmatrix}
,
\begin{pmatrix}
0\\1
\end{pmatrix}
,
\begin{pmatrix}
0\\2
\end{pmatrix}
\right\}
\]

\subsubsection*{(b)}
Niet waar. Tegenvoorbeeld:\\
Kies $V = \mathbb{R}$, $W=\mathbb{R}$ en $\alpha$ en $\beta$ als volgt.
\[
\alpha = \{1\} \text{ en } \beta = \{2\}
\]
\[
\alpha \cap \beta = \emptyset
\]
De doorsnede van $V$ en $W$ is $\mathbb{R}$, maar $\alpha \cap \beta$ vormt enkel een basis voor $\{0\}$.

\subsubsection*{(c)}
Niet waar. Tegenvoorbeeld:\\
Kies $V= \mathbb{R}$, $v=-1$ en $w=1$.
Nu geldt dat $v+w \in \{0\}$ en dat $\{0\}$ een deelruimte is van $V$. maar $v$ en $w$ zitten niet in $\{0\}$.

\subsubsection*{(d)}
Niet waar. Tegenvoorbeeld:\\
Kies $V = \mathbb{R}^{2}$. Kies als drie deelruimten de deelruimten opgespannen door respectievelijk de volgende vectoren.
\[
v_1 = 
\begin{pmatrix}
1\\0
\end{pmatrix}
,
v_2 = 
\begin{pmatrix}
1\\0.5
\end{pmatrix}
,
v_3 = 
\begin{pmatrix}
0.5\\1
\end{pmatrix}
,
v_4 = 
\begin{pmatrix}
0\\1
\end{pmatrix}
\]
Deze deelruimten zijn niet gelijk, maar het zijn er wel  meer dan $2+1$.

\subsubsection*{(e)}
Waar. Voorbeeld.\\
Kies $V = \mathbb{R}[X]_{\le 2}$ en $W = \mathbb{R}[X]_{\le 1}$.
Kies $e_1,e_2,e_3$ als volgt. 
\[
e_1 = 1, e_2 = X, e_3= X^2
\]
Nu bestaat er een basis voor $V$ waarvan geen enkel element in $W$ zit, namelijk de volgende bijvoorbeeld.
\[
\left\{
X^2, X^2+1, X^2+X
\right\}
\]


\subsection{Oefening 30}
\begin{proof}
Eerst de optelling.
De optelling is inwendig en overal bepaald zoals uit $R_{0}^{+} x R_{0}^{+} \rightarrow R_{0}^{+} $ 
De optelling is associatief.
$ (x + y) + z = x + (y+z) $
$ xy + z = x + yz$
$ xyz = xyz $

Er is een neutraal element.
$x \cdot 1 = x $

GEEN Tegengesteld element.
Het is niet mogenlijk om een uniek tegengesteld element te vinden voor elk element dat gelijk is aan 0.

Commutatief!
$xy = yx$

Alle eigenschappen voor de optelling zijn niet voldaan, dus het is geen geldige definitie.

Bij $ \bigotimes $ geldt distr1 ook al niet.

\end{proof}

\subsection{Oefening 31}
\subsubsection*{a)}
Neen, dit is geen deelruimte; dit is zelfs geen vectorruimte. Een simpel tegenvoorbeeld:\\

$(1,0,1)$ is een element van de verzameling, alsook $(1,0,-1)$, dit is makkelijk na te gaan door invulling. Hun som daarentegen, $(2,0,0)$ voldoet niet aan het voorschrift.\\

Hierdoor is niet voldaan aan Stelling 3.11 en is $K$ bijgevolg geen deelruimte.

\subsubsection*{b)}

De dimensie van $\mathbb{R}^3$ is 3, dit komt triviaal voort uit de standaardbasis $\{(1,0,0),(0,1,0),(0,0,1)\}$.\\

De verzameling $\{(1,0,1),(0,1,1),(0,0,1)\}$ van elementen uit $K$ is lineair onafhankelijk, dit is triviaal te bewijzen met de stelsels

\[
\left\{
\begin{array}{l}
a = 0\\
0 = 1\\
a = 1
\end{array} \right.
\]

en

\[
\left\{
\begin{array}{l}
a = 0\\
b = 0\\
a + b = 1
\end{array} \right.
\]

die we met het uitdunningsalgoritme bekomen. Beide zijn strijdig, dus is de verzameling vrij. Roepen we weer Stelling 3.11 aan, weten we dat voor $\mathbb{R}^3$ een vrije verzameling met 3 elementen een basis is. Aangezien de verzameling een basis bevat van $\mathbb{R}^3$, wordt $\mathbb{R}^3$ ook voortgebracht door $K$ zelf.

\subsection{Oefening 32}
We berekenen alle $a$ waarvoor de vier gegeven basisvectoren lineair afhankelijk zijn.
\[
\begin{vmatrix}
4+a & 3 & 3 & 0\\
2 & a-1 & 5 & 10+a\\
0 & 0 & a+1 & 0\\
-2 & -1 & -5 & 0
\end{vmatrix}
= -(2-a)(10+a)(a+1)=0
\]
\[
\Leftrightarrow 
\left\lbrace
\begin{array}{c c}
a = 2 &\vee\\
a = -10 &\vee\\
a = -1\\
\end{array}
\right.
\]
Nu weten we dus dat als $a$ \textit{niet} \'e\'en van die waarden aanneemt, de doorsnede van $U$ en $W$, $\{\vec{0}\}$ zal zijn. Een basis van $\{\vec{0}\}$ is de lege verzameling en de dimensie ervan is $0$.
Tenslotte bepalen we nog voor de speciale gevallen van $a$ de dimensie en een basis.
\subsubsection*{$a = 2$}
\[
\left\lbrace
\begin{pmatrix}
6\\2\\0\\-2
\end{pmatrix}
,
\begin{pmatrix}
3\\1\\0\\-1
\end{pmatrix}
\right\rbrace
\left\lbrace
\begin{pmatrix}
3\\5\\3\\-5
\end{pmatrix}
,
\begin{pmatrix}
0\\12\\0\\0
\end{pmatrix}
\right\rbrace
\]
Als $a=2$ dan zien we dat de twee vectoren die $U$ opspannen lineair afhankelijk zijn. $U_2$ is dus \'e\'endimensionaal. Als we de eerste vector verwijderen zijn de overblijvende vectoren lineair onafhankelijk. De dimensie van de somruimte is dus $3$. Bij gevolg is de dimensie van de doorsnede $0$ en de basis ervoor $\emptyset$.
\subsubsection*{$a=-10$}
\[
\left\lbrace
\begin{pmatrix}
5\\2\\0\\-2
\end{pmatrix}
,
\begin{pmatrix}
3\\-11\\0\\-1
\end{pmatrix}
\right\rbrace
\left\lbrace
\begin{pmatrix}
3\\5\\-5\\-5
\end{pmatrix}
,
\begin{pmatrix}
0\\0\\0\\0
\end{pmatrix}
\right\rbrace
\]
Hier is de dimensie van $W_a$ $1$. De als we de laatste vector verwijderen blijven er weer drie lineair onafhankelijke vectoren over. De dimensie vande doorsnede is weer $0$ en de basis daarvoor weer $\emptyset$.
\subsubsection*{$a=-1$}
\[
\left\lbrace
\begin{pmatrix}
3\\2\\0\\-2
\end{pmatrix}
,
\begin{pmatrix}
3\\-2\\0\\-1
\end{pmatrix}
\right\rbrace
\left\lbrace
\begin{pmatrix}
3\\5\\0\\-5
\end{pmatrix}
,
\begin{pmatrix}
0\\9\\0\\0
\end{pmatrix}
\right\rbrace
\]
We weten dat voor elke vector $u = \begin{pmatrix}
u_1\\u_2\\u_3\\u_4
\end{pmatrix} \in V\cap W$ geldt:
\[\exists \lambda_1,\lambda_2 :u = 
\lambda_1
\begin{pmatrix}
3\\2\\0\\-2
\end{pmatrix} + 
\lambda_2
\begin{pmatrix}
3\\-2\\0\\-1
\end{pmatrix}
\wedge
\exists \lambda_3,\lambda_4 :u = 
\lambda_3
\begin{pmatrix}
3\\5\\0\\-5
\end{pmatrix} + 
\lambda_4
\begin{pmatrix}
0\\9\\0\\0
\end{pmatrix}
\]
\[
\left\lbrace
\begin{array}{r l l}
u_1 &= 3\lambda_1+3\lambda_2 &= 3\lambda_3\\
u_2 &= 2\lambda_1-2\lambda_2 &= 5\lambda_3+9\lambda_4\\
u_3 &= 0 &= 0\\
u_4 &= -2\lambda_1 -1\lambda_2 &= -5\lambda_3
\end{array}
\right.
\]
%TODO <wat nu?!>

\subsection{Extra Oefening}
\subsubsection*{1.1}
\underline{+} is de som van de functies\\
Neem $f,g,h \in \mathbb{R}^{\mathbb{R}}$ willekeurig.\\
Neem $x \in \mathbb{R}$ willekeurig dan is:
\begin{align*}
((f\underline{+}g)+h)(x) = (f\underline{+} g)(x) +h(x)\\
=(f(x)+g(x))+h(x)\\
=f(x)+(g(x)+h(x))\\
=f(x)+(g\underline{+}h)(x)\\
=(f\underline{+}(g\underline{+}h)(x))\\
\end{align*}
Dus:
\begin{align*}
(f\underline{+}g)\underline{+}h = f\underline{+}(g\underline{+}h)
\end{align*}
\subsubsection*{1.3}
Tegengesteld element:
\[
\forall v \in V:\;\exists v' \in V:\; v+v'=v'+v=0
\]
Stel $g'$ is het tegengesteld element van $g$.
Noteer het neutraal element als $\odot$.
\[
g' = (-1)\bullet g
\]
\begin{proof}
\[
g\textbf{+}g' = g\textbf{+} (-1)\bullet g
\]
We evalueren dit in een willekeurige $x \in R$:
\[ 
(g\textbf{+} (-1)\bullet g)(x) \overset{def\;1}{=} g(x) -g(x) = 0 = \odot
\]
Vanwege de commutativiteit geldt ook:
\[
g(x) -g(x) = -g(x) + g(x) = 0 = \odot
\]
\end{proof}
\subsubsection{1.4}
Stel $f:\mathbb{R} \rightarrow \mathbb{R}: x \mapsto f(x)$ en $g:\mathbb{R} \rightarrow \mathbb{R}: x \mapsto g(x)$, dan geldt voor iedere $x \in \mathbb{R}$ dat
\[ (\lambda \bullet (f \boldsymbol{+} g))(x) 
    = \lambda (f \boldsymbol{+} g)(x) 
    = \lambda (f(x) + g(x))
    = \lambda \cdot f(x) + \lambda \cdot g(x)
    = (\lambda \bullet f)(x) + (\lambda \bullet g)(x)
    = (\lambda \bullet f \boldsymbol{+} \lambda \bullet g)(x)\]
en dus $\lambda \bullet (f \boldsymbol{+} g) = \lambda \bullet f \boldsymbol{+} \lambda \bullet g $


\section{Opdrachten}

\subsection{Opdracht 3.9 p 94}
\label{3.9}
Neem aan dat $\lambda \cdot v = 0$ voor $\lambda \in \mathbb{R}$ en $v \in V$.\\
Er zijn dan twee mogelijkheden:\\
\begin{enumerate}
\item $\lambda = 0$ dan is het in orde.
\item $\lambda \neq 0$
\begin{align*}
v = 1\cdot v \tag{co\"effici\"ent}
\\
= \frac{\lambda}{\lambda}\cdot v
\\
= \frac{1}{\lambda}(\lambda \cdot v) \tag{gemegde associativiteit}
\\
= \frac{1}{\lambda}(0) \tag{lemma 3.8.1}
\\
= 0
\end{align*}
\end{enumerate}


\subsection{Opdracht 3.17 p 97}
\label{3.17}
\begin{enumerate}
\item
\begin{enumerate}
\item
\[
D_1 = \{(1,0,0),(0,0,1)\}
\]

\item
\[
D_2 = \{(1,1,1)\}
\]
\end{enumerate}

\item De gepunte driedimensionale ruimte is een deelruimte van zichzelf.
Elk vlak is een deelruimte, elke rechte is een deelruimte en de triviale vectorruimte $\{\vec{0}\}$ is ook een deelruimte.

\item
\[
f_1 : \mathbb{R} \rightarrow \mathbb{R} : x \mapsto 1
\]
\[
f_2 : \mathbb{R} \rightarrow \mathbb{R} : x \mapsto x
\]
\[
f_3 : \mathbb{R} \rightarrow \mathbb{R} : x \mapsto x^2
\]
\subsubsection*{Te Bewijzen}
\[
vct\{f_1,f_2,f_3\} = \mathbb{R}[X]_{\le 2}
\]
\subsubsection*{Bewijs}
\begin{proof}
Door lineaire combinaties te nemen van $f_1$, $f_2$ en $f_3$ kunnen we heel $\mathbb{R}[X]_{\le 2}$ bekomen.
\[
vct\{f_1,f_2,f_3\} = \{ \lambda f_1+ \mu f_2 + \nu f_3| \lambda,\mu,\nu \in \mathbb{R}\}
\]
\[ = \{f\ |\ f : \mathbb{R} \rightarrow \mathbb{R} : x \mapsto (\lambda 1+ \mu x + \nu x^2) ,\ \lambda,\mu,\nu \in \mathbb{R}\}= \mathbb{R}[X]_{\le 2}
\]
\end{proof}
\end{enumerate}

\subsection{Opdracht 3.18 p 98}
\label{3.18}
Zij $(\mathbb{R},V,+)$ een vectorruimte. Zij $D$ een deelverzameling van $V$ en $U_i$ deelruimten van $V$ die $D$ omvatten.

\subsubsection*{Te Bewijzen}
\[\displaystyle
vct(D) = \bigcap_{i}U_i
\]

\subsubsection*{Bewijs}
\begin{proof}
Direct bewijs.\\
Elke deelruimte $U_i$ moet minstens elke lineaire combinatie van de vectoren in $D$ bevatten.
De kleinste deelruimte $U_i$ die $D$ omvat is precies $vct(D)$. Hiervoor geldt dan $U_i = vct(D)$.
Voor elke $U_i$ geldt dus het volgende.
\[
vct(D) \subseteq U_i 
\]
Dit houdt precies in dat $vct(D) = \displaystyle\bigcap_iU_i$ geldt.
\end{proof}


\subsection{Opdracht 3.21 p 99}
\label{3.21}
Om aan te tonen dat $U_1 + U_2$ deelruimten zijn van $V$ dan moeten we de drie eigenschappen van een deelruimte aantonen:
\begin{enumerate}
\item We moeten aantonen dat $0 \in U_1 + U_2$.\\
We weten dat $U_1$ en $U_2$ deelruimten zijn van $V$.\\
Dus $0 \in U_1$ en $0 \in U_2$ samen met het feit dat $0 = 0 + 0$ volgt dat $0 \in U_1 + U_2$.

\item We moeten aantonen $\forall x,y \in U_1 + U_2$ geldt: $x + y \in U_1 + U_2$.\\
Veronderstel dat twee willekeurige vectoren $x$ en $y \in U_1 + U_2$ bestaan.\\
Dan volgt uit de definitie van $U_1 + U_2$ dat er twee vectoren $x_1$ en $y_1 \in U_1$ en $x_2$ en $y_2 \in U_2$ bestaan zodat $x = x_1 + x_2$ en $y = y_1 + y_2$.\\
Hieruit volgt:
$$x + y = (x_1 + x_2) + (y_1 + y_2) = (x_1 + y_1) + (x_2 + y_2)$$
Omdat $x_1, y_1 \in U_1$ en $U_1$ is een deelruimte van V, $x_1 + y_1 \in U_1$.\\
Omdat $x_2, y_2 \in U_2$ en $U_2$ is een deelruimte van V, $x_2 + y_2 \in U_2$.\\
Nu hebben we aangetoond dat $x+y$ de som is van een vector in $U_1$ en een vector in $U_2$. 
Dus $x + y \in U_1 + U_2$ door de definitie van $U_1 + U_2$ en aangezien $x$ en $y$ willekeurig waren geldt dit voor alle $x,y \in U_1 + U_2$.

\item We moeten aantonen dat $\forall x \in U_1 + U_2$ en $\forall r \in \mathbb{R}$ geldt: $rx \in U$.

Veronderstel dat een willekeurige $x \in U_1 + U_2$ en $r \in \mathbb{R}$ bestaan. Dan volgt uit de definitie van $U_1 + U_2$ dat er een $x_1 \in U_1$ en $x_2 \in U_2$ bestaan zodat $x = x_1 + x_2$.\\
Hieruit volgt:
$$rx = r(x_1+x_2) = rx_1+rx_2$$
Omdat $x_1 \in U_1$ en $U_1$ is een deelruimte van V, $rx_1 \in U_1$.\\
Omdat $x_2 \in U_2$ en $U_2$ is een deelruimte van V, $rx_2 \in U_2$.\\
Nu hebben we aangetoond dat $rx$ de som is van een vector in $U_1$ en een vector in $U_2$. 
Dus $rx \in U_1 + U_2$ door de definitie van $U_1 + U_2$ en aangezien $x$ en $r$ willekeurig waren geldt dit voor alle $r \in \mathbb{R}$ en $x \in U_1 + U_2$.
\end{enumerate}


\subsection{Opdracht 3.24 p 100}
\label{3.24}
\begin{enumerate}
\item
Kies $U_i$ als volgt.
$$U_1 = \{(x,0,0)\} $$
$$U_2 =  \{ (0,y,0)\} $$
$$U_3 = \{ (x,y,0)\} $$

\item
\[
U + W = \mathbb{R}[X]_{\le 4}
\]
$U+W$ is niet de directe som van $U$ en $W$ omdat sommige vectoren op verschillende manieren geschreven kunnen worden als de sum van een $u\in U$ en een $w\in W$.\\
BvB:
$u_1 = x$, $u_2 = 1$, $w_1 = 1$ en $w_2 = x$.
\[
u_1 + w_1 = u_2 = w_2 = x+1 \in U+W
\]
\end{enumerate}


\subsection{Opdracht 3.29 p 103}
\label{3.29}
\begin{enumerate}
\item
We lossen hiervoor het volgende homogeen stelsel op.
\[
\left\{
\begin{array}{r r r r r}
u &+ 2x &+ 0y &+ z =& 0\\
0u &- x &- y &+ z =& 0\\
-u &+ 0x &+ y &+z =& 0\\
u &+ x &+ y &+ e =& 0\\
\end{array}
\right.
\rightarrow
A=
\begin{pmatrix}
1 & 2 & 0 & 1\\
0 & -1 & -1 & 1\\
-1 & 0 & 1 & 1\\
1 & 1 & 1 & 1
\end{pmatrix}
\]
Berekenen we van deze matrix de determinant (om te zien of het homogene stelsel niet-triviale oplossingen heeft) dan vinden we $\det(A) = 7\neq 0$.
Dit houdt in dat de gegeven vectoren lineair afhankelijk zijn.

\item
Geen enkele $X^i$ kan geschreven worden als een lineaire combinatie van $X^j$ als $i\neq j$.
\end{enumerate}

%\subsection{Opdracht 3.48}
%\label{3.48}
%\begin{enumerate}
%\item


%\item


%\item


%\item


%\end{enumerate}


\subsection{Opdracht 3.51 p 115}
\label{3.51}
$U+W$ is de deelruimte van $R[X]_{\le4}$ waarbij $a_3 = a_4$.
\[U+W = \{a_0 + a_1 X + a_2 X^2 +a_3 X^3 + a_4 X^4 | a_3 = a_4\}\]
Een basis van $U$ is simpelweg $\beta_U = \{1,X,X^2\}$. Een basis van $W$ is $\beta_W = \{X+X^2,X^3+X^4\}$. Een basis voor $U\cap W$ is $\beta_{U\cap W} = \{X+X^2\}$. Tenslotte is $\beta_{U+W} = \{1,X,X^2,X^3+X^4\}$ een basis van $U+W$. We verifi\"eren nu de dimensie stelling\footnote{Zie stelling 3.49 p 114.}.
\[
dim(U+W) + dim(U\cap W) = dimU+dimW
\]
\[
4 + 1 = 3+2
\]
Dit ziet er juist uit.


%\subsection{Opdracht 3.55 p 116}
%\label{3.55}
%TODO
%\begin{enumerate}
%\item
%TODO
%
%\item
%TODO

%\end{enumerate}


\subsection{Opdracht 3.59 p 118}
\label{3.59}
\subsubsection*{Te Bewijzen}
\[
\left\lbrace
\begin{array}{c c}
a_1x_1 + b_1x_2 + c_1x_3 + d_1x_4 &= 0 \\
a_1x_1 + b_2x_2+ c_2x_3 + d_2x_4 &= 0 \\
a_1x_1 + b_3x_2+ c_3x_3 + d_3x_4 &= 0 \\
\end{array}
\right.
\]
Het bovenstaande stelsel heeft een niet-triviale oplossing. (Oneindig veel niet-triviale oplossingen zelfs.)

\subsubsection*{Bewijs}
\begin{proof}
Het originele stelsel ziet er als volgt uit in matrixvorm.
\[
\begin{pmatrix}
a_1 & b_1 & c_1 & d_1\\
a_2 & b_2 & c_5 & d_2\\
a_3 & b_3 & c_3 & d_3\\
\end{pmatrix}
\cdot
\begin{pmatrix}
x_1\\ x_2\\x_3
\end{pmatrix}
= \vec{0}
= 
\begin{pmatrix}
a & b & c & d
\end{pmatrix}
\cdot
\begin{pmatrix}
x_1\\ x_2\\x_3
\end{pmatrix}
\]
Beschouwen we vectoren $(a_1,a_2,a_3)$, $(b_1,b_2,b_3)$, $(c_1,c_2,c_3)$ en $(d_1,d_2,d_3)$. Volgens het Lemma van Steinitz\footnote{Zie Stelling 3.32 p 104} is elke verzameling van vectoren in $\mathbb{R}^3$ met meer dan drie elementen lineaire afhankelijkheid. Dit geldt dus ook voor onze vier beschouwde vectoren. Als we nu stellen dat minstens \'e\'en van de vectoren samengesteld wordt uit de anderen, dan kunnen we onze onbekenden $x_1$, $x_2$, $x_3$ en $x_4$ zo kiezen dat we het tegengestelde van de samengestelde (lineair afhankelijke) vector(en) vormen uit het lineair onafhankelijk en (eventueel deels) voortbrengend deel. Stel dat $d$ lineair afhankelijk is van $a, b$ en $c$ (welke vector dit is maakt niet uit omwille van de commutativiteit van de optelling in $\mathbb{R}$.), dan ziet dit er uit als volgt.
\[
\left\lbrace
\begin{array}{c c}
-d_1x_4 + d_1x_4 &= 0 \\
-d_2x_4 + d_2x_4 &= 0 \\
-d_3x_4 + d_3x_4 &= 0 \\
\end{array}
\right.
\]
Dan is voldaan aan ons homogeen stelsel omdat beide elkaar zullen opheffen. Deze oplossing van het homogeen stelsel is niet triviaal omdat een eigenschap van lineaire onafhankelijkheid\footnote{Zie Propositie 2.26 p 101} zegt dat de co\"ffici\"enten $x_1,x_2,x_3$ niet nul mogen zijn.
Deze redenering geldt voor elke keuze van $x_4\in \mathbb{R}$, dus er zijn oneindig veel triviale oplossingen.
\end{proof}


%\subsection{Opdracht 3.64 p 123}
%\label{3.64}
%TODO


\end{document}