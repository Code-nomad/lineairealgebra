\documentclass[lineaire_algebra_oplossingen.tex]{subfiles}
\begin{document}

\section{Examen Augustus 2010}
\subsection{Vraag 1 (Theorie)}
Formulering:
\[
dim(V) = dim(Ker(A)) + dim(Im(A))
\]
\begin{proof}
%TODO
\end{proof}
Dit bewijs staat letterlijk in de cursus. Zie Stelling 4.41 p. 157 (\ref{4.31}).

\subsection{Vraag 2: (Theorie)}
\subsubsection*{a)}
Fout: (tegenvoorbeeld) Als we mogen aannemen dat $V$ eindigdimensionaal is dan is het wel juist.
\begin{proof}
$A$ is injectief \footnote{Zie Propositie 6.59 p. 254 (\ref{6.59})}.
Als $A$ eindigdimensionaal is, dan is $A$ een isomorfisme, dus inverteerbaar.
\end{proof}

\subsubsection*{b)}
Fout. Tegenvoorbeeld:\\
Als de matrix van een lineaire afbeelding symmetrisch is, dan is die lineaire afbeelding symmetrisch. Als tegenvoorbeeld geven we $O$. Deze matrix is symmetrisch, maar niet inverteerbaar. Elke lineaire afbeelding die $O$ als matrix heeft is dus symmetrisch, maar niet inverteerbaar.

\end{document}