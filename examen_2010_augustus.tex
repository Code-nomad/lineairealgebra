\documentclass[lineaire_algebra_oplossingen.tex]{subfiles}
\begin{document}

\section{Examen Augustus 2010}
\subsection{Vraag 1 (Theorie)}
Formulering:
\[
dim(V) = dim(Ker(A)) + dim(Im(A))
\]
\begin{proof}
%TODO
\end{proof}
Dit bewijs staat letterlijk in de cursus. Zie Stelling 4.41 p. 157 (\ref{4.31}).

\subsection{Vraag 2: (Theorie)}
\subsubsection*{a)}
Fout: (tegenvoorbeeld) Als we mogen aannemen dat $V$ eindigdimensionaal is dan is het wel juist.
\begin{proof}
$A$ is injectief \footnote{Zie Propositie 6.59 p. 254 (\ref{6.59})}.
Als $A$ eindigdimensionaal is, dan is $A$ een isomorfisme, dus inverteerbaar.
\end{proof}

\subsubsection*{b)}
Fout. Tegenvoorbeeld:\\
Als de matrix van een lineaire afbeelding symmetrisch is, dan is die lineaire afbeelding symmetrisch. Als tegenvoorbeeld geven we $O$. Deze matrix is symmetrisch, maar niet inverteerbaar. Elke lineaire afbeelding die $O$ als matrix heeft is dus symmetrisch, maar niet inverteerbaar.

\subsection{Vraag 3}
Ik veronderstel dat $\varphi^{-1}(U)$ de verzameling van vectoren is die op $U$ afgebeeld worden met $\varphi$.\\
We lossen volgende stelsels op.
\[
\begin{pmatrix}[c c c | c]
2 & 1 & -1 & 0\\
0 & 1 & -2 & 0\\
-2 & 0 &-1 & 1\\
\end{pmatrix}
\rightarrow
\begin{pmatrix}[c c c | c]
1 & 0 & \frac{1}{2} & 0\\
0 & 1 & -2 & 0\\
0 & 0 & 0 & 1\\
\end{pmatrix}
\]
Dit stelsel is echter strijdig. %TODO wtf??
\[
\begin{pmatrix}[c c c | c]
2 & 1 & -1 & 1\\
0 & 1 & -2 & 1\\
-2 & 0 &-1 & 1\\
\end{pmatrix}
\rightarrow
\begin{pmatrix}[c c c | c]
1 & 0 & \frac{1}{2} & 0\\
0 & 1 & -2 & 0\\
0 & 0 & 0 & 1\\
\end{pmatrix}
\]
Dit ook %TODO huh??


\subsection{Vraag 4}
\subsubsection*{(a)}
Ja. In U gelden $a_0+a_3=0$, $a_2+a_4=0$ en $a_5=0$.
Een voorbeeld van een basis voor $U$ is $\beta$
\[
\beta = \{x^2-x,xy,y^2-y\}
\]

\subsubsection*{(b)}
\[
V' = vct(\{1,x,y\})
\]

\subsection{Vraag 5}
%TODO


\subsection{Vraag 6}
\subsubsection*{(a)}
Fout, Tegenvoorbeeld:
Kies vier verschillende rechten van het vlak als $U_1$, $U_2$, $U_3$ en $U_4$. $4 > n+1$ maar geen enkele $U_i$ is gelijk.

\subsubsection*{(b)}
Fout.
\begin{proof}
Bewijs uit het ongerijmde.\\
Stel dat er zo een basis $V$ bestaat, dan zitten $v_1,v_2$ en $v_3$ alledrie in $V\diagdown W$. Alle $v_i$ zijn dus een veelvoud van $e_3$. $V$ is bijgevolg niet vrij, dus geen basis. Hieruit concluderen we dat $V$ niet kan bestaan.
\end{proof}

\subsection{Vraag 7}

\end{document}