\documentclass[lineaire_algebra_oplossingen.tex]{subfiles}
\begin{document}

\section{Examen Januari 2010}
\subsection{Vraag 1 (Theorie)}
\subsubsection*{(a)}
Dit bewijs staat letterlijk in de cursus. Zie Stelling 3.37 p. 107 (\ref{3.37}).

\subsubsection*{(b)}
Afgezien van het feit dat een algoritme een probleem zou hebben met oneindigdimensionale vectorruimten, zou dit geen probleem mogen vormen.

\subsection{Vraag 2 (Theorie)}
Dit bewijs staat letterlijk in de cursus. Zie Stelling 5.18 p. 190 (\ref{5.18}).

\subsection{Vraag 3}
\subsubsection*{(a)}
We lossen volgend stelsel op.
\[
\begin{pmatrix}[cccccccccccccccc|c]
1 & 1 & 0 & 0 & 0 & 0 & 0 & 0 & 0 & 0 & 0 & 0 & 0 & 0 & 0 & 0 & 0\\
0 & 0 & 0 & 0 & 1 & 1 & 0 & 0 & 0 & 0 & 0 & 0 & 0 & 0 & 0 & 0 & 1\\
0 & 0 & 0 & 0 & 0 & 0 & 0 & 0 & 1 & 1 & 0 & 0 & 0 & 0 & 0 & 0 & 0\\
0 & 0 & 0 & 0 & 0 & 0 & 0 & 0 & 0 & 0 & 0 & 0 & 1 & 1 & 0 & 0 & -1\\
1 & 0 & 1 & 0 & 0 & 0 & 0 & 0 & 0 & 0 & 0 & 0 & 0 & 0 & 0 & 0 & 1\\
0 & 0 & 0 & 0 & 1 & 0 & 1 & 0 & 0 & 0 & 0 & 0 & 0 & 0 & 0 & 0 & 1\\
0 & 0 & 0 & 0 & 0 & 0 & 0 & 0 & 1 & 0 & 1 & 0 & 0 & 0 & 0 & 0 & 1\\
0 & 0 & 0 & 0 & 0 & 0 & 0 & 0 & 0 & 0 & 0 & 0 & 1 & 0 & 1 & 0 & 0\\
0 & 1 & 0 & -1 & 0 & 0 & 0 & 0 & 0 & 0 & 0 & 0 & 0 & 0 & 0 & 0 & 0\\
0 & 0 & 0 & 0 & 0 & 1 & 0 & -1 & 0 & 0 & 0 & 0 & 0 & 0 & 0 & 0 & 0\\
0 & 0 & 0 & 0 & 0 & 0 & 0 & 0 & 0 & 1 & 0 & -1 & 0 & 0 & 0 & 0 & 0\\
0 & 0 & 0 & 0 & 0 & 0 & 0 & 0 & 0 & 0 & 0 & 0 & 0 & 1 & 0 & -1 & 0\\
1 & 1 & 1 & 0 & 0 & 0 & 0 & 0 & 0 & 0 & 0 & 0 & 0 & 0 & 0 & 0 & 0\\
0 & 0 & 0 & 0 & 1 & 1 & 1 & 0 & 0 & 0 & 0 & 0 & 0 & 0 & 0 & 0 & 0\\
0 & 0 & 0 & 0 & 0 & 0 & 0 & 0 & 1 & 1 & 1 & 0 & 0 & 0 & 0 & 0 & 0\\
0 & 0 & 0 & 0 & 0 & 0 & 0 & 0 & 0 & 0 & 0 & 0 & 1 & 1 & 1 & 0 & 0\\
\end{pmatrix}
\]
De matrix van $A$ is dus de volgende:
\[
A =
\begin{pmatrix}
1 & -1 & 0 & -1\\
2 & -1 & -1 & -1\\
1 & -1 & 0 & -1\\
-1 & 0 & 1 & 0
\end{pmatrix}
\]

\subsubsection*{(b)}
Ja.


\subsection{Vraag 4}
\subsubsection*{(a)}
\subsubsection*{(b)}


\subsection{Vraag 5}


\subsection{Vraag 6}
Dit is een oefening die letterlijk in de cursus staat. Zie oefening 18 van hoofdstuk 6 (\ref{oef:6.18}).


\subsection{Vraag 7}

\end{document}