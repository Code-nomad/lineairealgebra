\documentclass[lineaire_algebra_oplossingen.tex]{subfiles}
\begin{document}

\newpage
\section*{Inleiding}

\noindent Allereerst zou de auteur u willen bedanken voor het vertrouwen wanneer u dit boek leest.\\

\noindent Dit boek begint met alle theorie van de cursus gedetailleerd uit te leggen en alle bewijzen (ook de analoge en eenvoudige) volledig uit te schrijven. Vervolgens staan alle oplossingen van alle oefeningen in de cursus opgesomd per hoofdstuk. Verder staan er nog een aantal extraatjes. Tenslotte geeft het boek ook nog de oplossingen van de gekregen zelfreflecties, de huistaken en vorige examens.\\

\noindent Dit boek zal zaken vaak veel uitgebreider uitleggen dan nodig is omdat de cursus vaak te weinig in detail gaat om meteen te kunnen volgen. Dit betekent natuurlijk niet dat u zelf zo gedetailleerd te werk moet gaan, als het maar duidelijk is dat u de stof beheerst.\\

\noindent Lineaire algebra vergt een open geest maar vooral een gemotiveerde student.
Als het niet interessant lijkt, doet u dan tenminste alsof want dit deel van de wiskunde is prachtig wanneer u het kan doorgronden.
Als student informatica kan het zijn dat u vindt dat het lijkt alsof er gefoefeld wordt in bewijzen en de auteur zou u dat zeker niet kwalijk nemen. Dit is helaas eigen aan de wiskunde.
Een mooie vuistregel voor bewijzen van die aard is dat wanneer er gefoefeld lijkt te worden, u van iets niet-triviaal uitgaat of iets mooi nog niet heeft opgemerkt.\\

\noindent Ten slotte wilt de auteur u nog veel succes wensen met dit vak. Hij raadt u aan om niet op te geven voordat u het begrijpt en minstens elke dag het vak te bekijken.


\end{document}

