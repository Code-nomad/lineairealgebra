\documentclass[lineaire_algebra_oplossingen.tex]{subfiles}
\begin{document}

\newpage
\section*{Inleiding}
Elk jaar is het slaagpercentage voor het vak lineaire algebra lager dan het zou kunnen zijn. Dit boek is geschreven om studenten informatica te helpen met lineaire algebra. Het boek van Professor Veys en Professor Igodt (hierna "de cursus") houdt er namelijk geen rekening mee dat het ook voor tweedejaars informatici dient. Dit betekent niet dat u het helemaal overboord moet gooien.

\subsection*{Wat er niet goed is aan de cursus}
\begin{itemize}
\item De cursus gaat uit van kennis die informatici niet mee krijgen in het eerste jaar. Informatici hebben doorgaans namelijk niet het vak bewijzen en redeneren gehad.
\item De cursus laat zeer veel bewijzen weg die niet altijd triviaal zijn.
\item De cursus gaat uit van vrij veel wiskundig inzicht, terwijl dit vak voor informatici het eerste vak is waar dat zeer noodzakelijk bij is.
\item Er staan geen oplossingen en geen algoritmes in de cursus zodat het soms moeilijk is om te beginnen aan de oefeningen.
\item U kan niet in het midden van de cursus beginnen lezen want dan begrijpt u niet alles.
\item De notatie is vaak verwarrend. Er wordt namelijk geen onderscheid gemaakt tussen $0\in \mathbb{R}$ en de nulvector.
\end{itemize}
\subsection*{Wat er wel goed is aan de cursus}
\begin{itemize}
\item De opbouw in de cursus is zeer goed. Je zal nooit verder vooruit moeten kijken om iets te vinden dat je nodig hebt.
\end{itemize}
\subsection*{Wat dit boek probeert}
\begin{itemize}
\item Oplossingen voorzien van de oefeningen uit de cursus
\item Bewijzen die in de cursus staan, maar vooral bewijzen die niet in de cursus staan, beter uit te leggen.
\item Bewijzen tot in veel groter detail uit te leggen zodat ze makkelijker te begrijpen zijn. 
\item Als controleermiddel dienen wanneer je niet zeker bent van je oplossing/bewijs.
\item Bewijzen encapsuleren zodat u het bewijs begrijpt wanneer u de rest van cursus ervoor niet opnieuw wilt lezen. 
\item Duidelijke notatie hanteren.
\end{itemize}
In dit boek staan zeker fouten en garandeert daarom geen juistheid.
\textbf{Vindt u een fout in dit boek of een bewijs dat niet duidelijk genoeg is, aarzel dan niet om de auteurs te contacteren.}
\subsection*{Hoe dit boek tot stand is gekomen}
Dit boek is begonnen als een manier om te zorgen voor ordelijke genoteerde oplossingen van dit vak. Dit idee kwam van \textbf{Ward Schodts} en werd enthousiast onthaald door mezelf. Met de opensource ideologie in het achterhoofd zou het in principe zo zijn dat iedereen, elke week na de oefenzitting een oefening of twee zou moeten toevoegen in \LaTeX \ aan de git-repository. Maar zoals al eerder vermeld bleek de cursus ook zelfs eens een grodig onderhoud nodig te hebben. Het is begonnen als een gesloten git-repository waarbij iedereen die wou meewerken, de broncode mocht hebben. Vervolgens is dit boek volledig opensource geworden door het idee dat iedere student hier wel ook hulp aan zou hebben. Maar ook omdat Ward Schodts, de oorspronkelijke medewerks en ik ervan overtuigd zijn dat door de kracht van een community we de kwaliteit van het boek kunnen verbeteren en vervolledigen.

\end{document}