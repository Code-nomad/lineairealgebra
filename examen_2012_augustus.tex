\documentclass[lineaire_algebra_oplossingen.tex]{subfiles}
\begin{document}

\section{Examen Augustus 2012}

\subsection{Vraag 1 (Theorie)}
Zie Stelling 3.49 p. 114 (\ref{3.49}).

\subsection{Vraag 2 (Theorie)}
Zie Stelling 5.18 p. 190 (\ref{5.18}).

\subsection{Vraag 3}
\[
A^n + a_{n-1}A^{n-1} + ... + a_1A + a_0\mathbb{I}_n = 0
\]
\subsection*{(a)}
\[
A^n + a_{n-1}A^{n-1} + ... + a_1A + a_0\mathbb{I}_n = 0
\]
\begin{proof}
Bewijs van een equivalentie.
\[
A(A^{n-1} + a_{n-1}A^{n-2} + ... + a_1\mathbb{I}_n) = -a_0\mathbb{I}_n
\]
\begin{itemize}
\item $\Rightarrow$
Stel $A$ is inverteerbaar en $a_0$ is \emph{wel} nul.
\[
A^{n-1} + a_{n-1}A^{n-2} + ... + a_2A + a_1\mathbb{I}_n = 0
\]
Maar $n$ is het kleinste getal waarvoor dit soort gelijkheid geldt, contradictie.

\item $\Leftarrow$
Stel $a_0 \neq 0$ en $A$ \emph{niet} inverteerbaar. $\det(A) = 0$
\[
\det(A(A^{n-1} + a_{n-1}A^{n-2} + ... + a_1\mathbb{I}_n)) = \det(-a_0\mathbb{I}_n)
\]
\[
\det(A)\det((A^{n-1} + a_{n-1}A^{n-2} + ... + a_1\mathbb{I}_n)) = \det(-a_0\mathbb{I}_n)
\]
\[
0 = -a_0
\]
Contradictie.
\end{itemize}
\end{proof}

\subsubsection*{(b)}
\[
(A^n + a_{n-1}A^{n-1} + ... + a_1A + a_0\mathbb{I}_n)v = 0
\]
\[
A^nv + a_{n-1}A^{n-1}v + ... + a_1Av + a_0\mathbb{I}_nv = 0
\]
\[
\lambda^nv + a_{n-1}\lambda^{n-1}v + ... + a_1\lambda v + a_0v = 0
\]
\[
(\lambda^n + a_{n-1}\lambda^{n-1} + ... + a_1\lambda + a_0)v = 0
\]
\[
\lambda^n + a_{n-1}\lambda^{n-1} + ... + a_1\lambda + a_0 = 0
\]
$\lambda$, een eigenwaarde van $A$, voldoet aan de vergelijking.

\subsection{Vraag 4}


\subsection{Vraag 5}


\subsection{Vraag 6}

\end{document}