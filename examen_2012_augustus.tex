\documentclass[lineaire_algebra_oplossingen.tex]{subfiles}
\begin{document}

\section{Examen Augustus 2012}

\subsection{Vraag 1 (Theorie)}
Zie Stelling 3.49 p. 114 (\ref{3.49}).

\subsection{Vraag 2 (Theorie)}
Zie Stelling 5.18 p. 190 (\ref{5.18}).

\subsection{Vraag 3}
\[
A^n + a_{n-1}A^{n-1} + ... + a_1A + a_0\mathbb{I}_n = 0
\]
\subsection*{(a)}
\[
A^n + a_{n-1}A^{n-1} + ... + a_1A + a_0\mathbb{I}_n = 0
\]
\begin{proof}
Bewijs van een equivalentie.
\[
A(A^{n-1} + a_{n-1}A^{n-2} + ... + a_1\mathbb{I}_n) = -a_0\mathbb{I}_n
\]
\begin{itemize}
\item $\Rightarrow$
Stel $A$ is inverteerbaar en $a_0$ is \emph{wel} nul.
\[
A^{n-1} + a_{n-1}A^{n-2} + ... + a_2A + a_1\mathbb{I}_n = 0
\]
Maar $n$ is het kleinste getal waarvoor dit soort gelijkheid geldt, contradictie.

\item $\Leftarrow$
Stel $a_0 \neq 0$ en $A$ \emph{niet} inverteerbaar. $\det(A) = 0$
\[
\det(A(A^{n-1} + a_{n-1}A^{n-2} + ... + a_1\mathbb{I}_n)) = \det(-a_0\mathbb{I}_n)
\]
\[
\det(A)\det((A^{n-1} + a_{n-1}A^{n-2} + ... + a_1\mathbb{I}_n)) = \det(-a_0\mathbb{I}_n)
\]
\[
0 = -a_0
\]
Contradictie.
\end{itemize}
\end{proof}

\subsubsection*{(b)}
\[
(A^n + a_{n-1}A^{n-1} + ... + a_1A + a_0\mathbb{I}_n)v = 0
\]
\[
A^nv + a_{n-1}A^{n-1}v + ... + a_1Av + a_0\mathbb{I}_nv = 0
\]
\[
\lambda^nv + a_{n-1}\lambda^{n-1}v + ... + a_1\lambda v + a_0v = 0
\]
\[
(\lambda^n + a_{n-1}\lambda^{n-1} + ... + a_1\lambda + a_0)v = 0
\]
\[
\lambda^n + a_{n-1}\lambda^{n-1} + ... + a_1\lambda + a_0 = 0
\]
$\lambda$, een eigenwaarde van $A$, voldoet aan de vergelijking.

\subsection{Vraag 4}
\subsubsection*{(a)}
\begin{proof}
De kern van $L$ is gelijk aan de nulruimte van $A$. Bovendien is de nulruimte van $A$ het orthogonaal complement van de rijruimte van $A$, of de kolomruimte van $A^T$. De kolomruimte van $A^T$ is precies het beeld van $L^T$.
\[
Ker(L) = Im(L^T)^\bot
\]
\end{proof}


\subsubsection*{(b)}
\[
Im(L^T) \oplus Im(L^T)^\bot = \mathbb{R}^n
\]
\[
dim(Im(L^T)) + dim(Im(L^T)^\bot) = \mathbb{R}^n
\]
\[
dim(Im(L^T)) + dim(Ker(L)) = \mathbb{R}^n
\]
\[
dim(Im(L^T)) = \mathbb{R}^n - dim(Ker(L))
\]
\[
dim(Im(L^T)) = dim(Im(L))
\]

\subsection{Vraag 5}
We rijreduceren eerst volgende matrix om de dimensie van $vct(D)$ te vinden.
\[
\begin{pmatrix}
1 & 0 & 2012 & 7\\
0 & a & 0 & 0\\
a & 0 & a & 3\\
0 & a^2 & 2011 & 1\\
2 & a & a-9 & 5
\end{pmatrix}
\rightarrow
\begin{pmatrix}
1 & 0 & 2012 & 7\\
0 & a & 0 & 0\\
a & 0 & a & 3\\
0 & 0 & 2011 & 1\\
2 & 0 & a-9 & 5
\end{pmatrix}
\rightarrow
\begin{pmatrix}
1 & 0 & 2012 & 7\\
0 & a & 0 & 0\\
0 & 0 & -2011a & 3\\
0 & 0 & 2011 & 1\\
0 & 0 & a-4033 & -9
\end{pmatrix}
\]
Lelijk he? Met een beetje inzicht kan het simpeler. We onderscheiden $a=0$ en $a\neq 0$
\subsubsection*{Geval 1: $a=0$}
\[
\begin{pmatrix}
1 & 0 & 2012 & 7\\
0 & 0 & 0 & 0\\
0 & 0 & 0 & 3\\
0 & 0 & 2011 & 1\\
2 & 0 & -9 & 5
\end{pmatrix}
\]
Neem de eerste, derde en vierde vector als basis.

\subsubsection*{Geval 2: $a\neq0$}
\[
\begin{pmatrix}
1 & 0 & 2012 & 7\\
0 & 1 & 0 & 0\\
a & 0 & a & 3\\
0 & 0 & 2011 & 1\\
2 & 0 & a-9 & 5
\end{pmatrix}
\]
Neem alle vier de vectoren als basis.

\subsection{Vraag 6}
\subsubsection*{(a)}
\subsubsection*{(b)}
Te Bewijzen:
\[
\begin{vmatrix}
1 & a & a^2 & bcd\\
1 & b & b^2 & acd\\
1 & c & c^2 & abd\\
1 & d & d^2 & abc
\end{vmatrix}
=
-
\begin{vmatrix}
1 & a & a^2 & a^3\\
1 & b & b^2 & b^3\\
1 & c & c^2 & c^3\\
1 & d & d^2 & d^3
\end{vmatrix}
\]
\begin{proof}
We bewijzen dat het linkerlid tegengesteld is aan het rechterlid door eerst beide leden uit te werken, en dan te tonen dat ze tegengesteld zijn.\\
Als \'e\'en paar elementen gelijk is, is de determinant nul. Voor de verdere redenering gaan we er daarom van uit dat $a$, $b$, $c$ en $d$ ondeling verschillend zijn.
Zie p. 70 in de cursus (Determinant van Vandermonde).
\begin{itemize}
\item Linkerlid
\[
\begin{vmatrix}
1 & a & a^2 & bcd\\
1 & b & b^2 & acd\\
1 & c & c^2 & abd\\
1 & d & d^2 & abc
\end{vmatrix}
\]
\[
=
\begin{vmatrix}
1 & a & a^2 & bcd\\
0 & b-a & b^2-a^2 & acd-bcd\\
0 & c-a & c^2-a^2 & abd-bcd\\
0 & d-a & d^2-a^2 & abc-bcd
\end{vmatrix}
=
\begin{vmatrix}
b-a & b^2-a^2 & acd-bcd\\
c-a & c^2-a^2 & abd-bcd\\
d-a & d^2-a^2 & abc-bcd
\end{vmatrix}
\]
\[
(b-a)(c-a)(d-a)
\begin{vmatrix}
1 & b+a & -cd\\
1 & c+a & -bd\\
1 & d+a & -bc
\end{vmatrix}
=
(b-a)(c-a)(d-a)
\begin{vmatrix}
1 & b+a & -cd\\
0 & c-b & cd-bd\\
0 & d-b & cd-bc
\end{vmatrix}
\]
\[
=
(b-a)(c-a)(d-a)
\begin{vmatrix}
c-b & cd-bd\\
d-b & cd-bc
\end{vmatrix}
=
(b-a)(c-a)(d-a)(c-b)(d-b)
\begin{vmatrix}
1 & d\\
1 & c
\end{vmatrix}
\]
\[
=
(b-a)(c-a)(d-a)(c-b)(d-b)(c-d)
\]
\[
=
-(a-b)(a-c)(a-d)(b-c)(b-d)(c-d)
\]

\item Rechterlid
\[
\begin{vmatrix}
1 & a & a^2 & a^3\\
1 & b & b^2 & b^3\\
1 & c & c^2 & c^3\\
1 & d & d^2 & d^3
\end{vmatrix}
=
(a-b)(a-c)(a-d)(b-c)(b-d)(c-d)
\]

\end{itemize}
\end{proof}


\end{document}