\documentclass[10pt,a4paper]{article}
\usepackage[utf8]{inputenc}
\usepackage[dutch]{babel}
\usepackage{amsmath}
\usepackage{amsfonts}
\usepackage{amssymb}
\usepackage{amsthm}

\title{Huistaak 1: 2.4 - 9 p 80}
\date{14 oktober 2013}
\author{Tom Sydney Kerckhove}
\begin{document}
\maketitle

Zij $F_{n}$ het $n$-de Fibonacci-getal, dus $F_{1}=1$, $F_{2}=1$ en $F_{n}=F_{n-1}+F_{n-2}$ als $n \ge 3$.
\section*{Te bewijzen}
$\forall n \in \mathbb{N}\setminus\{0,1\}$ geldt dat
\[
\begin{vmatrix}
F_{n}   & F_{n+1}\\
F_{n-1} & F_{n}
\end{vmatrix}
=
(-1)^{n-1}
\]
\section*{Bewijs}
\begin{proof}
Bewijs door volledige inductie.
\subsubsection*{Basisstap}
Voor $n=2$ houdt deze stelling in dat
\[
\begin{vmatrix}
F_{2}   & F_{3}\\
F_{1} & F_{2}
\end{vmatrix}
=
(-1)^{1}
\]
Dit is waar want
\[
\begin{vmatrix}
1 & 2\\
1 & 1
\end{vmatrix}
=
-1
\]
is inderdaad waar.\\
Hiermee is de basisstap bewezen.

\subsubsection*{Inductiestap}
Dit is de inductiehypothese: We nemen aan dat de stelling geldt voor een bepaalde $k$. Dit betekent:
\[
\begin{vmatrix}
F_{k}   & F_{k+1}\\
F_{k-1} & F_{k}
\end{vmatrix}
=
(-1)^{k-1}
\]
We moeten nu bewijzen dat de stelling dan ook geldt voor $k+1$.
We bewijzen dus dat:
\[
\begin{vmatrix}
F_{k+1}   & F_{k+2}\\
F_{k} & F_{k+1}
\end{vmatrix}
=
(-1)^{k}
\]
als volgt.

\[
\begin{vmatrix}
F_{k+1}   & F_{k+2}\\
F_{k} & F_{k+1}
\end{vmatrix}
\overset{\overset{(1)}{R1 \longmapsto R1-R2}}{=}
\begin{vmatrix}
F_{k-1}   & F_{k}\\
F_{k} & F_{k+1}
\end{vmatrix}
\overset{(2)}{=}
(-1)
\begin{vmatrix}
F_{k} & F_{k+1}\\
F_{k-1}   & F_{k}
\end{vmatrix}
\]
(1): Stelling 2.3 op p 58\\
Als men op een matrix $A$ een elementaire rijoperatie uitvoert van het type $R_{i} \longmapsto R_{i}+\lambda R_{j}$ met $(i \neq j)$, dan verandert $det(A)$ niet.\\\\
(2): Definitie 2.1 op p 57\\
$det(A)$ verandert van teken als men in de matrix $A$ twee rijen van plaats wisselt.

\[
(-1)
\begin{vmatrix}
F_{k} & F_{k+1}\\
F_{k-1}   & F_{k}
\end{vmatrix}
= (-1)\cdot(-1)^{k-1}
= (-1)^{k}
\]
Hiermee is ook de inductiestap bewezen.

\subsubsection*{Conclusie}
Omdat de basisstap en de inductiestap bewezen zijn, volgt met het principe van volledige inductie dat
\[
\begin{vmatrix}
F_{n}   & F_{n+1}\\
F_{n-1} & F_{n}
\end{vmatrix}
=
(-1)^{n-1}
\]
waar is voor elk getal $n \in \mathbb{N}\setminus\{0,1\}$. 
\end{proof}

\end{document}