\documentclass[lineaire_algebra_oplossingen.tex]{subfiles}
\begin{document}

\chapter{Notatie}
Dit hoofdstuk is tot stand gekomen toen er iemand die de les niet mee gevolgd heeft verward was door de notatie.
Omdat dit \'e\'en van de dingen is die de cursus nalaat duidelijk te maken wordt het hier zeer expliciet opnieuw vermeld.\\\\
De belangrijkste regel omtrent notatie is om steeds duidelijk te maken wat u bedoelt met elk teken.
Zo wordt de benaming van elk onderdeel van een bewijs het best expliciet verklaard.
Sommige van deze verklaringen lijken waarschijnlijk overbodig, omdat u de notatie kent, maar het is toch belangrijk om het expliciet te vermelden.

\section{Algemeen}
\subsection{Indices}
Vaak spreken we over meerdere gelijkaardige elementen met indices. Beschouw bijvoorbeeld de verzameling $\{a_1,a_2,...,a_n\}$.
Soms spreken we van 'alle $a_i$'.
Dit betekent dan 'Elk element uit de verzameling', of nog: 'Elk element $a$ waarbij we een zinvolle index $i$ kunnen zetten'.
 
\subsection{Sommaties, Producten en andere}
De betekenis van het sommatie- en productteken is u waarschijnlijk bekend, maar wordt hier toch nog eens expliciet vermeld als inleiding tot wat volgt. 
\[
\sum_{i=1}^n
\]
Er zijn echter nog reeksen bewerkingen die we korter kunnen noteren. 

\section{Vectoren}
Vectoren worden meestal met een kleine letter aangeduid en meestal $v$, $w$ of $u$ als letter.
Wanneer andere letters worden gebruikt wordt er soms een vector pijltje boven de letter gezet: $\vec{a}$. (Vooral in natuurkunde is dit onderscheid belangrijk.)
Elementen uit vectoren worden meestal met een index aangeduid. Dit is natuurlijk enkel zinvol wanneer we over co\"ordinaat vectoren spreken.
\[
\vec{v} = 
\begin{pmatrix}
v_{1}&v_{2}&\cdots&v_{n}
\end{pmatrix}
\ 
\vec{a} = 
\begin{pmatrix}
a_{1}\\a_{2}\\\vdots\\a_{n}
\end{pmatrix}
\]
Onbekende vectoren worden meestal als $X$ en $Y$ aangeduid, of $\vec{x}$ en $\vec{y}$. Deze worden meestal gebruikt in stelsels.\\\\
De verzameling van alle re\"ele co\"ordinaat vectoren met $n$ elementen word met $\mathbb{R}^n$ aangeduid. Analoog wordt de verzameling van de complexe co\"ordinaatvectoren met $\mathbb{C}^n$ aangeduid.

\section{Matrices}
Matrices worden meestal met een drukletter aangeduid.
Vaak worden $A$, $B$ en $C$ gebruikt om onderscheid te maken van vectorruimten (zie verder). Elementen in matrices worden meestal aangeduid met een index, al dan niet met haakjes.
\[
A = 
\begin{pmatrix}
a_{11} & a_{12} & \cdots & a_{1n}\\
a_{21} & a_{22} & \cdots & a_{2n}\\
\vdots & \vdots & \ddots & \vdots\\
a_{m1} & a_{m2} & \cdots & a_{mn}\\
\end{pmatrix}
= 
\begin{pmatrix}
A_{11} & A_{12} & \cdots & A_{1n}\\
A_{21} & A_{22} & \cdots & A_{2n}\\
\vdots & \vdots & \ddots & \vdots\\
A_{m1} & A_{m2} & \cdots & A_{mn}\\
\end{pmatrix}
=
\begin{pmatrix}
(A)_{11} & (A)_{12} & \cdots & (A)_{1n}\\
(A)_{21} & (A)_{22} & \cdots & (A)_{2n}\\
\vdots & \vdots & \ddots & \vdots\\
(A)_{m1} & (A)_{m2} & \cdots & (A)_{mn}\\
\end{pmatrix}
\]
De verzameling van alle re\"ele $m\times n$ matrices word met $\mathbb{R}^n$ aangeduid.
Analoog wordt de verzameling van de complexe matrices met $\mathbb{C}^{m\times n}$ aangeduid.
Voor de duidelijkheid, een $m \times n$ matrix heeft $m$ rijen en $n$ kolommen.
De kolommen en rijen hebben respectievelijk lengte $n$ en $m$.
Element $a_{ij}$ uit een matrix $A$ is het element op \emph{rij} $i$ en \emph{kolom} $j$.


\section{Stelsels}
De volgorde waarin de vergelijkingen in een stelsel staan maakt niet uit.
Een stelsel kunnen we een naam geven, maar hier zijn geen conventies voor in dit boek, omdat dit meestal niet gebeurt.
\[
\left\{
\begin{array}{c c c c c}
a_{11}x_{1} &+ a_{12}x_{2} & \cdots &+ a_{1n}x_{n} &= b_1\\
a_{21}x_{1} &+ a_{22}x_{2} & \cdots &+ a_{2n}x_{n} &= b_2\\
\vdots & \vdots & \ddots & \vdots & \vdots \\
a_{m1}x_{1} &+ a_{m2}x_{2} & \cdots &+ a_{mn}x_{n} &= b_n\\
\end{array}
\right.
\]
Stelsels kunnen we echter eenvoudiger opschrijven. Dit kan bijvoorbeeld als een co\"effici\"entenmatrix, al dan niet uitgebreid. Wanneer alle $b_i$ nul zijn laten we soms die kolom weg.
\[
\left(
\begin{array}{c c c c | c}
a_{11} & a_{12} & \cdots & a_{1n} & b_1\\
a_{21} & a_{22} & \cdots & a_{2n} & b_2\\
\vdots & \vdots & \ddots & \vdots & \vdots \\
a_{m1} & a_{m2} & \cdots & a_{mn} & b_n\\
\end{array}
\right)
\ 
\text{ of }
\ 
\left(
\begin{array}{c c c c}
a_{11} & a_{12} & \cdots & a_{1n}\\
a_{21} & a_{22} & \cdots & a_{2n}\\
\vdots & \vdots & \ddots &   \\
a_{m1} & a_{m2} & \cdots & a_{mn}\\
\end{array}
\right)
\]


\section{Vectorruimten}
%TODO



\end{document}