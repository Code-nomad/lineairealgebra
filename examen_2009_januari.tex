\documentclass[lineaire_algebra_oplossingen.tex]{subfiles}
\begin{document}

\section{Examen Januari 2009}
\subsection{Vraag 1 (Theorie)}
Formulering:
\[
dim(V) = dim(Ker(A)) + dim(Im(L))
\]
\begin{proof}
%TODO
\end{proof}
Dit bewijs staat letterlijk in de cursus. Zie Stelling 4.41 p. 157 (\ref{4.31}).

\subsection{Vraag 2 (Theorie)}
\subsubsection*{a)}
Dit staat letterlijk in de cursus. Zie Stelling 6.27 p. 238 (\ref{6.27}).

\subsubsection*{b)}

\subsection{Vraag 3}
\subsubsection*{a)}
Te Bewijzen:\\
\[
\forall x,y \in \mathbb{R}\ u,v \in H: x\cdot u + y\cdot v \in H
\]
\begin{proof}
Kies een willekeurige $u = (u_1,u_2,u_3)$ en $v = (v_1,v_2,v_3)$ in $H$ en noem $a = (a_1,a_2,a_3)$ en $b = (b_1,b_2,b_3)$.
\[
x\cdot u + y\cdot v = (xu_1 + yv_1, xu_2 + yv_2, xu_3 + yv_3)
\]
Nu rest er ons nog volgende twee vergelijkingen te controleren.
\[ 
\left\{
\begin{array}{c}
(xu_1 + yv_1)a_1 + (xu_2 + yv_2)a_2 + (xu_3 + yv_3)a_3 = 0\\
(xu_1 + yv_1)b_1 + (xu_2 + yv_2)b_2 + (xu_3 + yv_3)b_3 = 0 
\end{array}
\right.
\]
\[ 
\left\{
\begin{array}{c}
xu_1a_1 + yv_1a_1 + xu_2a_2 + yv_2a_2 + xu_3a_3 + yv_3a_3 = 0\\
xu_1b_1 + yv_1b_1 + xu_2b_2 + yv_2b_2 + xu_3b_3 + yv_3b_3 = 0 
\end{array}
\right.
\]
\[ 
\left\{
\begin{array}{c}
x(u_1a_1 + u_2a_2 + u_3a_3) + y(v_1a_1 + v_2a_2 + v_3a_3) = 0\\
x(u_1b_1 + u_2b_2 + u_3b_3) + y(v_1b_1 + v_2b_2 + v_3b_3) = 0 
\end{array}
\right.
\]
\[ 
\left\{
\begin{array}{c}
x\cdot 0 + y\cdot 0 = 0\\
x\cdot 0 + y\cdot 0 = 0 
\end{array}
\right.
\]
\end{proof}


\subsubsection*{b)}


\subsubsection*{c)}

\subsection{Vraag 4}
\subsubsection*{a)}


\subsubsection*{b)}

\subsection{Vraag 5}

\subsection{Vraag 6}
\subsubsection*{a)}


\subsubsection*{b)}

\end{document}