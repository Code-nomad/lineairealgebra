\documentclass[lineaire_algebra_oplossingen.tex]{subfiles}
\begin{document}

\section{Examen Januari 2009}
\subsection{Vraag 1 (Theorie)}
Formulering:
\[
dim(V) = dim(Ker(A)) + dim(Im(L))
\]
\begin{proof}
%TODO
\end{proof}
Dit bewijs staat letterlijk in de cursus. Zie Stelling 4.41 p. 157 (\ref{4.31}).

\subsection{Vraag 2 (Theorie)}
\subsubsection*{a)}
Dit staat letterlijk in de cursus. Zie Stelling 6.27 p. 238 (\ref{6.27}).

\subsubsection*{b)}

\subsection{Vraag 3}
\subsubsection*{a)}
Te Bewijzen:\\
\[
\forall x,y \in \mathbb{R}\ u,v \in H: x\cdot u + y\cdot v \in H
\]
\begin{proof} (rechtstreeks bewijs)\\
Kies een willekeurige $u = (u_1,u_2,u_3)$ en $v = (v_1,v_2,v_3)$ in $H$ en noem $a = (a_1,a_2,a_3)$ en $b = (b_1,b_2,b_3)$.
\[
x\cdot u + y\cdot v = (xu_1 + yv_1, xu_2 + yv_2, xu_3 + yv_3)
\]
Nu rest er ons nog volgende twee vergelijkingen te controleren.
\[ 
\left\{
\begin{array}{c}
(xu_1 + yv_1)a_1 + (xu_2 + yv_2)a_2 + (xu_3 + yv_3)a_3 = 0\\
(xu_1 + yv_1)b_1 + (xu_2 + yv_2)b_2 + (xu_3 + yv_3)b_3 = 0 
\end{array}
\right.
\]
\[ 
\left\{
\begin{array}{c}
xu_1a_1 + yv_1a_1 + xu_2a_2 + yv_2a_2 + xu_3a_3 + yv_3a_3 = 0\\
xu_1b_1 + yv_1b_1 + xu_2b_2 + yv_2b_2 + xu_3b_3 + yv_3b_3 = 0 
\end{array}
\right.
\]
\[ 
\left\{
\begin{array}{c}
x(u_1a_1 + u_2a_2 + u_3a_3) + y(v_1a_1 + v_2a_2 + v_3a_3) = 0\\
x(u_1b_1 + u_2b_2 + u_3b_3) + y(v_1b_1 + v_2b_2 + v_3b_3) = 0 
\end{array}
\right.
\]
\[ 
\left\{
\begin{array}{c}
x\cdot 0 + y\cdot 0 = 0\\
x\cdot 0 + y\cdot 0 = 0 
\end{array}
\right.
\]
\end{proof}
\noindent Zie Stelling 3.11 p 94 (\ref{3.11}).

\subsubsection*{b)}
Aangezien een dimensie nooit negatief kan zijn, moeten we enkel bewijzen dat de dimensie van $H$ niet nul is. Als de dimensie van $H$ nul zou zijn dan zou $H$ gelijk zijn aan $\{\vec{0}\}$. We moeten dus aantonen dat er een niet-nul vector bestaat in $H$.\\\\
Bekijken we $H$ eens met het standaard inproduct in het achterhoofd, dan zien we dat $H$ bestaat uit alle vectoren die loodrecht staat op precies twee vectoren $a$ en $b$. Intu\"itief zien we dus al dat $H$ een rechte vormt wanneer $a$ en $b$ lineair onafhankelijk zijn dat $H$ een vlak vormt wanneer $a$ en $b$ lineair afhankelijk zijn en dat $H$ de driedimensionale ruimte vormt wanneer $a$ en $b$ beide nulvectoren zijn. De dimensie van $H$ is dus $1$, $2$ of $3$, maar zeker niet nul.
%TODO formeel bewijs.

\begin{proof}
We construeren een vector $u = (u_1,u_2,u_3)$ zodat $u$ geen nulvector is, en toch in $H$ zit.
\[
u = a \times b =
(a_2b_3-b_2a_3, b_1a_1-a_1b_3, a_1b_2-b_1a_2)
\]
We gaan nu na of $u$ tot $H$ behoort.
\[
\left\{
\begin{array}{c}
a_1(a_2b_3-b_2a_3) + a_2(b_1a_1-a_1b_3) + a_3(a_1b_2-b_1a_2) = 0\\
b_1(a_2b_3-b_2a_3) + b_2(b_1a_1-a_1b_3) + b_3(a_1b_2-b_1a_2) = 0
\end{array}
\right.
\]
\[
\left\{
\begin{array}{c}
a_1a_2b_3-a_1b_2a_3 + a_2b_1a_1-a_2a_1b_3 + a_3a_1b_2-a_3b_1a_2 = 0\\
b_1a_2b_3-b_1b_2a_3 + b_2b_1a_1-b_2a_1b_3 + b_3a_1b_2-b_3b_1a_2 = 0
\end{array}
\right.
\]
Nu zien we dat alle termen mooi wegvallen en dat $u$ bijgevolg in $H$ zit. Merk op dat $u$ enkel nul is wanneer $a$ en $b$ beide nulvectoren zijn. In dat geval kunnen we voor $u$ een willekeurige vector uit $\mathbb{R}^3$ kiezen.
\end{proof}

\subsubsection*{c)}
Kies bijvoorbeeld $a= (1,0,0)$ en $b=(0,1,0)$. $H$ heeft dan dimensie $1$ en $H : \{(0,0,1)\}$.

\subsection{Vraag 4}
\subsubsection*{a)}
Een afbeelding is lineair als ze een lineaire combinatie behoudt\footnote{Zie Lemma 4.2 p 130 (\ref{4.2})}.
\[
\forall \lambda_1,\lambda_2 \in \mathbb{R} v_1,v_2 \in \mathbb{R}^{2}:\  A(\lambda_1v_1+\lambda_2v_2) = \lambda_1A(v_1)+\lambda_2A(v_2)
\]
We hoeven dit echter niet formeel uit te rekenen. Het is makkelijk in te zien dat $A$ enkel lineair is wanneer de rechte waarrond gespiegeld wordt door de oorsprong gaat. $b$ moet dus nul zijn.

\subsubsection*{b)}
De matrix van $A$ is de volgende. Ik heb geen idee hoe u dit uit uw duim zou moeten zuigen. Ik heb het ook maar opgezocht.
\[
\frac{1}{a^2 + 1}
\begin{pmatrix}
1-a^2 & 2a\\
2a & a^2-1
\end{pmatrix}
\]
%TODO

\subsection{Vraag 5}
\[
A_a:
\begin{pmatrix}
x\\y\\z
\end{pmatrix}
\mapsto
\begin{pmatrix}
ax+z\\
x+(1-a)y+az\\
(2a-1)x+(1-a^2)y+a^2z\\
\end{pmatrix}
\]
\[
A_a \leftrightarrow
\begin{pmatrix}
a & 0 & 1\\
1 & (1-a) & a\\
(2a-1) & (1-a^2) & a^2
\end{pmatrix}
\]
\[
Ker(A_a) \leftrightarrow
\begin{pmatrix}
a & 0 & 1\\
1 & (1-a) & a\\
(2a-1) & (1-a^2) & a^2
\end{pmatrix}
\begin{pmatrix}
x\\y\\z
\end{pmatrix}
=
\begin{pmatrix}
0\\0\\0
\end{pmatrix}
\]
%TODO

\subsection{Vraag 6}
\subsubsection*{a)}
\[
dim(Im(g\circ f)) \ge dim(Im(g)) + dim(Im(f)) - dim(W)
\]
De dimensie van het beeld van $g \circ f$ is zeker kleiner of gelijk aan de dimensie van het beeld van $g$. Zo is ook de dimensie van de kern van $g \circ f$ groter of gelijk aan de dimensie van de kern van $f$ omdat de nulvector steeds op zichzelf wordt afgebeeld. ($f$ en $g$ zijn lineaire afbeeldingen).
%TODO is dit juist?
\begin{proof}
\[
dim(Im(g)) \ge dim(Im(g\circ f)) \ge dim(Im(g)) + dim(Im(f)) - dim(W)
\]
\[
dim(Im(g)) \ge dim(Im(g)) + dim(Im(f)) - dim(W)
\]
\[
0 \ge dim(Im(f)) - dim(W)
\]
\[
dim(W) \ge dim(Im(f))
\]
\end{proof}

\subsubsection*{b)}
Zij $f$ en $g$ de volgende transformatie van $\mathbb{R}^2$
\[
f = g : \mathbb{R}^2 \rightarrow \mathbb{R}^2: (x,y) \mapsto (x,0)
\]
Nu zien we het volgende:
\[
dim(Im(f)) = dim(Im(g)) = 1
\]
\[
dim(Im(g\circ f)) = 1
\]
\[
dim(W) = 2
\]
\[
1 \ge 1 + 1 - 2
\]



\end{document}