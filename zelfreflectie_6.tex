\documentclass[lineaire_algebra_oplossingen.tex]{subfiles}
\begin{document}

\section{Zelfreflectie 6}
\subsection{Oefening 1}
Zie opmerking 6.2 \ref{6.2}. (Lineairiteit in beide componenten en symmetrie).

\subsection{Oefening 2}
\[
\Vert x+y \Vert^2 = \Vert x\Vert^2 + 2\langle x,y \rangle+ \Vert y\Vert^2 \overset{\langle x,y \rangle = 0}{=}\Vert x\Vert^2 + \Vert y\Vert^2
\]

\subsection{Oefening 3}
Zie \ref{6.27}.

\subsection{Oefening 4}
Zie de spectraalstelling \ref{6.42}.

\subsection{Oefening 5}
Juist, zie \ref{6.46}.

\subsection{Oefening 6}
%TODO wut?

\subsection{Oefening 7}
We weten dat de $Tr(A^T A)$ als inproduct kan dienen. Dan passen we de ongelijkheid van cauchy toe \footnote{Zie Stelling 6.14 p 229.}

\subsection{Oefening 8}
Juist\footnote{Zie Lemma 3.38 p 244} \footnote{Zie Lemma 6.40 p 345} \footnote{Zie Stelling 6.42 p 246}

\subsection{Oefening 9}
Een vergroting/verkleining in elke dimensie.

\subsection{Oefening 10}
Zie \ref{6.43}.

\subsection{Oefening 11}
De matrix van een symmetrische lineaire transformatie is een symmetrische matrix. Pas daar de spectraalstelling op toe. Zie \ref{6.42}.

\subsection{Oefening 12}
Juist. Zie \ref{6.46}.

\subsection{Oefening 13}
$\int_a^bf(x)g(x)$ is geen inproduct want het is niet definitie. Beschouw bijvoorbeeld $f(x) = \sqrt{\sin(x)}$. Het `inproduct' van deze functie met zichzelf is $0$ maar de functie niet \footnote{Zie Definitie 6.1 p 222.}

\subsection{Oefening 14}
\[
A = 
\begin{pmatrix}
a & b & c\\
d & e & f
\end{pmatrix}
\text{ en }
B = 
\begin{pmatrix}
g & h & i\\
j & k & l
\end{pmatrix}
\]
\[
Tr(A\cdot B^T) = 
Tr(
ag + bh + ci + dj + ek + fl
= Tr(A^T\cdot B)
\]

\subsection{Oefening 15}
Zie \ref{6.26}.


\end{document}