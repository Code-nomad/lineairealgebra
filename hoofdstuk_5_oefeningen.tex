\documentclass[lineaire_algebra_oplossingen.tex]{subfiles}
\begin{document}

\chapter{Oefeningen Hoofdstuk 5}

\section{Oefeningen 5.9}
\subsection{Oefening 1}
\subsubsection*{(a)}
Zij $A \in \mathbb{R}^{n\times n}$ een idempotente vierkante matrix. ($A^2 = A$).
We berekenen nu de eigenwaarden van $A$ en $A^2$. Deze eigenwaarden zijn gelijk want $A= A^2$. Noem een willekeurige eigenwaarde van $A$ $\lambda$ met bijhorende eigenvector $v$.
\[
A\cdot v = A^2\cdot v
\]
\[
\lambda v = A \cdot A \cdot v
\]
\[
\lambda v = A \cdot \lambda v
\]
\[
\lambda v = \lambda \cdot A \cdot v
\]
\[
\lambda v = \lambda \cdot A \cdot v
\]
\[
\lambda v = \lambda^2 v
\]
\[
\lambda = \lambda^2
\]
\[
\lambda = 1 \wedge \lambda = 0
\]

\subsubsection*{(b)}
Dit is een beetje gefoefel. Noem een willekeurige eigenwaarde van $A$ $\lambda$ met bijhorende eigenvector $v$.
\[
A \cdot v = \lambda v
\]
\[
A^{-1} \cdot A \cdot v = A^{-1}\cdot \lambda v
\]
\[
\mathbb{I}_n v = \lambda \cdot A^{-1}\cdot v
\]
\[
\frac{1}{\lambda} v = A^{-1} \cdot v
\]
Omdat $A$ inverteerbaar is heeft $A$ zeker geen eigenwaarden die nul zijn.

\subsection{Oefening 2}
\subsubsection*{(a)}
We gaan er maar van uit dat $A$ diagonalieerbaar is, anders is dit niet op te lossen.
De som van de eigenwaarden is nu het spoor, en het product de determinant \footnote{Zie Propositie 5.25 p 194 (\ref{5.25}).}. 
\[
\lambda_1 \lambda_2 = -4 \text{ en } \lambda_1 + \lambda_2 = 3
\]
\[
\left\{
\begin{array}{c c}
\lambda_1 &= 4\\
\lambda_2 &= -1\\
\end{array}
\right.
\]

\subsubsection*{(b)}
De oplossingen van de karakteristieke veelterm van $A$ zijn de volgende.
\[
\left\{
\begin{array}{c c}
\lambda_1 &= 2\\
\lambda_2 &= 4\\
\end{array}
\right.
\]
Omdat $\phi_{A}$ volledig te ontbinden valt int een product van eerstegraads factoren moeten we enkel de eigenwaarden optellen/vermenigvuldigen om de determinant/het spoor van $A$ te vinden.
\[
\left\{
\begin{array}{r l l}
\det(A) &= 2\cdot 4 &=  8\\
Tr(A) &= 2+4 &= 6\\
\end{array}
\right.
\]

\subsection{Oefening 3}
\subsubsection*{Eigenwaarden}
Om de eigenwaarden van $A$ te vinden zoeken we de nulpunten van de karakteristieke veelterm van $A$.
\[
\phi_A = \det(A-\lambda\mathbb{I}_n) =
\begin{vmatrix}
1-\lambda & 1 & 1 & 1\\
1 & 1-\lambda & 1 & 1\\
1 & 1 & 1-\lambda & 1\\
1 & 1 & 1 & 1-\lambda\\
\end{vmatrix}
\]
We kunnen deze determinant domweg uitrekenen. Dit vraagt vrij veel rekenwerk en is heel saai. We kunnen het ook iets intelligenter aanpakken door eerst de eerste en de laatste en daarna de middelste twee rijen te verwisselen. De determinant verandert dan niet. Daarna rijreduceren we de matrix zonder rijen te verwisselen. Dan verandert de determinant ook niet. Nu kunnen we in elke rij $\lambda$ afzonderen. Zo bekomen we een veel simpelere determinant.
\[
=
\begin{vmatrix}
1 & 1 & 1 & 1-\lambda\\
1 & 1 & 1-\lambda & 1\\
1 & 1-\lambda & 1 & 1\\
1-\lambda & 1 & 1 & 1\\
\end{vmatrix}
=
\begin{vmatrix}
1 & 1 & 1 & 1-\lambda\\
0 & 0 & -\lambda & \lambda\\
0 & -\lambda & 0 & \lambda\\
0 & \lambda & \lambda & 2\lambda-\lambda^2\\
\end{vmatrix}
=
\begin{vmatrix}
0 & -\lambda & \lambda\\
-\lambda & 0 & \lambda\\
\lambda & \lambda & 2\lambda-\lambda^2\\
\end{vmatrix}
\]
\[
= \lambda^3\cdot
\begin{vmatrix}
0 & -1 & 1\\
-1 & 0 & 1\\
1 & 1 & 2-\lambda\\
\end{vmatrix}
=
\lambda^3
\left(
\begin{vmatrix}
-1 & 1\\
1 & 2-\lambda\\
\end{vmatrix}
+
\begin{vmatrix}
-1 & 1\\
0 & 1\\
\end{vmatrix}
\right)
=
\lambda^3
\left(
-3+\lambda-1
\right)
=
\lambda^3(\lambda-4)
\]
$A$ heeft dus de eigenwaarde $0$ met multipliciteit $m(\lambda) = 4$.

\subsubsection*{Eigenvector(en)}
De eigenruimte $E_\lambda$ is de nulruimte van $A = A-\lambda\mathbb{I}_n$. We hoeven dus enkel $A-\lambda\mathbb{I}_n$ te rijreduceren om een oplossing te zien staan.
\[
\begin{pmatrix}
1 & 1 & 1 & 1\\
1 & 1 & 1 & 1\\
1 & 1 & 1 & 1\\
1 & 1 & 1 & 1\\
\end{pmatrix}
\rightarrow
\begin{pmatrix}
1 & 1 & 1 & 1\\
0 & 0 & 0 & 0\\
0 & 0 & 0 & 0\\
0 & 0 & 0 & 0\\
\end{pmatrix}
\Longrightarrow
E_0 = 
\left\lbrace
\begin{pmatrix}
-\lambda-\mu-\nu\\\lambda\\\mu\\\nu
\end{pmatrix}
\middle| \lambda,\mu,\nu\in\mathbb{R}
\right\rbrace
\]
\[
\begin{pmatrix}
-3 & 1 & 1 & 1\\
1 & -3 & 1 & 1\\
1 & 1 & -3 & 1\\
1 & 1 & 1 & -3\\
\end{pmatrix}
\rightarrow
\begin{pmatrix}
1 & 0 & 0 & -1\\
0 & 1 & 0 & -1\\
0 & 0 & 1 & -1\\
0 & 0 & 0 & 0\\
\end{pmatrix}
\Longrightarrow
E_4 = 
\left\lbrace
\begin{pmatrix}
\lambda\\\lambda\\\lambda\\\lambda
\end{pmatrix}
\middle| \lambda\in\mathbb{R}
\right\rbrace
\]

\subsection{Oefening 9}
We stellen allereerst de matrix van $L$ op ten opzichte van de standaardbasis. De lineaire combinatie die we zien in het voorschrift zetten we in de rijen van een $3\times 3$ matric.
\[
L_\epsilon = 
\begin{pmatrix}
0 & 1 & 1\\
1 & 1 & 0\\
1 & 0 & 1
\end{pmatrix}
\]
Uit de theorie weten we dat dit iets met de eigenwaarden en eigenvectoren te maken heeft. Sterker nog, dit is precies waarvoor eigenwaarden en eigenvectoren dienen. We zoeken de matrix van eigenvectoren $P$ zodat $AP = PB$, dan is $B$ een diagonaalmatrix \footnote{Zie het bewijs van Steling 5.7}.
Nu moeten we dus eerst een basis van eigenvectoren van $\mathbb{R}^n$ zoeken.
We lossen de karakteristieke vergelijking van $L$ op.
\[
\phi_L = 
\begin{vmatrix}
-\lambda & 1 & 1\\
1 & 1-\lambda & 0\\
1 & 0 & 1-\lambda
\end{vmatrix}
= 0
\]
\[
(1-\lambda)((-\lambda(1-\lambda)-1) +\lambda-1) = 0 \Leftrightarrow (1-\lambda)(\lambda^2-2)-=0
\]
De eigenwaarden van $L$ zijn dus $\{-1,1,2\}$. $L$ heeft een enkelvoudig spectrum dus $L$ is diagonaliseerbaar.
We zoeken nu de bijhorende eigenruimten.

\emph{$E_{-1}$}\\
\[
\begin{pmatrix}
1 & 1 & 1\\
1 & 2 & 0\\
1 & 0 & 2
\end{pmatrix}
\rightarrow
\begin{pmatrix}
1 & 0 & 2\\
0 & 1 & -1\\
0 & 0 & 0
\end{pmatrix}
\]
\[
\Rightarrow E_{-1} = 
\left\{ 
\lambda
\begin{pmatrix}
-2\\1\\1\\
\end{pmatrix}
| \lambda \in \mathbb{R}
\right\}
\]

\emph{$E_1$}\\
\[
\begin{pmatrix}
-1 & 1 & 1\\
1 & 0 & 0\\
1 & 0 & 0
\end{pmatrix}
\rightarrow
\begin{pmatrix}
1 & 0 & 0\\
0 & 1 & 1\\
0 & 0 & 0
\end{pmatrix}
\]
\[
\Rightarrow E_1 = 
\left\{ 
\lambda
\begin{pmatrix}
0\\-1\\1\\
\end{pmatrix}
| \lambda \in \mathbb{R}
\right\}
\]

\emph{$E_2$}\\
\[
\begin{vmatrix}
-2 & 1 & 1\\
1 & -1& 0\\
1 & 0 & -1
\end{vmatrix}
\rightarrow
\begin{pmatrix}
1 & 0 & -1\\
0 & 1 & -1\\
0 & 0 & 0
\end{pmatrix}
\]
\[
\Rightarrow E_2 = 
\left\{ 
\lambda
\begin{pmatrix}
1\\1\\1\\
\end{pmatrix}
| \lambda \in \mathbb{R}
\right\}
\]
Kiezen we nu drie eigenvectoren, elk uit een andere eigenruimte, dan en gebruiken we deze als basis $\beta$ voor $(\mathbb{R},\mathbb{R}[X]_{\le 2},+)$ dan is de matrix van $L$ ten opzichte van deze basis een diagonaalmatrix.
(Deze diagonaalmatrix is de matrix waarbij de eigenwaarden op de diagonaal staan)
\[
L_\beta = 
\begin{pmatrix}
-1 & 0 & 0\\
 0 & 1 & 0\\
 0 & 0 & 2
\end{pmatrix}
\]


\subsection{Oefening 10}

\subsubsection*{a}
De gegeven formules voor $T^\alpha_\alpha$ vallen te herschrijven in een matrix als:
\[
T^\alpha_\alpha = \left(
\begin{array}{c c c}
5 & 2 & 3 \\
2 & -1 & 0 \\
3 & 0 & 1 \\
\end{array}
\right)
\]
Hieruit kunnen we de karakteristieke veelterm halen met de formule $\det(\lambda * I_3 - T^\alpha_\alpha)$, waarbij $I_n$ de eenheidsmatrix uit $\mathbb{R}^{n\times n}$ is.

\[\det(\lambda * I_3 - T^\alpha_\alpha)=\]
\[\det\left(
\begin{array}{c c c}
\lambda-5 & -2 & -3 \\
-2 & \lambda+1 & 0 \\
-3 & 0 & \lambda-1 \\
\end{array}
\right)=\]
\[-3*\det\left(
\begin{array}{c c}
-2 & -3 \\
\lambda+1 & 0 \\
\end{array}
\right) + (\lambda-1)*\det\left(
\begin{array}{c c}
\lambda-5 & -2 \\
-2 & \lambda+1 \\
\end{array}
\right)=\]
\[\lambda^3 - 5*\lambda^2 - 14*\lambda=\]
\[\lambda*(\lambda-7)*(\lambda+2)\]

De nulpunten van deze vergelijking zijn de eigenwaarden van $T$. Deze zijn dus 0, -2 en 7.

\subsubsection*{b}

Om de eigenruimten te berekenen per eigenwaarde gaan we een eigenwaarde invullen in de matrix waar we de determinant van berekend hebben om onze eigenwaarden te bekomen. We maken hier vervolgens een homogeen stelsel van en bepalen het vectorvoorschrift voor de oplossingen van het stelsel. Dit voorschrift bepaalt de eigenruimte.\\

Voor eigenwaarde 0 bekomen we:
\[\left(
\begin{array}{c c c | c}
\lambda-5 & -2 & -3 & 0\\
-2 & \lambda+1 & 0 & 0\\
-3 & 0 & \lambda-1 & 0\\
\end{array}
\right)\]
\[\mapsto\left(
\begin{array}{c c c | c}
-5 & -2 & -3 & 0 \\
-2 & 1 & 0 & 0\\
-3 & 0 & -1 & 0\\
\end{array}
\right)\]
waar we na wat rijreduceren komen tot
\[\mapsto\left(
\begin{array}{c c c | c}
1 & 4 & 3 & 0 \\
0 & 3 & 2 & 0\\
0 & 0 & 0 & 0\\
\end{array}
\right)\]
De oplossingsverzameling van dit stelsel kunnen we schrijven als $\{(-\lambda,-2*\lambda,3*\lambda)|\lambda\in\mathbb{R}\}$, hetgeen een voorschrift is voor de co\"ordinaten van de eigenruimte met de gegeven basis $\alpha$. We bekomen dus:
\[E_0 = \{(-\lambda*v_1-2*\lambda*v_2+3*\lambda*v_3)|\lambda\in\mathbb{R}\}\]
\[=vct\{-v_1-2*v_2+3*v_3\}\]
Dit is de eigenruimte bij eigenwaarde 0.

Voor $-2$ en $7$ vinden we op een gelijkaardige manier de eigenruimten, en bekomen we:
\[E_{-2} = \{(-\lambda*v_1+2*\lambda*v_2+\lambda*v_3)|\lambda\in\mathbb{R}\}\]
\[=vct\{-v_1+2*v_2+v_3\}\]
\[E_{7} = \{(4*\lambda*v_1+\lambda*v_2+2*\lambda*v_3)|\lambda\in\mathbb{R}\}\]
\[=vct\{4*v_1+v_2+2*v_3\}\]

\subsubsection*{c}

Aangezien we een enkelvoudig spectrum hebben door de drie verschillende eigenwaarden kunnen we een basis kiezen van de gevonden eigenvectoren. Stellen we hiermee een basisveranderingsmatrix op met de vectoren als kolommen, dan krijgen we een diagonaalmatrix voor de afbeelding met de eigenwaarden op de diagonaal, in de volgorde waarin de eigenvectoren als kolommen gebruikt zijn.

\subsubsection*{d}

De matrix van de afbeelding wordt dus (we nemen hier de volgorde 0,-2 en 7 voor de eigenwaarden):

\[
T^\beta_\beta = \left(
\begin{array}{c c c}
0 & 0 & 0 \\
0 & -2 & 0 \\
0 & 0 & 7 \\
\end{array}
\right)
\]

\subsubsection*{e}

Voor de matrix van basisverandering gebruiken we dus de gevonden eigenruimten in volgorde van de eigenwaarden hierboven. We bekomen dus:

\[
Id^\alpha_\beta = \left(
\begin{array}{c c c}
-1 & -1 & 4 \\
-2 & 2 & 1 \\
3 & 1 & 2 \\
\end{array}
\right)
\]

\subsection{Oefening 11}
We beschouwen allereerst het voorschrift als co\"ordinaten volgens de standaardbasis.
\[
\begin{pmatrix}
a & b\\
c & d
\end{pmatrix}
\rightarrow
\begin{pmatrix}
a \\ b \\ c \\ d
\end{pmatrix}
\text{ en }
\begin{pmatrix}
a+b+d & a+b+c\\
b+c+d & a+c+d
\end{pmatrix}
\rightarrow
\begin{pmatrix}
a+b+d \\ a+b+c \\ b+c+d \\ a+c+d
\end{pmatrix}
\]
$T$ volgens de standaardbasis ziet er als volgt uit. 
\[
T_\epsilon =
\begin{pmatrix}
1 & 1 & 0 & 1\\
1 & 1 & 1 & 0\\
0 & 1 & 1 & 1\\
1 & 0 & 1 & 1
\end{pmatrix}
\]
De karakteristieke veelterm van $T$ is dus de volgende.
\[
\begin{vmatrix}
1-\lambda & 1 & 0 & 1\\
1 & 1-\lambda & 1 & 0\\
0 & 1 & 1-\lambda & 1\\
1 & 0 & 1 & 1-\lambda
\end{vmatrix}
= 0
\]
%TODO


\subsection{Oefening 16}
Met een rekenmachine vinden we dat het antwoord de volgende matrix is.
\[
A^{10}=
\begin{pmatrix}
1 & 0\\
-1023 & 1024\\
\end{pmatrix}
\]
Op het examen mag je echter geen rekenmachine gebruiken.
Het domweg manueel doen is natuurlijk niet wat je moet doen.

Stel dat $A$ diagonaliseerbaar is. Dan geldt dat er een inverteerbare $P$ bestaat zodat $B$ een diagonaalmatrix is en het volgende geldt.
\[
P^{-1}AP = B \Rightarrow A = PBP^{-1}
\]
$P$ is hier een matrix waarin een eigenbasis van $A$ in de kolommen staat. $B$ is de diagonaalmatrix waarbij de eigenwaarden van $A$ op de diagonaal staan.
We zoeken de tiende macht van $A$.
\[
A^{10} = (PBP^{-1})^{10} = PBP^{-1}PBP^{-1}...PBP^{-1} = PB\mathbb{I}_2B\mathbb{I}_2...\mathbb{I}_2BP^{-1} = PB^{10}P^{-1}
\]
Dit zou het wel heel makkelijk maken want een macht van een diagonaalmatrix is makkelijker te berekenen dan een macht van een willekeurige matrix \footnote{Zie de formule p 176 bovenaan.}.
We zoeken eerst de eigenwaarden van $A$.
\[
\phi_A =
\begin{vmatrix}
1-\lambda & 0\\
-1 & 2-\lambda
\end{vmatrix}
= 0
\Rightarrow (1-\lambda)(2-\lambda)=0
\]
De eigenwaarden van $A$ zijn dus $\{1,2\}$. Mits het spectrum van $A$ enkelvoudig is is $A$ diagonaliseerbaar.
We zoeken nu de eigenruimten van $A$.\\

\emph{$E_1$}\\
\[
\begin{pmatrix}
0 & 0\\
-1 & 1
\end{pmatrix}
\rightarrow
\begin{pmatrix}
1 & -1\\
0 & 0
\end{pmatrix}
\]
\[
\Rightarrow 
E_1 = 
\left\{
\lambda
\begin{pmatrix}
1\\1
\end{pmatrix}
| \lambda \in \mathbb{R}
\right\}
\]

\emph{$E_2$}\\
\[
\begin{pmatrix}
-1 & 0\\
-1 & 0
\end{pmatrix}
\rightarrow
\begin{pmatrix}
1 & 0\\
0 & 0
\end{pmatrix}
\]
\[
\Rightarrow
E_2 =
\left\{
\lambda
\begin{pmatrix}
0\\1
\end{pmatrix}
| \lambda \in \mathbb{R}
\right\}
\]
We bekomen nu $P$ en $B$.
\[
P = 
\begin{pmatrix}
1 & 0\\
1 & 1
\end{pmatrix}
\text{ en }
B = 
\begin{pmatrix}
1 & 0\\
0 & 2
\end{pmatrix}
\]
Om $P^{-1}$ te berekenen rijreduceren we de volgende matrix.
\[
\begin{pmatrix}
1 & 0 & 1 & 0\\
1 & 1 & 0 & 1
\end{pmatrix}
\rightarrow
\begin{pmatrix}
1 & 0 & 1 & 0\\
0 & 1 & -1 & 1
\end{pmatrix}
\text{ dus }
P^{-1} = 
\begin{pmatrix}
1 & 0\\
-1 & 1
\end{pmatrix}
\]
Nu kunnen we eenvoudig $A^{10}$ berekenen.
\[
A^{10}=
P
\cdot
\begin{pmatrix}
1 & 0\\
0 & 2
\end{pmatrix}^{10}
\cdot 
P^{-1}
=
P
\cdot
\begin{pmatrix}
1^{10} & 0\\
0 & 2^{10}
\end{pmatrix}
\cdot 
P^{-1}
=
\begin{pmatrix}
1 & 0\\
1 & 1
\end{pmatrix}
\cdot
\begin{pmatrix}
1 & 0\\
0 & 1024
\end{pmatrix}
\cdot 
\begin{pmatrix}
1 & 0\\
-1 & 1
\end{pmatrix}
\]
Als we dit uitrekenen zien we dat de rekenmachine gelijk had.
\[
A^{10}=
\begin{pmatrix}
1 & 0\\
-1023 & 1024\\
\end{pmatrix}
\]

\subsection{Oefening 17}

Te Bewijzen:
Als $P = NP$ en $P$ is diagonaliseerbaar, dan heeft $N$ een eigenruimte die minstens rang($P$)-dimensionaal is.
\begin{proof}
We weten dat P diagonaliseerbaar is. Hieruit kunnen we stellen dat Spec($P$) gelijk is aan $\{\lambda_1, \lambda_2,\cdots,\lambda_n\}$ met bijhorende verzameling eigenvectoren $\{v_1, v_2,\cdots,v_n\}$.\\
We kunnen voor elk van deze eigenwaarden stellen:
\[P*v_i = \lambda_i*v_i\]
en daaruit
\[N*P*v_i = N*\lambda_i*v_i\]
Aangezien geldt $P=NP$ kunnen we dit gelijkstellen tot:
\[\lambda_i*v_i = N*\lambda_i*v_i\]
\[\Rightarrow \lambda_i * (N*v_i - v_i) = 0\]
Nu geldt voor elke eigenwaarde $\lambda_i$ die niet nul is dat $(N*v_i - v_i)$ dus gelijk moet zijn aan nul. Dit valt te herschrijven naar:
\[N*v_i = v_i\]
Hieruit besluiten we dat voor elke eigenvector $v_i$ van $P$ met eigenwaarde $\lambda_i$ niet gelijk aan nul dat dit ook een eigenvector van $N$ is met eigenwaarde 1.
We zien in dat $N$ minstens evenveel eigenvectoren heeft als $P$ eigenwaarden heeft die niet nul zijn. Dit aantal is immers de rang van $P$. \\
De eigenruimte van $N$ is dus minstens rang($P$)-dimensionaal.
\end{proof}


\subsection{Oefening 18}
\subsubsection*{a)}
We weten dat de dimensies van $E_{\lambda_i}$ alledrie $1$ zijn.
We kiezen $v_1$ als basis van $E_{\lambda_1}$ zo ook $v_2$ en $v_3$ respectievelijk als basissen van $E_{\lambda_2}$ en $E_{\lambda_3}$.
Vervolgens kiezen we $\{v_1,v_2\}$ en $v_3$ respectievelijk als basissen van $E_{\mu_1}$ en $E_{\mu_2}$.
Beschouw nu $v_1$ als volgt.
\[
A\cdot B\cdot v_1 = A\cdot\mu_1v_1 = \mu_1Av_1= \mu_1\lambda_1v_1
\]
Dit houdt in dat $\mu_1\lambda_1$ een eigenwaarde is van $A\cdot B$ met eigenvector $v_1$ en eigenruimte $E_{\mu_1\lambda_1} = vct(\{v_1\})$. Op dezelfde manier vinden we dat $\mu_1*lambda_2$ en $\mu_2*lambda_3$ eigenwaarden zijn van $A\cdot B$, met respectievelik eigenvector $v2$, $v3$ en eigenruimten $E_{\mu_1*\lambda_2} = vct(\{v_2\})$ en $E_{\mu_2*  \lambda_3} = vct(\{v_3\})$

\subsubsection*{b)}
\[
B\cdot A\cdot v1
= B\cdot \lambda_1\cdot v1
= \lambda_1\cdot B\cdot v1
= \lambda_1\cdot \mu_1\cdot v1
= \mu_1\cdot \lambda_1\cdot v1
= \mu_1\cdot A\cdot v1
= A\cdot \mu_1\cdot v1
= A\cdot B\cdot v1
\]
\[
B\cdot A = A\cdot B
\]


\section{Opdrachten}

%\subsection{Opdracht 5.10 p 185}
%\label{5.10}
%TODO

\subsection{Opdracht 5.34 p 206}
\label{5.34}
Zij $\lambda$ een complexe niet-re\"ele eigenwaarde van een re\"ele vierkante matrix $A$.
\[
\exists x,y \in \mathbb{R}: \lambda = x +yi
\]

\subsubsection*{Te Bewijzen}
$\lambda$ heeft geen re\"ele eigenvectoren.

\subsubsection*{Bewijs}
\begin{proof}
Bewijs uit het ongerijmde.\\
We gaan ervan uit dat $v$ een re\"ele eigenvector is van $A$ met $\lambda$ als eigenwaarde.
\[
A\cdot v = \lambda v
\]
\[
A\cdot v = (x+yi) v= xv + yiv
\]
\[
A\cdot v - xv = yiv
\]
In bovenstaande gelijkheid is het linkerlid zeker re\"eel en het rechterlid zeker complex.
De gelijkheid geldt dus niet. 
\end{proof}


\subsection{Opdracht 5.42 p 211}
\label{5.42}
Zij $L_n$ een Leslie-matrix.
\[
L_n =
\begin{pmatrix}
f_1 & f_2 & \cdots & \cdots & f_2\\
o_1 & 0 & \cdots & \cdots & 0\\
0 & o_2 & \cdots & \cdots & 0\\
\vdots & \vdots & \ddots & & \vdots\\
\vdots & \vdots & & & \vdots\\
0 & 0 & \cdots & o_{n-1} & 0
\end{pmatrix}
\]

\subsubsection*{Te Bewijzen}
\[
\phi_{L_{n}} =X^{n} -\sum_{i=1}^n \left(\prod_{j=1}^{i-1} o_{j} \right)f_1X^{n-i}
\]

\subsubsection*{Bewijs}
\begin{proof}
\[
\begin{vmatrix}
f_1-\lambda\mathbb{I}_n & f_2 & \cdots & \cdots & f_2\\
o_1 & 0-\lambda\mathbb{I}_n & \cdots & \cdots & 0\\
0 & o_2 & \cdots & \cdots & 0\\
\vdots & \vdots & \ddots & & \vdots\\
\vdots & \vdots & & & \vdots\\
0 & 0 & \cdots & o_{n-1} & \lambda\mathbb{I}_n
\end{vmatrix}
= 0
\]
We bekomen de gezochtte veelterm door simpelweg de bovenstaande determinant recursief naar de onderste rij te ontwikkelen.
\end{proof}


\end{document}