\documentclass[lineaire_algebra_oplossingen.tex]{subfiles}
\begin{document}

\chapter{Oefeningen Hoofdstuk 6}

\section{Oefeningen 6.8}


\subsection{Oefening 1}
\subsubsection*{Te Bewijzen}
\[
(\mathbb{R},\mathbb{R}^{2\times 2},+,\langle \cdot,\cdot \rangle ) \text{ is een euclidische ruimte}
\]
Waarbij het inproduct als volgt gedefinieerd is.
\[
\langle A,B \rangle = Tr(A^T\cdot B)
\]

\subsubsection*{Bewijs}
Dit is een heel langdradig bewijs, maar het is hier toch eens volledig uitgeschreven bij wijze van voorbeeld.
\begin{proof}
$(\mathbb{R},\mathbb{R}^{2\times 2},+,Tr)$ is een eindig dimensionale vectorruimte.
We moeten dus enkel de eigenschappen van $\langle A,B \rangle = Tr(A^T,B)$ als inproduct nagaan.\\
Kies willekeurige $V_1,V_2,V, W \in \mathbb{R}^{2\times 2}$ en $\lambda_1,\lambda_2 \in \mathbb{R}^{2\times 2}$.

\begin{itemize}
\item Lineariteit in de eerste component.\\
\[
\langle \lambda_1V_1 + \lambda_2V_2,W \rangle
= Tr((\lambda_1V_1 + \lambda_2V_2)^T\cdot W)
= Tr((\lambda_1V_1^T + \lambda_2V_2^T)\cdot W)
\]
Bovenstaande gelijkheden gelden omwille van de gegeven definitie en de lineariteit van het transponeren van een matrix \footnote{Zie Eigenschap 1.16 p 28.}.
Volgende gelijkheid geldt omdat de spoorafbeelding een lineaire afbeelding is \footnote{Zie Voorbeeld 4.4 p 131 9.}.
\[
= Tr(\lambda_1V_1^T\cdot W + \lambda_2V_2^T\cdot W)
= \lambda_1Tr(V_1^T\cdot W) + \lambda_2Tr(V_2^T\cdot W)
= \lambda_1\langle V_1, W\rangle + \lambda_2\langle V_2, W\rangle
\]
Samengevat staat hierboven het volgende, wat precies betekent dat het gegeven inproduct lineair is in de eerste component.

\item Symmetrie\\
\[
\langle V,W\rangle = Tr(V^T\cdot W) = Tr(V\cdot W^T) = \langle V,W\rangle
\]

\item Positief\\
\[
\langle V,V\rangle = Tr(V^T\cdot V)
= \sum_{i=1}^n(V^T\cdot V)_{ii}
= \sum_{i=1}^n\sum_{j=1}^nV^T_{ij}V_{ji} \ge 0
\]

\item Definiet\\
\[
\langle V,V\rangle
= 0 \Leftrightarrow V=\vec{0}
\]
\begin{itemize}
\item $\Rightarrow$\\
\[
\langle V,V\rangle
= 0 \Rightarrow \sum_{i=1}^n\sum_{j=1}^nV^T_{ij}V_{ji} = 0 \Rightarrow V = \vec{0}
\]

\item $\Leftarrow$\\
\[
v = \vec{0} \Rightarrow \langle V,V\rangle = \langle \vec{0},\vec{0}\rangle
= \sum_{i=1}^n\sum_{j=1}^n\vec{0}_{ij}\vec{0}_{ji} = 0
\]

\end{itemize}

\end{itemize}
\end{proof}
\subsubsection*{Extra}
\[
\Vert A_1 \Vert 
= 1
\]
\[
\Vert A_2 \Vert 
= \sqrt{5}
\]
\[
\Vert A_3 \Vert
= \sqrt{14}
\]
\[
\Vert A_4 \Vert
= 2
\]
\[
\langle A_1,A_2 \rangle
= 2
\]
%TODO Inproducten
\[
\langle A_1,A_3 \rangle
=
\]
\[
\langle A_1,A_4 \rangle
=
\]
\[
\langle A_2,A_3 \rangle
=
\]
\[
\langle A_2,A_4 \rangle
=
\]
\[
\langle A_3,A_4 \rangle
=
\]
%TODO hoeken
\[
\widehat{A_1,A_2}
= \arccos\frac{2}{1\cdot \sqrt{5}}
= \arccos\frac{2\sqrt{5}}{5}
\]
\[
\widehat{A_1,A_3}
=
\]
\[
\widehat{A_1,A_4}
=
\]
\[
\widehat{A_2,A_3}
=
\]
\[
\widehat{A_3,A_4}
=
\]
%TODO afstanden
\[
\Vert A_2-A_1 \Vert
=
\]
\[
\Vert A_3-A_1 \Vert
=
\]
\[
\Vert A_4-A_1 \Vert
=
\]
\[
\Vert A_3-A_2 \Vert
=
\]
\[
\Vert A_4-A_3 \Vert
=
\]

\subsection{Oefening 8}
\subsubsection*{c}
We zoeken voor de deelruimte bepaald door $vct\{\bigl(\begin{smallmatrix}
1 & 0 \\
1 & 2
\end{smallmatrix}\bigr),\bigl(\begin{smallmatrix}
1 & 1 \\
1 & 1
\end{smallmatrix}\bigr),\bigl(\begin{smallmatrix}
2 & 2 \\
0 & 1
\end{smallmatrix}\bigr)\}$  het orthogonaal complement. Dit betekent dat we een andere deelruimte van $\mathbb{R}^{2\times2}$ zoeken waarvoor elk element loodrecht staat op de elementen uit de originele vectorruimte.

Deze wordt dus voortgebracht door een vector of alle vectoren $\bigl(\begin{smallmatrix}
a & b \\
c & d
\end{smallmatrix}\bigr)$ die voldoen aan:
\[ <\bigl(\begin{smallmatrix}
a & b \\
c & d
\end{smallmatrix}\bigr),\bigl(\begin{smallmatrix}
1 & 0 \\
1 & 2
\end{smallmatrix}\bigr)> = 0\]
\[ <\bigl(\begin{smallmatrix}
a & b \\
c & d
\end{smallmatrix}\bigr),\bigl(\begin{smallmatrix}
1 & 1 \\
1 & 1
\end{smallmatrix}\bigr)> = 0\]
\[ <\bigl(\begin{smallmatrix}
a & b \\
c & d
\end{smallmatrix}\bigr),\bigl(\begin{smallmatrix}
2 & 2 \\
0 & 1
\end{smallmatrix}\bigr)> = 0\]

Wat betekent dat een element van onze nieuwe deelruimte loodrecht staat op alle elementen van de basis van de oude, en dus ook op al hun lineaire combinaties. Dit geldt door de definitie van het standaardinproduct in $\mathbb{R}^{2\times2}$ als en slechts als:
\[\left\{
\begin{array}{l}
a+c+2*d=0\\
a+b+c+d=0\\
2*a+2*b+d=0
\end{array} \right.\]
We kunnen dit stelsel van gelijkheden omzetten in de matrix
\[\left(
\begin{array}{c c c c | c}
1 & 0 & 1 & 2 & 0\\
1 & 1 & 1 & 1 & 0\\
2 & 2 & 0 & 1 & 0
\end{array}
\right)\]
hetgeen we door rijreductie omzetten in
\[\left(
\begin{array}{c c c c | c}
1 & 0 & 0 & \frac{3}{2} & 0\\
0 & 1 & 0 & -1 & 0\\
0 & 0 & 1 & \frac{1}{2} & 0
\end{array}
\right)\]

Kiezen we $d$ als vrije variabele, dan moeten alle elementen van de nieuwe deelruimte voldoen aan het voorschrift $\bigl(\begin{smallmatrix}
\frac{-3}{2}*\lambda & \lambda \\
\frac{-1}{2}*\lambda & \lambda
\end{smallmatrix}\bigr)$ met $\lambda \in \mathbb{R}$.

We kunnen dus door een invulling van lambda stellen dat het orthogonaal complement de deelruimte is met als voorschrift:
\[
vct\{\bigl(\begin{smallmatrix}
-3 & 2 \\
-1 & 2
\end{smallmatrix}\bigr)\}
\]

\section{Opdrachten}
\subsection{Opdracht 6.8 p 227}
Kies willekeurige elementen $x_1,x_2,x,y \in \mathbb{R}^{2}$ en $\lambda_1,\lambda_2,\lambda \in \mathbb{R}$.
\begin{enumerate}
\item
\begin{proof}
\begin{enumerate}
\item Lineair in de eerste component
\[
\langle \lambda_1x_1 + \lambda_2x_2,y \rangle_1
\]
\[
= 3(\lambda_1x_1 + \lambda_2x_2)_1y_1 - 2(\lambda_1x_1 + \lambda_2x_2)_1y_2-2(\lambda_1x_1 + \lambda_2x_2)_2y_1+2(\lambda_1x_1 + \lambda_2x_2)_2y_2
\]
\[
= \lambda_1(3x_{11}y_1-2x_{11}y_2-2x_{12}y_1+2x_{12}y_2) + \lambda_2(3x_{21}y_1-2x_{21}y_2-2x_{22}y_1+2x_{22}y_2)
\]
\[
= \lambda_1\langle x_1,y \rangle_1
+  \lambda_2\langle x_2,y \rangle_1
\]

\item Symmetrisch
\[
\langle x , y \rangle_1
=
3x_{1}y_1-2x_{1}y_2-2x_{2}y_1+2x_{2}
= \langle y , x \rangle_1
\]

\item Positief
\[
\langle x,x \rangle_1  = x_1^2\ge 0
\]

\item Definiet
\[
\langle x,x \rangle_1 = 0 \Leftrightarrow x= \vec{0}
\]

\begin{itemize}
\item $\Rightarrow$
\[
\langle x,x \rangle_1 = 0 
\]
\[
x_1^2=0
\]
\[
\Rightarrow x= \vec{0}
\]

\item $\Leftarrow$
\[
x= \vec{0}
\]
\[
x_1^2=0
\]
\[
 \Rightarrow \langle x,x \rangle_1 = 0
\]
\end{itemize}
\end{enumerate}
\end{proof}

\item
Dit inproduct is niet positief. 
Kies bevoorbeeld $x$ als volgt.
\[
x = 
\begin{pmatrix}
1\\1
\end{pmatrix}
\]
\[
\langle x,x \rangle_1 = -1 <0
\]
\end{enumerate}




\end{document}