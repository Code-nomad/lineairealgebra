\documentclass[lineaire_algebra_oplossingen.tex]{subfiles}
\begin{document}
\noindent De errata van de cursus zijn hier nog eens opgesomd omdat ik ze persoonlijk niet helemaal duidelijk vond op toledo.

\section{Voorbeelden p. 18 puntje (2)}
In de uitdrukking voor $z$ staat een fout. Dit is de juiste uitdrukking.
\[
z = -\frac{1}{3}\mu + \frac{2}{3}
\]
De juiste oplossingsverzameling wordt dan de volgende.
\[
V = \left\{\left(-\frac{7}{3}\mu+\frac{5}{3},\mu,-\frac{1}{3}\mu + \frac{2}{3}\right) | \mu \in \mathbb{R}\right\}
\]

\section{Bewijs van Stelling 2.10 p. 62}
In de matrix waarboven ``Eerst een speciaal geval...'' staat een fout. Dit is de juiste matrix.
\[
\begin{pmatrix}
1 & 2 & \cdots & i & i+1 & \cdots & n\\
a_1 & a_2 & \cdots & a_{i+1} & a_i & \cdots & a_n\\ 
\end{pmatrix}
\]

\section{Voorbeeld bij de determinant van Wronski p. 70}
Deze fout staat nog niet in de errata-lijst op Toledo.\\
De uitkomst van (b) is fout, er moet nog een minteken voor. Dit wordt dus het volgende.
\[
W(e^{\lambda_1x}, e^{\lambda_2x}, e^{\lambda_3x}) = -(\lambda_1-\lambda_2) (\lambda_1-\lambda_3) (\lambda_2-\lambda_3)
e^{(\lambda_1 + \lambda_2 + \lambda_3)x}
\]

\section{Inleiding van sectie 3.2 p. 94}
Op de tweede regel staat een fout. Er staat $C^0[a,b] \subset C^1[a,b]$.
Dit zou betekenen dat de verzameling van alle functies in $C$ die nul keer afleidbaar zijn een deelverzameling is van de verzamling van alle functies in $C$ die \'e\'en keer afleidbaar zijn.
Dit deze inclusie geldt enkel in de omgekeerde richting. Wat er moest staan is dus het volgende.
\[
C^1[a,b] \subset C^0[a,b]
\]

\section{Stelling 3.45 p. 111}
In deze stelling staat een fout in de notatie. Er staat $\beta = v_1,v_2,...,v_n$.
In het bewijs wordt deze basis gebruikt maar heten de vectoren ervan $e_i$.
Wat er hoorde te staan is het volgende.
\[
\beta = e_1,e_2,...,e_n
\]

\section{Toepassing p. 115}
In de beide puntjes staat een verwarrende fout, een verwijzing naar iets onbestaand.
Er staat op het einde ``... zoals in (b)'' en ``zoals in (c)''.
Dit moet simpelweg verwijderd worden.

\section{Redenering boven 4.7 p. 162}
In de redenering op pagina 162 staat een fout. Er staat ``... een bijectief verband (1-1-verband) ...''.
Dit is een heel subtiele fout, maar wat er moet staan is het volgende. ``... op deze manier er een isomorfisme is ...''. 

\section{Oefening 6(a) p. 167}
In de opgave van oefening 4.8.6.a staat een fout die de opgave wel eens onbegrijpelijk zou kunnen maken.
Er staat ``... een voortbrengend deel voor $L$. ...''.
Een voortbrengend deel voor een lineaire afbeelding is natuurlijk geen zinvolle uitdrukking.
Wat er moet staan is het volgende. ``... een voortbrengend deel voor $V$. ...''.

\section{Oefening 23 p. 170}
In de opgave van oefening 4.8.23 staat een fout die de opgave wel eens onbegrijpelijk zou kunnen maken.
Er staat ``... Toon nu aan dat $L = ker(L) \oplus Im(L)$.''.
Wat er moet staan is het volgende.
``Toon nu aan dat $V = ker(L) \oplus Im(L)$ geldt.''.

\section{Voorbeeld 3(a) p. 198}
Op het einde van (a) op pagina 198 staat iets verwarrend.
Er staat ``... de matrix van basisverandering van de basis $\alpha$ naar de basis van eigenvectoren.''.
Dit is het omgekeerde van wat het moet zijn.
Wat er moet staan is het volgende.
``... de matrix van basisverandering van de basis van eigenvectoren naar $\alpha$.''

\section{Opdracht 5.42 p. 211}
In opdracht 5.42 staat een fout. De exponent van $X_{n-1}$ in de uitdrukking voor $\phi_{L_{n}}(X)$ moet $n-1$ zijn in plaats van $1$.

\end{document}

