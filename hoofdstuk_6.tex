\documentclass[lineaire_algebra_oplossingen.tex]{subfiles}
\begin{document}

\chapter{Hoofdstuk 6}

\section{Bewijzen uit de cursus}

\subsection{Opmerking 6.2 p 222}
Zij $(\mathbb{R},V,+)$ een reële vectorruimte.
\subsubsection*{Te Bewijzen}
Een inproduct is lineair in de tweede component.
\[
\forall w_1,w_2,v\in V, \forall \lambda_1,\lambda_2 \in \mathbb{R}:
\langle v,\lambda_1w_1+\lambda_2w_2\rangle = \lambda_1\langle v,w_1\rangle + \lambda_2\langle v,w_2\rangle
\]
\subsubsection*{Bewijs}
\begin{proof}
Kies drie willekeurige vectoren $w_1,w_2,v\in V$ en beschouw het volgende inproduct. Door lineariteit in de eerste component geldt de gelijkheid.
\[
\langle \lambda_1w_1+\lambda_2w_2,v\rangle = \lambda_1\langle w_1,v\rangle + \lambda_2\langle w_2,v\rangle
\]
Door symmetrie geldt dat deze twee leden ook gelijk zijn aan de volgende.
\[
\langle v,\lambda_1w_1+\lambda_2w_2\rangle = \lambda_1\langle v,w_1\rangle + \lambda_2\langle v,w_2\rangle
\]
\end{proof}

\subsection{Voorbeeld 6.6 p 224}
Definieer $\langle f,g\rangle$ als volgt.
\[
\langle f,g\rangle = \int_a^bf(x)\cdot g(x)dx
\]
Beschouw de vectorruimte $(\mathbb{R},C[a,b],+)$.
\subsubsection*{Te Bewijzen}
$\langle f,g\rangle$ is een inproduct.
\subsubsection*{Bewijs}
\begin{proof}
\begin{itemize}
\item $\langle f,g\rangle$ is lineair in de eerste component.
\[
\forall v_1,v_2,w\in V, \forall \lambda_1,\lambda_2 \in \mathbb{R}:
\langle \lambda_1v_1+\lambda_2v_2,w\rangle = \int_a^b\lambda_1 v_1(x)\cdot \lambda_2 v_2(x)dx
\]
\[
= \lambda_1\int_a^b v_1(x)\cdot \lambda_2\int_a^b v_2(x)dx = \lambda_1\langle v_1,w\rangle + \lambda_2\langle v_2,w\rangle
\]

\item $\langle f,g\rangle$ is symmetrisch.
\[
\forall v,w\in V: \langle v,w\rangle = \int_a^bv(x)\cdot w(x)dx= \int_a^bw(x)\cdot v(x)dx = \langle w,v\rangle
\]

\item $\langle f,g\rangle$ is positief.
\[
\forall v\in V: \langle v,v\rangle = \int_a^bv(x)\cdot v(x)dx \ge 0
\]
Dit is positief als want het is een kwadraat, namelijk van $\int_a^bv(x)dx$

\item $\langle f,g\rangle$ is definiet.
\[
\forall v\in V: \langle v,0\rangle = \vec{0}
\]
\[
\Leftrightarrow \int_a^bv(x)\vec{0}(x) dx = 0
\]
\end{itemize}
\end{proof}

\subsection{Stelling 6.11 p 228}
Zij $(\mathbb{R},V,+,\langle\cdot,\cdot\rangle)$ een inproductruimte met bijhorende norm.
\subsubsection*{Te Bewijzen}
\begin{enumerate}
\item 
\[
\forall \lambda\in\mathbb{R}, v\in V: \Vert\lambda v\Vert = \vert\lambda\vert\cdot \Vert v\Vert
\]

\item
\[
\forall v\in V: \Vert v\Vert\ge 0
\]

\item
\[
\forall v\in V: \Vert v\Vert\ge 0 \Leftrightarrow v=\vec{0}
\]
\end{enumerate}
\subsubsection*{Bewijs}
\begin{proof}
Direct bewijs.\\
Kies een willekeurige $v \in V$ en een $\lambda \in \mathbb{R}$
\begin{enumerate}
\item
\[
\Vert\lambda v\Vert = \sqrt{\langle \lambda v,\lambda v\rangle} = 
\sqrt{\lambda^2 \langle v,v\rangle}= \lambda\sqrt{\langle v,v\rangle} = \vert\lambda\vert\cdot \Vert v\Vert
\]

\item
\[
\Vert v\Vert= \sqrt{\langle v,v\rangle} \ge 0
\]
Een wortel is steeds positief en een inproduct is positief dus de wortel kan steeds getrokken worden \footnote{Zie Definitie 6.1 p 222}0.

\item
\item $\Rightarrow$\\
\[
\Vert v\Vert = 0 \Leftrightarrow \sqrt{\langle v,v\rangle}=0
\]
Dit kan enkel waar zijn als hetgeen onder de wortel ook nul is. Dit is zo omdat het inproduct definiet is \footnote{Zie Definitie 6.1 p 222}. 
\end{enumerate}
\end{proof}

\subsection{Stelling 6.14 p 229}
Zij $(\mathbb{R},V,+,\langle\cdot,\cdot\rangle)$ een inproductruimte.
\subsubsection*{Te Bewijzen}
\begin{enumerate}
\item 
\[
\forall v,w \in V: \vert\langle v,w\rangle\vert \le \Vert v\Vert\cdot \Vert w\Vert
\]
\item
\[
\forall v,w \in V: (\exists \lambda\in\mathbb{R} v = \lambda w) \Rightarrow \vert\langle v,w\rangle\vert = \Vert v\Vert\cdot \Vert w\Vert
\]
\end{enumerate}

\subsubsection*{Bewijs}
\begin{proof}
Bewijs door gevalsonderscheid.\\
Als $v$ en $w$ nulvectoren zijn gelden beide beweringen.
\[
\vert\langle \vec{0},\vec{0}\rangle\vert =0= \Vert \vec{0}\Vert\cdot \Vert \vec{0}\Vert
\]
We bewijzen de beweringen nu nog voor de andere gevallen.

\begin{enumerate}
\item
We weten dat voor willekeurige $v,w\in V$ geldt dat $\langle v + \lambda w,v+\lambda w\rangle \ge 0$. We werken dit uit.
\[
\langle v + \lambda w,v+\lambda w\rangle = \langle v,v+\lambda w \rangle + \lambda \langle w,v+\lambda w\rangle = \langle v,v \rangle + \lambda \langle v,w \rangle + \lambda \langle w,v\rangle + \lambda^2\langle w,w\rangle
\]
\[
\langle v,v \rangle + 2\lambda \langle v,w \rangle + \lambda^2\langle w,w\rangle \ge 0
\]
Het linker lid van deze ongelijkheid is een tweedegraads veelterm waarvan we weten dat ze altijd positief is. Bijgevolg is de discriminant ervan negatief.
\[
4\langle v,w\rangle^2-4\langle v,v\rangle\langle w,w\rangle \le 0
\]
\[
\langle v,w\rangle^2-\langle v,v\rangle\langle w,w\rangle \le 0
\]
\[
\langle v,w\rangle^2 \le \langle v,v\rangle\langle w,w\rangle 
\]
Voor diegenen die dit gefoefel vinden, u heeft gelijk.

\item
Als $w$ en $v$ lineair afhankelijk zijn dan bestaat er dus een $\lambda\in\mathbb{R}$ zodat $v=\lambda w$.
\[
|\langle \lambda w, w \rangle| = |\lambda  \langle w, w \rangle| = |\lambda\sqrt{\langle w, w \rangle^2}| = |\sqrt{\lambda^2\langle w, w \rangle^2}| = |\sqrt{\langle \lambda w, \lambda w \rangle \cdot \langle w, w \rangle }| \]
\[
= |\sqrt{\langle \lambda w, \lambda w \rangle} \sqrt{\langle w, w \rangle }| = |\sqrt{\langle v, v \rangle} \sqrt{\langle w, w \rangle }| = \Vert v\Vert\cdot \Vert w\Vert
\]
\end{enumerate}
\end{proof}

\subsection{Definitie 6.16 p 231}
Zij $(\mathbb{R},V,+,\langle\cdot,\cdot\rangle)$ een inproductruimte.
\subsubsection*{Te Bewijzen}
Voor elke $v,w\in V$ bestaat er een unieke hoek $\theta$ zodat de volgende gelijkheid geldt.
\[
\cos\theta = \frac{\langle v,w\rangle}{\Vert v\Vert\cdot \Vert w\Vert}
\]
\subsubsection*{Bewijs}
\begin{proof}
Direct bewijs\\
We weten dat voor elke $v,w\in V$ het volgende geldt.
\[
\vert\langle v,w\rangle\vert \le \Vert v\Vert\cdot \Vert w\Vert
\]
Hieruit volgt het volgende, zorg dat je dit begrijpt.
\[
-1 \le \frac{\langle v,w\rangle}{\Vert v\Vert\cdot \Vert w\Vert} \le 1
\]
We weten dat voor elk getal tussen $-1$ en $1$ er een hoek bestaat zodat de cosinus ervan dat getal is.
\end{proof}

\subsection{Stelling 6.19 p 232}
Zij $(\mathbb{R},V,+,\langle\cdot,\cdot\rangle)$ een inproductruimte met bijhorende norm $||\cdot ||$.
\subsubsection*{Te Bewijzen}
\[
\forall v,w \in V: \Vert v+w \Vert \le \Vert v\Vert + \Vert w\Vert
\]
\subsubsection*{Bewijs}
\begin{proof}
We weten dat $\Vert v+w \Vert$ en $\Vert v\Vert + \Vert w\Vert$ positief zijn. Bijgevolg is het voldoende te bewijzen dat $\Vert v+w \Vert^2 \le (\Vert v\Vert + \Vert w\Vert)^2$ geldt.
\[
 \Vert v+w \Vert^2 = \sqrt{\langle v+w,v+w\rangle}^2 = \langle v+w,v+w\rangle
\]
Let op, dit geldt enkel omdat $\Vert v+w \Vert$ positief is.
\[
= \langle v,v\rangle + \langle v,w\rangle + \langle w,v \rangle + \langle w,w \rangle =  \langle v,v\rangle + 2\langle v,w\rangle + \langle w,w \rangle
\]
Passen we nu drie keer de ongelijkheid van Cauchy-Schwartz toe dan verkrijgen we het volgende.
\[
\le \Vert v\Vert^2 + 2\Vert v\Vert \Vert w\Vert + \Vert w\Vert^2 = (\Vert v\Vert + \Vert w\Vert)^2
\]
\end{proof}

\subsection{Opmerking 6.20 p 232}
Zij $(\mathbb{C},V,+,\langle\cdot,\cdot\rangle)$ een hermetische ruimte.
\subsubsection*{Te Bewijzen}
\[
\forall v,w \in V: \Vert v+w \Vert \le \Vert v\Vert + \Vert w\Vert
\]
\subsubsection*{Bewijs}
\begin{proof}
Merk allereerst op dat het wel degelijk zin heeft om dit te bewijzen. Hoewel het hermetisch product beelden heeft in $\mathbb{C}$ heeft de norm nog steeds enkel beelden in $\mathbb{R}$.
We weten dat $\Vert v+w \Vert$ en $\Vert v\Vert + \Vert w\Vert$ positief zijn. Bijgevolg is het voldoende te bewijzen dat $\Vert v+w \Vert^2 \le (\Vert v\Vert + \Vert w\Vert)^2$ geldt.
\[
 \Vert v+w \Vert^2 = \sqrt{\langle v+w,v+w\rangle}^2 = \langle v+w,v+w\rangle = \langle v+w,v+w\rangle
\]
De lineairiteit in de tweede component geldt vanuit de definitie \footnote{Zie Definitie 6.3 p 223}. Over de eerste component weten we dat de toegevoegde lineariteit geldt \footnote{Zie Opmerking 6.4 p 223}. (In dit geval is de toegevoegde lineariteit voldoende omdat $\lambda = 1 \in \mathbb{R}$.)
\[
= \langle v+w,v\rangle + \langle v+w,w\rangle = \langle v,v\rangle + \langle v,w\rangle + \langle w,v \rangle + \langle w,w \rangle = \langle v,v\rangle + \langle v,w\rangle + \overline{\langle v,w \rangle} + \langle w,w \rangle
\]
We passen nu twee keer de ongelijkheid van Cauchy-Schwartz toe (op de eerste en de laatste term). Verder weten we ook nog dat de volgende gelijkheid geldt. We kunnen dezelfde ongelijkheid dus ook gebruiken voor de middelste termen.
\[
\langle v,w\rangle + \overline{\langle v,w \rangle} \le \vert\langle v,w\rangle\vert
\]
\[
\langle v,v\rangle + \langle v,w\rangle + \overline{\langle v,w \rangle} + \langle w,w \rangle \le \Vert v\Vert^2 + 2\Vert v\Vert \Vert w\Vert + \Vert w\Vert^2 = (\Vert v\Vert + \Vert w\Vert)^2
\]

\end{proof}

\subsection{Stelling 6.23 p 234}
Zij $(\mathbb{R},V,+,\langle\cdot,\cdot\rangle)$ een inproductruimte en zij $\alpha = \{v_1,v_2,...,v_k\}$ een deelverzameling van $k$ vectoren uit $\alpha$ zodat geen enkele $v_i$ de nulvector is en zodat voor twee verschillende $v_i,v_j (i\neq j)$ orthogonaal zijn ($\langle v_i,v_j \rangle = 0$). 
\subsubsection*{Te Bewijzen}
$\alpha$ is vrij.
\subsubsection*{Bewijs}
\begin{proof}
Bewijs uit het ongerijmde.\\
Stel dat de vectoren uit $\alpha$ lineaire afhankelijk zijn dan bestaan er $\lambda_i \in \mathbb{R}$ zodat de volgende gelijkheid opgaat.
\[
\sum_{i=1}^k \lambda_iv_i = \vec{0}
\]
Nu komt er gefoefel. Voor elke $i, j$ met $i\neq j$. (Noem $\mu_i =-\lambda_i $.)
De eerste gelijkheid geldt uit lineariteit van het inproduct \footnote{Zie opmerking 6.2 p 222 (ii)}.
\[
0 = \langle v_i, v_j \rangle = \langle v_i, \lambda_jv_j \rangle = \langle v_i, -\sum_{l=1, l\neq j}^k \lambda_lv_l \rangle = \sum_{l= 1}^k\langle v_i, \mu_lv_l\rangle = 0 + ... + 0 + \langle v_i,\mu_i v_i\rangle + 0 + ... + 0
\]
\[
= \mu_i \langle v_i,v_i\rangle = \mu_i \Vert v_i\Vert^2
\]
We weten dat $\Vert v_i\Vert^2$ strikt positief is aangezien $v_i \neq \vec{0}$. Hieruit volgt dat elke $\mu_i = -\lambda_i = 0$. Dit betekent dat $\alpha$ vrij is, en dat is in contradictie met de aanname.
\end{proof}

\subsection{Stelling 6.26 p 236}
Zij $(\mathbb{R}, V,+, \langle \cdot,\cdot \rangle)$ een euclidische ruimte en zij $\beta = \{v_1,v_2,...,v_n\}$ een \emph{orthonormale} basis van $V$. Zij $x,y\in \mathbb{R}^n$ twee willekeurige coördinaatvectoren ten opzichte van $\beta$. $x$ en $y$ zien er uit als volgt.
\[
x = (x_1,x_2,...,x_n) \text { en } y = (y_1,y_2,...,y_n)
\]
\subsubsection*{Te Bewijzen}
\[
\langle x,y\rangle = \sum_{i=1}^nx_iy_i
\]
\subsubsection*{Bewijs}
\begin{proof}
\[
\langle x,y\rangle = \left\langle \sum_{i=1}^nx_iv_i,\sum_{j=1}^ny_jv_j \right\rangle
\]
Het inproduct is lineaire in zowel de eerste als de tweede component\footnote{Zie Definitie 6.1 p 222 en opmerking 6.2 p 222.}.
\[
= \sum_{i=1}^nx_i \left\langle v_i,\sum_{j=1}^ny_jv_j \right\rangle
\]
\[
= \sum_{i=1}^nx_i \sum_{j=1}^n y_j \left\langle v_i,v_j \right\rangle
\]
We weten dat voor elke $i$ en $j$ zodat $i\neq j$ geldt dat $\langle v_j, v_i \rangle = 0$. Bovendien, als $i=j$ dan geldt $\langle v_j, v_i \rangle = \langle v_i, v_i \rangle = \Vert v\Vert^2$.
Omdat $\beta$ een orthonormale basis is is $i$-de component van de basisvector precies gelijk aan de norm van die vector en ook gelijk aan $1$. 
\[
= \sum_{i=1}^nx_i  (y_10+...+y_{i-1}0+y_i\Vert v_i\Vert^2+y_{i+1}0+...+y_n)
\]
\[
= \sum_{i=1}^n x_i y_i \Vert v_i\Vert^2
\]
\[
= \sum_{i=1}^nx_i y_j
\]
\end{proof}

\subsection{Stelling 6.27 p 238}
Zij $(\mathbb{R}, V,+, \langle \cdot,\cdot \rangle)$ een euclidische ruimte en zij $\beta = \{v_1,v_2,...,v_n\}$ een willekeurige basis van $V$.

\subsubsection*{Te Bewijzen}
$\beta$ kan omgevormd worden tot een orthonormale basis $\gamma = \{u_1,u_2,...,u_n\}$.

\subsubsection*{Bewijs}
\begin{proof}
Bewijs door constructie.\\
We zullen eenvoudigweg het algoritme beschrijven en de correctheid ervan bewijzen.\\\\
We normeren de eerste vector $v_1$ door ze te delen door zijn norm.
\[
u_1 = \frac{1}{\Vert v_1\Vert}v_1
\]
We construeren nu $v_2'$ zodat ze loodrecht staat op $u_1$.
\[
v_2' = v_2 - \langle v_2,v_1\rangle u_1
\]
$v_2'$ is zeker niet nul omdat $u_1$ ofwel niet op dezelfde rechte ligt als $v_2$, ofwel is $v_2' = \vec{0}$. $v_2'$ staat nu wel degelijk loodrecht op $u_1$.
\[
\langle v_2',u_1\rangle = \langle v_2 - \langle v_2,v_1\rangle u_1,u_1\rangle = \langle v_2,u_1\rangle - \langle\langle v_2,v_1\rangle u_1,u_1\rangle
\]
\[
= \langle v_2,u_1\rangle - \langle v_2,v_1\rangle \langle u_1,u_1\rangle = \langle v_2,u_1\rangle - \langle v_2,v_1\rangle \Vert u_1\Vert = \langle v_2,u_1\rangle - \langle v_2,v_1\rangle = 0
\]
Merk op dat $u_1$ echt genormeerd moet zijn, opdat dit zou werken. Tenslotte normeren we $v_2'$ nog om $u_2$ te bekomen.
\[
u_2 = \frac{1}{\Vert v_2'\Vert}v_2' = \frac{1}{\Vert v_2 - \langle v_2,v_1\rangle u_1\Vert}(v_2 - \langle v_2,v_1\rangle u_1)
\]
Dit proces kunnen we verder zetten. We beschrijven nu de stappen die je moet ondernemen in iteratie $k+1$.
We construeren $v_{k+1}'$ als volgt.
\[
v_{k+1}' = v_{k+1} - \langle v_{k+1},u_1 \rangle u_1 - \langle v_{k+1},u_2 \rangle u_2 - ... - \langle v_{k+1},u_k \rangle u_k
= v_{k+1} - \sum_{i=1}^k \langle v_{k+1},u_i \rangle
\]
Beschouw de sommatie in deze formule als \'e\'en vector. Nu is de redenering om te besluiten dat $v_{k+1}'$ dezelfde als die bij $v_2'$.
$v_{k+1}'$ is staat inderdaad loodrecht op alle vectoren $v_i$ met $i\le k$.

\[
\langle v_{k+1}',u_i\rangle  = \left\langle v_{k+1} - \sum_{j=1}^k \langle v_{k+1},u_j \rangle u_j,u_i\right\rangle
=
\langle v_{k+1},u_i\rangle -
\left\langle \sum_{j=1}^k \langle v_{k+1},u_j \rangle u_j,u_i\right\rangle
\]
\[
=
\langle v_{k+1},u_i\rangle -
\sum_{j=1}^k
\left\langle  \langle v_{k+1},u_j \rangle u_j,u_i\right\rangle
=
\langle v_{k+1},u_i\rangle -
\sum_{j=1}^k
\langle v_{k+1},u_j \rangle
\left\langle u_j,u_i\right\rangle
\]
In de laatste som is elke term behalve de term waarbij $i=j$ geldt nul.
\[
\langle v_{k+1},u_i\rangle -
\langle v_{k+1},u_i\rangle
=0
\]
We kunnen nu $v_{k+1}'$ ook normeren door te delen door de norm.
\[
u_{k+1} = \frac{1}{\Vert v_{k+1}'\Vert}v_{k+1}'
\]
Omdat $V$ eindig dimensionaal is zal dit algoritme zeker stoppen. Het resultaat is $\{u_1,u_2,...,u_n\}$, een orthonormale basis.


\end{proof}







\section{Oefeningen 6.8}






\end{document}