\documentclass[lineaire_algebra_oplossingen.tex]{subfiles}
\begin{document}

\chapter{Hoofdstuk 6}

\section{Bewijzen uit het boek}

\subsection{Opmerking 6.2 p 222}
Zij $(\mathbb{R},V,+)$ een reële vectorruimte.
\subsubsection*{Te Bewijzen}
Een inproduct is lineair in de tweede component.
\[
\forall w_1,w_2,v\in V, \forall \lambda_1,\lambda_2 \in \mathbb{R}:
\langle v,\lambda_1w_1+\lambda_2w_2\rangle = \lambda_1\langle v,w_1\rangle + \lambda_2\langle v,w_2\rangle
\]
\subsubsection*{Bewijs}
\begin{proof}
Kies drie willekeurige vectoren $w_1,w_2,v\in V$ en beschouw het volgende inproduct. Door lineariteit in de eerste component geldt de gelijkheid.
\[
\langle \lambda_1w_1+\lambda_2w_2,v\rangle = \lambda_1\langle w_1,v\rangle + \lambda_2\langle w_2,v\rangle
\]
Door symmetrie geldt dat deze twee leden ook gelijk zijn aan de volgende.
\[
\langle v,\lambda_1w_1+\lambda_2w_2\rangle = \lambda_1\langle v,w_1\rangle + \lambda_2\langle v,w_2\rangle
\]
\end{proof}

\subsection{Voorbeeld 6.6 p 224}
Definieer $\langle f,g\rangle$ als volgt.
\[
\langle f,g\rangle = \int_a^bf(x)\cdot g(x)dx
\]
Beschouw de vectorruimte $(\mathbb{R},C[a,b],+)$.
\subsubsection*{Te Bewijzen}
$\langle f,g\rangle$ is een inproduct.
\subsubsection*{Bewijs}
\begin{proof}
\begin{itemize}
\item $\langle f,g\rangle$ is lineair in de eerste component.
\[
\forall v_1,v_2,w\in V, \forall \lambda_1,\lambda_2 \in \mathbb{R}:
\langle \lambda_1v_1+\lambda_2v_2,w\rangle = \int_a^b\lambda_1 v_1(x)\cdot \lambda_2 v_2(x)dx
\]
\[
= \lambda_1\int_a^b v_1(x)\cdot \lambda_2\int_a^b v_2(x)dx = \lambda_1\langle v_1,w\rangle + \lambda_2\langle v_2,w\rangle
\]

\item $\langle f,g\rangle$ is symmetrisch.
\[
\forall v,w\in V: \langle v,w\rangle = \int_a^bv(x)\cdot w(x)dx= \int_a^bw(x)\cdot v(x)dx = \langle w,v\rangle
\]

\item $\langle f,g\rangle$ is positief.
\[
\forall v\in V: \langle v,v\rangle = \int_a^bv(x)\cdot v(x)dx \ge 0
\]
Dit is positief als want het is een kwadraat, namelijk van $\int_a^bv(x)dx$

\item $\langle f,g\rangle$ is definiet.
\[
\forall v\in V: \langle v,0\rangle = \vec{0}
\]
\[
\Leftrightarrow \int_a^bv(x)\vec{0}(x) dx = 0
\]
\end{itemize}
\end{proof}









\section{Oefeningen 6.8}






\end{document}