\documentclass[lineaire_algebra_oplossingen.tex]{subfiles}
\begin{document}

\chapter{Theorie Hoofdstuk 4}
\section{Bewijzen uit de cursus}
\subsection{Lemma 4.2 p 130}
\subsubsection*{Te bewijzen}
Een afbeelding $L:V\rightarrow W$ tussen re\"ele vectorruimten is een lineaire afbeelding a.s.a.
$$L(\lambda_1v_1+\lambda_2v_2) = \lambda_1L(v_1)+\lambda_2L(V_2)\ \text{ voor alle } \lambda_1,\lambda_2 \in \mathbb{R} \text{ en voor alle } v_1,v_2 \text{ in V.}$$
\subsubsection*{Bewijs}
We moeten dus aantonen dat:
\begin{enumerate}
\item $L(v_1+v_2) = L(v_1) + L(v_2) \ \forall v_1,v_2 \in V$
\item $L(\lambda v) = \lambda L(v) \ \forall v \in V \ \text{en} \ \forall \lambda \in \mathbb{R}$
\end{enumerate}
\begin{align*}L(\lambda_1v_1+\lambda_2v_2) = \lambda_1L(v_1)+\lambda_2L(V_2)\ \text{voor alle $\lambda_1,\lambda_2 \in \mathbb{R}$ en voor alle $v_1,v_2 \in V$}. \tag{3.}
\end{align*}

\subsubsection*{$\Rightarrow$}
Kies $\lambda_1,\lambda_2 \in \mathbb{R}$ en $v_1,v_2 \in V$ willekeurig.
\begin{align*}
L(\lambda_1 v_1 + \lambda_2 v_2) = L(\lambda_1 v_1) + L(\lambda_2 v_2)\tag{wegens 1.}\\
=\lambda_1 L(v_1)+ \lambda_2 L(v_2) \tag{wegens 2.}
\end{align*}

\subsubsection*{$\Leftarrow$ voor 1.}
Kies $v_1,v_2 \in V$ willekeurig.

\begin{align*}
L(v_1 + v_2) = L(1.v_1+1.v_2) \tag{co\"effici\"ent 1 in V}\\
=1L(v_1)+1L(v_2) \tag{wegens 3.}\\
=L(v_1)+L(v_2) \tag{co\"effici\"ent 1 in W}
\end{align*}

\subsubsection*{$\Leftarrow$ voor 2.}
Neem $v\in V, \lambda \in \mathbb{R}$ willekeurig.
\begin{align*}
L(\lambda v)=L(\lambda v + 0) \tag{neutraal element in V}\\
=L(\lambda v + 0v)\tag{Lemma 3.8}\\
=\lambda L(v) + 0 L(v) \tag{wegens 3.}\\
=\lambda L(v) + 0 \tag{Lemma 3.8 in W}\\
=\lambda L(v) \tag{neutraal element in W}
\end{align*}

\subsection{Gevolg 4.3 p 130}
Zij $L:V\rightarrow W$ een lineaire afbeelding.
\subsubsection*{Te Bewijzen}
\begin{enumerate}
\item
\[\forall v \in V: L(\vec{0})=\vec{0}\]
\item
\[\forall v \in V: L(-v)=-L(v)\]
\item
\[\forall v_i\in V, \lambda_i \in \mathbb{R}: L\left(\sum_{i=1}^n\lambda_iv_i\right) = \sum_{i=1}^n\lambda_iL(v_i)\]
\item
Een lineaire afbeelding ligt volledig vast door de beelden van een basis.
\end{enumerate}
\subsubsection*{Bewijs}
\begin{proof}
Direct bewijs.
\begin{enumerate}
\item
Uit lineariteit van een lineaire afbeelding\footnote{Zie Definitie 4.1 p 130} (duh) volgt het volgende.
\[
L(\vec{0}) = L(\vec{0} + \vec{0}) = L(\vec{0}) + L(\vec{0})
\]
Detail: het volgende geldt enkel omdat $L(\vec{0})$ een element is uit een vectorruimte\footnote{Zie Lemma 3.7 p 93}.
\[
L(\vec{0}) = L(\vec{0}) + L(\vec{0}) \Rightarrow L(\vec{0}) = \vec{0}
\]
\item
We beginnen met iets triviaal.
\[\vec{0}=\vec{0}\]
Zie vorig deel van dit bewijs.
\[L(\vec{0}) = \vec{0}\]
Tegengesteld element in een vectorruimte\footnote{Zie Definitie 3.2 p 88 puntje 3}
\[L(v + (-v)) = \vec{0}\]
Uit lineariteit van een lineaire afbeelding\footnote{Zie Definitie 4.1 p 130} (duh) volgt het volgende.
\[L(v) + L(-v) = \vec{0}\]
Deze overgang lijk triviaal maar het werkt omdat $L(v)$ een element van een vectorruimte is.
\[L(-v) = - L(v)\]
\item
Dit volgt eenvoudig uit de definitie van lineaire afbeelding\footnote{Zie Definitie 4.1 p 130}. Kies willekeurige $v_i$ en $\lambda_i$.
\[L\left(\sum_{i=1}^n\lambda_iv_i\right) =  \sum_{i=1}^nL(\lambda_iv_i)= \sum_{i=1}^n\lambda_iL(v_i)\]
\item
Elke vector $v\in V$ kan geschreven worden als lineaire combinatie van de basisvectoren van $\beta = \{e_1,e_2,...,e_n\}$. Als we nu de basis van $V$ afbeelden krijgen we $L(\beta) = \{L(e_1),L(e_2),...,L(e_n)\}$ We bewijzen nu dat elke vector in $W$ geschreven kan worden als lineaire combinatie van $L(\beta)$.
\[
\forall v \in V:\exists\lambda_i \in \mathbb{R}: v= \sum\lambda_iv_i
\]
\[
\forall v \in V:\exists\lambda_i \in \mathbb{R}: L(v)= L(\sum\lambda_iv_i)
\]
\[
\forall v \in V:\exists\lambda_i \in \mathbb{R}: L(v)= \sum\lambda_iL(v_i)
\]
\end{enumerate}
\end{proof}

\subsection{Voorbeeld 4.4 p 131}
\begin{enumerate}
\item
\begin{enumerate}[a)]
\item De projectie op een as is lineair.
\begin{proof}
\[
p_x \left(\lambda_1
\begin{pmatrix}
u_{1x}\\u_{1y}
\end{pmatrix} 
+ \lambda_2
\begin{pmatrix}
u_{2x}\\u_{2y}
\end{pmatrix} \right)
 = 
p_x \left(
\begin{pmatrix}
\lambda_1u_{1x}\\\lambda_1u_{1y}
\end{pmatrix} 
+ 
\begin{pmatrix}
\lambda_2u_{2x}\\\lambda_2u_{2y}
\end{pmatrix} \right)
=
p_x \left(
\begin{pmatrix}
\lambda_1u_{1x}+\lambda_2u_{2x}\\\lambda_1u_{1y}+\lambda_2u_{2y}\\
\end{pmatrix} \right)
\]
\[
= 
\begin{pmatrix}
\lambda_1u_{1x}+\lambda_2u_{2x}\\0\\
\end{pmatrix}
=
\begin{pmatrix}
\lambda_1u_{1x}\\0\\
\end{pmatrix}
+
\begin{pmatrix}
\lambda_2u_{2x}\\0\\
\end{pmatrix}
=
\lambda_1
\begin{pmatrix}
u_{1x}\\0\\
\end{pmatrix}
+
\lambda_2
\begin{pmatrix}
u_{2x}\\0\\
\end{pmatrix}
\]
\[
=
\lambda_1p_x(u_1) + \lambda_2p_x(u_2)
\]
\end{proof}
\item De spiegeling t.o.v. een as is lineair.
\begin{proof}
\[
s_x \left(\lambda_1
\begin{pmatrix}
u_{1x}\\u_{1y}
\end{pmatrix} 
+ \lambda_2
\begin{pmatrix}
u_{2x}\\u_{2y}
\end{pmatrix} \right)
 = 
s_x \left(
\begin{pmatrix}
\lambda_1u_{1x}\\\lambda_1u_{1y}
\end{pmatrix} 
+ 
\begin{pmatrix}
\lambda_2u_{2x}\\\lambda_2u_{2y}
\end{pmatrix} \right)
=
s_x \left(
\begin{pmatrix}
\lambda_1u_{1x}+\lambda_2u_{2x}\\\lambda_1u_{1y}+\lambda_2u_{2y}\\
\end{pmatrix} \right)
\]
\[
= 
\begin{pmatrix}
\lambda_1u_{1x}+\lambda_2u_{2x}\\-(\lambda_1u_{1y}+\lambda_2u_{2y})\\
\end{pmatrix} 
=
\begin{pmatrix}
\lambda_1u_{1x}\\-\lambda_2u_{2x}\\
\end{pmatrix}
+
\begin{pmatrix}
\lambda_2u_{2x}\\-\lambda_2u_{2y}\\
\end{pmatrix}
=
\lambda_1
\begin{pmatrix}
u_{1x}\\-u_{2x}\\
\end{pmatrix}
+
\lambda_2
\begin{pmatrix}
u_{2x}\\-u_{2y}\\
\end{pmatrix}
\]
\[
=
\lambda_1s_x(u_1) + \lambda_2s_x(u_2)
\]
\end{proof}
(De spiegeling t.o.v. een rechte die niet door de oorsprong gaat is niet lineair.)
\item De puntspiegeling t.o.v. de oorsprong is lineair.
\begin{proof}
\[
s_O \left(\lambda_1
\begin{pmatrix}
u_{1x}\\u_{1y}
\end{pmatrix} 
+ \lambda_2
\begin{pmatrix}
u_{2x}\\u_{2y}
\end{pmatrix} \right)
 = 
s_O \left(
\begin{pmatrix}
\lambda_1u_{1x}\\\lambda_1u_{1y}
\end{pmatrix} 
+ 
\begin{pmatrix}
\lambda_2u_{2x}\\\lambda_2u_{2y}
\end{pmatrix} \right)
=
s_O \left(
\begin{pmatrix}
\lambda_1u_{1x}+\lambda_2u_{2x}\\\lambda_1u_{1y}+\lambda_2u_{2y}\\
\end{pmatrix} \right)
\]
\[
= 
\begin{pmatrix}
-(\lambda_1u_{1x}+\lambda_2u_{2x})\\-(\lambda_1u_{1y}+\lambda_2u_{2y})\\
\end{pmatrix} 
=
\begin{pmatrix}
-\lambda_1u_{1x}\\-\lambda_2u_{2x}\\
\end{pmatrix}
+
\begin{pmatrix}
-\lambda_2u_{2x}\\-\lambda_2u_{2y}\\
\end{pmatrix}
=
\lambda_1
\begin{pmatrix}
-u_{1x}\\-u_{2x}\\
\end{pmatrix}
+
\lambda_2
\begin{pmatrix}
-u_{2x}\\-u_{2y}\\
\end{pmatrix}
\]
\[
=
\lambda_1s_O(u_1) + \lambda_2s_O(u_2)
\]
\end{proof}
(De puntspiegeling t.o.v. een ander punt is niet lineair.)
\item De homothetie met factor $\lambda$ en centrum de oorsprong is lineair.
\begin{proof}
\[
h_\lambda \left(\lambda_1
\begin{pmatrix}
u_{1x}\\u_{1y}
\end{pmatrix} 
+ \lambda_2
\begin{pmatrix}
u_{2x}\\u_{2y}
\end{pmatrix} \right)
 = 
h_\lambda \left(
\begin{pmatrix}
\lambda_1u_{1x}\\\lambda_1u_{1y}
\end{pmatrix} 
+ 
\begin{pmatrix}
\lambda_2u_{2x}\\\lambda_2u_{2y}
\end{pmatrix} \right)
=
h_\lambda \left(
\begin{pmatrix}
\lambda_1u_{1x}+\lambda_2u_{2x}\\\lambda_1u_{1y}+\lambda_2u_{2y}\\
\end{pmatrix} \right)
\]
\[
= 
\begin{pmatrix}
\lambda\lambda_1u_{1x}+\lambda\lambda_2u_{2x}\\\lambda\lambda_1u_{1y}+\lambda\lambda_2u_{2y})\\
\end{pmatrix} 
=
\begin{pmatrix}
\lambda\lambda_1u_{1x}\\\lambda\lambda_2u_{2x}\\
\end{pmatrix}
+
\begin{pmatrix}
\lambda\lambda_2u_{2x}\\\lambda\lambda_2u_{2y}\\
\end{pmatrix}
=
\lambda_1
\begin{pmatrix}
\lambda u_{1x}\\\lambda u_{2x}\\
\end{pmatrix}
+
\lambda_2
\begin{pmatrix}
\lambda u_{2x}\\\lambda u_{2y}\\
\end{pmatrix}
\]
\[
=
\lambda_1p_x(u_1) + \lambda_2p_x(u_2)
\]
\end{proof}
(een homothetie waarbij het centrum niet de oorsprong is is geen lineaire afbeelding.)
\end{enumerate}
\item Translaties volgens een vector verschillend van de nulvector zijn \textbf{geen} lineaire afbeeldingen.
\begin{proof}
De nulvector $\vec{0}$ wordt niet op zichzelf afgebeeld.
\end{proof}
\item
\begin{enumerate}[a)]
\item De identieke afbeelding is een lineaire afbeelding.
\begin{proof}
\[
Id(\lambda_1v_1+\lambda_2v_2) = \lambda_1v_1+\lambda_2v_2 = \lambda_1Id(v_1)+\lambda_2Id(v_2))
\]
\end{proof}
\item De nulafbeelding is een lineaire afbeelding.
\begin{proof}
\[
(0)(\lambda_1v_1+\lambda_2v_2) = \vec{0} = \lambda_1(0)(v_1)+\lambda_2(0)(v_2))
\]
\end{proof}
\item De $(+1)$ afbeelding is lineair in de binaire ringruimte.
\begin{proof}
Dit is geen belangrijk bewijs, maar het is makkelijk in te zien.
\[
(+1)(\lambda_1v_1+\lambda_2v_2) = \lambda_1(+1)(v_1)+\lambda_2(+1)(v_2))
\]
\end{proof}
\end{enumerate}
\item De afgeleide afbeelding is een lineaire afbeelding
\begin{proof}
De optelling en de scalaire vermenigvuldiging gaan door de afgeleide.
\[
D(\lambda_1v_1+\lambda_2v_2) = \lambda_1D(v_1)+\lambda_2D(v_2))
\]
\end{proof}
\item Elke lineaire combinatie van afgeleiden is een lineare afbeelding.
\begin{proof}
Dit volgt meteen uit het vorige bewijs.
\end{proof}
\item De bepaalde integraal is een lineaire afbeelding.
\begin{proof}
De optelling en de scalaire vermenigvuldiging gaan door de bepaalde integraal.
\[
\int_a^b(\lambda_1v_1+\lambda_2v_2) = \lambda_1\int_a^b(v_1)+\lambda_2\int_a^b(v_2))
\]
\end{proof}
\item 
\begin{itemize}
\item
De volgende afbeelding is een lineaire afbeelding.
\[f: \mathbb{R}^2 \rightarrow \mathbb{R}: (x,y) \mapsto ax+by\]
\begin{proof}
\[
f(\lambda_1v_1+\lambda_2v_2) = (a\lambda_1v_{1x}+b\lambda_1(v_{1y})+(a\lambda_2v_{2x}+b\lambda_2v_{2y})
\]
\[
= \lambda_1(av_{1x}+bv_{1y})+\lambda_2(av_{2x}+bv_{2y}) = \lambda_1f(v_1)+\lambda_2f(v_2))
\]
\end{proof}
\item De volgende afbeelding is niet lineair
\[g: \mathbb{R}^2 \rightarrow \mathbb{R}: (x,y) \mapsto ax+by+c\]
\begin{proof}
De nulvector $\vec{0}$ wordt niet op zichzelf afgebeeld maar op $c$.
\end{proof}
\end{itemize}
\item De projecties $p_i$ die co\"ordinaatvectoren afbeelden op individuele co\"ordinaten zijn lineaire vormen.
\begin{proof}
Het is duidelijk dat deze afbeelding op $(\mathbb{R},\mathbb{R},+)$ wordt afgebeeld. Nu rest er ons nog te bewijzen dat $p_i$ lineair is.
\[
p_i(\lambda_1v_1+\lambda_2v_2) = (\lambda_1v_1+\lambda_2v_2)_i
= \lambda_1v_{1i}+\lambda_2v_{2i} = \lambda_1p_i(v_1)+\lambda_2p_i(v_2))
\]
\end{proof}
\item De spoorafbeelding is een lineaire afbeelding.
\[
Tr:\mathbb{R}^{n\times n}\rightarrow \mathbb{R}:A\mapsto Tr(A)=a_{11}+a_{22}+...+a_{nn}
\]
\begin{proof}
\[
Tr(\lambda_1v_1+\lambda_2v_2) = (\lambda_1v_1+\lambda_2v_2)_{11}+(\lambda_1v_1+\lambda_2v_2)_{22}+...+(\lambda_1v_1+\lambda_2v_2)_{nn}
\]
\[
\lambda_1 (v_{1_{11}}+v_{1_{22}}+...+v_{1_{nn}}) + \lambda_2 (v_{2_{11}}+v_{2_{22}}+...+v_{2_{nn}}) = \lambda_1Tr(v_1)+\lambda_2Tr(v_2))
\]
\end{proof}
\item De overgang is een lineaire afbeelding.
Te Bewijzen:
\[
\forall \lambda_1, \lambda_2 \in \mathbb{R}, v_1,v_2 \in \mathbb{R}^3: S(\lambda_1v_1+\lambda_2v_2) = \lambda_1 S v_1+\lambda_2Sv_2)
\]
\begin{proof}
Kies willekeurige $\lambda_1, \lambda_2 \in \mathbb{R}$ en $v_1,v_2 \in \mathbb{R}^3$:
\[
S(\lambda_1v_1+\lambda_2v_2)
= S(\lambda_1v_1)+S(\lambda_2v_2)
= \lambda_1 S (v_1)+\lambda_2S(v_2)
\]
Dit geldt door de lineairiteit van de linkse vermenigvuldiging met een matrix van de juiste afmetingen \footnote{Zie Eigenschap 1.19 p 29.} \footnote{Zie Eigenschappen 1.22 p 31}.
\end{proof}

Zij $X$ de vector die voorstelt hoeveel klanten er van elk merk zijn.
$A\cdot X$ is dan diezelfde vector na \'e\'en jaar. $A^n\cdot X$ is de verwachtte vector na $n$ jaar.

\item 
\emph{Te Bewijzen}\\
Elke matrix $A \in \mathbb{R}^{m\times n}$ bepaalt een lineaire afbeelding.
\emph{Bewijs}\\
\begin{proof}
Dit is een vrij belangrijk maar eenvoudig bewijs.\\
Kies willekeurige $\lambda_1, \lambda_2 \in \mathbb{R}$ en $v_1,v_2 \in \mathbb{R}^3$:
\[
L(\lambda_1v_1+\lambda_2v_2) = A(\lambda_1v_1+\lambda_2v_2) = A(\lambda_1v_1)+A(\lambda_2v_2)
\]
\[
 = \lambda_1A(v_1)+\lambda_2A(v_2) =\lambda_1 L(v_1)+\lambda_2L(v_2)
\]
Zie de eigenschappen van de matrixvermenigvuldiging in hoofdstuk 1 om te zien waarom dit geldt.
\end{proof}

\item Bewijs dat dit een lineaire afbeelding is. In het vorige deel van deze voorbeelden hebben we bewezen dat elke matrix een lineaire afbeelding bepaald. Beschouw de rijvectorn $I$, $V$ en $W$ als rijen van een matrix $A$ dan is $A$ de matrix die de beschreven lineaire afbeelding bepaalt.

\item Bewijs dat $\Delta$ een lineare afbeelding is.
%TODO BEWIJS

\end{enumerate}

\subsection{Voorbeeld 4.6}
\subsubsection*{Te Bewijzen}
De co\"ordinaatafbeelding ten opzichte van elke basis is niet enkel bijectief, maar ook lineair en dus een isomorfisme.
\subsubsection*{Bewijs}
\begin{proof}
Zij $co_{\beta}$ de co\"ordinaatafbeelding van $V$ ten opzichte van een willekeurige basis $\beta = \{e_1,...,e_n\}$.
Kies willekeurige $\mu_1,\mu_2 \in \mathbb{R}$ en $v_1,v_2 \in V$. Zij $\nu_1,...,\nu_n$ de coordinaten van $\mu_1v_1+\mu_2v_2$ ten opzichte van $\beta$.
\[
(\nu_1,...,\nu_n)
= co_{\beta}(\mu_1v_1+\mu_2v_2)
= (\mu_1\lambda_{11}+ \mu_2\lambda_{21},...,\mu_1\lambda_{12}+ \mu_2\lambda_{22})
\]
\[
= \mu_1(\lambda_{11},...,\lambda_{12})
+ \mu_2(\lambda_{21},...,\lambda_{22})
= \mu_1co_{\beta}(v_1)+\mu_2co_{\beta}(v_2)
\]
\end{proof}

\subsection{Opmerking 4.7}
Zij $(\mathbb{R},V,+)$ een vectorruimte.
Zij $L_{\alpha}^{\beta} = A$ de matrix van basisverandering van een basis $\alpha$ van $V$ naar een basis $\beta$ van $V$.

\subsubsection*{Te Bewijzen}
$L_{\alpha}^{\beta}$ is uniek.

\subsubsection*{Bewijs}
\begin{proof}
Bewijs uit het ongerijmde.\\
Stel dat er nog een matrix $A'$ bestaat die $L$ bepaalt ten opzichte van de gegeven basissen. $(A\neq A)$
Kies een willekeurige co\"ordinaatvector $X$ van een vector $v\in V$ ten opzichte van $\alpha$. Beschouw $A$ en $A'$ als rijen van kolomvectoren $(a_1,...,a_n)$ en $(a_1',...,a_n')$.
\[
A \cdot X = L(X) = A'\cdot X
\]
\[
\sum_{i=1}^na_iX = \sum_{i=1}^na_i'X
\]
\[
A=A'
\]
Dit is in contradictie met de aanname dat $A$ en $A'$ verschillend zijn.
\end{proof}


\subsection{Propositie 4.10 p 141}
Zij $(\mathbb{R},V,+)$ en $(\mathbb{R},W,+)$ vectorruimten. Zij $f$ en $g$ lineaire afbeeldingen $V\rightarrow W$.
\subsubsection*{Te Bewijzen}
\begin{enumerate}
\item De volgende afbeelding is lineair.
\[f+g: V \rightarrow W: v\mapsto(f+g)(v)=f(v)+g(v)\]

\item De volgende afbeelding is eveneens lineair.
\[\lambda f: V\rightarrow W: v\mapsto (\lambda f)(v) = \lambda f(v)\]

\item De verzameling van alle lineaire afbeeldingen van $V$ naar $W$ is een lineaire afbeelding.

\end{enumerate}
\subsubsection*{Bewijs}
\begin{proof}
Direct bewijs.
\begin{enumerate}
\item
We veri\"eren de lineariteit van deze afbeelding.
\[
(f+g)(\lambda_1v_1+\lambda_2v_2) = f(\lambda_1v_1+\lambda_2v_2) + g(\lambda_1v_1+\lambda_2v_2)
\]
Nu geldt door de lineariteit van $f$ en $g$ het volgende.
\[
= \lambda_1f(v_1)+\lambda_2f(v_2) + \lambda_1g(v_1)+\lambda_2g(v_2) = \lambda_1(f(v_1)+g(v_1)) + \lambda_2(f(v_1)+g(v_1))
\]
Dit is precies wat nodig is voor lineariteit.
\[
\lambda_1(f+g)(v_1)+\lambda_2(f+g)(v_2)
\]

\item
We veri\"eren de lineariteit van deze afbeelding.
\[
(\lambda f)(\lambda_1v_1+\lambda_2v_2) = \lambda f(\lambda_1v_1+\lambda_2v_2)
\]
De volgende gelijkheid geldt door de lineariteit van $f$. 
\[
=  f(\lambda_1\lambda v_1+\lambda_2\lambda v_2) = \lambda_1\lambda f(v_1)+\lambda_2\lambda f(v_2) = 
\lambda_1(\lambda f)(v_1)+\lambda_2(\lambda f)(v_2)
\]

\item
De verzameling van alle lineaire afbeelding is een deelverzameling van alle afbeeldingen (wat een vectorruimte is.). De verzameling van alle lineaire afbeeldingen bevat de nulafbeelding (nulvector) en is intern \footnote{Zie Stelling 4.13 p 142}.

\end{enumerate}
\end{proof}


\subsection{Stelling 4.12 p 142}
Zij $(\mathbb{R},V,+)$ en  $(\mathbb{R},W,+)$ eindigdimensionale vecotrruimten van dimensie respectievelijk $n$ en $m$.

\subsubsection*{Te Bewijzen}
Na basiskeuze in $V$ en $W$ is de vectorruimte van de lineaire afbeeldingen van $V$ naar $W$, namelijk $Hom_{\mathbb{R}}(V,W)$, in bijectief verband met de vectorruimte van de matrices $\mathbb{R}^{m\times n}$.

\subsubsection*{Bewijs}
\begin{proof}
Na basiskeuze beschrijft elke lineaire afbeelding van $V$ naar $W$ precies \'e\'en matrix $A\in \mathbb{R}^{n\times m}$. Deze matrix is bovendien uniek \footnote{Zie Opmerking 4.7 p 138.}.
\end{proof}


\subsection{Stelling 4.13 p 142}
Zij $K:U\rightarrow V$ en $L:V\rightarrow W$ lineaire afbeeldingen.
\subsubsection*{Te Bewijzen}
De samenstelling van twee lineaire afbeeldingen is opnieuw een lineaire afbeelding.
\[
\forall u_1,u_2\in U \lambda_1,\lambda_2\in\mathbb{R}: (L\circ K)(\lambda_1u_1 + \lambda_2u_2) = \lambda_1(L\circ K)u_1 + \lambda_2(L\circ K)u_2
\]
\subsubsection*{Bewijs}
\begin{proof}
Kies willekeurige $u_1,u_2\in U$ en $\lambda_1,\lambda_2\in\mathbb{R}$
\[
(L\circ K)(\lambda_1u_1 + \lambda_2u_2) = L(K(\lambda_1u_1 + \lambda_2u_2))
\]
Bovenstaande gelijkheid geldt volgens de definitie van samengestelde afbeeldingen. Zie \ref{samenstelling_van_afbeeldingen}. De volgende gelijkheden gelden door de lineariteit van $L$ en $K$
\[
= L(\lambda_1K(u_1) + \lambda_2K(u_2)) = \lambda_1L(K(u_1)) + \lambda_2L(K(u_2))
\]
Opnieuw volgens de definitie van samengestelde afbeeldingen geldt de volgende vergelijking.
\[
= \lambda_1(L\circ K)u_1 + \lambda_2(L\circ K)u_2
\]
Dit betekent precies dat de samenstelling van lineaire afbeeldingen lineair is.
\end{proof}

\subsection{Stelling 4.16 p 145}
Zij $L:V\rightarrow W$ een bijectieve lineaire afbeelding van de vectorruimte $(\mathbb{R},V,+)$ naar de vectorruimte $(\mathbb{R},W,+)$.
\subsubsection*{Te Bewijzen}
De inverse afbeelding van $L$ is (noem het $K$) is linear.
\subsubsection*{Bewijs}
\begin{proof}
$K$ is ook bijectief. Te bewijzen is nu nog dat $K$ lineair is.
\[
(L\circ K)(\lambda_1u_1 + \lambda_2u_2) = \lambda_1u_1 + \lambda_2u_2 = L(\lambda_1K(u_1) + \lambda_2K(u_2))
\]
Omdat $L$ injectief is geldt de volgende vergelijking.
\[
K(\lambda_1u_1 + \lambda_2u_2) = \lambda_1K(u_1) + \lambda_2K(u_2)
\]
\end{proof}

\subsection{Propositie 4.27 p 155}
Zij $L: V\rightarrow W$ een lineaire afbeelding.
\subsubsection*{Te Bewijzen}
\begin{enumerate}
\item De kern van $L$ is een deelruimte van $V$.
\[
\forall l_1l_2\in ker(L), \lambda_1,\lambda_2\in \mathbb{R}: \lambda_1l_1+\lambda_2l_2 \in ker(L)
\]
\item Het beeld van $L$ is een deelruimte van $W$.
\[
\forall l_1l_2\in Im(L), \lambda_1,\lambda_2\in \mathbb{R}: \lambda_1l_1+\lambda_2l_2 \in Im(L)
\]
\item $Im(L)$ is de vectorruimte voortgebracht door de beelden van een basis van $V$.
\[
Im(L) = vct\{L(e)|e\in \beta\}
\]
\end{enumerate}
\subsubsection*{Bewijs}
\begin{proof}
Rechtstreeks bewijs.
\begin{enumerate}
\item Kies twee elementen $v,w \in ker(L)$. We weten nu dat $L(v)=L(w)=0$ geldt volgens de definitie van de kern van een lineaire afbeelding. Kies nu twee willekeurige $\lambda_1,\lambda_2 \in \mathbb{R}$.
\[
 L(0) = \lambda_1L(v)+\lambda_2L(w) = L(\lambda_1v+\lambda_2w) = \vec{0}
 \Rightarrow  (\lambda_1v+\lambda_2w) \in ker(L)
\]
Volgens de lineariteit van $L$ geldt de tweede gelijkheid\footnote{Zie Lemma 4.2 p 130}.
\item Kies twee elementen $w_1,w_2 \in Im(L)$ en twee willekeurige $\lambda_1,\lambda_2 \in \mathbb{R}$.
\[
\lambda_1w_1+\lambda_2w_2 = \lambda_1L(v_1) + \lambda_2L(v_1) = L(\lambda_1v_1+\lambda_2v_2) \in Im(L)
\]
De eerste gelijkheid geldt omwille van de definitie van $Im(L)$ en de tweede omwille van de lineariteit van $L$\footnote{Zie Lemma 4.2 p 130}.
\item Zij $\beta = \{e_1,e_2,...,e_n\}$ een basis van $V$, dan is elke vector $v\in V$ te schrijven als een lineaire combinatie van de vectoren in $\beta$\footnote{Zie Stelling 3.41 p 109}. We beschouwen nu het beeld van een willekeurige $v$.
\[
L(v) = L(\lambda_1e_1+\lambda_2e_2 + ... + \lambda_ne_n) = \lambda_1L(e_1)+\lambda_2L(e_2) + ... + \lambda_nL(e_n)
\] 
De tweede gelijkheid geldt volgens de lineariteit van $L$\footnote{Zie Lemma 4.2 p 130}.
\end{enumerate}
\end{proof}

\subsection{Stelling 4.29 p 156}
Zij $L:V\rightarrow W$ een lineaire afbeelding.
\subsubsection*{Te Bewijzen}
\begin{center}
$L$ is injectief.
\end{center}
\[\Leftrightarrow\]
\[
ker(L) = \{\vec{0}\}
\]
\subsubsection*{Bewijs}
\begin{proof}
Bewijs van een equivalentie in twee delen.
\begin{itemize}
\item $\Rightarrow$\\
Neem een willekeurige $v\in ker(L)$. Voor $v$ geldt dat $L(v) = \vec{0}$ omwille van de definitie van de kern van een lineaire afbeelding\footnote{Zie Definitie 4.26 p 155}. Omdat $L$ een lineaire afbeelding is geldt dat $L(\vec{0}) = \vec{0}$.
\[
L(v) = \vec{0} = L(\vec{0})
\]
Vanwege de injectiviteit\ref{injectief} van $L$ geldt nu dat $v=\vec{0}$ voor elke $v \in ker(L)$. Dit betekent dat in $ker(L)$ enkel de nulvector zit.
\item $\Leftarrow$\\
Kies twee vectoren $v_1,v_2 \in V$ zodat $L(v_1)=L(v_2)$. We moeten nu aantonen dat dit impliceert dat $v_1$ en $v_2$ gelijk zijn.
\[
L(v_1)=L(v_2)
\]
\[
L(v_1)-L(v_2) = \vec{0}
\]
\[
L(v_1-v_2) = \vec{0}
\]
Bovenstaande bewering houdt in dat $v_1-v_2 \in ker(L)$. Omdat $ker(L) = \{\vec{0}\}$ geldt het volgende.
\[
v_1-v_2 = \vec{0}
\]
\[
v_1=v_2
\]
Dit houdt in dat $L$ injectief is.
\end{itemize}
\end{proof}

\subsection{Stelling 4.31 p 157}
Zij $(\mathbb{R},V,+)$ en $(\mathbb{R},W,+)$ vectorruimten zijn waarvan $V$ eindig dimensionaal is. Zij $L:V\rightarrow W$ een lineaire afbeelding.
\subsubsection*{Te Bewijzen}
\[
dimV = dim(Ker(L)) + dim(Im(L))
\]
\subsubsection*{Bewijs}
\begin{proof}
Bewijs van dimensies.\\
We kiezen een basis van $Ker(L)$ en breiden deze uit tot een basis van $V$. Dit kan omdat $V$ eindig dimensionaal is en omdat $Ker(L)$ een deelruimte is van $V$\footnote{Zie Propositie 4.27 p 156}. Kies $\beta = \{v_1,v_2,...,v_r\}$ als basis van $Ker(L)$. We breiden $\beta$ nu uit tot $\beta'= \{v_1,v_2,...,v_r,v_{r+1},...,v_{n}\}$ door $n-r$ vectoren toe te voegen. 
We zullen nu bewijzen dat de uitbreiding $\{v_{r+1},...,v_{n}\}$ een basis is van $Im(L)$. We weten het volgende over het beeld van $L$.
\[
Im(L) = vct\{L(v_1),L(v_2),...,L(v_r),L(v_{r+1}),...,L(v_{n})\}
\]
Omdat $v_1,v_2,...,v_r$ in $Ker(L)$ zitten weten we het volgende.
\[
Im(L) = vct \{L(v_{r+1}),...,L(v_{n})\}
\]
We beweren nu dat de verzameling $\{L(v_{r+1}),...,L(v_{n})\}$ inderdaad vrij is.
\[
\vec{0}=\sum_{i=r+1}^nx_iL(v_i)=L\left(\sum_{i=r+1}^nx_iv_i\right)
\]
Bovenstaande bewering houdt precies in dat $\sum_{i=r+1}^nx_iv_i\in Ker(L)$ geldt. Omdat $\sum_{i=r+1}^nx_iv_i$ is dus een lineaire combinatie van $\{v_1,v_2,...,v_r\}$ en bijgevolg bestaan er $\lambda_i$ zodat het volgende geldt.
\[
\sum_{i=r+1}^nx_iv_i = \sum_{i=1}^r\lambda_iv_i
\]
\[
\sum_{i=r+1}^nx_iv_i + \sum_{i=1}^r(-\lambda_i)v_i = \vec{0}
\]
Omdat $\{v_1,v_2,...,v_r,v_{r+1},...,v_{n}\}$ een basis is van $V$ geldt dat alle $x_i$ en $\lambda_i$ gelijk zijn aan nul. $\{L(v_{r+1}),...,L(v_{n})\}$ is dus een basis van $Im(L)$
De basis van $Im(L)$ en $(Ker(L))$ bevatten respectievelijk $n-r$ en $r$ vectoren. Bijgevolg geldt het volgende en is de stelling bewezen.
\[
dimV = n = r + (n-r) = dim(Ker(L)) + dim(Im(L))
\]
\end{proof}

\subsection{Stelling 4.34 p 159}
Zij $(\mathbb{R},V,+)$ en $(\mathbb{R},W,+)$ eindig dimensionale vectorruimten en zijn $L:V\rightarrow W$ een lineaire afbeelding.
\subsubsection*{Te Bewijzen}
Er bestaat een basis $\alpha$ van $V$ en een basis $\beta$ van $W$ zodat de matrix van $L$ t.o.v. deze basissen van de volgende vorm is.
\[
\begin{pmatrix}
1 & 0 & \cdots & 0 & 0 & \cdots & 0\\
0 & 1 & \cdots & 0 & 0 & \cdots & 0\\
\vdots & \vdots & \ddots & \vdots & \vdots & \ddots & \vdots\\
0 & 0 & \cdots & 1 & 0 & \cdots & 0\\
0 & 0 & \cdots & 0 & 0 & \cdots & 0\\
\vdots & \vdots & \ddots & \vdots & \vdots & \ddots & \vdots\\
0 & 0 & \cdots & 0 & 0 & \cdots & 0\\
\end{pmatrix}
=
\begin{pmatrix}
\mathbb{I}_r & 0\\
0 & 0\\
\end{pmatrix}
\]
Hierin is $r$ de rang van de lineaire afbeelding $L$.

\subsubsection*{Bewijs}
\begin{proof}
TODO, hoe?
\end{proof}

\subsection{Gevolg 4.35 p 159}
Zij $L: V\rightarrow V : v\mapsto L(v)$ een lineaire transformatie is van de eindig dimensionale vectorruimte $(\mathbb{R},V,+)$.
\subsubsection*{Te Bewijzen}
\begin{center}
$L$ is injectief $\Leftrightarrow L$ is surjectief $\Leftrightarrow L$ is bijectief
\end{center}
\subsubsection*{Bewijs}
\begin{proof}
Bewijs door circulaire implicaties.
\begin{itemize}
\item $L$ is injectief $\Rightarrow L$ is surjectief.\\
Omdat $L$ injectief is, zal de kern van $L$ enkel de nulvector bevatten\footnote{Zie Stelling 4.29 p 156}.
\[Ker(L) = \{\vec{0}\}\]
$dim(Ker(L))$ is dus gelijk aan $0$. Uit de dimensiestelling volgt dan dat $dim(Im(L)) = dim(V)$ geldt. Omdat $Im(L)$ een deelruimte is van $V$\footnote{Zie Propositie 4.27 p 155} geldt dat $Im(L) = V$. Dit houdt in dat $L$ surjectief is.

\item $L$ is surjectief $\Rightarrow L$ is bijectief.\\
Omdat $L$ surjectief is geldt $Im(L)=V$. Uit de dimensiestelling volgt dan dat $dim(Ker(L)) = 0$\footnote{Zie Propositie 4.27 p 155}. Bijgevolg geldt $Ker(L) = \{\vec{0}\}$. Dit houdt precies in dat $V$ injectief is\footnote{Zie Stelling 4.29 p 156}. Omdat $L$ injectief en surjectief is geldt dat $L$ bijectief is(Zie \ref{bijectief}).

\item $L$ is bijectief $\Rightarrow L$ is injectief.\\
Dit volgt meteen uit de definitie van bijectiviteit (Zie \ref{bijectief}).
\end{itemize}
\end{proof}

\subsection{Gevolg 4.39 p 160}
\subsubsection*{Te Bewijzen}
\subsubsection*{Bewijs}
\begin{proof}

\end{proof}

\subsection{Stelling 4.40 p 161}
Zij $A \in \mathbb{R}^{m\times n}$.
\subsubsection*{Te Bewijzen}
\begin{center}
$A\cdot X = b$ is oplosbaar voor elke $b\in \mathbb{R}^m$
\end{center}
\[\Leftrightarrow\]
\[C(A) = \mathbb{R}^m\]
\[\Leftrightarrow\]
\begin{center}
De rang van $A$ is $m$
\end{center}
\[\Leftrightarrow\]
\begin{center}
De rijen van $A$ zijn lineair onafhankelijk.
\end{center}
\subsubsection*{Bewijs}
\begin{proof}
Bewijs door circulaire implicaties.
\begin{itemize}
\item
$A\cdot X = b$ is oplosbaar voor elke $b\in \mathbb{R}^m \Rightarrow C(A) = \mathbb{R}^m$.\\
Als elke $b\in \mathbb{R}^m$ gevormd kan worden door lineaire combinatie van de kolommen van $A$ houdt dit precies in dat de kolomruimte van $A$ heel $\mathbb{R}^m$ is.
\item
$C(A) = \mathbb{R}^m \Rightarrow $ De rang van $A$ is $m$.\\
We weten dat de kolomrang gelijk is aan de rijrang en de rang van $A$. De kolomrang is gelijk aan de dimensie van de kolomruimte. Deze is duidelijk $m$. De rang van $A$ dus ook.

\item
 De rang van $A$ is $m \Rightarrow$ De rijen van $A$ zijn lineair onafhankelijk.\\
Het aantal lineair onafhankelijk rijen van $A$ is precies gelijk aan de rang van $A$. Er zijn dus nul lineaire afhankelijke rijen in $A$. De rijen van $A$ zijn bijgevolg lineair onafhankelijk.

\item
De rijen van $A$ zijn lineair onafhankelijk. $\Rightarrow A\cdot X = b$ is oplosbaar voor elke $b\in \mathbb{R}^m$.

Als de rijen van $A$ lineair onafhankelijk zijn komen er geen nulrijen tevoorschijn na rijreductie onafhankelijk van $b$.
\end{itemize}
\end{proof}

\subsection{Stelling 4.41 p 161}
Zij $A \in \mathbb{R}^{m\times n}$
\subsubsection*{Te Bewijzen}
\begin{center}
Voor elke $b \in \mathbb{R}^m$ heeft het stelsel $A\cdot X = b$ hoogstens \'e\'en oplossing.
\end{center}
\[\Leftrightarrow\]
\[N(A) = \{\vec{0}\}\]
\[\Leftrightarrow\]
\begin{center}
De rang van $A$ is gelijk aan $m$.
\end{center}
\[\Leftrightarrow\]
\begin{center}
De kolommen van $A$ zijn lineair onafhankelijk.
\end{center}
\subsubsection*{Bewijs}
\begin{proof}
Bewijs door circulaire implicaties.
\begin{itemize}
\item
Voor elke $b \in \mathbb{R}^m$ heeft het stelsel $A\cdot X = b$ hoogstens \'e\'en oplossing.
 $\Rightarrow N(A) = \{\vec{0}\}$\\
$A\cdot X = b$ heeft dus ook hoogstens \'e\'en oplossing voor $b = \vec{0}$. Die \'ene oplossing moet de nulvector zijn omdat  p $N(A)$ een deelruimte is\footnote{Propositie 4.27 p 155} en de kleinste deelruimte enkel de nulvector bevat.
\item
$N(A) = \{\vec{0}\} \Rightarrow $ De rang van $A$ is gelijk aan $n$.\\
TODO
\item
De rang van $A$ is gelijk aan $m$. $\Rightarrow$ De kolommen van $A$ zijn lineair onafhankelijk.\\
TODO
\item
De kolommen van $A$ zijn lineair onafhankelijk. $\Rightarrow$ Voor elke $b \in \mathbb{R}^m$ heeft het stelsel $A\cdot X = b$ hoogstens \'e\'en oplossing.
TODO
\end{itemize}
\end{proof}

\subsection{Stelling 4.42 p 161}
Zij $A\in \mathbb{R}^{n \times n}$ een vierkante matrix.
\subsubsection*{Te Bewijzen}
\begin{center}
A is inverteerbaar.
\end{center}
\[\Leftrightarrow\]
\[det(A) \neq 0\]
\[\Leftrightarrow\]
\begin{center}
De rang van $A$ is gelijk aan $n$.
\end{center}
\[\Leftrightarrow\]
\begin{center}
De rijen van $A$ zijn lineair onafhankelijk.
\end{center}
\[\Leftrightarrow\]
\begin{center}
De kolommen van $A$ zijn lineair onafhankelijk.
\end{center}
\subsubsection*{Bewijs}
\begin{proof}
Bewijs door circulaire implicaties.
\begin{itemize}
\item
A is inverteerbaar. $\Rightarrow det(A) \neq 0$\\
Dit is een stelling die al bewezen is\footnote{Zie Stelling 2.4 p 59}.
\item
$det(A) \neq 0 \Rightarrow $ De rang van $A$ is gelijk aan $n$.\\
TODO
\item
De rang van $A$ is gelijk aan $n$. $\Rightarrow $ De rijen van $A$ zijn lineair onafhankelijk.\\
TODO
\item
De rijen van $A$ zijn lineair onafhankelijk. $\Rightarrow$ De kolommen van $A$ zijn lineair onafhankelijk.\\
Als de rijen van $A$ lineair onafhankelijk zijn dan geldt $det(A) \neq 0$ en omdat $det(A) = det(A^T)$ zijn de kolommen dus ook onafhankelijk.
\item
De kolommen van $A$ zijn lineair onafhankelijk. $\Rightarrow$ A is inverteerbaar.\\
TODO
\end{itemize}
\end{proof}

\subsection{Stelling 4.43 p 164}
Zij $L:V\rightarrow W$ een lineaire afbeelding van $(\mathbb{R},V,+)$ naar $(\mathbb{R},W,+)$ die het probleem $L(x) = w_0$ bepaalt.
\subsubsection*{Te Bewijzen}
\begin{enumerate}
\item Dit lineair probleem heeft een oplossing als en slechts als $w_0 \in Im(L)$.

\item
Als het probleem oplosbaar is en $x_p$ een oplossing is dan wordt de oplossingsverzameling de volgende.
\[
X = \{x_p + x_h | x_h \in Ker(L)\}
\]
\end{enumerate}

\subsubsection*{Bewijs}
\begin{proof}
\begin{enumerate}
\item Stel dat het probleem oplosbaar zou zijn met $w_0$ een oplossing maar dat $w_0 \not\in Im(L)$.

Omdat $w_0$ een oplossing is bestaat er een $x$ zodat $L(x) = w_0$ geldt. Omdat $w_0$ niet in $Im(L)$ zit bestaat er geen $x$ zodat $L(x) = w_0$. Dit is contradictorisch.
\item
We bewijzen eerst dat de verzameling $X$ een deelverzameling is van de oplossingsverzameling. Daarna bewijzen we dat de oplossingsverzameling een deelverzameling is van $X$.\\\\
Zij $x_p$ een particuliere oplossing van het lineair probleem, dan is elke vector van de vorm $x_p+x_h$ met $x_h\in Ker(L)$ ook een oplossing.
\[
L(x_p+x_h)=L(x_p)+L(x_h)=w_0+\vec{0} = w_0
\]
De eerste gelijkheid geldt omwille van de lineariteit van $L$\footnote{Zie Definitie 4.1 p 130}. De derde gelijkheid geldt omwille van het neutraal element van een vectorruimte\footnote{Zie Definitie 3.2 p 88}.\\\\
Zij $x$ een willekeurige oplossing van het lineaire probleem ($L(x)=w_0)$), dan kunnen we $x$ schrijven als volgt.
\[
x = x_p + (x-x_p)
\]
Hierboven is $x_p$ een particuliere oplossing van het lineair probleem. Omdat $L(x-x_p) = L(x) - L(x_p) = w_0-w_0=\vec{0}$ geldt, zit $(x-x_p)$ in de kern van $L$.
\end{enumerate}
\end{proof}

\end{document}